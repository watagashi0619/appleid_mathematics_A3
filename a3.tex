\documentclass[dvipdfmx,a4j,10pt]{jsarticle}
\usepackage{amsthm}%定理環境
\usepackage{newtxtext,newtxmath}%times font setting
\usepackage{amsmath,amsfonts}%数式font setting
\usepackage{mathrsfs}%花文字
\usepackage{pxrubrica}%ルビ
%\usepackage{lmodern}%font setting
\usepackage{mathtools}%数式の相互参照部分にのみ式番号をふる
%\mathtoolsset{showonlyrefs,showmanualtags}
\usepackage{empheq}%方程式
\usepackage{physics}%物理系
\usepackage{bm}%vector
\usepackage{mleftright}%カッコ
\usepackage{framed}%フレーム
\usepackage{color}%文字に色付け
\usepackage{ulem}%取り消し線
\usepackage{tcolorbox}
\tcbuselibrary{breakable,theorems,skins}%数式を「デコ」る
\usepackage{tikz}
\usetikzlibrary{cd}%矢印の図
\usetikzlibrary{decorations,decorations.pathreplacing}%tikz-cdを「デコ」る
\usepackage{bbm}%指示関数とかはこっちで出す
%\usepackage[right]{showlabels}%数式番号を左に表示
\usepackage{booktabs}%表の上線\toprule中線\midrule下線\bottomrule
%\usepackage{url}%参考文献とかでurlをかくとき
\usepackage[dvipdfmx]{hyperref}%ハイパーリンク
\usepackage{comment}
%\usepackage{lscape}%ページを横向きにする
\usepackage{listings,jlisting}
\lstset{
    language=python,
    frame=single,
    basicstyle=\ttfamily\footnotesize,
    commentstyle=\textit,
    classoffset=1,
    keywordstyle=\bfseries,
    framesep=5pt,
    showstringspaces=false,
    numbers=left,
    stepnumber=1,
    lineskip=-0.2ex,
    numberstyle=\tiny,
    tabsize=2,
    breaklines = true,
    framexleftmargin=17pt
}

\makeatletter%subsubsubsection定義
    \newcommand{\subsubsubsection}{\@startsection{paragraph}{4}{\z@}%
        {1.0\Cvs \@plus.5\Cdp \@minus.2\Cdp}%
        {.1\Cvs \@plus.3\Cdp}%
        {\reset@font\sffamily\normalsize}
    }
\makeatother

\setcounter{secnumdepth}{3}%セクションの深さレベル
%\headheight = 0mm
%\textheight = 205mm
%\footskip = 10mm
%\usepackage{fancyhdr}
%\pagestyle{fancy}
%穴埋め定義
\newcounter{BCntr}
\setcounter{BCntr}{1}
\def\blank#1{
\underline{{¥color{red}{#1}}{¥tiny ¥theBCntr}}
\stepcounter{BCntr}
}%

%physicsでrotもcurlで使う
\newcommand{\rot}{\curl}

%amsthm style1,2 微分積分学B準拠
%例
%定理1.1(中間値の定理)
%定理=\thmname{#1} 1.1=\thmnumber{#2} (中間値の定理)=\thmnote{(#3)}
\newtheoremstyle{mystyle1}% Name
    {}% Space above
    {}% Space below
    {\normalfont}% Body font
    {}% Indent amount
    {\bfseries\sffamily}% Theorem head font
    {\hspace{0.5em}}% Punctuation after theorem head
    { }% Space after theorem head, ‘ ‘, or \newline
    {\thmname{#1}\thmnumber{#2}\thmnote{:\,#3\\}}% Theorem head spec (can be left empty, meaning `normal')
\theoremstyle{mystyle1}
\newtheorem{definition}{定義}[section]
\newtheorem{theorem}[definition]{定理}
\newtheorem{lemma}[definition]{補題}
\newtheorem{proposition}[definition]{命題}
\newtheorem{formula}[definition]{公式}

\newtheoremstyle{mystyle3}% Name
    {}% Space above
    {}% Space below
    {\normalfont}% Body font
    {}% Indent amount
    {\bfseries\sffamily}% Theorem head font
    {\hspace{0.5em}}% Punctuation after theorem head
    { }% Space after theorem head, ‘ ‘, or \newline
    {\thmname{#1}\thmnumber{#2}\thmnote{:\,#3\\}}% Theorem head spec (can be left empty, meaning `normal')
\theoremstyle{mystyle3}
\newtheorem{example}{例}[section]

\newtheoremstyle{mystyle4}% Name
    {}% Space above
    {}% Space below
    {\normalfont}% Body font
    {}% Indent amount
    {\bfseries\sffamily}% Theorem head font
    {\hspace{0.5em}}% Punctuation after theorem head
    { }% Space after theorem head, ‘ ‘, or \newline
    {\thmname{#1}\thmnumber{#2}\thmnote{:\,#3\\}}% Theorem head spec (can be left empty, meaning `normal')
\theoremstyle{mystyle4}
\newtheorem{note}{注意}[section]


\newtheoremstyle{mystyle2}% Name
    {}% Space above
    {}% Space below
    {\normalfont}% Body font
    {}% Indent amount
    {\bfseries\sffamily}% Theorem head font
    {\hspace{0.5em}}% Punctuation after theorem head
    { }% Space after theorem head, ‘ ‘, or \newline
    {\thmname{#1}\thmnote{:\,#3\\}}% Theorem head spec (can be left empty, meaning `normal')
\theoremstyle{mystyle2}
\newtheorem{dfn*}{定義}
\newtheorem{thm*}{定理}
\newtheorem{exercise*}{演習問題}
\newtheorem{ex*}{例}
\newtheorem{qes*}{疑問}
\newtheorem{rem*}{注意}
\newtheorem{ans*}{解答例}
\newtheorem{answer*}{解答}
\newtheorem{note*}{注意}
\newtheorem{remark*}{補足}
\newtheorem{lem*}{補題}
\newtheorem{symbol*}{記号}
\newtheorem{premise*}{前提}
\newtheorem{proposition*}{命題}

%proofの後ろに黒四角をつける
\makeatletter
\renewenvironment{proof}[1][\proofname]{\par
  \pushQED{\qed}%
  \normalfont
  \topsep6\p@\@plus6\p@ \trivlist
  \item[\hskip\labelsep{\bfseries\sffamily #1}]\ignorespaces
}{%
  \popQED\endtrivlist\@endpefalse
}
\renewcommand\proofname{証明}
\renewcommand{\qedsymbol}{\ensuremath{\blacksquare}}
\makeatother

\makeatletter
\renewenvironment{ans*}[1][解答例]{\par
  \pushQED{\qed}%
  \normalfont
  \topsep6\p@\@plus6\p@ \trivlist
  \item[\hskip\labelsep{\bfseries\sffamily #1}]\ignorespaces
}{%
  \popQED\endtrivlist\@endpefalse
}
\renewcommand{\qedsymbol}{\ensuremath{\blacksquare}}
\makeatother

\definecolor{shadecolor}{gray}{0.9}

\newcommand{\defLeftrightarrow}{\overset{\text{def}}{\iff}}

% \prtlabel{label_strings}
\def\startpoint#1#2{
    {\hfill\rlap{\quad{$\overline{\tt{#1}\ \tt{#2}}$}}}\vspace{-1.5\baselineskip}
}

\newcommand\range{\operatorname{range}}
\newcommand\sgn{\operatorname{sgn}}

\newcommand{\blueunderline}[3][pos=0.5]{
    \tcboxmath[
        enhanced,
        frame hidden, % 枠を消す
        interior hidden, % 背景を消す
        size=minimal, % 余白を消す
        overlay={
                \draw[
                    blue,
                    decorate
                ] ([yshift=-4pt]frame.south west) -- ([yshift=-4pt]frame.south east)
                node[#1,scale=0.72,below] {#3};
            }
    ]{#2}
}

\newcommand{\bluenotearrow}[2]{
    \tcboxmath[
        enhanced,
        frame hidden, % 枠を消す
        interior hidden, % 背景を消す
        size=minimal, % 余白を消す
        overlay={
                \draw[<-,blue] ([yshift=-1ex]frame.south) to[out=-90,in=90] +(0,-3ex) %+(幅,上下の位置)
                node[below,scale=0.72] {#2};
            }
    ]{\,\,\, #1\,\,\,}
}

\newcommand{\bluenote}[2]{
    \tcboxmath[
        enhanced,
        frame hidden, % 枠を消す
        interior hidden, % 背景を消す
        size=minimal, % 余白を消す
        overlay={
                \node[blue,below,scale=0.72] at ([yshift=-1ex]frame.south) {#2};
            }
    ]{\,\,\, #1\,\,\,}
}

\newcommand{\bluenoteunderleft}[2]{
    \tcboxmath[
        enhanced,
        frame hidden, % 枠を消す
        interior hidden, % 背景を消す
        size=minimal, % 余白を消す
        overlay={
                \draw[<-,blue] ([yshift=-1ex,xshift=-0.5em]frame.south) to[out=-120,in=0] +(-2em,-2ex)
                node[left,scale=0.72] {#2};
            }
    ]{\,\,\, #1\,\,\,}
}

\newcommand{\bluenoteoverleft}[2]{
    \tcboxmath[
        enhanced,
        frame hidden, % 枠を消す
        interior hidden, % 背景を消す
        size=minimal, % 余白を消す
        overlay={
                \draw[<-,blue] ([yshift=1ex,xshift=-0.5em]frame.north) to[out=120,in=0] +(-2em,2ex)
                node[left,scale=0.72] {#2};
            }
    ]{\,\,\, #1\,\,\,}
}

%\renewcommand{\thepart}{\arabic{part}}
%\renewcommand{\thenote}{}
%\renewcommand{\thelem}{}

%\makeatletter
%\@addtoreset{section}{part}
%\makeatother

%\makeatletter
%\def\blfootnote{\xdef\@thefnmark{}\@footnotetext}
%\makeatother

\makeatletter
\@addtoreset{equation}{section}
\def\theequation{\thesection.\arabic{equation}}% renewcommand でもOK
\makeatother

%\setcounter{tocdepth}{1}%目次をsectionまでの表示にする

\renewcommand{\labelenumi}{(\roman{enumi})}

%デコりamsthm
\newenvironment{dfn}[1][]
{\begin{tcolorbox}[
    enhanced,
    boxrule=0pt,
    arc=0mm,
    frame hidden,
    borderline west={2pt}{-4pt}{green!60!black},
    breakable = true
    ]
    \begin{definition}[#1]
}
{\end{definition}\end{tcolorbox}}

\newenvironment{lem}[1][]
{\begin{tcolorbox}[
    enhanced,
    boxrule=0pt,
    arc=0mm,
    frame hidden,
    borderline west={2pt}{-4pt}{yellow!90!black},
    breakable = true
    ]
    \begin{lemma}[#1]
}
{\end{lemma}\end{tcolorbox}}

\newenvironment{prop}[1][]
{\begin{tcolorbox}[
    enhanced,
    boxrule=0pt,
    arc=0mm,
    frame hidden,
    borderline west={2pt}{-4pt}{blue!50!black},
    breakable = true
    ]
    \begin{proposition}[#1]
}
{\end{proposition}\end{tcolorbox}}

\newenvironment{thm}[1][]
{\begin{tcolorbox}[
    enhanced,
    boxrule=0pt,
    arc=0mm,
    frame hidden,
    borderline west={2pt}{-4pt}{red},
    breakable = true
    ]
    \begin{theorem}[#1]
}
{\end{theorem}\end{tcolorbox}}

\newenvironment{fml}[1][]
{\begin{tcolorbox}[
    enhanced,
    boxrule=0pt,
    arc=0mm,
    frame hidden,
    borderline west={2pt}{-4pt}{orange},
    breakable = true
    ]
    \begin{formula}[#1]
}
{\end{formula}\end{tcolorbox}}

\newenvironment{ex}[1][]
{\begin{tcolorbox}[
    enhanced,
    boxrule=0pt,
    arc=0mm,
    frame hidden,
    borderline west={0.25pt}{-4pt}{black},
    borderline west={0.25pt}{-2.25pt}{black},
    breakable = true
    ]
    \begin{example}[#1]
}
{\end{example}\end{tcolorbox}}

\newenvironment{prop*}[1][]
{\begin{tcolorbox}[
    enhanced,
    boxrule=0pt,
    arc=0mm,
    frame hidden,
    borderline west={2pt}{-4pt}{blue!50!black},
    breakable = true
    ]
    \begin{proposition*}[#1]
}
{\end{proposition*}\end{tcolorbox}}

\title{工業数学A3}
\author{}
\date{更新日時:\today}%コメントアウトした状態だと今日の日付が入る

\begin{document}
\maketitle
\tableofcontents%目次

\newpage

\section{フーリエ級数展開}

\subsection{導入}

$f(t)$を$\mathbb{R}$上の関数とする$(t\in\mathbb{R})$.

\begin{dfn*}[周期関数]
    $f$が周期$T$の周期関数であるとは,$\forall t\in\mathbb{R}$に対して
    \[
        f(t+T)=f(t)\quad(t\in\mathbb{R})
    \]
    が成り立つこと.
\end{dfn*}
このとき,$\forall n\in\mathbb{Z}$に対して
\[
    f(t+nT)=f(t)\quad(t\in\mathbb{R})
\]
が成り立つ.

\begin{dfn*}[フーリエ級数展開]
    \begin{equation}\label{eq:1-1}
        f(t)=c+\sum_{n=1}^\infty a[n]\cos(n\omega t)+\sum_{n=1}^\infty b[n]\sin(n\omega t)
    \end{equation}
    ここで$\omega=2\pi/t$,$c,a[n],b[n]$は定数.
\end{dfn*}

これが収束すれば周期$T$の周期関数である.広いクラスの,周期$T$の周期関数は\eqref{eq:1-1}の形に表現できる.

\subsection{三角関数の直交関係とフーリエ係数}

\eqref{eq:1-1}における$c,a[n],b[n]$をフーリエ係数とよぶ.

\begin{thm}[三角関数の直交関係]\label{thm:1-1}
    $n,m$を正の整数とする.
    \begin{equation}\label{eq:1-2}
        \frac{2}{T}\int_0^T\cos(n\omega t)\cos(m\omega t) dt=\delta_{nm}
    \end{equation}
    \begin{equation}\label{eq:1-3}
        \frac{2}{T}\int_0^T\sin(n\omega t)\sin(m\omega t) dt=\delta_{nm}
    \end{equation}
    \begin{equation}\label{eq:1-4}
        \frac{2}{T}\int_0^T\sin(n\omega t)\cos(m\omega t) dt=0
    \end{equation}
    \begin{equation}\label{eq:1-5}
        \frac{2}{T}\int_0^T\cos(n\omega t) dt=\frac{2}{T}\sin(n\omega t) dt=0
    \end{equation}
    ここで,$\delta_{nm}$はクロネッカーのデルタである.
\end{thm}

\begin{proof}
    省略.
\end{proof}

\begin{fml}\label{fml:1-2}
    周期$T$の周期関数$f(t)$が
    \begin{equation}\label{eq:1-6}
        f(t)=\frac{a[0]}{2}+\sum_{n=1}^\infty a[n]\cos(n\omega t)+\sum_{n=1}^\infty a[n]\sin(n\omega t)
    \end{equation}
    と表されるならば,フーリエ係数は
    \begin{equation}\label{eq:1-7}
        a[n]=\frac{2}{T}\int_0^T f(t)\cos(n\omega t)dt\quad(n=0,1,2,\ldots)
    \end{equation}
    \begin{equation}\label{eq:1-8}
        b[n]=\frac{2}{T}\int_0^T f(t)\sin(n\omega t)dt\quad(n=0,1,2,\ldots)
    \end{equation}
    と与えられる.
\end{fml}

周期性により
\begin{equation}\label{eq:1-9}
    a[n]=\frac{2}{T}\int_{-T/2}^{T/2} f(t)\cos(n\omega t)dt\quad(n=0,1,2,\ldots)
\end{equation}
\begin{equation}\label{eq:1-10}
    b[n]=\frac{2}{T}\int_{-T/2}^{T/2} f(t)\sin(n\omega t)dt\quad(n=0,1,2,\ldots)
\end{equation}
とかける.$f$が偶関数ならば$b[n]=0\quad(n=0,1,2,\ldots)$となり,\textbf{余弦フーリエ級数展開}:
\[
    f(t)=\frac{a[0]}{2}+\sum_{n=1}^\infty a[n]\cos(n\omega t)
\]
$f$が奇関数ならば$a[n]=0\quad(n=0,1,2,\ldots)$となり,\textbf{正弦フーリエ級数展開}:
\[
    f(t)=\sum_{n=1}^\infty b[n]\cos(n\omega t)
\]
をもつといわれる.

\subsection{複素フーリエ変換}

オイラーの公式を用いれば,フーリエ級数展開\eqref{eq:1-6}は
\[
    f(t)=\frac{a[0]}{2}+\sum_{n=1}^\infty\mleft(\frac{a[n]}{2}+\frac{b[n]}{2i}\mright)e^{in\omega t}+\sum_{n=1}^\infty\mleft(\frac{a[n]}{2}-\frac{b[n]}{2i}\mright)e^{-in\omega t}
\]
となる.そこで
\[
    c[n]=\begin{cases}
        a[0]/2 & (n=0) \\
        (a[n]-ib[n])/2 & (n>0) \\
        (a[-n]+ib[-n])/2 & (n<0)
    \end{cases}
\]
とすると\textbf{複素フーリエ変換}:
\begin{equation}\label{eq:1-11}
    f(t)=\sum_{n=-\infty}^\infty c[n]e^{in\omega t}
\end{equation}
が得られる.公式\ref{fml:1-2}もしくは直接計算により
\[
    c[n]=\frac{1}{T}\int_0^T f(t)e^{-in\omega t}dt\quad(n\in\mathbb{Z})
\]
が確かめられる.

\subsection{いくつかの実例}

\begin{ex}[三角多項式]
    三角多項式($\sin\omega t,\cos\omega t$の多項式).
\end{ex}

例えば
\[
    f_1(t)=(\cos\omega t)^3=\frac{3}{4}\cos\omega t+\frac{1}{4}\cos 3\omega t
\]

一般的な三角多項式に対しては,$N\in\mathbb{N}$が存在して
\[
    f(t)=\sum_{n=-N}^N c[n]e^{in\omega t}
\]
の形にかける.

\begin{ex}[三角波]
    \[
        f_2(t)=|T|\quad(|t|\leq T/2)
    \]
\end{ex}

$f_2(t)$は偶関数であるから,余弦フーリエ級数展開をもち
\[
    a[n]=\frac{2}{T}\int_{-T/2}^{T/2}|t|\cos(n\omega t)dt=\frac{4}{T}\int_0^{T/2}t\cos(n\omega t)dt
\]
$n=0$のとき
\[
    a[0]=\frac{4}{T}\mleft[\frac{t^2}{2}\mright]_0^{T/2}=\frac{T}{2}
\]
$n\neq 0$のとき
\[
    a[n]=\begin{cases}
        0 & (nは偶数) \\
        -\frac{4}{\pi n^2 \omega } & (nは奇数)
    \end{cases}
\]
つまり,$f_2(t)$のフーリエ級数展開は
\[
    f_2(t)=\frac{T}{4}-\frac{4}{\pi\omega}\sum_{m=0}^\infty \frac{1}{(2m+1)^2}\cos\{(2m+1)\omega t\}
\]
となる.

\begin{ex}{}
    \[
        f_3(t)=\frac{1}{\frac{5}{4}+\cos\omega t}
    \]
\end{ex}
留数定理を用いてフーリエ係数を計算すると,$n\geq 0$として
\[
    c[-n]=\frac{4}{3}(-2)^n
\]
$n>0$として$c[n]=c[-n]$より
\[
    c[n]=\frac{4}{3}(-2)^{-n}
\]
となるから,$f_3(t)$のフーリエ級数展開は
\[
    f_3(t)=\frac{4}{3}+\frac{8}{3}\sum_{n=1}^\infty (-2)^n\cos(n\omega t)
\]
となる.

\subsection{フーリエ級数の一様収束}

\begin{dfn*}[リプシッツ連続(Lipshitz continuous)]
    $\mathbb{R}$上の関数$f(t)$が\textbf{リプシッツ連続}であるとは,
\[
    \exists C>0\, \textrm{s.t.}\, \forall t,s\in\mathbb{R},\, |f(t)-f(s)|\leq C|t-s|
\]
が成り立つことである.
\end{dfn*}

\begin{thm}\label{thm:1-3}
    $f(t)$がリプシッツ連続ならば,フーリエ級数展開は$f(t)$に一様収束する.
\end{thm}

\begin{proof}
    は後回し.
\end{proof}

\subsection{有限フーリエ級数}

$X=\mathbb{C}^N$を$N$次元複素線形空間とする.$X$の元を$u=(u[0],u[1],\cdots,u[N-1])\in X$と書く.

\begin{dfn*}[エルミート内積]
    $u,v\in X$のエルミート内積:
\[
    \langle u,v\rangle=\sum_{n=0}^{N-1}u[n]\overline{v[n]}
\]
\end{dfn*}

\begin{dfn*}[長さ]
    $u\in X$の長さ:
    \[
        |u|=\sqrt{\langle u,u\rangle}
    \]
\end{dfn*}

\begin{dfn*}[正規直交系/正規直交基底]
    $u_0,\ldots,u_{M-1}\in X$が正規直交系であるとは
    \[
        \langle u_m,u_n\rangle=\delta_{mn}\quad(n,m=0,1,\ldots,M-1)
    \]
    が成り立つこと.  

    $M=N=\dim X$のとき正規直交基底とよばれ,$\forall u\in X$は
    \[
        u=\sum_{n=0}^{N-1}\langle u,u_n \rangle u_n
    \]
    と直交分解できる.$\mathbb{C}^N$の標準的な基底$\bm{e}_n[k]=\delta_{kn}\quad(n,k=0,\ldots,N-1)$は正規直交基底である.
\end{dfn*}

有限フーリエ変換の定義に用いられる正規直交基底は,$\alpha=2\pi/N$として
\[
    \varphi_n[k]=\frac{1}{\sqrt{N}}\exp(i\alpha nk)\quad(n,k=0,1,\ldots,N-1)
\]
で定義される.

\begin{prop}\label{prop:1-4}
    $\{\varphi_0,\ldots,\varphi_{N-1}\}$は$X(=\mathbb{C}^N)$の正規直交基底である.
\end{prop}

\begin{proof}
    $n=m$のとき
    \[
        |\varphi_n|^2=\langle\varphi_n,\varphi_n\rangle=\frac{1}{N}\sum_{k=0}^{N-1}|e^{i\alpha nk}|^2=1
    \]
    $n\neq m$のとき
    \[
        \langle\varphi_n,\varphi_m\rangle=\frac{1}{N}\sum_{k=0}^{N-1}e^{i\alpha nk}e^{-i\alpha mk}=\frac{1}{N}\frac{1-e^{i\alpha(n-m)N}}{1-e^{i\alpha n-m}}\bluenotearrow{=}{$\alpha N=2\pi$}0
    \]
\end{proof}

$u\in X$に対して\textbf{有限フーリエ変換}:
\begin{equation}\label{eq:1-12}
    \hat u[n]=\langle u,\varphi_n \rangle=\frac{1}{\sqrt{N}}\sum_{k=0}^{N-1}u[k]e^{-i\alpha nk}\quad(n=0,\ldots,N-1)
\end{equation}
とおくと,\textbf{有限フーリエ級数展開}($\hat u[n]$の\textbf{逆有限フーリエ変換}):
\begin{equation}\label{eq:1-13}
    u[k]=\sum_{n=0}^{N-1}\hat u[n]\varphi_n[k]=\frac{1}{\sqrt{N}}\sum_{n=0}^{N-1}\hat u[n]e^{i\alpha nk}\quad(n=0,\ldots,N-1)
\end{equation}
と直交展開される.

有限フーリエ変換,逆有限フーリエ変換はユニタリーな線形変換である.すなわち:
\begin{equation}\label{eq:1-14}
    |u|=|\hat u|
\end{equation}
が成り立つ.これは
\begin{equation}\label{eq:1-15}
    \langle u,v\rangle=\langle \hat u,\hat v\rangle
\end{equation}
であることから従う.

\subsection{有限フーリエ変換の連続極限}

\begin{thm}\label{thm:1-5}
    $f(t)$を周期$T$のリプシッツ連続な周期関数とし,$\omega=2\pi/T, \alpha=2\pi/N$とする.
    \[
        f_N(t)=\sum_{-N/2\leq n<N/2} c_N[n]e^{in\omega t}
    \]
    \[
        c_N[n]=\frac{1}{N}\sum_{m=0}^{N-1}f\mleft(\frac{m}{N}T\mright)e^{-i\alpha nm}\quad(n\in\mathbb{Z})
    \]
    と定める.このとき,$f_N(t)$は$f(t)$に一様収束する.
\end{thm}

\begin{note}\label{note:1-1}
    \[
        f_N\mleft(\frac{k}{N}T\mright)=\sum_{-N/2\leq n<N/2} c_N[n]e^{i\alpha nk}=\sum_{n=0}^{N-1}e_N[n]e^{i\alpha nk}=f\mleft(\frac{k}{N}T\mright)
    \]
    $f_N(t)$は$f(t_k)$の値を与えたときの補間.
\end{note}

\begin{proof}
    まず
    \[
        d_N[n]=\frac{1}{N}\sum_{m=0}^{N-1}\mleft\{f\mleft(\frac{m}{N}T\mright)-f\mleft(\frac{m-1}{N}T\mright)\mright\}e^{-i2\pi (m/N)n}
    \]
    とおく.$\{d_N[n]\}$は$f(t_n)-f(t_n-(T/N))$の有限フーリエ変換の$1/\sqrt{N}$倍である.リプシッツ連続性の仮定より,$\exists C>0$,
    \[
        \mleft|f\mleft(\frac{m}{N}T\mright)-f\mleft(\frac{m-1}{N}T\mright)\mright|\leq\frac{C}{N}\quad(m\in\mathbb{Z})
    \]
    が成り立つ.したがって有限フーリエ変換の等長性\eqref{eq:1-14}より
    \[
        N\sum_{k=0}^{N-1}|d_N[k]|^2\leq \sum_{m=0}^{N-1}\mleft|\frac{C}{N}\mright|^2=\frac{C^2}{N}
    \]
    すなわち
    \[
        \sum_{k=0}^{N-1}|d_N[k]|^2\leq \frac{C^2}{N^2}
    \]
    が従う.一方,$d_N[k]$の定義により
    \[
        d_N[k]=(1-e^{-i 2\pi(k/N)})c_N[k]=2i e^{-i\pi(k/N)}\sin(\pi k/N)c_N[k]
    \]
    これとジョルダンの不等式:
    \begin{equation}\label{eq:1-16}
        |\sin\theta|\geq\frac{2}{\pi}|\theta|\quad(|\theta|\leq \pi/2)
    \end{equation}
    を用いれば
    \[
        |d_N[k]|\geq\frac{4|k|}{N}|c_N[k]|\quad\mleft(-\frac{N}{2}\leq k<\frac{N}{2}\mright)
    \]
    が従う.$d_N[k]$が周期$N$を持つことに注意して,これらを組み合わせると
    \[
        \sum_{-N/2\leq k< N/2}|k|^2|c_N[k]|^2\leq \frac{C^2}{16}
    \]
    が導かれる.
    
    また,
    \[
        f_N'(t)=\sum_n i\omega n c_N[n]e^{i\omega n t}
    \]
    の絶対値を考えると
    \begin{equation}\label{eq:1-17}
        \begin{split}
            |f_N'(t)| & \leq \omega \sqrt{N}\sum_n \frac{n}{\sqrt{N}}|c_N[n]| \\
            & \bluenoteunderleft{\leq}{Schwarzの不等式} \omega\sqrt{N}\mleft(\sum_n\mleft(\frac{1}{\sqrt{N}}\mright)^2\mright)^{1/2} \mleft(\sum_n|n|^2|c_N[n]|^2\mright)^{1/2} \\
            & \leq \sqrt{N}C'\quad (C'はある定数)
        \end{split}
    \end{equation}
    となる.さて,$\forall t\in [0,T]$に対して$|t-t_k|\leq T/(2N)$であるような$t_k=(k/N)T$が存在する.$f(t_k)=f_N(t_k)$に注意して(注意\ref{note:1-1})
    \[
        f(t)-f_N(t)=(f(t)-f(t_k))+(f_N(t_k)-f_N(t))
    \]
    と分解して考える.$f(t)$のリプシッツ連続性と$f_N'$の微分の評価\eqref{eq:1-17}から
    \[
        \begin{split}
            |f(t)-f_N(t)| & \leq |f(t)-f(t_k)|+|f_N(t_k)-f_N(t)| \\
            & \leq \blueunderline{C|t-t_k|}{リプシッツ連続性}+\blueunderline{C'\sqrt{N}|t-t_k|}{平均値の定理} \\[3ex] 
            & =O(1/\sqrt{N})
        \end{split}
    \]
    が得られる.つまり,$N\to\infty$のとき$f_N(t)$が$f(t)$に一様収束することが示された.
\end{proof}

\subsection{関数空間の内積と直交関数系}

%正確にはルベーグ積分が必要だけどリーマン積分でなんとかします

$X$を周期$T$で周期的で有界かつ$[0,T]$上で積分可能な関数全体とする.

\begin{dfn*}[内積]
    $f,g\in X$に対して内積を
    \[
        \langle f,g \rangle=\frac{1}{T}\int_0^T f(t)\overline{g(t)}dt
    \]
    で定義する.
\end{dfn*}

$f,g,h\in X$,$a,b\in\mathbb{C}$のとき
\[
    \langle af+bg,h\rangle=a\langle f,h\rangle +b\langle g,h\rangle
\]
\[
    \langle f,ag+bh\rangle=\overline{a}\langle f,g\rangle +\overline{b}\langle f,h\rangle
\]
\[
    \langle f,g\rangle=\overline{\langle g,f\rangle}
\]

\begin{dfn*}[$L^2$ノルム]
    $f\in X$に対して
    \[
        \|f\|=\sqrt{\langle f,f\rangle}\geq 0
    \]
\end{dfn*}

$\|f\|=0\,\leftrightarrow\,f=0$ $\mu$-$a.e.$である.

\subparagraph{シュワルツの不等式}

\begin{equation}\label{eq:1-18}
    |\langle f,g\rangle|\leq\|f\|\|g\|
\end{equation}

\subparagraph{三角不等式}

\begin{equation}\label{eq:1-19}
    \|f+g\|\leq\|f\|+\|g\|
\end{equation}

\begin{dfn*}[直交]
    $f,g$が直交するとは,
    \[
        \langle f,g\rangle=0
    \]
    となること.
\end{dfn*}

\begin{dfn*}[直交関数系]
    $\{f_n\}_{n=1}^\infty \subset X$が正規直交系であるとは,
    \[
        \langle f_n,f_m\rangle=\delta_{nm}\quad(n,m=1,2,\ldots)
    \]
    となること.
\end{dfn*}

\textbf{フーリエ関数系}を
\[
    \varphi_n(t)=e^{in\omega t}\quad(n\in\mathbb{Z},t\in\mathbb{R})
\]
で定義すると,$\{\varphi_n\}_{n=-\infty}^\infty$は正規直交系となる.このとき,フーリエ級数展開
\[
    f(t)=\sum_{-\infty}^\infty c[n]\varphi_n(t)
\]
\[
    c[n]=\langle f,\varphi_n \rangle=\frac{1}{T}\int_0^T f(t)e^{-in\omega t}dt\quad(n\in\mathbb{Z})
\]
は正規直交系$\{\varphi_n\}$に関する展開であり,フーリエ係数は座標成分である.実フーリエ級数\eqref{eq:1-6}も同様である.

\subsection{正規直交基底とフーリエ級数の平均収束}
$\{f_n\}_{n=1}^\infty$を$X$の正規直交系とする.
\begin{dfn*}[正規直交基底(完全正規直交系)]
    $\{f_n\}_{n=1}^\infty$が正規直交基底もしくは完全正規直交系であるとは,$\forall f\in X$に対して
    \[
        \lim_{N\to\infty} \mleft\|f-\sum_{n=1}^N \langle f,f_n\rangle f_n\mright\|=0
    \]
    であること.\footnote{
        ちなみに,完全のつかない正規直交系として$\{\cos x,\cos 2x,\ldots,\sin x,\sin 2x,\ldots\}$が考えられる.これは例えば$f$に定数をとればわかるように,完全正規直交系(正規直交基底)にはならない.$\{1,\cos x,\cos 2x,\ldots,\sin x,\sin 2x,\ldots\}$なら完全正規直交系(正規直交基底)になる.
    }
\end{dfn*}

\begin{thm}[パーセバルの等式]\label{thm:1-6}
    フーリエ関数系$\{\varphi_n\}_{n=-\infty}^\infty$は$X$の正規直交基底で,$\forall f\in X,\, c[n]=\langle f,\varphi_n\rangle$として
    \begin{equation}\label{eq:1-20}
        \|f\|^2=\sum_{n=-\infty}^\infty |c[n]|^2
    \end{equation}
    が成り立つ.
\end{thm}
\begin{proof}
    はあとで.
\end{proof}

定理\ref{thm:1-6}より,フーリエ級数展開の\textbf{平均収束($L^2$-収束)}:
\[
    \mleft\|f-\sum_{n=-N}^N c[n]\varphi_n\mright\|\to 0\quad(N\to\infty)
\]
がわかる.

\begin{prop}[ベッセルの不等式]\label{prop:1-7}
    $\{f_n\}_{n=1}^\infty$が正規直交系ならば,$\forall f\in X$に対して
    \begin{equation}\label{eq:1-22}
        \sum_{n=1}^\infty |\langle f,f_n\rangle|\leq \|f\|^2
    \end{equation}が
    が成り立つ.
\end{prop}

\begin{proof}
    \[
        \begin{split}
            \mleft\|f-\sum_{n=1}^N\langle f,f_n\rangle f_n\mright\|^2
            &= \mleft\langle f-\sum_{n=1}^N\langle f,f_n\rangle f_n, f-\sum_{n=1}^N\langle f,f_n\rangle f_n\mright\rangle \\
            &=\|f\|^2-\sum_{n=1}^N\langle f,f_n\rangle\langle f_n,f\rangle-\sum_{n=1}^N\overline{\langle f,f_n\rangle}\langle f_n,f\rangle+\sum_{n=1}^N\sum_{m=1}^N\langle f,f_n\rangle\overline{\langle f,f_m\rangle}\langle f_n,f_m\rangle \\
            &=\|f\|^2-\sum_{n=1}^N|\langle f,f_n\rangle|^2 \\
            &\geq 0
        \end{split}
    \]
    より結論を得る.
\end{proof}

\begin{prop}\label{prop:1-8}
    $\{f_n\}_{n=1}^\infty$を正規直交系とする.$\{f_n\}_{n=1}^\infty$が正規直交基底であるための必要十分条件は,$\forall f\in X$について
    \[
        \|f\|^2=\sum_{n=1}^\infty|\langle f,f_n\rangle|^2
    \]
    が成り立つことである.
\end{prop}

\begin{proof}
    命題\ref{prop:1-7}をみると,$\{f_n\}$が正規直交基底ならば
    \[
        \mleft\|f-\sum_{n=1}^N\langle f,f_n\rangle f_n\mright\|^2
        =\|f\|^2-\sum_{n=1}^N|\langle f,f_n\rangle|^2 \overset{n\to\infty}{\longrightarrow} 0
    \]
    が成り立つ.逆に,
    \[
        \lim_{n\to\infty}\mleft(\|f\|^2-\sum_{n=1}^N|\langle f,f_n\rangle|^2\mright)=0
    \]
    ならば
    \[
        \lim_{n\to\infty}\,\mleft\|f-\sum_{n=1}^N\langle f,f_n\rangle f_n\mright\|^2=0
    \]
    であることがわかり,$\{f_n\}$は正規直交基底である.
\end{proof}


\begin{lem}\label{lem:1-9}
    $\{f_n\}$を正規直交系,$a_1,a_2,\ldots,a_N\in\mathbb{C}$,$f\in X$とする.このとき
    \begin{equation}\label{eq:1-23}
        \mleft\|f-\sum_{n=1}^N\langle f,f_n\rangle f_n\mright\|\leq\mleft\|f-\sum_{n=1}^N a_n f_n\mright\|
    \end{equation}
    が成り立つ.
\end{lem}

\begin{proof}
    $g=f-\sum_{n=1}^N\langle f,f_n\rangle f_n$とする.$\langle g,f_m\rangle=0$に注意すると
    \[
        \begin{split}
            \mleft\|f-\sum_n a_n f_n\mright\|^2
            &=\mleft\|g+\sum_n(\langle f,f_n\rangle-a_n)f_n\mright\|^2 \\
            &=\|g\|^2+\mleft\langle g,\sum_n(\langle f,f_n\rangle-a_n)f_n\mright\rangle +\mleft\langle \sum_n(\langle f,f_n\rangle-a_n)f_n,g\mright\rangle +\mleft\|\sum_n(\langle f,f_n\rangle-a_n)f_n\mright\|^2 \\
            &=\|g\|^2+\mleft\|\sum_n(f_n\rangle-a_n)f_n\mright\|^2 \\
            &\geq \|g\|^2
        \end{split}
    \]
    となる.
\end{proof}

\begin{prop*}
    $f$を周期$T$で周期的かつ有界で$[0,T]$上で積分可能な関数とする.任意の$\varepsilon>0$に対して$\|f-g\|<\varepsilon$となるリプシッツ連続な周期関数$g$が存在する.
\end{prop*}

(メモ:ルベーグ積分論を使った証明があれば追記したい)

\begin{proof}
    区間$[0,T]$の分割$\Delta:0=t_0<t_1<\cdots<t_n=T$をとり,周期$T$で周期的な階段関数$f_\Delta (t)$を次式により定める:
    \[
        f_\Delta (t)=\inf_{t_j\leq t<t_{j+1}} f(t)\quad(t_j\leq t<t_{j+1}のとき)
    \]
    $\Delta$を十分細かく取れば,積分の定義により
    \begin{equation}
        \|f-f_\Delta\|=\frac{1}{T}\int_0^T|f(t)-f_\Delta(t)|dt<\frac{\varepsilon}{2} \tag{1}
    \end{equation}
    とできる.上式が成立するように$\Delta$をとって固定する.また,
    \[
        \lim_{s\to 0}\int_0^T|f_\Delta(t-s)-f_\Delta(t)|dt=0
    \]
    であるから,$n_0>0$を十分大きくとれば,$n\geq n_0$のとき
    \begin{equation}
        \sup_{|s|\leq 1/n}\frac{1}{T}\int_0^T|f_\Delta(t-s)-f_\Delta(t)|dt<\frac{\varepsilon}{2} \tag{2}
    \end{equation}
    が成立する.

    任意の$t\in\mathbb{R}$に対して$\phi(t)\geq 0$で,$|t|\geq 1$のとき$\phi(t)=0$かつ
    \[
        \int_{-1}^1 \phi(t)dt=1
    \]
    を満たすリプシッツ連続な関数$\phi(t)$をとり,$\phi_n(t)=n\phi(nt)$とする.このとき
    \begin{equation}
        |t|\geq 1/nのとき\phi_n(t)=0かつ\int_{-1/n}^{1/n}\phi_n(t)dt=1 \tag{3}
    \end{equation}
    となる.関数$f_n(t)$を次式により定める.
    \[
        f_n(t)=\int_{-\infty}^\infty \phi_n(t-s)f_\Delta(s)ds=\int_{-1/n}^{1/n}\phi_n(s)f_\Delta(t-s)ds
    \]
    $f_\Delta (t)$の周期性より
    \[
        f_n(T)=\int_{-1/n}^{1/n}\phi_n(s)f_\Delta(T-s)ds=\int_{-1/n}^{1/n}\phi_n(s)f_\Delta(-s)ds=f_n(0)
    \]
    かつ,十分大きな$n>0$に対して,$L_\phi$を関数$\phi$のリプシッツ定数として
    \[
        |f_n(t_1)-f_n(t_2)|\leq\int_{-\infty}^\infty|\phi_n(t_1-s)-\phi_n(t_2-s)||f_\Delta(s)|ds\leq 4nL_\phi|t_1-t_2|\sup_{0\leq t\leq T}|f(t)|
    \]
    となるから,$f_n(t)$は周期$T$のリプシッツ連続な周期関数である.ここで,$t_1\neq t_2$ならば,$n>0$が十分大きいとき,
    \[
        \max\mleft(t_1-\frac{1}{n},t_2-\frac{1}{n}\mright)>\min\mleft(t_1+\frac{1}{n},t_2+\frac{1}{n}\mright)
    \]
    であること,および式(3)より,
    \[
        \begin{split}
            \int_{-\infty}^\infty|\phi_n(t_1-s)-\phi_n(t_2-s)|ds
            &\leq \int_{t_1-1/n}^{t_1+1/n}|\phi_n(t_1-s)|ds+\int_{t_2-1/n}^{t_2+1/n}|\phi_n(t_2-s)|ds \\
            & =\int_{nt_1-1}^{nt_1+1}|\phi(nt_1-u)|du+\int_{nt_2-1}^{nt_2+1}|\phi(nt_2-u)|du \\
            &\leq \int_{nt_1-1}^{nt_1+1}|\phi(nt_1-u)-\phi(nt_2-u)|du+\int_{nt_2-1}^{nt_2+1}|\phi(nt_1-u)-\phi(nt_2-u)|du \\
            &\leq \int_{nt_1-1}^{nt_1+1} nL_\phi|t_1-t_2|du +\int_{nt_2}^{nt_2+1}nL_\phi|t_1-t_2|du \\
            &\leq 4nL_\phi|t_1-t_2|
        \end{split}
    \]
    となることを用いた.

    式(3)より
    \[
        f_n(t)-f_\Delta(t)=\int_{-1/n}^{1/n} \phi_n(s)(f_\Delta(t-s)-f_\Delta(t))ds
    \]
    であるから,
    \[
        |f_n(t)-f_\Delta(t)|\leq\int_{-1/n}^{1/n}\phi_n(s)|f_\Delta(t-s)-f_\Delta(t)|ds
    \]
    となる.さらに,上式を$t$で積分し,積分を交換すると,$n\geq n_0$のとき
    \[
        \|f_n-f_\Delta\|\leq\frac{1}{T}\int_{-1/n}^{1/n}\phi_n(s)\int_0^T|f_\Delta(t-s)-f_\Delta(s)|dtds<\frac{\varepsilon}{2}
    \]
    が成立する.ここで,式(2)と(3)を用いた.三角不等式により,上式と式(1)とから,$n\geq n_0$のとき,
    \[
        \|f-f_n\|\leq\|f-f_\Delta\|+\|f_\Delta-f_n\|<\varepsilon
    \]
    となり,$g=f_n$とおけば結論を得る.
\end{proof}

\begin{note*}
    上の証明で関数$\phi(t)$を$C^\infty$に取れば,近似関数$g$は$C^\infty$級となる.
\end{note*}

\setcounter{definition}{5}
\begin{thm}[パーセバルの等式(再掲)]
    フーリエ関数系$\{\varphi_n\}_{n=-\infty}^\infty$は$X$の正規直交基底で,$\forall f\in X,\, c[n]=\langle f,\varphi_n\rangle$として
    \begin{equation}\tag{\ref{eq:1-20}}
        \|f\|^2=\sum_{n=-\infty}^\infty |c[n]|^2
    \end{equation}
    が成り立つ.
\end{thm}

\begin{proof}
    $f\in X$,$\forall\varepsilon>0$をとる.すると,上の命題より,リプシッツ連続な周期関数$g$で,$\|f-g\|<\varepsilon/2$をみたすものをとれる.すると,定理\ref{thm:1-5}より,$N$を十分大きくとれば,
    \[
        \mleft|g(t)-\sum_{|n|\leq N}c_N[n]\varphi_n(t)\mright|<\frac{\varepsilon}{2}\quad(t\in [0,T])
    \]
    が成り立つ\footnote{1.6節で$\varphi_n=e^{in\omega t}$で定めている.}.ここで$\{c_N[n]\}$は
    \[
        c_N[n]=\frac{1}{N}\sum_{m=0}^{N-1}g\mleft(\frac{m}{N}T\mright)e^{-i2\pi(m/N)n}\quad(n\in\mathbb{Z})
    \]
    である.これより
    \[
        \mleft\|g-\sum_{|n|\leq N}c_N[n]\varphi_n\mright\|<\frac{\varepsilon}{2}
    \]
    がしたがう.ゆえに
    \[
        \mleft\|f-\sum_{|n|\leq N}c_N[n]\varphi_n\mright\|\leq\|f-g\|+\mleft\|g-\sum_{|n|\leq N}c_N[n]\varphi_n\mright\|<\varepsilon
    \]
    がわかる.ここで,補題\ref{lem:1-9}を用いると
    \[
        \mleft\|f-\sum_{|n|\leq N}\langle f,\varphi_n\rangle\varphi_n\mright\|\leq\mleft\|f-\sum_{|n|\leq N}c_N[n]\varphi_n\mright\|<\varepsilon
    \]
    が導かれる.
\end{proof}

\setcounter{definition}{9}

\subsection{定理\ref{thm:1-3}の証明}

\begin{lem}\label{lem:1-10}
    $f$を周期$T$のリプシッツ連続な周期関数,$\{c[n]\}$を$f$のフーリエ係数とする.このとき
    \[
        \sum_{n=-\infty}^\infty |n|^2|c[n]|^2<\infty
    \]
    が成り立つ.
\end{lem}

\begin{proof}
    リプシッツ連続性より,$C>0$が存在して
    \[
        |f(t+h)-f(t)|\leq C|h|\quad(t,h\in\mathbb{R})
    \]
    が成り立つ.$N\geq 1$に対して$h=T/2N$とおき,$f$の$h$だけ差分関数を
    \[
        g(t)=\frac{1}{h}\{f(t+h)-f(t)\}=\frac{2N}{T}\mleft\{f\mleft(t+\frac{T}{2N}\mright)-f(t)\mright\}
    \]
    と定義すれば,$|g(t)|\leq C$である.$g$のフーリエ係数を$\{d[n]\}$と書くことにしよう.すると,
    \[
        \begin{split}
            d[n]&=\frac{1}{hT}\int_0^T\{f(t+h)-f(t)\}e^{-i\omega nt}dt \\
            &=\frac{1}{h}(e^{i\omega nh}-1)c[n] \\
            &=\frac{2N}{T}e^{i\omega nh/2}\cdot 2i\sin\mleft(\frac{\pi n}{2N}\mright)c[n]
        \end{split}
    \]
    が成り立つ.$g$についてパーセバルの等式を用いれば
    \[
        \|g\|^2=\sum_{n=-\infty}^\infty|d[n]|^2=\mleft(\frac{4N}{T}\mright)^2\sum_{n=-\infty}^\infty\mleft|\sin\mleft(\frac{\pi n}{2N}\mright)\mright|^2|c[n]|^2
    \]
    がわかる.一方,\eqref{eq:1-16}を用いると,
    \[
        |n|\leq N\quad ならば\quad \mleft|\sin\mleft(\frac{\pi n}{2N}\mright)\mright|\geq\frac{2}{\pi}\cdot\frac{\pi n}{2N}=\frac{n}{N}
    \]
    だから,
    \[
        \|g\|^2\geq \frac{16}{T^2}\sum_{n=-N}^N |n|^2|c[n]|^2
    \]
    を得る.$|g(t)|^2\leq C^2$だったので
    \[
        \sum_{n=-N}^N |n|^2|c[n]|^2\leq\frac{T^2C^2}{16}
    \]
    がわかる.右辺は$N$に依らないので,$N\to\infty$として求める不等式が導かれる.
\end{proof}


\begin{lem}\label{lem:1-11}
    $f$を周期$T$のリプシッツ連続な周期関数,$\{c[n]\}$を$f$のフーリエ係数とする.このとき
    \[
        \sum_{n=-\infty}^\infty |c[n]|<\infty
    \]
    が成り立つ.
\end{lem}

\begin{proof}
    $\mathbb{C}^N$でのシュワルツの不等式から
    \[
        \begin{split}
            \sum_{1\leq|n|\leq N}|c[n]|
            &\leq\mleft(2\sum_{n=1}^N\frac{1}{n^2}\mright)^{1/2}\mleft(\sum_{1\leq|n|\leq N}|n|^2|c[n]|^2\mright) \\
            &\leq\mleft(2\sum_{n=1}^\infty\frac{1}{n^2}\mright)^{1/2}\mleft(\sum_{n=-\infty}^\infty |n|^2|c[n]|^2\mright) 
        \end{split}
    \]
    がわかる.右辺は補題\ref{lem:1-10}より,有限の定数で$N$に依らない.したがって,$N\to\infty$として補題の主張が成り立つ.
\end{proof}

\begin{lem}\label{lem:1-12}
    $f$を周期$T$の連続な周期関数,$\{c[n]\}$を$f$のフーリエ係数とする.\\ このとき,$\sum_{n=-\infty}^\infty |c[n]|<\infty$ならば,フーリエ部分和:
    \[
        S_N(t)=\sum_{n=-N}^N c[n]e^{i\omega nt}
    \]
    は$N\to\infty$のとき$f$に一様収束する.
\end{lem}

\begin{proof}
    $N<M$とすると,
    \begin{equation}\label{eq:1-24}
        |S_N(t)-S_M(t)|\leq\sum_{N<|n|\leq M}|c[n]|\leq\sum_{|n|>N}|c[n]|
    \end{equation}
    である.仮定により,$N\to\infty$のとき$\sum_{|n|>N}|c[n]|\to 0$だから,\eqref{eq:1-24}の右辺は$0$に収束する.つまり,各$t$ごとに,$\{S_N(t)\}$はコーシー列であり,極限が存在する.そこで$g(t)\equiv\lim_{N\to\infty}S_N(t)$とおく.\eqref{eq:1-24}で$M\to\infty$とすると
    \[
        |S_N(t)-g(t)|\leq\sum_{|n|>N}|c[n]|
    \]
    となる.右辺は$t$によらず,$N\to\infty$のとき$0$に収束するのだから,$S_N(t)$は$g$に一様収束することになる.これより$\|S_N-g\|\to 0$が従う.一方,定理\ref{thm:1-6}より,$\|S_N-f\|\to 0$だから,$f=g$でなければならない.以上により,$S_N$が$f$に一様収束することが示された. 
\end{proof}

\begin{note*}
    補題\ref{lem:1-12}は定理\ref{thm:1-3}よりも強い主張であり,重要である.
\end{note*}

\begin{note}
    $\{g_N(t)\}$を連続関数列で$N\to\infty$のとき$g_N(t)$は$g(t)$に一様収束するならば,$g(t)$は連続である.
\end{note}

\setcounter{definition}{2}

\begin{thm}[(再掲)]
    $f(t)$がリプシッツ連続ならば,フーリエ級数展開は$f(t)$に一様収束する.
\end{thm}

\begin{proof}
    補題\ref{lem:1-11}と補題\ref{lem:1-12}から直ちに導かれる.
\end{proof}

\setcounter{definition}{12}

\subsection{ギッブス現象と総和法}

\end{document}
