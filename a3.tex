\documentclass[dvipdfmx,a4j,10pt]{jsarticle}
\usepackage{amsthm}%定理環境
\usepackage{newtxtext,newtxmath}%times font setting
\usepackage{amsmath,amsfonts}%数式font setting
\usepackage{mathrsfs}%花文字
\usepackage{pxrubrica}%ルビ
%\usepackage{lmodern}%font setting
\usepackage{mathtools}%数式の相互参照部分にのみ式番号をふる
%\mathtoolsset{showonlyrefs,showmanualtags}
\usepackage{empheq}%方程式
\usepackage{physics}%物理系
\usepackage{bm}%vector
\usepackage{mleftright}%カッコ
\usepackage{framed}%フレーム
\usepackage{color}%文字に色付け
\usepackage{ulem}%取り消し線
\usepackage{tcolorbox}
\tcbuselibrary{breakable,theorems,skins}%数式を「デコ」る
\usepackage{tikz}
\usetikzlibrary{cd}%矢印の図
\usetikzlibrary{decorations,decorations.pathreplacing}%tikz-cdを「デコ」る
\usepackage{fig/gnuplot-lua-tikz}%gnuplot経由画像
\usepackage{bbm}%指示関数とかはこっちで出す
%\usepackage[right]{showlabels}%数式番号を左に表示
\usepackage{booktabs}%表の上線\toprule中線\midrule下線\bottomrule
%\usepackage{url}%参考文献とかでurlをかくとき
\usepackage[dvipdfmx]{hyperref}%ハイパーリンク
\usepackage{comment}
%\usepackage{lscape}%ページを横向きにする

\makeatletter%subsubsubsection定義
    \newcommand{\subsubsubsection}{\@startsection{paragraph}{4}{\z@}%
        {1.0\Cvs \@plus.5\Cdp \@minus.2\Cdp}%
        {.1\Cvs \@plus.3\Cdp}%
        {\reset@font\sffamily\normalsize}
    }
\makeatother

\setcounter{secnumdepth}{3}%セクションの深さレベル
%\headheight = 0mm
%\textheight = 205mm
%\footskip = 10mm
%\usepackage{fancyhdr}
%\pagestyle{fancy}
%穴埋め定義
\newcounter{BCntr}
\setcounter{BCntr}{1}
\def\blank#1{
\underline{{¥color{red}{#1}}{¥tiny ¥theBCntr}}
\stepcounter{BCntr}
}%

%physicsでrotもcurlで使う
\newcommand{\rot}{\curl}

%amsthm style1,2
%style1 連番
\newtheoremstyle{mystyle1}% Name
    {}% Space above
    {}% Space below
    {\normalfont}% Body font
    {}% Indent amount
    {\bfseries\sffamily}% Theorem head font
    {\hspace{0.5em}}% Punctuation after theorem head
    { }% Space after theorem head, ‘ ‘, or \newline
    {\thmname{#1}\thmnumber{#2}\thmnote{:\,#3\\}}% Theorem head spec (can be left empty, meaning `normal')
\theoremstyle{mystyle1}
\newtheorem{definition}{定義}[section]
\newtheorem{theorem}[definition]{定理}
\newtheorem{lemma}[definition]{補題}
\newtheorem{proposition}[definition]{命題}
\newtheorem{corollary}[definition]{系}
\newtheorem{formula}[definition]{公式}

\newtheoremstyle{mystyle3}% Name
    {}% Space above
    {}% Space below
    {\normalfont}% Body font
    {}% Indent amount
    {\bfseries\sffamily}% Theorem head font
    {\hspace{0.5em}}% Punctuation after theorem head
    { }% Space after theorem head, ‘ ‘, or \newline
    {\thmname{#1}\thmnumber{#2}\thmnote{:\,#3\\}}% Theorem head spec (can be left empty, meaning `normal')
\theoremstyle{mystyle3}
\newtheorem{example}{例}[section]

\newtheoremstyle{mystyle4}% Name
    {}% Space above
    {}% Space below
    {\normalfont}% Body font
    {}% Indent amount
    {\bfseries\sffamily}% Theorem head font
    {\hspace{0.5em}}% Punctuation after theorem head
    { }% Space after theorem head, ‘ ‘, or \newline
    {\thmname{#1}\thmnumber{#2}\thmnote{:\,#3\\}}% Theorem head spec (can be left empty, meaning `normal')
\theoremstyle{mystyle4}
\newtheorem{note}{注意}[section]

%style2 番号なし
\newtheoremstyle{mystyle2}% Name
    {}% Space above
    {}% Space below
    {\normalfont}% Body font
    {}% Indent amount
    {\bfseries\sffamily}% Theorem head font
    {\hspace{0.5em}}% Punctuation after theorem head
    { }% Space after theorem head, ‘ ‘, or \newline
    {\thmname{#1}\thmnote{:\,#3\\}}% Theorem head spec (can be left empty, meaning `normal')
\theoremstyle{mystyle2}
\newtheorem{dfn*}{定義}
\newtheorem{thm*}{定理}
\newtheorem{exercise*}{演習問題}
\newtheorem{ex*}{例}
\newtheorem{qes*}{疑問}
\newtheorem{rem*}{注意}
\newtheorem{ans*}{解答例}
\newtheorem{answer*}{解答}
\newtheorem{note*}{注意}
\newtheorem{remark*}{補足}
\newtheorem{lem*}{補題}
\newtheorem{symbol*}{記号}
\newtheorem{premise*}{前提}
\newtheorem{proposition*}{命題}

%proofの後ろに黒四角をつける
\makeatletter
\renewenvironment{proof}[1][\proofname]{\par
  \pushQED{\qed}%
  \normalfont
  \topsep6\p@\@plus6\p@ \trivlist
  \item[\hskip\labelsep{\bfseries\sffamily #1}]\ignorespaces
}{%
  \popQED\endtrivlist\@endpefalse
}
\renewcommand\proofname{証明}
\renewcommand{\qedsymbol}{\hfill\ensuremath{\blacksquare}}
\makeatother

\definecolor{shadecolor}{gray}{0.9}

\newcommand{\defLeftrightarrow}{\overset{\text{def}}{\iff}}

% \prtlabel{label_strings}
\def\startpoint#1#2{
    {\hfill\rlap{\quad{$\overline{\tt{#1}\ \tt{#2}}$}}}\vspace{-1.5\baselineskip}
}

\newcommand\range{\operatorname{range}}
\newcommand\sgn{\operatorname{sgn}}

\newcommand{\blueunderline}[3][pos=0.5]{
    \tcboxmath[
        enhanced,
        frame hidden, % 枠を消す
        interior hidden, % 背景を消す
        size=minimal, % 余白を消す
        overlay={
                \draw[
                    blue,
                    decorate
                ] ([yshift=-4pt]frame.south west) -- ([yshift=-4pt]frame.south east)
                node[#1,scale=0.72,below] {#3};
            }
    ]{#2}
}

\newcommand{\bluenotearrow}[2]{
    \tcboxmath[
        enhanced,
        frame hidden, % 枠を消す
        interior hidden, % 背景を消す
        size=minimal, % 余白を消す
        overlay={
                \draw[<-,blue] ([yshift=-1ex]frame.south) to[out=-90,in=90] +(0,-3ex) %+(幅,上下の位置)
                node[below,scale=0.72] {#2};
            }
    ]{\,\,\, #1\,\,\,}
}

\newcommand{\bluenote}[2]{
    \tcboxmath[
        enhanced,
        frame hidden, % 枠を消す
        interior hidden, % 背景を消す
        size=minimal, % 余白を消す
        overlay={
                \node[blue,below,scale=0.72] at ([yshift=-1ex]frame.south) {#2};
            }
    ]{\,\,\, #1\,\,\,}
}

\newcommand{\bluenoteunderleft}[2]{
    \tcboxmath[
        enhanced,
        frame hidden, % 枠を消す
        interior hidden, % 背景を消す
        size=minimal, % 余白を消す
        overlay={
                \draw[<-,blue] ([yshift=-1ex,xshift=-0.5em]frame.south) to[out=-120,in=0] +(-2em,-2ex)
                node[left,scale=0.72] {#2};
            }
    ]{\,\,\, #1\,\,\,}
}

\newcommand{\bluenoteoverleft}[2]{
    \tcboxmath[
        enhanced,
        frame hidden, % 枠を消す
        interior hidden, % 背景を消す
        size=minimal, % 余白を消す
        overlay={
                \draw[<-,blue] ([yshift=1ex,xshift=-0.5em]frame.north) to[out=120,in=0] +(-2em,2ex)
                node[left,scale=0.72] {#2};
            }
    ]{\,\,\, #1\,\,\,}
}

%\renewcommand{\thepart}{\arabic{part}}
%\renewcommand{\thenote}{}
%\renewcommand{\thelem}{}

%\makeatletter
%\@addtoreset{section}{part}
%\makeatother

%\makeatletter
%\def\blfootnote{\xdef\@thefnmark{}\@footnotetext}
%\makeatother


\makeatletter
\renewcommand{\thefigure}{% 図番号の付け方
\thesection.\arabic{figure}}
\@addtoreset{figure}{section}
\makeatother

\makeatletter
\@addtoreset{equation}{section}
\def\theequation{\thesection.\arabic{equation}}% renewcommand でもOK
\makeatother

%\setcounter{tocdepth}{1}%目次をsectionまでの表示にする

\renewcommand{\labelenumi}{(\roman{enumi})}

%デコりamsthm
\newenvironment{dfn}[1][]
{\begin{tcolorbox}[
    enhanced,
    boxrule=0pt,
    arc=0mm,
    frame hidden,
    borderline west={2pt}{-4pt}{green!60!black},
    breakable = true
    ]
    \begin{definition}[#1]
}
{\end{definition}\end{tcolorbox}}

\newenvironment{lem}[1][]
{\begin{tcolorbox}[
    enhanced,
    boxrule=0pt,
    arc=0mm,
    frame hidden,
    borderline west={2pt}{-4pt}{yellow!90!black},
    breakable = true
    ]
    \begin{lemma}[#1]
}
{\end{lemma}\end{tcolorbox}}

\newenvironment{prop}[1][]
{\begin{tcolorbox}[
    enhanced,
    boxrule=0pt,
    arc=0mm,
    frame hidden,
    borderline west={2pt}{-4pt}{blue!50!black},
    breakable = true
    ]
    \begin{proposition}[#1]
}
{\end{proposition}\end{tcolorbox}}

\newenvironment{cor}[1][]
{\begin{tcolorbox}[
    enhanced,
    boxrule=0pt,
    arc=0mm,
    frame hidden,
    borderline west={2pt}{-4pt}{blue!50!black},
    breakable = true
    ]
    \begin{corollary}[#1]
}
{\end{corollary}\end{tcolorbox}}
\newenvironment{thm}[1][]
{\begin{tcolorbox}[
    enhanced,
    boxrule=0pt,
    arc=0mm,
    frame hidden,
    borderline west={2pt}{-4pt}{red},
    breakable = true
    ]
    \begin{theorem}[#1]
}
{\end{theorem}\end{tcolorbox}}

\newenvironment{fml}[1][]
{\begin{tcolorbox}[
    enhanced,
    boxrule=0pt,
    arc=0mm,
    frame hidden,
    borderline west={2pt}{-4pt}{orange},
    breakable = true
    ]
    \begin{formula}[#1]
}
{\end{formula}\end{tcolorbox}}

\newenvironment{ex}[1][]
{\begin{tcolorbox}[
    enhanced,
    boxrule=0pt,
    arc=0mm,
    frame hidden,
    borderline west={0.25pt}{-4pt}{black},
    borderline west={0.25pt}{-2.25pt}{black},
    breakable = true
    ]
    \begin{example}[#1]
}
{\end{example}\end{tcolorbox}}

\newenvironment{prop*}[1][]
{\begin{tcolorbox}[
    enhanced,
    boxrule=0pt,
    arc=0mm,
    frame hidden,
    borderline west={2pt}{-4pt}{blue!50!black},
    breakable = true
    ]
    \begin{proposition*}[#1]
}
{\end{proposition*}\end{tcolorbox}}

\DeclareMathOperator{\supp}{supp}

\title{工業数学A3}
\author{}
\date{更新日時:\today}%コメントアウトした状態だと今日の日付が入る

\begin{document}
\maketitle
\tableofcontents%目次

\newpage

\section{フーリエ級数展開}

\subsection{導入}

$f(t)$を$\mathbb{R}$上の関数とする$(t\in\mathbb{R})$.

\begin{dfn*}[周期関数]
    $f$が周期$T$の周期関数であるとは,$\forall t\in\mathbb{R}$に対して
    \[
        f(t+T)=f(t)\quad(t\in\mathbb{R})
    \]
    が成り立つこと.
\end{dfn*}
このとき,$\forall n\in\mathbb{Z}$に対して
\[
    f(t+nT)=f(t)\quad(t\in\mathbb{R})
\]
が成り立つ.

\begin{dfn*}[フーリエ級数展開]
    \begin{equation}\label{eq:1-1}
        f(t)=c+\sum_{n=1}^\infty a[n]\cos(n\omega t)+\sum_{n=1}^\infty b[n]\sin(n\omega t)
    \end{equation}
    ここで$\omega=2\pi/T$,$c,a[n],b[n]$は定数.
\end{dfn*}

これが収束すれば周期$T$の周期関数である.広いクラスの,周期$T$の周期関数は\eqref{eq:1-1}の形に表現できる.

\subsection{三角関数の直交関係とフーリエ係数}

\eqref{eq:1-1}における$c,a[n],b[n]$をフーリエ係数とよぶ.

\begin{thm}[三角関数の直交関係]\label{thm:1-1}
    $n,m$を正の整数とする.
    \begin{equation}\label{eq:1-2}
        \frac{2}{T}\int_0^T\cos(n\omega t)\cos(m\omega t) dt=\delta_{nm}
    \end{equation}
    \begin{equation}\label{eq:1-3}
        \frac{2}{T}\int_0^T\sin(n\omega t)\sin(m\omega t) dt=\delta_{nm}
    \end{equation}
    \begin{equation}\label{eq:1-4}
        \frac{2}{T}\int_0^T\sin(n\omega t)\cos(m\omega t) dt=0
    \end{equation}
    \begin{equation}\label{eq:1-5}
        \frac{2}{T}\int_0^T\cos(n\omega t) dt=\frac{2}{T}\sin(n\omega t) dt=0
    \end{equation}
    ここで,$\delta_{nm}$はクロネッカーのデルタである.
\end{thm}

\begin{proof}
    省略.
\end{proof}

\begin{fml}\label{fml:1-2}
    周期$T$の周期関数$f(t)$が
    \begin{equation}\label{eq:1-6}
        f(t)=\frac{a[0]}{2}+\sum_{n=1}^\infty a[n]\cos(n\omega t)+\sum_{n=1}^\infty a[n]\sin(n\omega t)
    \end{equation}
    と表されるならば,フーリエ係数は
    \begin{equation}\label{eq:1-7}
        a[n]=\frac{2}{T}\int_0^T f(t)\cos(n\omega t)dt\quad(n=0,1,2,\ldots)
    \end{equation}
    \begin{equation}\label{eq:1-8}
        b[n]=\frac{2}{T}\int_0^T f(t)\sin(n\omega t)dt\quad(n=0,1,2,\ldots)
    \end{equation}
    と与えられる.
\end{fml}

周期性により
\begin{equation}\label{eq:1-9}
    a[n]=\frac{2}{T}\int_{-T/2}^{T/2} f(t)\cos(n\omega t)dt\quad(n=0,1,2,\ldots)
\end{equation}
\begin{equation}\label{eq:1-10}
    b[n]=\frac{2}{T}\int_{-T/2}^{T/2} f(t)\sin(n\omega t)dt\quad(n=0,1,2,\ldots)
\end{equation}
とかける.$f$が偶関数ならば$b[n]=0\quad(n=0,1,2,\ldots)$となり,\textbf{余弦フーリエ級数展開}:
\[
    f(t)=\frac{a[0]}{2}+\sum_{n=1}^\infty a[n]\cos(n\omega t)
\]
$f$が奇関数ならば$a[n]=0\quad(n=0,1,2,\ldots)$となり,\textbf{正弦フーリエ級数展開}:
\[
    f(t)=\sum_{n=1}^\infty b[n]\cos(n\omega t)
\]
をもつといわれる.

\subsection{複素フーリエ変換}

オイラーの公式を用いれば,フーリエ級数展開\eqref{eq:1-6}は
\[
    f(t)=\frac{a[0]}{2}+\sum_{n=1}^\infty\mleft(\frac{a[n]}{2}+\frac{b[n]}{2i}\mright)e^{in\omega t}+\sum_{n=1}^\infty\mleft(\frac{a[n]}{2}-\frac{b[n]}{2i}\mright)e^{-in\omega t}
\]
となる.そこで
\[
    c[n]=\begin{cases}
        a[0]/2           & (n=0) \\
        (a[n]-ib[n])/2   & (n>0) \\
        (a[-n]+ib[-n])/2 & (n<0)
    \end{cases}
\]
とすると\textbf{複素フーリエ変換}:
\begin{equation}\label{eq:1-11}
    f(t)=\sum_{n=-\infty}^\infty c[n]e^{in\omega t}
\end{equation}
が得られる.公式\ref{fml:1-2}もしくは直接計算により
\[
    c[n]=\frac{1}{T}\int_0^T f(t)e^{-in\omega t}dt\quad(n\in\mathbb{Z})
\]
が確かめられる.

\subsection{いくつかの実例}

\begin{ex}[三角多項式]
    三角多項式($\sin\omega t,\cos\omega t$の多項式).
\end{ex}

例えば
\[
    f_1(t)=(\cos\omega t)^3=\frac{3}{4}\cos\omega t+\frac{1}{4}\cos 3\omega t
\]

一般的な三角多項式に対しては,$N\in\mathbb{N}$が存在して
\[
    f(t)=\sum_{n=-N}^N c[n]e^{in\omega t}
\]
の形にかける.

\begin{ex}[三角波]
    \[
        f_2(t)=|T|\quad(|t|\leq T/2)
    \]
\end{ex}

$f_2(t)$は偶関数であるから,余弦フーリエ級数展開をもち
\[
    a[n]=\frac{2}{T}\int_{-T/2}^{T/2}|t|\cos(n\omega t)dt=\frac{4}{T}\int_0^{T/2}t\cos(n\omega t)dt
\]
$n=0$のとき
\[
    a[0]=\frac{4}{T}\mleft[\frac{t^2}{2}\mright]_0^{T/2}=\frac{T}{2}
\]
$n\neq 0$のとき
\[
    a[n]=\begin{cases}
        0                          & (nは偶数) \\
        -\frac{4}{\pi n^2 \omega } & (nは奇数)
    \end{cases}
\]
つまり,$f_2(t)$のフーリエ級数展開は
\[
    f_2(t)=\frac{T}{4}-\frac{4}{\pi\omega}\sum_{m=0}^\infty \frac{1}{(2m+1)^2}\cos\{(2m+1)\omega t\}
\]
となる.

\begin{ex}{}
    \[
        f_3(t)=\frac{1}{\frac{5}{4}+\cos\omega t}
    \]
\end{ex}
留数定理を用いてフーリエ係数を計算すると,$n\geq 0$として
\[
    c[-n]=\frac{4}{3}(-2)^n
\]
$n>0$として$c[n]=c[-n]$より
\[
    c[n]=\frac{4}{3}(-2)^{-n}
\]
となるから,$f_3(t)$のフーリエ級数展開は
\[
    f_3(t)=\frac{4}{3}+\frac{8}{3}\sum_{n=1}^\infty (-2)^n\cos(n\omega t)
\]
となる.

\subsection{フーリエ級数の一様収束}

\begin{dfn*}[リプシッツ連続(Lipshitz continuous)]
    $\mathbb{R}$上の関数$f(t)$が\textbf{リプシッツ連続}であるとは,
    \[
        \exists C>0\, \textrm{s.t.}\, \forall t,s\in\mathbb{R},\, |f(t)-f(s)|\leq C|t-s|
    \]
    が成り立つことである.
\end{dfn*}

\begin{thm}\label{thm:1-3}
    $f(t)$がリプシッツ連続ならば,フーリエ級数展開は$f(t)$に一様収束する.
\end{thm}

\begin{proof}
    は\ref{sec:1-10}節に後回し.
\end{proof}

\subsection{有限フーリエ級数}

$X=\mathbb{C}^N$を$N$次元複素線形空間とする.$X$の元を$u=(u[0],u[1],\cdots,u[N-1])\in X$と書く.

\begin{dfn*}[エルミート内積]
    $u,v\in X$のエルミート内積:
    \[
        \langle u,v\rangle=\sum_{n=0}^{N-1}u[n]\overline{v[n]}
    \]
\end{dfn*}

\begin{dfn*}[長さ]
    $u\in X$の長さ:
    \[
        |u|=\sqrt{\langle u,u\rangle}
    \]
\end{dfn*}

\begin{dfn*}[正規直交系/正規直交基底]
    $u_0,\ldots,u_{M-1}\in X$が正規直交系であるとは
    \[
        \langle u_m,u_n\rangle=\delta_{mn}\quad(n,m=0,1,\ldots,M-1)
    \]
    が成り立つこと.

    $M=N=\dim X$のとき正規直交基底とよばれ,$\forall u\in X$は
    \[
        u=\sum_{n=0}^{N-1}\langle u,u_n \rangle u_n
    \]
    と直交分解できる.$\mathbb{C}^N$の標準的な基底$\bm{e}_n[k]=\delta_{kn}\quad(n,k=0,\ldots,N-1)$は正規直交基底である.
\end{dfn*}

有限フーリエ変換の定義に用いられる正規直交基底は,$\alpha=2\pi/N$として
\[
    \varphi_n[k]=\frac{1}{\sqrt{N}}\exp(i\alpha nk)\quad(n,k=0,1,\ldots,N-1)
\]
で定義される.

\begin{prop}\label{prop:1-4}
    $\{\varphi_0,\ldots,\varphi_{N-1}\}$は$X(=\mathbb{C}^N)$の正規直交基底である.
\end{prop}

\begin{proof}
    $n=m$のとき
    \[
        |\varphi_n|^2=\langle\varphi_n,\varphi_n\rangle=\frac{1}{N}\sum_{k=0}^{N-1}|e^{i\alpha nk}|^2=1
    \]
    $n\neq m$のとき
    \[
        \langle\varphi_n,\varphi_m\rangle=\frac{1}{N}\sum_{k=0}^{N-1}e^{i\alpha nk}e^{-i\alpha mk}=\frac{1}{N}\frac{1-e^{i\alpha(n-m)N}}{1-e^{i\alpha n-m}}\bluenotearrow{=}{$\alpha N=2\pi$}0
    \]
\end{proof}

$u\in X$に対して\textbf{有限フーリエ変換}:
\begin{equation}\label{eq:1-12}
    \hat u[n]=\langle u,\varphi_n \rangle=\frac{1}{\sqrt{N}}\sum_{k=0}^{N-1}u[k]e^{-i\alpha nk}\quad(n=0,\ldots,N-1)
\end{equation}
とおくと,\textbf{有限フーリエ級数展開}($\hat u[n]$の\textbf{逆有限フーリエ変換}):
\begin{equation}\label{eq:1-13}
    u[k]=\sum_{n=0}^{N-1}\hat u[n]\varphi_n[k]=\frac{1}{\sqrt{N}}\sum_{n=0}^{N-1}\hat u[n]e^{i\alpha nk}\quad(n=0,\ldots,N-1)
\end{equation}
と直交展開される.

有限フーリエ変換,逆有限フーリエ変換はユニタリーな線形変換である.すなわち:
\begin{equation}\label{eq:1-14}
    |u|=|\hat u|
\end{equation}
が成り立つ.これは
\begin{equation}\label{eq:1-15}
    \langle u,v\rangle=\langle \hat u,\hat v\rangle
\end{equation}
であることから従う.

\subsection{有限フーリエ変換の連続極限}

\begin{thm}\label{thm:1-5}
    $f(t)$を周期$T$のリプシッツ連続な周期関数とし,$\omega=2\pi/T, \alpha=2\pi/N$とする.
    \[
        f_N(t)=\sum_{-N/2\leq n<N/2} c_N[n]e^{in\omega t}
    \]
    \[
        c_N[n]=\frac{1}{N}\sum_{m=0}^{N-1}f\mleft(\frac{m}{N}T\mright)e^{-i\alpha nm}\quad(n\in\mathbb{Z})
    \]
    と定める.このとき,$f_N(t)$は$f(t)$に一様収束する.
\end{thm}

\begin{note}\label{note:1-1}
    \[
        f_N\mleft(\frac{k}{N}T\mright)=\sum_{-N/2\leq n<N/2} c_N[n]e^{i\alpha nk}=\sum_{n=0}^{N-1}e_N[n]e^{i\alpha nk}=f\mleft(\frac{k}{N}T\mright)
    \]
    $f_N(t)$は$f(t_k)$の値を与えたときの補間.
\end{note}

\begin{proof}
    まず
    \[
        d_N[n]=\frac{1}{N}\sum_{m=0}^{N-1}\mleft\{f\mleft(\frac{m}{N}T\mright)-f\mleft(\frac{m-1}{N}T\mright)\mright\}e^{-i2\pi (m/N)n}
    \]
    とおく.$\{d_N[n]\}$は$f(t_n)-f(t_n-(T/N))$の有限フーリエ変換の$1/\sqrt{N}$倍である.リプシッツ連続性の仮定より,$\exists C>0$,
    \[
        \mleft|f\mleft(\frac{m}{N}T\mright)-f\mleft(\frac{m-1}{N}T\mright)\mright|\leq\frac{C}{N}\quad(m\in\mathbb{Z})
    \]
    が成り立つ.したがって有限フーリエ変換の等長性\eqref{eq:1-14}より
    \[
        N\sum_{k=0}^{N-1}|d_N[k]|^2\leq \sum_{m=0}^{N-1}\mleft|\frac{C}{N}\mright|^2=\frac{C^2}{N}
    \]
    すなわち
    \[
        \sum_{k=0}^{N-1}|d_N[k]|^2\leq \frac{C^2}{N^2}
    \]
    が従う.一方,$d_N[k]$の定義により
    \[
        d_N[k]=(1-e^{-i 2\pi(k/N)})c_N[k]=2i e^{-i\pi(k/N)}\sin(\pi k/N)c_N[k]
    \]
    これとジョルダンの不等式:
    \begin{equation}\label{eq:1-16}
        |\sin\theta|\geq\frac{2}{\pi}|\theta|\quad(|\theta|\leq \pi/2)
    \end{equation}
    を用いれば
    \[
        |d_N[k]|\geq\frac{4|k|}{N}|c_N[k]|\quad\mleft(-\frac{N}{2}\leq k<\frac{N}{2}\mright)
    \]
    が従う.$d_N[k]$が周期$N$を持つことに注意して,これらを組み合わせると
    \[
        \sum_{-N/2\leq k< N/2}|k|^2|c_N[k]|^2\leq \frac{C^2}{16}
    \]
    が導かれる.

    また,
    \[
        f_N'(t)=\sum_n i\omega n c_N[n]e^{i\omega n t}
    \]
    の絶対値を考えると
    \begin{equation}\label{eq:1-17}
        \begin{split}
            |f_N'(t)| & \leq \omega \sqrt{N}\sum_n \frac{n}{\sqrt{N}}|c_N[n]| \\
            & \bluenoteunderleft{\leq}{Schwarzの不等式} \omega\sqrt{N}\mleft(\sum_n\mleft(\frac{1}{\sqrt{N}}\mright)^2\mright)^{1/2} \mleft(\sum_n|n|^2|c_N[n]|^2\mright)^{1/2} \\
            & \leq \sqrt{N}C'\quad (C'はある定数)
        \end{split}
    \end{equation}
    となる.さて,$\forall t\in [0,T]$に対して$|t-t_k|\leq T/(2N)$であるような$t_k=(k/N)T$が存在する.$f(t_k)=f_N(t_k)$に注意して(注意\ref{note:1-1})
    \[
        f(t)-f_N(t)=(f(t)-f(t_k))+(f_N(t_k)-f_N(t))
    \]
    と分解して考える.$f(t)$のリプシッツ連続性と$f_N'$の微分の評価\eqref{eq:1-17}から
    \[
        \begin{split}
            |f(t)-f_N(t)| & \leq |f(t)-f(t_k)|+|f_N(t_k)-f_N(t)| \\
            & \leq \blueunderline{C|t-t_k|}{リプシッツ連続性}+\blueunderline{C'\sqrt{N}|t-t_k|}{平均値の定理} \\[3ex]
            & =O(1/\sqrt{N})
        \end{split}
    \]
    が得られる.つまり,$N\to\infty$のとき$f_N(t)$が$f(t)$に一様収束することが示された.
\end{proof}

\subsection{関数空間の内積と直交関数系}

%正確にはルベーグ積分が必要だけどリーマン積分でなんとかします

$X$を周期$T$で周期的で有界かつ$[0,T]$上で積分可能な関数全体とする.

\begin{dfn*}[内積]
    $f,g\in X$に対して内積を
    \[
        \langle f,g \rangle=\frac{1}{T}\int_0^T f(t)\overline{g(t)}dt
    \]
    で定義する.
\end{dfn*}

$f,g,h\in X$,$a,b\in\mathbb{C}$のとき
\[
    \langle af+bg,h\rangle=a\langle f,h\rangle +b\langle g,h\rangle
\]
\[
    \langle f,ag+bh\rangle=\overline{a}\langle f,g\rangle +\overline{b}\langle f,h\rangle
\]
\[
    \langle f,g\rangle=\overline{\langle g,f\rangle}
\]

\begin{dfn*}[$L^2$ノルム]
    $f\in X$に対して
    \[
        \|f\|=\sqrt{\langle f,f\rangle}\geq 0
    \]
\end{dfn*}

$\|f\|=0\,\Leftrightarrow\,f=0$ $\mu$-$a.e.$である.

\subparagraph{シュワルツの不等式}

\begin{equation}\label{eq:1-18}
    |\langle f,g\rangle|\leq\|f\|\|g\|
\end{equation}

\subparagraph{三角不等式}

\begin{equation}\label{eq:1-19}
    \|f+g\|\leq\|f\|+\|g\|
\end{equation}

\begin{dfn*}[直交]
    $f,g$が直交するとは,
    \[
        \langle f,g\rangle=0
    \]
    となること.
\end{dfn*}

\begin{dfn*}[直交関数系]
    $\{f_n\}_{n=1}^\infty \subset X$が正規直交系であるとは,
    \[
        \langle f_n,f_m\rangle=\delta_{nm}\quad(n,m=1,2,\ldots)
    \]
    となること.
\end{dfn*}

\textbf{フーリエ関数系}を
\[
    \varphi_n(t)=e^{in\omega t}\quad(n\in\mathbb{Z},t\in\mathbb{R})
\]
で定義すると,$\{\varphi_n\}_{n=-\infty}^\infty$は正規直交系となる.このとき,フーリエ級数展開
\[
    f(t)=\sum_{-\infty}^\infty c[n]\varphi_n(t)
\]
\[
    c[n]=\langle f,\varphi_n \rangle=\frac{1}{T}\int_0^T f(t)e^{-in\omega t}dt\quad(n\in\mathbb{Z})
\]
は正規直交系$\{\varphi_n\}$に関する展開であり,フーリエ係数は座標成分である.実フーリエ級数\eqref{eq:1-6}も同様である.

\subsection{正規直交基底とフーリエ級数の平均収束}
$\{f_n\}_{n=1}^\infty$を$X$の正規直交系とする.
\begin{dfn*}[正規直交基底(完全正規直交系)]
    $\{f_n\}_{n=1}^\infty$が正規直交基底もしくは完全正規直交系であるとは,$\forall f\in X$に対して
    \[
        \lim_{N\to\infty} \mleft\|f-\sum_{n=1}^N \langle f,f_n\rangle f_n\mright\|=0
    \]
    であること.\footnote{
        ちなみに,完全のつかない正規直交系として$\{\cos x,\cos 2x,\ldots,\sin x,\sin 2x,\ldots\}$が考えられる.これは例えば$f$に定数をとればわかるように,完全正規直交系(正規直交基底)にはならない.$\{1,\cos x,\cos 2x,\ldots,\sin x,\sin 2x,\ldots\}$なら完全正規直交系(正規直交基底)になる.
    }
\end{dfn*}

\begin{thm}[パーセバルの等式]\label{thm:1-6}
    フーリエ関数系$\{\varphi_n\}_{n=-\infty}^\infty$は$X$の正規直交基底で,$\forall f\in X,\, c[n]=\langle f,\varphi_n\rangle$として
    \begin{equation}\label{eq:1-20}
        \|f\|^2=\sum_{n=-\infty}^\infty |c[n]|^2
    \end{equation}
    が成り立つ.
\end{thm}
\begin{proof}
    はあとで.
\end{proof}

定理\ref{thm:1-6}より,フーリエ級数展開の\textbf{平均収束($L^2$-収束)}:
\[
    \mleft\|f-\sum_{n=-N}^N c[n]\varphi_n\mright\|\to 0\quad(N\to\infty)
\]
がわかる.

\begin{prop}[ベッセルの不等式]\label{prop:1-7}
    $\{f_n\}_{n=1}^\infty$が正規直交系ならば,$\forall f\in X$に対して
    \begin{equation}\label{eq:1-22}
        \sum_{n=1}^\infty |\langle f,f_n\rangle|\leq \|f\|^2
    \end{equation}が
    が成り立つ.
\end{prop}

\begin{proof}
    \[
        \begin{split}
            \mleft\|f-\sum_{n=1}^N\langle f,f_n\rangle f_n\mright\|^2
            &= \mleft\langle f-\sum_{n=1}^N\langle f,f_n\rangle f_n, f-\sum_{n=1}^N\langle f,f_n\rangle f_n\mright\rangle \\
            &=\|f\|^2-\sum_{n=1}^N\langle f,f_n\rangle\langle f_n,f\rangle-\sum_{n=1}^N\overline{\langle f,f_n\rangle}\langle f_n,f\rangle+\sum_{n=1}^N\sum_{m=1}^N\langle f,f_n\rangle\overline{\langle f,f_m\rangle}\langle f_n,f_m\rangle \\
            &=\|f\|^2-\sum_{n=1}^N|\langle f,f_n\rangle|^2 \\
            &\geq 0
        \end{split}
    \]
    より結論を得る.
\end{proof}

\begin{prop}\label{prop:1-8}
    $\{f_n\}_{n=1}^\infty$を正規直交系とする.$\{f_n\}_{n=1}^\infty$が正規直交基底であるための必要十分条件は,$\forall f\in X$について
    \[
        \|f\|^2=\sum_{n=1}^\infty|\langle f,f_n\rangle|^2
    \]
    が成り立つことである.
\end{prop}

\begin{proof}
    命題\ref{prop:1-7}をみると,$\{f_n\}$が正規直交基底ならば
    \[
        \mleft\|f-\sum_{n=1}^N\langle f,f_n\rangle f_n\mright\|^2
        =\|f\|^2-\sum_{n=1}^N|\langle f,f_n\rangle|^2 \overset{n\to\infty}{\longrightarrow} 0
    \]
    が成り立つ.逆に,
    \[
        \lim_{n\to\infty}\mleft(\|f\|^2-\sum_{n=1}^N|\langle f,f_n\rangle|^2\mright)=0
    \]
    ならば
    \[
        \lim_{n\to\infty}\,\mleft\|f-\sum_{n=1}^N\langle f,f_n\rangle f_n\mright\|^2=0
    \]
    であることがわかり,$\{f_n\}$は正規直交基底である.
\end{proof}


\begin{lem}\label{lem:1-9}
    $\{f_n\}$を正規直交系,$a_1,a_2,\ldots,a_N\in\mathbb{C}$,$f\in X$とする.このとき
    \begin{equation}\label{eq:1-23}
        \mleft\|f-\sum_{n=1}^N\langle f,f_n\rangle f_n\mright\|\leq\mleft\|f-\sum_{n=1}^N a_n f_n\mright\|
    \end{equation}
    が成り立つ.
\end{lem}

\begin{proof}
    $g=f-\sum_{n=1}^N\langle f,f_n\rangle f_n$とする.$\langle g,f_m\rangle=0$に注意すると
    \[
        \begin{split}
            \mleft\|f-\sum_n a_n f_n\mright\|^2
            &=\mleft\|g+\sum_n(\langle f,f_n\rangle-a_n)f_n\mright\|^2 \\
            &=\|g\|^2+\mleft\langle g,\sum_n(\langle f,f_n\rangle-a_n)f_n\mright\rangle +\mleft\langle \sum_n(\langle f,f_n\rangle-a_n)f_n,g\mright\rangle +\mleft\|\sum_n(\langle f,f_n\rangle-a_n)f_n\mright\|^2 \\
            &=\|g\|^2+\mleft\|\sum_n(f_n\rangle-a_n)f_n\mright\|^2 \\
            &\geq \|g\|^2
        \end{split}
    \]
    となる.
\end{proof}

\begin{prop*}
    $f$を周期$T$で周期的かつ有界で$[0,T]$上で積分可能な関数とする.任意の$\varepsilon>0$に対して$\|f-g\|<\varepsilon$となるリプシッツ連続な周期関数$g$が存在する.\footnotemark
\end{prop*}

\footnotetext{ルベーグ積分論を用いた証明としては$L^2$空間における連続関数の稠密性が関係していそう.ルベーグ積分(伊藤)の定理24.2など.}

\begin{proof}
    区間$[0,T]$の分割$\Delta:0=t_0<t_1<\cdots<t_n=T$をとり,周期$T$で周期的な階段関数$f_\Delta (t)$を次式により定める:
    \[
        f_\Delta (t)=\inf_{t_j\leq t<t_{j+1}} f(t)\quad(t_j\leq t<t_{j+1}のとき)
    \]
    $\Delta$を十分細かく取れば,積分の定義により
    \begin{equation}
        \|f-f_\Delta\|=\frac{1}{T}\int_0^T|f(t)-f_\Delta(t)|dt<\frac{\varepsilon}{2} \tag{1}
    \end{equation}
    とできる.上式が成立するように$\Delta$をとって固定する.また,
    \[
        \lim_{s\to 0}\int_0^T|f_\Delta(t-s)-f_\Delta(t)|dt=0
    \]
    であるから,$n_0>0$を十分大きくとれば,$n\geq n_0$のとき
    \begin{equation}
        \sup_{|s|\leq 1/n}\frac{1}{T}\int_0^T|f_\Delta(t-s)-f_\Delta(t)|dt<\frac{\varepsilon}{2} \tag{2}
    \end{equation}
    が成立する.

    任意の$t\in\mathbb{R}$に対して$\phi(t)\geq 0$で,$|t|\geq 1$のとき$\phi(t)=0$かつ
    \[
        \int_{-1}^1 \phi(t)dt=1
    \]
    を満たすリプシッツ連続な関数$\phi(t)$をとり,$\phi_n(t)=n\phi(nt)$とする.このとき
    \begin{equation}
        |t|\geq 1/nのとき\phi_n(t)=0かつ\int_{-1/n}^{1/n}\phi_n(t)dt=1 \tag{3}
    \end{equation}
    となる.関数$f_n(t)$を次式により定める.
    \[
        f_n(t)=\int_{-\infty}^\infty \phi_n(t-s)f_\Delta(s)ds=\int_{-1/n}^{1/n}\phi_n(s)f_\Delta(t-s)ds
    \]
    $f_\Delta (t)$の周期性より
    \[
        f_n(T)=\int_{-1/n}^{1/n}\phi_n(s)f_\Delta(T-s)ds=\int_{-1/n}^{1/n}\phi_n(s)f_\Delta(-s)ds=f_n(0)
    \]
    かつ,十分大きな$n>0$に対して,$L_\phi$を関数$\phi$のリプシッツ定数として
    \[
        |f_n(t_1)-f_n(t_2)|\leq\int_{-\infty}^\infty|\phi_n(t_1-s)-\phi_n(t_2-s)||f_\Delta(s)|ds\leq 4nL_\phi|t_1-t_2|\sup_{0\leq t\leq T}|f(t)|
    \]
    となるから,$f_n(t)$は周期$T$のリプシッツ連続な周期関数である.ここで,$t_1\neq t_2$ならば,$n>0$が十分大きいとき,
    \[
        \max\mleft(t_1-\frac{1}{n},t_2-\frac{1}{n}\mright)>\min\mleft(t_1+\frac{1}{n},t_2+\frac{1}{n}\mright)
    \]
    であること,および式(3)より,
    \[
        \begin{split}
            \int_{-\infty}^\infty|\phi_n(t_1-s)-\phi_n(t_2-s)|ds
            &\leq \int_{t_1-1/n}^{t_1+1/n}|\phi_n(t_1-s)|ds+\int_{t_2-1/n}^{t_2+1/n}|\phi_n(t_2-s)|ds \\
            & =\int_{nt_1-1}^{nt_1+1}|\phi(nt_1-u)|du+\int_{nt_2-1}^{nt_2+1}|\phi(nt_2-u)|du \\
            &\leq \int_{nt_1-1}^{nt_1+1}|\phi(nt_1-u)-\phi(nt_2-u)|du+\int_{nt_2-1}^{nt_2+1}|\phi(nt_1-u)-\phi(nt_2-u)|du \\
            &\leq \int_{nt_1-1}^{nt_1+1} nL_\phi|t_1-t_2|du +\int_{nt_2}^{nt_2+1}nL_\phi|t_1-t_2|du \\
            &\leq 4nL_\phi|t_1-t_2|
        \end{split}
    \]
    となることを用いた.

    式(3)より
    \[
        f_n(t)-f_\Delta(t)=\int_{-1/n}^{1/n} \phi_n(s)(f_\Delta(t-s)-f_\Delta(t))ds
    \]
    であるから,
    \[
        |f_n(t)-f_\Delta(t)|\leq\int_{-1/n}^{1/n}\phi_n(s)|f_\Delta(t-s)-f_\Delta(t)|ds
    \]
    となる.さらに,上式を$t$で積分し,積分を交換すると,$n\geq n_0$のとき
    \[
        \|f_n-f_\Delta\|\leq\frac{1}{T}\int_{-1/n}^{1/n}\phi_n(s)\int_0^T|f_\Delta(t-s)-f_\Delta(s)|dtds<\frac{\varepsilon}{2}
    \]
    が成立する.ここで,式(2)と(3)を用いた.三角不等式により,上式と式(1)とから,$n\geq n_0$のとき,
    \[
        \|f-f_n\|\leq\|f-f_\Delta\|+\|f_\Delta-f_n\|<\varepsilon
    \]
    となり,$g=f_n$とおけば結論を得る.
\end{proof}

\begin{note*}
    上の証明で関数$\phi(t)$を$C^\infty$に取れば,近似関数$g$は$C^\infty$級となる.
\end{note*}

\setcounter{definition}{5}
\begin{thm}[パーセバルの等式(再掲)]
    フーリエ関数系$\{\varphi_n\}_{n=-\infty}^\infty$は$X$の正規直交基底で,$\forall f\in X,\, c[n]=\langle f,\varphi_n\rangle$として
    \begin{equation}\tag{\ref{eq:1-20}}
        \|f\|^2=\sum_{n=-\infty}^\infty |c[n]|^2
    \end{equation}
    が成り立つ.
\end{thm}

\begin{proof}
    $f\in X$,$\forall\varepsilon>0$をとる.すると,上の命題より,リプシッツ連続な周期関数$g$で,$\|f-g\|<\varepsilon/2$をみたすものをとれる.すると,定理\ref{thm:1-5}より,$N$を十分大きくとれば,
    \[
        \mleft|g(t)-\sum_{|n|\leq N}c_N[n]\varphi_n(t)\mright|<\frac{\varepsilon}{2}\quad(t\in [0,T])
    \]
    が成り立つ\footnote{1.6節で$\varphi_n=e^{in\omega t}$で定めている.}.ここで$\{c_N[n]\}$は
    \[
        c_N[n]=\frac{1}{N}\sum_{m=0}^{N-1}g\mleft(\frac{m}{N}T\mright)e^{-i2\pi(m/N)n}\quad(n\in\mathbb{Z})
    \]
    である.これより
    \[
        \mleft\|g-\sum_{|n|\leq N}c_N[n]\varphi_n\mright\|<\frac{\varepsilon}{2}
    \]
    がしたがう.ゆえに
    \[
        \begin{split}
            \mleft\|f-\sum_{|n|\leq N}\langle f,\varphi_n\rangle\varphi_n\mright\|
            &\bluenotearrow{\leq}{補題\ref{lem:1-9}}
            \mleft\|f-\sum_{|n|\leq N}c_N[n]\varphi_n\mright\| \\
            &\leq\|f-g\|+\mleft\|g-\sum_{|n|\leq N}c_N[n]\varphi_n\mright\| \\
            &<\varepsilon
        \end{split}
    \]
    となる.
\end{proof}

\setcounter{definition}{9}

\subsection{定理\ref{thm:1-3}の証明}\label{sec:1-10}

\begin{lem}\label{lem:1-10}
    $f$を周期$T$のリプシッツ連続な周期関数,$\{c[n]\}$を$f$のフーリエ係数とする.このとき
    \[
        \sum_{n=-\infty}^\infty |n|^2|c[n]|^2<\infty
    \]
    が成り立つ.
\end{lem}

\begin{proof}
    リプシッツ連続性より,$C>0$が存在して
    \[
        |f(t+h)-f(t)|\leq C|h|\quad(t,h\in\mathbb{R})
    \]
    が成り立つ.$N\geq 1$に対して$h=T/2N$とおき,$f$の$h$だけ差分関数を
    \[
        g(t)=\frac{1}{h}\{f(t+h)-f(t)\}=\frac{2N}{T}\mleft\{f\mleft(t+\frac{T}{2N}\mright)-f(t)\mright\}
    \]
    と定義すれば,$|g(t)|\leq C$である.$g$のフーリエ係数を$\{d[n]\}$と書くことにしよう.すると,
    \[
        \begin{split}
            d[n]&=\frac{1}{hT}\int_0^T\{f(t+h)-f(t)\}e^{-i\omega nt}dt \\
            &=\frac{1}{h}(e^{i\omega nh}-1)c[n] \\
            &=\frac{2N}{T}e^{i\omega nh/2}\cdot 2i\sin\mleft(\frac{\pi n}{2N}\mright)c[n]
        \end{split}
    \]
    が成り立つ.$g$についてパーセバルの等式を用いれば
    \[
        \|g\|^2=\sum_{n=-\infty}^\infty|d[n]|^2=\mleft(\frac{4N}{T}\mright)^2\sum_{n=-\infty}^\infty\mleft|\sin\mleft(\frac{\pi n}{2N}\mright)\mright|^2|c[n]|^2
    \]
    がわかる.一方,\eqref{eq:1-16}を用いると,
    \[
        |n|\leq N\quad ならば\quad \mleft|\sin\mleft(\frac{\pi n}{2N}\mright)\mright|\geq\frac{2}{\pi}\cdot\frac{\pi n}{2N}=\frac{n}{N}
    \]
    だから,
    \[
        \|g\|^2\geq \frac{16}{T^2}\sum_{n=-N}^N |n|^2|c[n]|^2
    \]
    を得る.$|g(t)|^2\leq C^2$だったので
    \[
        \sum_{n=-N}^N |n|^2|c[n]|^2\leq\frac{T^2C^2}{16}
    \]
    がわかる.右辺は$N$に依らないので,$N\to\infty$として求める不等式が導かれる.
\end{proof}


\begin{lem}\label{lem:1-11}
    $f$を周期$T$のリプシッツ連続な周期関数,$\{c[n]\}$を$f$のフーリエ係数とする.このとき
    \[
        \sum_{n=-\infty}^\infty |c[n]|<\infty
    \]
    が成り立つ.
\end{lem}

\begin{proof}
    $\mathbb{C}^N$でのシュワルツの不等式から
    \[
        \begin{split}
            \sum_{1\leq|n|\leq N}|c[n]|
            &\leq\mleft(2\sum_{n=1}^N\frac{1}{n^2}\mright)^{1/2}\mleft(\sum_{1\leq|n|\leq N}|n|^2|c[n]|^2\mright) \\
            &\leq\mleft(2\sum_{n=1}^\infty\frac{1}{n^2}\mright)^{1/2}\mleft(\sum_{n=-\infty}^\infty |n|^2|c[n]|^2\mright)
        \end{split}
    \]
    がわかる.右辺は補題\ref{lem:1-10}より,有限の定数で$N$に依らない.したがって,$N\to\infty$として補題の主張が成り立つ.
\end{proof}

\begin{lem}\label{lem:1-12}
    $f$を周期$T$の連続な周期関数,$\{c[n]\}$を$f$のフーリエ係数とする.\\ このとき,$\sum_{n=-\infty}^\infty |c[n]|<\infty$ならば,フーリエ部分和:
    \[
        S_N(t)=\sum_{n=-N}^N c[n]e^{i\omega nt}
    \]
    は$N\to\infty$のとき$f$に一様収束する.
\end{lem}

\begin{proof}
    $N<M$とすると,
    \begin{equation}\label{eq:1-24}
        |S_N(t)-S_M(t)|\leq\sum_{N<|n|\leq M}|c[n]|\leq\sum_{|n|>N}|c[n]|
    \end{equation}
    である.仮定により,$N\to\infty$のとき$\sum_{|n|>N}|c[n]|\to 0$だから,\eqref{eq:1-24}の右辺は$0$に収束する.つまり,各$t$ごとに,$\{S_N(t)\}$はコーシー列であり,極限が存在する.そこで$g(t)\equiv\lim_{N\to\infty}S_N(t)$とおく.\eqref{eq:1-24}で$M\to\infty$とすると
    \[
        |S_N(t)-g(t)|\leq\sum_{|n|>N}|c[n]|
    \]
    となる.右辺は$t$によらず,$N\to\infty$のとき$0$に収束するのだから,$S_N(t)$は$g$に一様収束することになる.これより$\|S_N-g\|\to 0$が従う.一方,定理\ref{thm:1-6}より,$\|S_N-f\|\to 0$だから,$f=g$でなければならない.以上により,$S_N$が$f$に一様収束することが示された.
\end{proof}

\begin{note*}
    補題\ref{lem:1-12}は定理\ref{thm:1-3}よりも強い主張であり,重要である.
\end{note*}

\begin{note}
    $\{g_N(t)\}$を連続関数列で$N\to\infty$のとき$g_N(t)$は$g(t)$に一様収束するならば,$g(t)$は連続である.
\end{note}

\setcounter{definition}{2}

\begin{thm}[(再掲)]
    $f(t)$がリプシッツ連続ならば,フーリエ級数展開は$f(t)$に一様収束する.
\end{thm}

\begin{proof}
    補題\ref{lem:1-11}と補題\ref{lem:1-12}から直ちに導かれる.
\end{proof}

\setcounter{definition}{12}

\subsection{ギッブス現象と総和法}

%連続じゃないけれども連続からは離れてない関数を考える.

\begin{thm}
    $f(t)$を周期$T$の周期関数で,$[0,T]$上では区分的に滑らか\footnotemark な関数であるとする.このとき,フーリエ部分和$S_N(t)$は
    \[
        \lim_{N\to\infty} S_N(t)=\begin{cases}
            f(t)                                      & (t:連続)   \\
            \displaystyle\frac{1}{2}\{f(t+0)+f(t-0)\} & (t:不連続)
        \end{cases}
    \]
    が成り立つ.\footnotemark
\end{thm}

\addtocounter{footnote}{-1}

\footnotetext{$a_j (j=1,\ldots ,n-1)$であって,$f(t)$は$[a_j,a_{j+1}]$で$C^1$級で,$f(a_j+0)\,(j<n)$,$f(a_j-0)\, (j>0)$となるものが存在する($a_0=0,a_n=T$)}

\addtocounter{footnote}{1}

\footnotetext{$f(t\pm 0)=\lim_{\varepsilon\to\pm 0}f(t\pm\varepsilon)$}

\begin{proof}
    \[
        \begin{split}
            S_N(t)
            &=\sum_{n=-N}^N c[n]e^{i\omega nt} \\
            &=\sum_{n=-N}^N \mleft(\frac{1}{T}\int_0^T f(s)e^{-in\omega s}ds\mright)e^{i\omega nt} \\
            &=\frac{1}{T}\sum_{n=-N}^N \int_0^T f(s)e^{in\omega (t-s)}ds
        \end{split}
    \]
    と変形できる.$f(s)$は$\mathbb{R}$上の関数なので,$f(s)e^{in\omega (t-s)}$の虚部$f(s)\sin\{n\omega(t-s)\}$を積分したものは$0$になる.したがって
    \[
        \lim_{N\to\infty}\frac{1}{T}\sum_{n=-N}^N \int_0^T f(s)\cos\{n\omega (t-s)\}ds =\begin{cases}
            f(t)                                      & (t:連続)   \\
            \displaystyle\frac{1}{2}\{f(t+0)+f(t-0)\} & (t:不連続)
        \end{cases}
    \]
    を示せばよい.ここで,
    \[
        2\sin\frac{\theta}{2}\cos k\theta=\sin \mleft(k+\frac{1}{2}\mright)\theta-\sin\mleft(k-\frac{1}{2}\mright)\theta
    \]
    より
    \[
        \sum_{k=0}^N\cos k\theta=\frac{\sin\mleft(N+\frac{1}{2}\mright)+\sin\frac{\theta}{2}}{2\sin\frac{\theta}{2}}
    \]
    両辺$1/2$を引いて
    \[
        \frac{1}{2}+\sum_{k=1}^N\cos k\theta=\frac{\sin\mleft(N+\frac{1}{2}\mright)\theta}{2\sin\frac{\theta}{2}}
    \]
    であること\footnote{
        ちなみに,$D_n(x)=\sum_{k=-n}^n e^{ikx}=1+2\sum_{k=1}^N\cos(kx)=\sin\{(n+1/2)x\}/\sin(x/2)$はディリクレ核とよばれ,これを用いると畳み込みを用いて$S_N(t)$が表現できる.
    }を利用すると
    \[
        \begin{split}
            \frac{1}{T}\sum_{n=-N}^N \int_0^T f(s)& \cos\{n\omega (t-s)\}ds \\
            &=\frac{1}{T} \int_0^T f(s)\sum_{n=-N}^N\cos\{n\omega (t-s)\} \\
            &=\frac{2}{T} \int_0^T f(s)\frac{\sin\frac{(2N+1)\omega(t-s)}{2}}{2\sin\frac{\omega(t-s)}{2}}ds \\
            &\bluenoteunderleft{=}{周期性の利用} \frac{2}{T} \int_t^{t+T} f(s)\frac{\sin\frac{(2N+1)\omega(t-s)}{2}}{2\sin\frac{\omega(t-s)}{2}}ds \\
            &=\frac{2}{T} \mleft(\int_t^{t+T/2} + \int_{t+T/2}^{t+T}\mright) f(s)\frac{\sin\frac{(2N+1)\omega(t-s)}{2}}{2\sin\frac{\omega(t-s)}{2}}ds \\
            &\bluenoteunderleft{=}{$-u=\omega(t-s)/2$,$u=\omega(t-s)/2$と置換} \frac{2}{T}\int_0^{\omega T/4} f(t+2u/\omega)\frac{\sin(2N+1)u}{\sin u}\frac{2}{\omega}du
            +\frac{2}{T}\int_0^{\omega T/4} f(t-2u/\omega)\frac{\sin(2N+1)u}{\sin u}\frac{2}{\omega}du \\
            &= \frac{4}{\omega T}\int_0^{\omega T/4} f(t+2u/\omega)\frac{\sin(2N+1)u}{\sin u}du
            +\frac{4}{\omega T}\int_0^{\omega T/4} f(t-2u/\omega)\frac{\sin(2N+1)u}{\sin u}du
        \end{split}
    \]
    ここまでの議論は$f(s)=1$としても成立していることに注意すると,不連続点$t$では(連続点では$f(t+0)=f(t-0)=f(t)$として考える)
    \[
        \begin{split}
            S_N(t)&-\mleft[\frac{1}{2}\{f(t+0)+f(t-0)\}\mright] \\
            &=\frac{4}{\omega T}\int_0^{\omega T/4} f(t+2u/\omega)\frac{\sin(2N+1)u}{\sin u}du
            +\frac{4}{\omega T}\int_0^{\omega T/4} f(t-2u/\omega)\frac{\sin(2N+1)u}{\sin u}du \\
            & \qquad -\frac{1}{2}\{f(t+0)+f(t-0)\}\mleft[\frac{4}{\omega T}\int_0^{\omega T/4} \frac{\sin(2N+1)u}{\sin u}du
                +\frac{4}{\omega T}\int_0^{\omega T/4} \frac{\sin(2N+1)u}{\sin u}du\mright] \\
            &= \frac{4}{\omega T}\int_0^{\omega T/4} \{f(t+2u/\omega)-f(t+0)\}\frac{\sin(2N+1)u}{\sin u}du \\
            & \qquad +\frac{4}{\omega T}\int_0^{\omega T/4} \{f(t-2u/\omega)-f(t-0)\}\frac{\sin(2N+1)u}{\sin u}du
        \end{split}
    \]
    ここで,$f$は区分的に滑らかであり,$u/\sin u\to 1$($u\to +0$)に注意すれば,$t$を固定して区間$(0,\omega T/4)$で定義された$u$の関数
    \[
        g_\pm(u)=\frac{f(t\pm 2u/\omega)-f(t\pm 0)}{\sin u}=\frac{f(t\pm 2u/\omega)-f(t\pm 0)}{\pm u}\frac{\pm u}{\sin u}
    \]
    は有界で有限個の点を除いて連続である(可積分).したがって,リーマン-ルベーグの定理を利用して
    \[
        \lim_{N\to\infty}\int_0^{\omega T/4}g_\pm(u)\sin(2N+1)u\, du=0
    \]
    がわかる.したがって
    \[
        \lim_{N\to\infty}\mleft|S_N(t)-\mleft[\frac{1}{2}\{f(t+0)+f(t-0)\}\mright]\mright|=0
    \]
\end{proof}

\begin{ex}[方形波]
    \[
        f_4(t)=\begin{cases}
            -1 & (-T/2<t<0) \\
            1  & (0<t<T/2)
        \end{cases}
    \]
\end{ex}

$f_4(t)$は奇関数で,フーリエ展開は
\[
    \begin{split}
        f_4(t)
        &=\frac{4}{\pi}\sum_{n=0}^\infty \frac{\sin(2n+1)\omega t}{2n+1} \\
        &=\frac{4}{\pi}\mleft(\sin\omega t+\frac{1}{3}\sin3\omega t+\frac{1}{5}\sin5\omega t+\cdots\mright)
    \end{split}
\]
となる.この展開の部分和$S_N(t)$をグラフにプロットすると図\ref{fig:ex1-4}のようになる.

\begin{figure}
    \begin{center}
        \begin{tikzpicture}[gnuplot]
  %% generated with GNUPLOT 5.4p1 (Lua 5.4; terminal rev. Jun 2020, script rev. 114)
  %% 火  6/ 1 20:35:16 2021
  \path (0.000,0.000) rectangle (12.000,8.000);
  \gpcolor{color=gp lt color border}
  \gpsetlinetype{gp lt border}
  \gpsetdashtype{gp dt solid}
  \gpsetlinewidth{1.00}

  \draw[->,>=stealth,semithick] (0,4.154)--(12,4.154)node[above]{$t$};
  \draw[->,>=stealth,semithick] (6.322,0)--(6.322,8)node[above]{$y$};

  \draw[gp path] (6.232,1.206)--(6.412,1.206);
  \node[gp node right] at (6.052,1.206) {$-1$};
  \draw[gp path] (6.232,2.680)--(6.412,2.680);
  \node[gp node right] at (6.052,2.680) {$-0.5$};
  \draw[gp path] (6.232,4.154)--(6.412,4.154);
  \node[gp node right] at (6.052,3.884) {$0$};
  \draw[gp path] (6.232,5.627)--(6.412,5.627);
  \node[gp node right] at (6.052,5.627) {$0.5$};
  \draw[gp path] (6.232,7.101)--(6.412,7.101);
  \node[gp node right] at (6.052,7.101) {$1$};

  \draw[gp path] (0.953,4.064)--(0.953,4.244);
  \node[gp node center] at (0.953,3.884) {$-4$};
  \draw[gp path] (2.295,4.064)--(2.295,4.244);
  \node[gp node center] at (2.025,3.884) {$-3$};
  \draw[gp path] (3.638,4.064)--(3.638,4.244);
  \node[gp node center] at (3.638,3.884) {$-2$};
  \draw[gp path] (4.980,4.064)--(4.980,4.244);
  \node[gp node center] at (4.980,3.884) {$-1$};
  \draw[gp path] (6.322,4.064)--(6.322,4.244);
  %\node[gp node center] at (6.322,3.884) {$0$};
  \draw[gp path] (7.664,4.064)--(7.664,4.244);
  \node[gp node center] at (7.664,3.884) {$1$};
  \draw[gp path] (9.006,4.064)--(9.006,4.244);
  \node[gp node center] at (9.006,3.884) {$2$};
  \draw[gp path] (10.349,4.064)--(10.349,4.244);
  \node[gp node center] at (10.169,3.884) {$3$};
  \draw[gp path] (11.691,4.064)--(11.691,4.244);
  \node[gp node center] at (11.691,3.884) {$4$};

  \draw[gp path] (7.137,1.448)--(7.137,2.680)--(9.525,2.680)--(9.525,1.448)--cycle;
  \gpcolor{rgb color={0.000,0.000,0.000}}
  \gpsetlinewidth{4.00}
  \draw[gp path] (1.196,7.101)--(1.199,7.101)--(1.203,7.101)--(1.206,7.101)--(1.210,7.101)%
  --(1.213,7.101)--(1.217,7.101)--(1.220,7.101)--(1.223,7.101)--(1.227,7.101)--(1.230,7.101)%
  --(1.234,7.101)--(1.237,7.101)--(1.240,7.101)--(1.244,7.101)--(1.247,7.101)--(1.251,7.101)%
  --(1.254,7.101)--(1.258,7.101)--(1.261,7.101)--(1.264,7.101)--(1.268,7.101)--(1.271,7.101)%
  --(1.275,7.101)--(1.278,7.101)--(1.281,7.101)--(1.285,7.101)--(1.288,7.101)--(1.292,7.101)%
  --(1.295,7.101)--(1.299,7.101)--(1.302,7.101)--(1.305,7.101)--(1.309,7.101)--(1.312,7.101)%
  --(1.316,7.101)--(1.319,7.101)--(1.322,7.101)--(1.326,7.101)--(1.329,7.101)--(1.333,7.101)%
  --(1.336,7.101)--(1.340,7.101)--(1.343,7.101)--(1.346,7.101)--(1.350,7.101)--(1.353,7.101)%
  --(1.357,7.101)--(1.360,7.101)--(1.363,7.101)--(1.367,7.101)--(1.370,7.101)--(1.374,7.101)%
  --(1.377,7.101)--(1.381,7.101)--(1.384,7.101)--(1.387,7.101)--(1.391,7.101)--(1.394,7.101)%
  --(1.398,7.101)--(1.401,7.101)--(1.405,7.101)--(1.408,7.101)--(1.411,7.101)--(1.415,7.101)%
  --(1.418,7.101)--(1.422,7.101)--(1.425,7.101)--(1.428,7.101)--(1.432,7.101)--(1.435,7.101)%
  --(1.439,7.101)--(1.442,7.101)--(1.446,7.101)--(1.449,7.101)--(1.452,7.101)--(1.456,7.101)%
  --(1.459,7.101)--(1.463,7.101)--(1.466,7.101)--(1.469,7.101)--(1.473,7.101)--(1.476,7.101)%
  --(1.480,7.101)--(1.483,7.101)--(1.487,7.101)--(1.490,7.101)--(1.493,7.101)--(1.497,7.101)%
  --(1.500,7.101)--(1.504,7.101)--(1.507,7.101)--(1.510,7.101)--(1.514,7.101)--(1.517,7.101)%
  --(1.521,7.101)--(1.524,7.101)--(1.528,7.101)--(1.531,7.101)--(1.534,7.101)--(1.538,7.101)%
  --(1.541,7.101)--(1.545,7.101)--(1.548,7.101)--(1.551,7.101)--(1.555,7.101)--(1.558,7.101)%
  --(1.562,7.101)--(1.565,7.101)--(1.569,7.101)--(1.572,7.101)--(1.575,7.101)--(1.579,7.101)%
  --(1.582,7.101)--(1.586,7.101)--(1.589,7.101)--(1.593,7.101)--(1.596,7.101)--(1.599,7.101)%
  --(1.603,7.101)--(1.606,7.101)--(1.610,7.101)--(1.613,7.101)--(1.616,7.101)--(1.620,7.101)%
  --(1.623,7.101)--(1.627,7.101)--(1.630,7.101)--(1.634,7.101)--(1.637,7.101)--(1.640,7.101)%
  --(1.644,7.101)--(1.647,7.101)--(1.651,7.101)--(1.654,7.101)--(1.657,7.101)--(1.661,7.101)%
  --(1.664,7.101)--(1.668,7.101)--(1.671,7.101)--(1.675,7.101)--(1.678,7.101)--(1.681,7.101)%
  --(1.685,7.101)--(1.688,7.101)--(1.692,7.101)--(1.695,7.101)--(1.698,7.101)--(1.702,7.101)%
  --(1.705,7.101)--(1.709,7.101)--(1.712,7.101)--(1.716,7.101)--(1.719,7.101)--(1.722,7.101)%
  --(1.726,7.101)--(1.729,7.101)--(1.733,7.101)--(1.736,7.101)--(1.739,7.101)--(1.743,7.101)%
  --(1.746,7.101)--(1.750,7.101)--(1.753,7.101)--(1.757,7.101)--(1.760,7.101)--(1.763,7.101)%
  --(1.767,7.101)--(1.770,7.101)--(1.774,7.101)--(1.777,7.101)--(1.781,7.101)--(1.784,7.101)%
  --(1.787,7.101)--(1.791,7.101)--(1.794,7.101)--(1.798,7.101)--(1.801,7.101)--(1.804,7.101)%
  --(1.808,7.101)--(1.811,7.101)--(1.815,7.101)--(1.818,7.101)--(1.822,7.101)--(1.825,7.101)%
  --(1.828,7.101)--(1.832,7.101)--(1.835,7.101)--(1.839,7.101)--(1.842,7.101)--(1.845,7.101)%
  --(1.849,7.101)--(1.852,7.101)--(1.856,7.101)--(1.859,7.101)--(1.863,7.101)--(1.866,7.101)%
  --(1.869,7.101)--(1.873,7.101)--(1.876,7.101)--(1.880,7.101)--(1.883,7.101)--(1.886,7.101)%
  --(1.890,7.101)--(1.893,7.101)--(1.897,7.101)--(1.900,7.101)--(1.904,7.101)--(1.907,7.101)%
  --(1.910,7.101)--(1.914,7.101)--(1.917,7.101)--(1.921,7.101)--(1.924,7.101)--(1.927,7.101)%
  --(1.931,7.101)--(1.934,7.101)--(1.938,7.101)--(1.941,7.101)--(1.945,7.101)--(1.948,7.101)%
  --(1.951,7.101)--(1.955,7.101)--(1.958,7.101)--(1.962,7.101)--(1.965,7.101)--(1.968,7.101)%
  --(1.972,7.101)--(1.975,7.101)--(1.979,7.101)--(1.982,7.101)--(1.986,7.101)--(1.989,7.101)%
  --(1.992,7.101)--(1.996,7.101)--(1.999,7.101)--(2.003,7.101)--(2.006,7.101)--(2.010,7.101)%
  --(2.013,7.101)--(2.016,7.101)--(2.020,7.101)--(2.023,7.101)--(2.027,7.101)--(2.030,7.101)%
  --(2.033,7.101)--(2.037,7.101)--(2.040,7.101)--(2.044,7.101)--(2.047,7.101)--(2.051,7.101)%
  --(2.054,7.101)--(2.057,7.101)--(2.061,7.101)--(2.064,7.101)--(2.068,7.101)--(2.071,7.101)%
  --(2.074,7.101)--(2.078,7.101)--(2.081,7.101)--(2.085,7.101)--(2.088,7.101)--(2.092,7.101)%
  --(2.095,7.101)--(2.098,7.101)--(2.102,7.101)--(2.105,7.101)--(2.109,7.101)--(2.112,7.101)%
  --(2.115,7.101)--(2.119,7.101)--(2.122,7.101)--(2.126,7.101)--(2.129,7.101)--(2.133,7.101)%
  --(2.136,7.101)--(2.139,7.101)--(2.143,7.101)--(2.146,7.101)--(2.150,7.101)--(2.153,7.101)%
  --(2.156,7.101)--(2.160,7.101)--(2.163,7.101)--(2.167,7.101)--(2.170,7.101)--(2.174,7.101)%
  --(2.177,7.101)--(2.180,7.101)--(2.184,7.101)--(2.187,7.101)--(2.191,7.101)--(2.194,7.101)%
  --(2.198,7.101)--(2.201,7.101)--(2.204,7.101)--(2.208,7.101)--(2.211,7.101)--(2.215,7.101)%
  --(2.218,7.101)--(2.221,7.101)--(2.225,7.101)--(2.228,7.101)--(2.232,7.101)--(2.235,7.101)%
  --(2.239,7.101)--(2.242,7.101)--(2.245,7.101)--(2.249,7.101)--(2.252,7.101)--(2.256,7.101)%
  --(2.259,7.101)--(2.262,7.101)--(2.266,7.101)--(2.269,7.101)--(2.273,7.101)--(2.276,7.101)%
  --(2.280,7.101)--(2.283,7.101)--(2.286,7.101)--(2.290,7.101)--(2.293,7.101)--(2.297,1.206)%
  --(2.300,1.206)--(2.303,1.206)--(2.307,1.206)--(2.310,1.206)--(2.314,1.206)--(2.317,1.206)%
  --(2.321,1.206)--(2.324,1.206)--(2.327,1.206)--(2.331,1.206)--(2.334,1.206)--(2.338,1.206)%
  --(2.341,1.206)--(2.344,1.206)--(2.348,1.206)--(2.351,1.206)--(2.355,1.206)--(2.358,1.206)%
  --(2.362,1.206)--(2.365,1.206)--(2.368,1.206)--(2.372,1.206)--(2.375,1.206)--(2.379,1.206)%
  --(2.382,1.206)--(2.386,1.206)--(2.389,1.206)--(2.392,1.206)--(2.396,1.206)--(2.399,1.206)%
  --(2.403,1.206)--(2.406,1.206)--(2.409,1.206)--(2.413,1.206)--(2.416,1.206)--(2.420,1.206)%
  --(2.423,1.206)--(2.427,1.206)--(2.430,1.206)--(2.433,1.206)--(2.437,1.206)--(2.440,1.206)%
  --(2.444,1.206)--(2.447,1.206)--(2.450,1.206)--(2.454,1.206)--(2.457,1.206)--(2.461,1.206)%
  --(2.464,1.206)--(2.468,1.206)--(2.471,1.206)--(2.474,1.206)--(2.478,1.206)--(2.481,1.206)%
  --(2.485,1.206)--(2.488,1.206)--(2.491,1.206)--(2.495,1.206)--(2.498,1.206)--(2.502,1.206)%
  --(2.505,1.206)--(2.509,1.206)--(2.512,1.206)--(2.515,1.206)--(2.519,1.206)--(2.522,1.206)%
  --(2.526,1.206)--(2.529,1.206)--(2.532,1.206)--(2.536,1.206)--(2.539,1.206)--(2.543,1.206)%
  --(2.546,1.206)--(2.550,1.206)--(2.553,1.206)--(2.556,1.206)--(2.560,1.206)--(2.563,1.206)%
  --(2.567,1.206)--(2.570,1.206)--(2.574,1.206)--(2.577,1.206)--(2.580,1.206)--(2.584,1.206)%
  --(2.587,1.206)--(2.591,1.206)--(2.594,1.206)--(2.597,1.206)--(2.601,1.206)--(2.604,1.206)%
  --(2.608,1.206)--(2.611,1.206)--(2.615,1.206)--(2.618,1.206)--(2.621,1.206)--(2.625,1.206)%
  --(2.628,1.206)--(2.632,1.206)--(2.635,1.206)--(2.638,1.206)--(2.642,1.206)--(2.645,1.206)%
  --(2.649,1.206)--(2.652,1.206)--(2.656,1.206)--(2.659,1.206)--(2.662,1.206)--(2.666,1.206)%
  --(2.669,1.206)--(2.673,1.206)--(2.676,1.206)--(2.679,1.206)--(2.683,1.206)--(2.686,1.206)%
  --(2.690,1.206)--(2.693,1.206)--(2.697,1.206)--(2.700,1.206)--(2.703,1.206)--(2.707,1.206)%
  --(2.710,1.206)--(2.714,1.206)--(2.717,1.206)--(2.720,1.206)--(2.724,1.206)--(2.727,1.206)%
  --(2.731,1.206)--(2.734,1.206)--(2.738,1.206)--(2.741,1.206)--(2.744,1.206)--(2.748,1.206)%
  --(2.751,1.206)--(2.755,1.206)--(2.758,1.206)--(2.762,1.206)--(2.765,1.206)--(2.768,1.206)%
  --(2.772,1.206)--(2.775,1.206)--(2.779,1.206)--(2.782,1.206)--(2.785,1.206)--(2.789,1.206)%
  --(2.792,1.206)--(2.796,1.206)--(2.799,1.206)--(2.803,1.206)--(2.806,1.206)--(2.809,1.206)%
  --(2.813,1.206)--(2.816,1.206)--(2.820,1.206)--(2.823,1.206)--(2.826,1.206)--(2.830,1.206)%
  --(2.833,1.206)--(2.837,1.206)--(2.840,1.206)--(2.844,1.206)--(2.847,1.206)--(2.850,1.206)%
  --(2.854,1.206)--(2.857,1.206)--(2.861,1.206)--(2.864,1.206)--(2.867,1.206)--(2.871,1.206)%
  --(2.874,1.206)--(2.878,1.206)--(2.881,1.206)--(2.885,1.206)--(2.888,1.206)--(2.891,1.206)%
  --(2.895,1.206)--(2.898,1.206)--(2.902,1.206)--(2.905,1.206)--(2.908,1.206)--(2.912,1.206)%
  --(2.915,1.206)--(2.919,1.206)--(2.922,1.206)--(2.926,1.206)--(2.929,1.206)--(2.932,1.206)%
  --(2.936,1.206)--(2.939,1.206)--(2.943,1.206)--(2.946,1.206)--(2.950,1.206)--(2.953,1.206)%
  --(2.956,1.206)--(2.960,1.206)--(2.963,1.206)--(2.967,1.206)--(2.970,1.206)--(2.973,1.206)%
  --(2.977,1.206)--(2.980,1.206)--(2.984,1.206)--(2.987,1.206)--(2.991,1.206)--(2.994,1.206)%
  --(2.997,1.206)--(3.001,1.206)--(3.004,1.206)--(3.008,1.206)--(3.011,1.206)--(3.014,1.206)%
  --(3.018,1.206)--(3.021,1.206)--(3.025,1.206)--(3.028,1.206)--(3.032,1.206)--(3.035,1.206)%
  --(3.038,1.206)--(3.042,1.206)--(3.045,1.206)--(3.049,1.206)--(3.052,1.206)--(3.055,1.206)%
  --(3.059,1.206)--(3.062,1.206)--(3.066,1.206)--(3.069,1.206)--(3.073,1.206)--(3.076,1.206)%
  --(3.079,1.206)--(3.083,1.206)--(3.086,1.206)--(3.090,1.206)--(3.093,1.206)--(3.096,1.206)%
  --(3.100,1.206)--(3.103,1.206)--(3.107,1.206)--(3.110,1.206)--(3.114,1.206)--(3.117,1.206)%
  --(3.120,1.206)--(3.124,1.206)--(3.127,1.206)--(3.131,1.206)--(3.134,1.206)--(3.138,1.206)%
  --(3.141,1.206)--(3.144,1.206)--(3.148,1.206)--(3.151,1.206)--(3.155,1.206)--(3.158,1.206)%
  --(3.161,1.206)--(3.165,1.206)--(3.168,1.206)--(3.172,1.206)--(3.175,1.206)--(3.179,1.206)%
  --(3.182,1.206)--(3.185,1.206)--(3.189,1.206)--(3.192,1.206)--(3.196,1.206)--(3.199,1.206)%
  --(3.202,1.206)--(3.206,1.206)--(3.209,1.206)--(3.213,1.206)--(3.216,1.206)--(3.220,1.206)%
  --(3.223,1.206)--(3.226,1.206)--(3.230,1.206)--(3.233,1.206)--(3.237,1.206)--(3.240,1.206)%
  --(3.243,1.206)--(3.247,1.206)--(3.250,1.206)--(3.254,1.206)--(3.257,1.206)--(3.261,1.206)%
  --(3.264,1.206)--(3.267,1.206)--(3.271,1.206)--(3.274,1.206)--(3.278,1.206)--(3.281,1.206)%
  --(3.284,1.206)--(3.288,1.206)--(3.291,1.206)--(3.295,1.206)--(3.298,1.206)--(3.302,1.206)%
  --(3.305,1.206)--(3.308,1.206)--(3.312,1.206)--(3.315,1.206)--(3.319,1.206)--(3.322,1.206)%
  --(3.326,1.206)--(3.329,1.206)--(3.332,1.206)--(3.336,1.206)--(3.339,1.206)--(3.343,1.206)%
  --(3.346,1.206)--(3.349,1.206)--(3.353,1.206)--(3.356,1.206)--(3.360,1.206)--(3.363,1.206)%
  --(3.367,1.206)--(3.370,1.206)--(3.373,1.206)--(3.377,1.206)--(3.380,1.206)--(3.384,1.206)%
  --(3.387,1.206)--(3.390,1.206)--(3.394,1.206)--(3.397,1.206)--(3.401,1.206)--(3.404,1.206)%
  --(3.408,1.206)--(3.411,1.206)--(3.414,1.206)--(3.418,1.206)--(3.421,1.206)--(3.425,1.206)%
  --(3.428,1.206)--(3.431,1.206)--(3.435,1.206)--(3.438,1.206)--(3.442,1.206)--(3.445,1.206)%
  --(3.449,1.206)--(3.452,1.206)--(3.455,1.206)--(3.459,1.206)--(3.462,1.206)--(3.466,1.206)%
  --(3.469,1.206)--(3.472,1.206)--(3.476,1.206)--(3.479,1.206)--(3.483,1.206)--(3.486,1.206)%
  --(3.490,1.206)--(3.493,1.206)--(3.496,1.206)--(3.500,1.206)--(3.503,1.206)--(3.507,1.206)%
  --(3.510,1.206)--(3.513,1.206)--(3.517,1.206)--(3.520,1.206)--(3.524,1.206)--(3.527,1.206)%
  --(3.531,1.206)--(3.534,1.206)--(3.537,1.206)--(3.541,1.206)--(3.544,1.206)--(3.548,1.206)%
  --(3.551,1.206)--(3.555,1.206)--(3.558,1.206)--(3.561,1.206)--(3.565,1.206)--(3.568,1.206)%
  --(3.572,1.206)--(3.575,1.206)--(3.578,1.206)--(3.582,1.206)--(3.585,1.206)--(3.589,1.206)%
  --(3.592,1.206)--(3.596,1.206)--(3.599,1.206)--(3.602,1.206)--(3.606,1.206)--(3.609,1.206)%
  --(3.613,1.206)--(3.616,1.206)--(3.619,1.206)--(3.623,1.206)--(3.626,1.206)--(3.630,1.206)%
  --(3.633,1.206)--(3.637,1.206)--(3.640,1.206)--(3.643,1.206)--(3.647,1.206)--(3.650,1.206)%
  --(3.654,1.206)--(3.657,1.206)--(3.660,1.206)--(3.664,1.206)--(3.667,1.206)--(3.671,1.206)%
  --(3.674,1.206)--(3.678,1.206)--(3.681,1.206)--(3.684,1.206)--(3.688,1.206)--(3.691,1.206)%
  --(3.695,1.206)--(3.698,1.206)--(3.701,1.206)--(3.705,1.206)--(3.708,1.206)--(3.712,1.206)%
  --(3.715,1.206)--(3.719,1.206)--(3.722,1.206)--(3.725,1.206)--(3.729,1.206)--(3.732,1.206)%
  --(3.736,1.206)--(3.739,1.206)--(3.743,1.206)--(3.746,1.206)--(3.749,1.206)--(3.753,1.206)%
  --(3.756,1.206)--(3.760,1.206)--(3.763,1.206)--(3.766,1.206)--(3.770,1.206)--(3.773,1.206)%
  --(3.777,1.206)--(3.780,1.206)--(3.784,1.206)--(3.787,1.206)--(3.790,1.206)--(3.794,1.206)%
  --(3.797,1.206)--(3.801,1.206)--(3.804,1.206)--(3.807,1.206)--(3.811,1.206)--(3.814,1.206)%
  --(3.818,1.206)--(3.821,1.206)--(3.825,1.206)--(3.828,1.206)--(3.831,1.206)--(3.835,1.206)%
  --(3.838,1.206)--(3.842,1.206)--(3.845,1.206)--(3.848,1.206)--(3.852,1.206)--(3.855,1.206)%
  --(3.859,1.206)--(3.862,1.206)--(3.866,1.206)--(3.869,1.206)--(3.872,1.206)--(3.876,1.206)%
  --(3.879,1.206)--(3.883,1.206)--(3.886,1.206)--(3.889,1.206)--(3.893,1.206)--(3.896,1.206)%
  --(3.900,1.206)--(3.903,1.206)--(3.907,1.206)--(3.910,1.206)--(3.913,1.206)--(3.917,1.206)%
  --(3.920,1.206)--(3.924,1.206)--(3.927,1.206)--(3.931,1.206)--(3.934,1.206)--(3.937,1.206)%
  --(3.941,1.206)--(3.944,1.206)--(3.948,1.206)--(3.951,1.206)--(3.954,1.206)--(3.958,1.206)%
  --(3.961,1.206)--(3.965,1.206)--(3.968,1.206)--(3.972,1.206)--(3.975,1.206)--(3.978,1.206)%
  --(3.982,1.206)--(3.985,1.206)--(3.989,1.206)--(3.992,1.206)--(3.995,1.206)--(3.999,1.206)%
  --(4.002,1.206)--(4.006,1.206)--(4.009,1.206)--(4.013,1.206)--(4.016,1.206)--(4.019,1.206)%
  --(4.023,1.206)--(4.026,1.206)--(4.030,1.206)--(4.033,1.206)--(4.036,1.206)--(4.040,1.206)%
  --(4.043,1.206)--(4.047,1.206)--(4.050,1.206)--(4.054,1.206)--(4.057,1.206)--(4.060,1.206)%
  --(4.064,1.206)--(4.067,1.206)--(4.071,1.206)--(4.074,1.206)--(4.077,1.206)--(4.081,1.206)%
  --(4.084,1.206)--(4.088,1.206)--(4.091,1.206)--(4.095,1.206)--(4.098,1.206)--(4.101,1.206)%
  --(4.105,1.206)--(4.108,1.206)--(4.112,1.206)--(4.115,1.206)--(4.119,1.206)--(4.122,1.206)%
  --(4.125,1.206)--(4.129,1.206)--(4.132,1.206)--(4.136,1.206)--(4.139,1.206)--(4.142,1.206)%
  --(4.146,1.206)--(4.149,1.206)--(4.153,1.206)--(4.156,1.206)--(4.160,1.206)--(4.163,1.206)%
  --(4.166,1.206)--(4.170,1.206)--(4.173,1.206)--(4.177,1.206)--(4.180,1.206)--(4.183,1.206)%
  --(4.187,1.206)--(4.190,1.206)--(4.194,1.206)--(4.197,1.206)--(4.201,1.206)--(4.204,1.206)%
  --(4.207,1.206)--(4.211,1.206)--(4.214,1.206)--(4.218,1.206)--(4.221,1.206)--(4.224,1.206)%
  --(4.228,1.206)--(4.231,1.206)--(4.235,1.206)--(4.238,1.206)--(4.242,1.206)--(4.245,1.206)%
  --(4.248,1.206)--(4.252,1.206)--(4.255,1.206)--(4.259,1.206)--(4.262,1.206)--(4.265,1.206)%
  --(4.269,1.206)--(4.272,1.206)--(4.276,1.206)--(4.279,1.206)--(4.283,1.206)--(4.286,1.206)%
  --(4.289,1.206)--(4.293,1.206)--(4.296,1.206)--(4.300,1.206)--(4.303,1.206)--(4.307,1.206)%
  --(4.310,1.206)--(4.313,1.206)--(4.317,1.206)--(4.320,1.206)--(4.324,1.206)--(4.327,1.206)%
  --(4.330,1.206)--(4.334,1.206)--(4.337,1.206)--(4.341,1.206)--(4.344,1.206)--(4.348,1.206)%
  --(4.351,1.206)--(4.354,1.206)--(4.358,1.206)--(4.361,1.206)--(4.365,1.206)--(4.368,1.206)%
  --(4.371,1.206)--(4.375,1.206)--(4.378,1.206)--(4.382,1.206)--(4.385,1.206)--(4.389,1.206)%
  --(4.392,1.206)--(4.395,1.206)--(4.399,1.206)--(4.402,1.206)--(4.406,1.206)--(4.409,1.206)%
  --(4.412,1.206)--(4.416,1.206)--(4.419,1.206)--(4.423,1.206)--(4.426,1.206)--(4.430,1.206)%
  --(4.433,1.206)--(4.436,1.206)--(4.440,1.206)--(4.443,1.206)--(4.447,1.206)--(4.450,1.206)%
  --(4.453,1.206)--(4.457,1.206)--(4.460,1.206)--(4.464,1.206)--(4.467,1.206)--(4.471,1.206)%
  --(4.474,1.206)--(4.477,1.206)--(4.481,1.206)--(4.484,1.206)--(4.488,1.206)--(4.491,1.206)%
  --(4.495,1.206)--(4.498,1.206)--(4.501,1.206)--(4.505,1.206)--(4.508,1.206)--(4.512,1.206)%
  --(4.515,1.206)--(4.518,1.206)--(4.522,1.206)--(4.525,1.206)--(4.529,1.206)--(4.532,1.206)%
  --(4.536,1.206)--(4.539,1.206)--(4.542,1.206)--(4.546,1.206)--(4.549,1.206)--(4.553,1.206)%
  --(4.556,1.206)--(4.559,1.206)--(4.563,1.206)--(4.566,1.206)--(4.570,1.206)--(4.573,1.206)%
  --(4.577,1.206)--(4.580,1.206)--(4.583,1.206)--(4.587,1.206)--(4.590,1.206)--(4.594,1.206)%
  --(4.597,1.206)--(4.600,1.206)--(4.604,1.206)--(4.607,1.206)--(4.611,1.206)--(4.614,1.206)%
  --(4.618,1.206)--(4.621,1.206)--(4.624,1.206)--(4.628,1.206)--(4.631,1.206)--(4.635,1.206)%
  --(4.638,1.206)--(4.641,1.206)--(4.645,1.206)--(4.648,1.206)--(4.652,1.206)--(4.655,1.206)%
  --(4.659,1.206)--(4.662,1.206)--(4.665,1.206)--(4.669,1.206)--(4.672,1.206)--(4.676,1.206)%
  --(4.679,1.206)--(4.683,1.206)--(4.686,1.206)--(4.689,1.206)--(4.693,1.206)--(4.696,1.206)%
  --(4.700,1.206)--(4.703,1.206)--(4.706,1.206)--(4.710,1.206)--(4.713,1.206)--(4.717,1.206)%
  --(4.720,1.206)--(4.724,1.206)--(4.727,1.206)--(4.730,1.206)--(4.734,1.206)--(4.737,1.206)%
  --(4.741,1.206)--(4.744,1.206)--(4.747,1.206)--(4.751,1.206)--(4.754,1.206)--(4.758,1.206)%
  --(4.761,1.206)--(4.765,1.206)--(4.768,1.206)--(4.771,1.206)--(4.775,1.206)--(4.778,1.206)%
  --(4.782,1.206)--(4.785,1.206)--(4.788,1.206)--(4.792,1.206)--(4.795,1.206)--(4.799,1.206)%
  --(4.802,1.206)--(4.806,1.206)--(4.809,1.206)--(4.812,1.206)--(4.816,1.206)--(4.819,1.206)%
  --(4.823,1.206)--(4.826,1.206)--(4.829,1.206)--(4.833,1.206)--(4.836,1.206)--(4.840,1.206)%
  --(4.843,1.206)--(4.847,1.206)--(4.850,1.206)--(4.853,1.206)--(4.857,1.206)--(4.860,1.206)%
  --(4.864,1.206)--(4.867,1.206)--(4.870,1.206)--(4.874,1.206)--(4.877,1.206)--(4.881,1.206)%
  --(4.884,1.206)--(4.888,1.206)--(4.891,1.206)--(4.894,1.206)--(4.898,1.206)--(4.901,1.206)%
  --(4.905,1.206)--(4.908,1.206)--(4.912,1.206)--(4.915,1.206)--(4.918,1.206)--(4.922,1.206)%
  --(4.925,1.206)--(4.929,1.206)--(4.932,1.206)--(4.935,1.206)--(4.939,1.206)--(4.942,1.206)%
  --(4.946,1.206)--(4.949,1.206)--(4.953,1.206)--(4.956,1.206)--(4.959,1.206)--(4.963,1.206)%
  --(4.966,1.206)--(4.970,1.206)--(4.973,1.206)--(4.976,1.206)--(4.980,1.206)--(4.983,1.206)%
  --(4.987,1.206)--(4.990,1.206)--(4.994,1.206)--(4.997,1.206)--(5.000,1.206)--(5.004,1.206)%
  --(5.007,1.206)--(5.011,1.206)--(5.014,1.206)--(5.017,1.206)--(5.021,1.206)--(5.024,1.206)%
  --(5.028,1.206)--(5.031,1.206)--(5.035,1.206)--(5.038,1.206)--(5.041,1.206)--(5.045,1.206)%
  --(5.048,1.206)--(5.052,1.206)--(5.055,1.206)--(5.058,1.206)--(5.062,1.206)--(5.065,1.206)%
  --(5.069,1.206)--(5.072,1.206)--(5.076,1.206)--(5.079,1.206)--(5.082,1.206)--(5.086,1.206)%
  --(5.089,1.206)--(5.093,1.206)--(5.096,1.206)--(5.100,1.206)--(5.103,1.206)--(5.106,1.206)%
  --(5.110,1.206)--(5.113,1.206)--(5.117,1.206)--(5.120,1.206)--(5.123,1.206)--(5.127,1.206)%
  --(5.130,1.206)--(5.134,1.206)--(5.137,1.206)--(5.141,1.206)--(5.144,1.206)--(5.147,1.206)%
  --(5.151,1.206)--(5.154,1.206)--(5.158,1.206)--(5.161,1.206)--(5.164,1.206)--(5.168,1.206)%
  --(5.171,1.206)--(5.175,1.206)--(5.178,1.206)--(5.182,1.206)--(5.185,1.206)--(5.188,1.206)%
  --(5.192,1.206)--(5.195,1.206)--(5.199,1.206)--(5.202,1.206)--(5.205,1.206)--(5.209,1.206)%
  --(5.212,1.206)--(5.216,1.206)--(5.219,1.206)--(5.223,1.206)--(5.226,1.206)--(5.229,1.206)%
  --(5.233,1.206)--(5.236,1.206)--(5.240,1.206)--(5.243,1.206)--(5.246,1.206)--(5.250,1.206)%
  --(5.253,1.206)--(5.257,1.206)--(5.260,1.206)--(5.264,1.206)--(5.267,1.206)--(5.270,1.206)%
  --(5.274,1.206)--(5.277,1.206)--(5.281,1.206)--(5.284,1.206)--(5.288,1.206)--(5.291,1.206)%
  --(5.294,1.206)--(5.298,1.206)--(5.301,1.206)--(5.305,1.206)--(5.308,1.206)--(5.311,1.206)%
  --(5.315,1.206)--(5.318,1.206)--(5.322,1.206)--(5.325,1.206)--(5.329,1.206)--(5.332,1.206)%
  --(5.335,1.206)--(5.339,1.206)--(5.342,1.206)--(5.346,1.206)--(5.349,1.206)--(5.352,1.206)%
  --(5.356,1.206)--(5.359,1.206)--(5.363,1.206)--(5.366,1.206)--(5.370,1.206)--(5.373,1.206)%
  --(5.376,1.206)--(5.380,1.206)--(5.383,1.206)--(5.387,1.206)--(5.390,1.206)--(5.393,1.206)%
  --(5.397,1.206)--(5.400,1.206)--(5.404,1.206)--(5.407,1.206)--(5.411,1.206)--(5.414,1.206)%
  --(5.417,1.206)--(5.421,1.206)--(5.424,1.206)--(5.428,1.206)--(5.431,1.206)--(5.434,1.206)%
  --(5.438,1.206)--(5.441,1.206)--(5.445,1.206)--(5.448,1.206)--(5.452,1.206)--(5.455,1.206)%
  --(5.458,1.206)--(5.462,1.206)--(5.465,1.206)--(5.469,1.206)--(5.472,1.206)--(5.476,1.206)%
  --(5.479,1.206)--(5.482,1.206)--(5.486,1.206)--(5.489,1.206)--(5.493,1.206)--(5.496,1.206)%
  --(5.499,1.206)--(5.503,1.206)--(5.506,1.206)--(5.510,1.206)--(5.513,1.206)--(5.517,1.206)%
  --(5.520,1.206)--(5.523,1.206)--(5.527,1.206)--(5.530,1.206)--(5.534,1.206)--(5.537,1.206)%
  --(5.540,1.206)--(5.544,1.206)--(5.547,1.206)--(5.551,1.206)--(5.554,1.206)--(5.558,1.206)%
  --(5.561,1.206)--(5.564,1.206)--(5.568,1.206)--(5.571,1.206)--(5.575,1.206)--(5.578,1.206)%
  --(5.581,1.206)--(5.585,1.206)--(5.588,1.206)--(5.592,1.206)--(5.595,1.206)--(5.599,1.206)%
  --(5.602,1.206)--(5.605,1.206)--(5.609,1.206)--(5.612,1.206)--(5.616,1.206)--(5.619,1.206)%
  --(5.622,1.206)--(5.626,1.206)--(5.629,1.206)--(5.633,1.206)--(5.636,1.206)--(5.640,1.206)%
  --(5.643,1.206)--(5.646,1.206)--(5.650,1.206)--(5.653,1.206)--(5.657,1.206)--(5.660,1.206)%
  --(5.664,1.206)--(5.667,1.206)--(5.670,1.206)--(5.674,1.206)--(5.677,1.206)--(5.681,1.206)%
  --(5.684,1.206)--(5.687,1.206)--(5.691,1.206)--(5.694,1.206)--(5.698,1.206)--(5.701,1.206)%
  --(5.705,1.206)--(5.708,1.206)--(5.711,1.206)--(5.715,1.206)--(5.718,1.206)--(5.722,1.206)%
  --(5.725,1.206)--(5.728,1.206)--(5.732,1.206)--(5.735,1.206)--(5.739,1.206)--(5.742,1.206)%
  --(5.746,1.206)--(5.749,1.206)--(5.752,1.206)--(5.756,1.206)--(5.759,1.206)--(5.763,1.206)%
  --(5.766,1.206)--(5.769,1.206)--(5.773,1.206)--(5.776,1.206)--(5.780,1.206)--(5.783,1.206)%
  --(5.787,1.206)--(5.790,1.206)--(5.793,1.206)--(5.797,1.206)--(5.800,1.206)--(5.804,1.206)%
  --(5.807,1.206)--(5.810,1.206)--(5.814,1.206)--(5.817,1.206)--(5.821,1.206)--(5.824,1.206)%
  --(5.828,1.206)--(5.831,1.206)--(5.834,1.206)--(5.838,1.206)--(5.841,1.206)--(5.845,1.206)%
  --(5.848,1.206)--(5.852,1.206)--(5.855,1.206)--(5.858,1.206)--(5.862,1.206)--(5.865,1.206)%
  --(5.869,1.206)--(5.872,1.206)--(5.875,1.206)--(5.879,1.206)--(5.882,1.206)--(5.886,1.206)%
  --(5.889,1.206)--(5.893,1.206)--(5.896,1.206)--(5.899,1.206)--(5.903,1.206)--(5.906,1.206)%
  --(5.910,1.206)--(5.913,1.206)--(5.916,1.206)--(5.920,1.206)--(5.923,1.206)--(5.927,1.206)%
  --(5.930,1.206)--(5.934,1.206)--(5.937,1.206)--(5.940,1.206)--(5.944,1.206)--(5.947,1.206)%
  --(5.951,1.206)--(5.954,1.206)--(5.957,1.206)--(5.961,1.206)--(5.964,1.206)--(5.968,1.206)%
  --(5.971,1.206)--(5.975,1.206)--(5.978,1.206)--(5.981,1.206)--(5.985,1.206)--(5.988,1.206)%
  --(5.992,1.206)--(5.995,1.206)--(5.998,1.206)--(6.002,1.206)--(6.005,1.206)--(6.009,1.206)%
  --(6.012,1.206)--(6.016,1.206)--(6.019,1.206)--(6.022,1.206)--(6.026,1.206)--(6.029,1.206)%
  --(6.033,1.206)--(6.036,1.206)--(6.040,1.206)--(6.043,1.206)--(6.046,1.206)--(6.050,1.206)%
  --(6.053,1.206)--(6.057,1.206)--(6.060,1.206)--(6.063,1.206)--(6.067,1.206)--(6.070,1.206)%
  --(6.074,1.206)--(6.077,1.206)--(6.081,1.206)--(6.084,1.206)--(6.087,1.206)--(6.091,1.206)%
  --(6.094,1.206)--(6.098,1.206)--(6.101,1.206)--(6.104,1.206)--(6.108,1.206)--(6.111,1.206)%
  --(6.115,1.206)--(6.118,1.206)--(6.122,1.206)--(6.125,1.206)--(6.128,1.206)--(6.132,1.206)%
  --(6.135,1.206)--(6.139,1.206)--(6.142,1.206)--(6.145,1.206)--(6.149,1.206)--(6.152,1.206)%
  --(6.156,1.206)--(6.159,1.206)--(6.163,1.206)--(6.166,1.206)--(6.169,1.206)--(6.173,1.206)%
  --(6.176,1.206)--(6.180,1.206)--(6.183,1.206)--(6.186,1.206)--(6.190,1.206)--(6.193,1.206)%
  --(6.197,1.206)--(6.200,1.206)--(6.204,1.206)--(6.207,1.206)--(6.210,1.206)--(6.214,1.206)%
  --(6.217,1.206)--(6.221,1.206)--(6.224,1.206)--(6.228,1.206)--(6.231,1.206)--(6.234,1.206)%
  --(6.238,1.206)--(6.241,1.206)--(6.245,1.206)--(6.248,1.206)--(6.251,1.206)--(6.255,1.206)%
  --(6.258,1.206)--(6.262,1.206)--(6.265,1.206)--(6.269,1.206)--(6.272,1.206)--(6.275,1.206)%
  --(6.279,1.206)--(6.282,1.206)--(6.286,1.206)--(6.289,1.206)--(6.292,1.206)--(6.296,1.206)%
  --(6.299,1.206)--(6.303,1.206)--(6.306,1.206)--(6.310,1.206)--(6.313,1.206)--(6.316,1.206)%
  --(6.320,1.206)--(6.323,7.101)--(6.327,7.101)--(6.330,7.101)--(6.333,7.101)--(6.337,7.101)%
  --(6.340,7.101)--(6.344,7.101)--(6.347,7.101)--(6.351,7.101)--(6.354,7.101)--(6.357,7.101)%
  --(6.361,7.101)--(6.364,7.101)--(6.368,7.101)--(6.371,7.101)--(6.374,7.101)--(6.378,7.101)%
  --(6.381,7.101)--(6.385,7.101)--(6.388,7.101)--(6.392,7.101)--(6.395,7.101)--(6.398,7.101)%
  --(6.402,7.101)--(6.405,7.101)--(6.409,7.101)--(6.412,7.101)--(6.415,7.101)--(6.419,7.101)%
  --(6.422,7.101)--(6.426,7.101)--(6.429,7.101)--(6.433,7.101)--(6.436,7.101)--(6.439,7.101)%
  --(6.443,7.101)--(6.446,7.101)--(6.450,7.101)--(6.453,7.101)--(6.457,7.101)--(6.460,7.101)%
  --(6.463,7.101)--(6.467,7.101)--(6.470,7.101)--(6.474,7.101)--(6.477,7.101)--(6.480,7.101)%
  --(6.484,7.101)--(6.487,7.101)--(6.491,7.101)--(6.494,7.101)--(6.498,7.101)--(6.501,7.101)%
  --(6.504,7.101)--(6.508,7.101)--(6.511,7.101)--(6.515,7.101)--(6.518,7.101)--(6.521,7.101)%
  --(6.525,7.101)--(6.528,7.101)--(6.532,7.101)--(6.535,7.101)--(6.539,7.101)--(6.542,7.101)%
  --(6.545,7.101)--(6.549,7.101)--(6.552,7.101)--(6.556,7.101)--(6.559,7.101)--(6.562,7.101)%
  --(6.566,7.101)--(6.569,7.101)--(6.573,7.101)--(6.576,7.101)--(6.580,7.101)--(6.583,7.101)%
  --(6.586,7.101)--(6.590,7.101)--(6.593,7.101)--(6.597,7.101)--(6.600,7.101)--(6.603,7.101)%
  --(6.607,7.101)--(6.610,7.101)--(6.614,7.101)--(6.617,7.101)--(6.621,7.101)--(6.624,7.101)%
  --(6.627,7.101)--(6.631,7.101)--(6.634,7.101)--(6.638,7.101)--(6.641,7.101)--(6.645,7.101)%
  --(6.648,7.101)--(6.651,7.101)--(6.655,7.101)--(6.658,7.101)--(6.662,7.101)--(6.665,7.101)%
  --(6.668,7.101)--(6.672,7.101)--(6.675,7.101)--(6.679,7.101)--(6.682,7.101)--(6.686,7.101)%
  --(6.689,7.101)--(6.692,7.101)--(6.696,7.101)--(6.699,7.101)--(6.703,7.101)--(6.706,7.101)%
  --(6.709,7.101)--(6.713,7.101)--(6.716,7.101)--(6.720,7.101)--(6.723,7.101)--(6.727,7.101)%
  --(6.730,7.101)--(6.733,7.101)--(6.737,7.101)--(6.740,7.101)--(6.744,7.101)--(6.747,7.101)%
  --(6.750,7.101)--(6.754,7.101)--(6.757,7.101)--(6.761,7.101)--(6.764,7.101)--(6.768,7.101)%
  --(6.771,7.101)--(6.774,7.101)--(6.778,7.101)--(6.781,7.101)--(6.785,7.101)--(6.788,7.101)%
  --(6.791,7.101)--(6.795,7.101)--(6.798,7.101)--(6.802,7.101)--(6.805,7.101)--(6.809,7.101)%
  --(6.812,7.101)--(6.815,7.101)--(6.819,7.101)--(6.822,7.101)--(6.826,7.101)--(6.829,7.101)%
  --(6.833,7.101)--(6.836,7.101)--(6.839,7.101)--(6.843,7.101)--(6.846,7.101)--(6.850,7.101)%
  --(6.853,7.101)--(6.856,7.101)--(6.860,7.101)--(6.863,7.101)--(6.867,7.101)--(6.870,7.101)%
  --(6.874,7.101)--(6.877,7.101)--(6.880,7.101)--(6.884,7.101)--(6.887,7.101)--(6.891,7.101)%
  --(6.894,7.101)--(6.897,7.101)--(6.901,7.101)--(6.904,7.101)--(6.908,7.101)--(6.911,7.101)%
  --(6.915,7.101)--(6.918,7.101)--(6.921,7.101)--(6.925,7.101)--(6.928,7.101)--(6.932,7.101)%
  --(6.935,7.101)--(6.938,7.101)--(6.942,7.101)--(6.945,7.101)--(6.949,7.101)--(6.952,7.101)%
  --(6.956,7.101)--(6.959,7.101)--(6.962,7.101)--(6.966,7.101)--(6.969,7.101)--(6.973,7.101)%
  --(6.976,7.101)--(6.979,7.101)--(6.983,7.101)--(6.986,7.101)--(6.990,7.101)--(6.993,7.101)%
  --(6.997,7.101)--(7.000,7.101)--(7.003,7.101)--(7.007,7.101)--(7.010,7.101)--(7.014,7.101)%
  --(7.017,7.101)--(7.021,7.101)--(7.024,7.101)--(7.027,7.101)--(7.031,7.101)--(7.034,7.101)%
  --(7.038,7.101)--(7.041,7.101)--(7.044,7.101)--(7.048,7.101)--(7.051,7.101)--(7.055,7.101)%
  --(7.058,7.101)--(7.062,7.101)--(7.065,7.101)--(7.068,7.101)--(7.072,7.101)--(7.075,7.101)%
  --(7.079,7.101)--(7.082,7.101)--(7.085,7.101)--(7.089,7.101)--(7.092,7.101)--(7.096,7.101)%
  --(7.099,7.101)--(7.103,7.101)--(7.106,7.101)--(7.109,7.101)--(7.113,7.101)--(7.116,7.101)%
  --(7.120,7.101)--(7.123,7.101)--(7.126,7.101)--(7.130,7.101)--(7.133,7.101)--(7.137,7.101)%
  --(7.140,7.101)--(7.144,7.101)--(7.147,7.101)--(7.150,7.101)--(7.154,7.101)--(7.157,7.101)%
  --(7.161,7.101)--(7.164,7.101)--(7.167,7.101)--(7.171,7.101)--(7.174,7.101)--(7.178,7.101)%
  --(7.181,7.101)--(7.185,7.101)--(7.188,7.101)--(7.191,7.101)--(7.195,7.101)--(7.198,7.101)%
  --(7.202,7.101)--(7.205,7.101)--(7.209,7.101)--(7.212,7.101)--(7.215,7.101)--(7.219,7.101)%
  --(7.222,7.101)--(7.226,7.101)--(7.229,7.101)--(7.232,7.101)--(7.236,7.101)--(7.239,7.101)%
  --(7.243,7.101)--(7.246,7.101)--(7.250,7.101)--(7.253,7.101)--(7.256,7.101)--(7.260,7.101)%
  --(7.263,7.101)--(7.267,7.101)--(7.270,7.101)--(7.273,7.101)--(7.277,7.101)--(7.280,7.101)%
  --(7.284,7.101)--(7.287,7.101)--(7.291,7.101)--(7.294,7.101)--(7.297,7.101)--(7.301,7.101)%
  --(7.304,7.101)--(7.308,7.101)--(7.311,7.101)--(7.314,7.101)--(7.318,7.101)--(7.321,7.101)%
  --(7.325,7.101)--(7.328,7.101)--(7.332,7.101)--(7.335,7.101)--(7.338,7.101)--(7.342,7.101)%
  --(7.345,7.101)--(7.349,7.101)--(7.352,7.101)--(7.355,7.101)--(7.359,7.101)--(7.362,7.101)%
  --(7.366,7.101)--(7.369,7.101)--(7.373,7.101)--(7.376,7.101)--(7.379,7.101)--(7.383,7.101)%
  --(7.386,7.101)--(7.390,7.101)--(7.393,7.101)--(7.397,7.101)--(7.400,7.101)--(7.403,7.101)%
  --(7.407,7.101)--(7.410,7.101)--(7.414,7.101)--(7.417,7.101)--(7.420,7.101)--(7.424,7.101)%
  --(7.427,7.101)--(7.431,7.101)--(7.434,7.101)--(7.438,7.101)--(7.441,7.101)--(7.444,7.101)%
  --(7.448,7.101)--(7.451,7.101)--(7.455,7.101)--(7.458,7.101)--(7.461,7.101)--(7.465,7.101)%
  --(7.468,7.101)--(7.472,7.101)--(7.475,7.101)--(7.479,7.101)--(7.482,7.101)--(7.485,7.101)%
  --(7.489,7.101)--(7.492,7.101)--(7.496,7.101)--(7.499,7.101)--(7.502,7.101)--(7.506,7.101)%
  --(7.509,7.101)--(7.513,7.101)--(7.516,7.101)--(7.520,7.101)--(7.523,7.101)--(7.526,7.101)%
  --(7.530,7.101)--(7.533,7.101)--(7.537,7.101)--(7.540,7.101)--(7.543,7.101)--(7.547,7.101)%
  --(7.550,7.101)--(7.554,7.101)--(7.557,7.101)--(7.561,7.101)--(7.564,7.101)--(7.567,7.101)%
  --(7.571,7.101)--(7.574,7.101)--(7.578,7.101)--(7.581,7.101)--(7.585,7.101)--(7.588,7.101)%
  --(7.591,7.101)--(7.595,7.101)--(7.598,7.101)--(7.602,7.101)--(7.605,7.101)--(7.608,7.101)%
  --(7.612,7.101)--(7.615,7.101)--(7.619,7.101)--(7.622,7.101)--(7.626,7.101)--(7.629,7.101)%
  --(7.632,7.101)--(7.636,7.101)--(7.639,7.101)--(7.643,7.101)--(7.646,7.101)--(7.649,7.101)%
  --(7.653,7.101)--(7.656,7.101)--(7.660,7.101)--(7.663,7.101)--(7.667,7.101)--(7.670,7.101)%
  --(7.673,7.101)--(7.677,7.101)--(7.680,7.101)--(7.684,7.101)--(7.687,7.101)--(7.690,7.101)%
  --(7.694,7.101)--(7.697,7.101)--(7.701,7.101)--(7.704,7.101)--(7.708,7.101)--(7.711,7.101)%
  --(7.714,7.101)--(7.718,7.101)--(7.721,7.101)--(7.725,7.101)--(7.728,7.101)--(7.731,7.101)%
  --(7.735,7.101)--(7.738,7.101)--(7.742,7.101)--(7.745,7.101)--(7.749,7.101)--(7.752,7.101)%
  --(7.755,7.101)--(7.759,7.101)--(7.762,7.101)--(7.766,7.101)--(7.769,7.101)--(7.773,7.101)%
  --(7.776,7.101)--(7.779,7.101)--(7.783,7.101)--(7.786,7.101)--(7.790,7.101)--(7.793,7.101)%
  --(7.796,7.101)--(7.800,7.101)--(7.803,7.101)--(7.807,7.101)--(7.810,7.101)--(7.814,7.101)%
  --(7.817,7.101)--(7.820,7.101)--(7.824,7.101)--(7.827,7.101)--(7.831,7.101)--(7.834,7.101)%
  --(7.837,7.101)--(7.841,7.101)--(7.844,7.101)--(7.848,7.101)--(7.851,7.101)--(7.855,7.101)%
  --(7.858,7.101)--(7.861,7.101)--(7.865,7.101)--(7.868,7.101)--(7.872,7.101)--(7.875,7.101)%
  --(7.878,7.101)--(7.882,7.101)--(7.885,7.101)--(7.889,7.101)--(7.892,7.101)--(7.896,7.101)%
  --(7.899,7.101)--(7.902,7.101)--(7.906,7.101)--(7.909,7.101)--(7.913,7.101)--(7.916,7.101)%
  --(7.919,7.101)--(7.923,7.101)--(7.926,7.101)--(7.930,7.101)--(7.933,7.101)--(7.937,7.101)%
  --(7.940,7.101)--(7.943,7.101)--(7.947,7.101)--(7.950,7.101)--(7.954,7.101)--(7.957,7.101)%
  --(7.960,7.101)--(7.964,7.101)--(7.967,7.101)--(7.971,7.101)--(7.974,7.101)--(7.978,7.101)%
  --(7.981,7.101)--(7.984,7.101)--(7.988,7.101)--(7.991,7.101)--(7.995,7.101)--(7.998,7.101)%
  --(8.002,7.101)--(8.005,7.101)--(8.008,7.101)--(8.012,7.101)--(8.015,7.101)--(8.019,7.101)%
  --(8.022,7.101)--(8.025,7.101)--(8.029,7.101)--(8.032,7.101)--(8.036,7.101)--(8.039,7.101)%
  --(8.043,7.101)--(8.046,7.101)--(8.049,7.101)--(8.053,7.101)--(8.056,7.101)--(8.060,7.101)%
  --(8.063,7.101)--(8.066,7.101)--(8.070,7.101)--(8.073,7.101)--(8.077,7.101)--(8.080,7.101)%
  --(8.084,7.101)--(8.087,7.101)--(8.090,7.101)--(8.094,7.101)--(8.097,7.101)--(8.101,7.101)%
  --(8.104,7.101)--(8.107,7.101)--(8.111,7.101)--(8.114,7.101)--(8.118,7.101)--(8.121,7.101)%
  --(8.125,7.101)--(8.128,7.101)--(8.131,7.101)--(8.135,7.101)--(8.138,7.101)--(8.142,7.101)%
  --(8.145,7.101)--(8.148,7.101)--(8.152,7.101)--(8.155,7.101)--(8.159,7.101)--(8.162,7.101)%
  --(8.166,7.101)--(8.169,7.101)--(8.172,7.101)--(8.176,7.101)--(8.179,7.101)--(8.183,7.101)%
  --(8.186,7.101)--(8.190,7.101)--(8.193,7.101)--(8.196,7.101)--(8.200,7.101)--(8.203,7.101)%
  --(8.207,7.101)--(8.210,7.101)--(8.213,7.101)--(8.217,7.101)--(8.220,7.101)--(8.224,7.101)%
  --(8.227,7.101)--(8.231,7.101)--(8.234,7.101)--(8.237,7.101)--(8.241,7.101)--(8.244,7.101)%
  --(8.248,7.101)--(8.251,7.101)--(8.254,7.101)--(8.258,7.101)--(8.261,7.101)--(8.265,7.101)%
  --(8.268,7.101)--(8.272,7.101)--(8.275,7.101)--(8.278,7.101)--(8.282,7.101)--(8.285,7.101)%
  --(8.289,7.101)--(8.292,7.101)--(8.295,7.101)--(8.299,7.101)--(8.302,7.101)--(8.306,7.101)%
  --(8.309,7.101)--(8.313,7.101)--(8.316,7.101)--(8.319,7.101)--(8.323,7.101)--(8.326,7.101)%
  --(8.330,7.101)--(8.333,7.101)--(8.336,7.101)--(8.340,7.101)--(8.343,7.101)--(8.347,7.101)%
  --(8.350,7.101)--(8.354,7.101)--(8.357,7.101)--(8.360,7.101)--(8.364,7.101)--(8.367,7.101)%
  --(8.371,7.101)--(8.374,7.101)--(8.378,7.101)--(8.381,7.101)--(8.384,7.101)--(8.388,7.101)%
  --(8.391,7.101)--(8.395,7.101)--(8.398,7.101)--(8.401,7.101)--(8.405,7.101)--(8.408,7.101)%
  --(8.412,7.101)--(8.415,7.101)--(8.419,7.101)--(8.422,7.101)--(8.425,7.101)--(8.429,7.101)%
  --(8.432,7.101)--(8.436,7.101)--(8.439,7.101)--(8.442,7.101)--(8.446,7.101)--(8.449,7.101)%
  --(8.453,7.101)--(8.456,7.101)--(8.460,7.101)--(8.463,7.101)--(8.466,7.101)--(8.470,7.101)%
  --(8.473,7.101)--(8.477,7.101)--(8.480,7.101)--(8.483,7.101)--(8.487,7.101)--(8.490,7.101)%
  --(8.494,7.101)--(8.497,7.101)--(8.501,7.101)--(8.504,7.101)--(8.507,7.101)--(8.511,7.101)%
  --(8.514,7.101)--(8.518,7.101)--(8.521,7.101)--(8.524,7.101)--(8.528,7.101)--(8.531,7.101)%
  --(8.535,7.101)--(8.538,7.101)--(8.542,7.101)--(8.545,7.101)--(8.548,7.101)--(8.552,7.101)%
  --(8.555,7.101)--(8.559,7.101)--(8.562,7.101)--(8.566,7.101)--(8.569,7.101)--(8.572,7.101)%
  --(8.576,7.101)--(8.579,7.101)--(8.583,7.101)--(8.586,7.101)--(8.589,7.101)--(8.593,7.101)%
  --(8.596,7.101)--(8.600,7.101)--(8.603,7.101)--(8.607,7.101)--(8.610,7.101)--(8.613,7.101)%
  --(8.617,7.101)--(8.620,7.101)--(8.624,7.101)--(8.627,7.101)--(8.630,7.101)--(8.634,7.101)%
  --(8.637,7.101)--(8.641,7.101)--(8.644,7.101)--(8.648,7.101)--(8.651,7.101)--(8.654,7.101)%
  --(8.658,7.101)--(8.661,7.101)--(8.665,7.101)--(8.668,7.101)--(8.671,7.101)--(8.675,7.101)%
  --(8.678,7.101)--(8.682,7.101)--(8.685,7.101)--(8.689,7.101)--(8.692,7.101)--(8.695,7.101)%
  --(8.699,7.101)--(8.702,7.101)--(8.706,7.101)--(8.709,7.101)--(8.712,7.101)--(8.716,7.101)%
  --(8.719,7.101)--(8.723,7.101)--(8.726,7.101)--(8.730,7.101)--(8.733,7.101)--(8.736,7.101)%
  --(8.740,7.101)--(8.743,7.101)--(8.747,7.101)--(8.750,7.101)--(8.754,7.101)--(8.757,7.101)%
  --(8.760,7.101)--(8.764,7.101)--(8.767,7.101)--(8.771,7.101)--(8.774,7.101)--(8.777,7.101)%
  --(8.781,7.101)--(8.784,7.101)--(8.788,7.101)--(8.791,7.101)--(8.795,7.101)--(8.798,7.101)%
  --(8.801,7.101)--(8.805,7.101)--(8.808,7.101)--(8.812,7.101)--(8.815,7.101)--(8.818,7.101)%
  --(8.822,7.101)--(8.825,7.101)--(8.829,7.101)--(8.832,7.101)--(8.836,7.101)--(8.839,7.101)%
  --(8.842,7.101)--(8.846,7.101)--(8.849,7.101)--(8.853,7.101)--(8.856,7.101)--(8.859,7.101)%
  --(8.863,7.101)--(8.866,7.101)--(8.870,7.101)--(8.873,7.101)--(8.877,7.101)--(8.880,7.101)%
  --(8.883,7.101)--(8.887,7.101)--(8.890,7.101)--(8.894,7.101)--(8.897,7.101)--(8.900,7.101)%
  --(8.904,7.101)--(8.907,7.101)--(8.911,7.101)--(8.914,7.101)--(8.918,7.101)--(8.921,7.101)%
  --(8.924,7.101)--(8.928,7.101)--(8.931,7.101)--(8.935,7.101)--(8.938,7.101)--(8.942,7.101)%
  --(8.945,7.101)--(8.948,7.101)--(8.952,7.101)--(8.955,7.101)--(8.959,7.101)--(8.962,7.101)%
  --(8.965,7.101)--(8.969,7.101)--(8.972,7.101)--(8.976,7.101)--(8.979,7.101)--(8.983,7.101)%
  --(8.986,7.101)--(8.989,7.101)--(8.993,7.101)--(8.996,7.101)--(9.000,7.101)--(9.003,7.101)%
  --(9.006,7.101)--(9.010,7.101)--(9.013,7.101)--(9.017,7.101)--(9.020,7.101)--(9.024,7.101)%
  --(9.027,7.101)--(9.030,7.101)--(9.034,7.101)--(9.037,7.101)--(9.041,7.101)--(9.044,7.101)%
  --(9.047,7.101)--(9.051,7.101)--(9.054,7.101)--(9.058,7.101)--(9.061,7.101)--(9.065,7.101)%
  --(9.068,7.101)--(9.071,7.101)--(9.075,7.101)--(9.078,7.101)--(9.082,7.101)--(9.085,7.101)%
  --(9.088,7.101)--(9.092,7.101)--(9.095,7.101)--(9.099,7.101)--(9.102,7.101)--(9.106,7.101)%
  --(9.109,7.101)--(9.112,7.101)--(9.116,7.101)--(9.119,7.101)--(9.123,7.101)--(9.126,7.101)%
  --(9.130,7.101)--(9.133,7.101)--(9.136,7.101)--(9.140,7.101)--(9.143,7.101)--(9.147,7.101)%
  --(9.150,7.101)--(9.153,7.101)--(9.157,7.101)--(9.160,7.101)--(9.164,7.101)--(9.167,7.101)%
  --(9.171,7.101)--(9.174,7.101)--(9.177,7.101)--(9.181,7.101)--(9.184,7.101)--(9.188,7.101)%
  --(9.191,7.101)--(9.194,7.101)--(9.198,7.101)--(9.201,7.101)--(9.205,7.101)--(9.208,7.101)%
  --(9.212,7.101)--(9.215,7.101)--(9.218,7.101)--(9.222,7.101)--(9.225,7.101)--(9.229,7.101)%
  --(9.232,7.101)--(9.235,7.101)--(9.239,7.101)--(9.242,7.101)--(9.246,7.101)--(9.249,7.101)%
  --(9.253,7.101)--(9.256,7.101)--(9.259,7.101)--(9.263,7.101)--(9.266,7.101)--(9.270,7.101)%
  --(9.273,7.101)--(9.276,7.101)--(9.280,7.101)--(9.283,7.101)--(9.287,7.101)--(9.290,7.101)%
  --(9.294,7.101)--(9.297,7.101)--(9.300,7.101)--(9.304,7.101)--(9.307,7.101)--(9.311,7.101)%
  --(9.314,7.101)--(9.317,7.101)--(9.321,7.101)--(9.324,7.101)--(9.328,7.101)--(9.331,7.101)%
  --(9.335,7.101)--(9.338,7.101)--(9.341,7.101)--(9.345,7.101)--(9.348,7.101)--(9.352,7.101)%
  --(9.355,7.101)--(9.359,7.101)--(9.362,7.101)--(9.365,7.101)--(9.369,7.101)--(9.372,7.101)%
  --(9.376,7.101)--(9.379,7.101)--(9.382,7.101)--(9.386,7.101)--(9.389,7.101)--(9.393,7.101)%
  --(9.396,7.101)--(9.400,7.101)--(9.403,7.101)--(9.406,7.101)--(9.410,7.101)--(9.413,7.101)%
  --(9.417,7.101)--(9.420,7.101)--(9.423,7.101)--(9.427,7.101)--(9.430,7.101)--(9.434,7.101)%
  --(9.437,7.101)--(9.441,7.101)--(9.444,7.101)--(9.447,7.101)--(9.451,7.101)--(9.454,7.101)%
  --(9.458,7.101)--(9.461,7.101)--(9.464,7.101)--(9.468,7.101)--(9.471,7.101)--(9.475,7.101)%
  --(9.478,7.101)--(9.482,7.101)--(9.485,7.101)--(9.488,7.101)--(9.492,7.101)--(9.495,7.101)%
  --(9.499,7.101)--(9.502,7.101)--(9.505,7.101)--(9.509,7.101)--(9.512,7.101)--(9.516,7.101)%
  --(9.519,7.101)--(9.523,7.101)--(9.526,7.101)--(9.529,7.101)--(9.533,7.101)--(9.536,7.101)%
  --(9.540,7.101)--(9.543,7.101)--(9.547,7.101)--(9.550,7.101)--(9.553,7.101)--(9.557,7.101)%
  --(9.560,7.101)--(9.564,7.101)--(9.567,7.101)--(9.570,7.101)--(9.574,7.101)--(9.577,7.101)%
  --(9.581,7.101)--(9.584,7.101)--(9.588,7.101)--(9.591,7.101)--(9.594,7.101)--(9.598,7.101)%
  --(9.601,7.101)--(9.605,7.101)--(9.608,7.101)--(9.611,7.101)--(9.615,7.101)--(9.618,7.101)%
  --(9.622,7.101)--(9.625,7.101)--(9.629,7.101)--(9.632,7.101)--(9.635,7.101)--(9.639,7.101)%
  --(9.642,7.101)--(9.646,7.101)--(9.649,7.101)--(9.652,7.101)--(9.656,7.101)--(9.659,7.101)%
  --(9.663,7.101)--(9.666,7.101)--(9.670,7.101)--(9.673,7.101)--(9.676,7.101)--(9.680,7.101)%
  --(9.683,7.101)--(9.687,7.101)--(9.690,7.101)--(9.693,7.101)--(9.697,7.101)--(9.700,7.101)%
  --(9.704,7.101)--(9.707,7.101)--(9.711,7.101)--(9.714,7.101)--(9.717,7.101)--(9.721,7.101)%
  --(9.724,7.101)--(9.728,7.101)--(9.731,7.101)--(9.735,7.101)--(9.738,7.101)--(9.741,7.101)%
  --(9.745,7.101)--(9.748,7.101)--(9.752,7.101)--(9.755,7.101)--(9.758,7.101)--(9.762,7.101)%
  --(9.765,7.101)--(9.769,7.101)--(9.772,7.101)--(9.776,7.101)--(9.779,7.101)--(9.782,7.101)%
  --(9.786,7.101)--(9.789,7.101)--(9.793,7.101)--(9.796,7.101)--(9.799,7.101)--(9.803,7.101)%
  --(9.806,7.101)--(9.810,7.101)--(9.813,7.101)--(9.817,7.101)--(9.820,7.101)--(9.823,7.101)%
  --(9.827,7.101)--(9.830,7.101)--(9.834,7.101)--(9.837,7.101)--(9.840,7.101)--(9.844,7.101)%
  --(9.847,7.101)--(9.851,7.101)--(9.854,7.101)--(9.858,7.101)--(9.861,7.101)--(9.864,7.101)%
  --(9.868,7.101)--(9.871,7.101)--(9.875,7.101)--(9.878,7.101)--(9.881,7.101)--(9.885,7.101)%
  --(9.888,7.101)--(9.892,7.101)--(9.895,7.101)--(9.899,7.101)--(9.902,7.101)--(9.905,7.101)%
  --(9.909,7.101)--(9.912,7.101)--(9.916,7.101)--(9.919,7.101)--(9.923,7.101)--(9.926,7.101)%
  --(9.929,7.101)--(9.933,7.101)--(9.936,7.101)--(9.940,7.101)--(9.943,7.101)--(9.946,7.101)%
  --(9.950,7.101)--(9.953,7.101)--(9.957,7.101)--(9.960,7.101)--(9.964,7.101)--(9.967,7.101)%
  --(9.970,7.101)--(9.974,7.101)--(9.977,7.101)--(9.981,7.101)--(9.984,7.101)--(9.987,7.101)%
  --(9.991,7.101)--(9.994,7.101)--(9.998,7.101)--(10.001,7.101)--(10.005,7.101)--(10.008,7.101)%
  --(10.011,7.101)--(10.015,7.101)--(10.018,7.101)--(10.022,7.101)--(10.025,7.101)--(10.028,7.101)%
  --(10.032,7.101)--(10.035,7.101)--(10.039,7.101)--(10.042,7.101)--(10.046,7.101)--(10.049,7.101)%
  --(10.052,7.101)--(10.056,7.101)--(10.059,7.101)--(10.063,7.101)--(10.066,7.101)--(10.069,7.101)%
  --(10.073,7.101)--(10.076,7.101)--(10.080,7.101)--(10.083,7.101)--(10.087,7.101)--(10.090,7.101)%
  --(10.093,7.101)--(10.097,7.101)--(10.100,7.101)--(10.104,7.101)--(10.107,7.101)--(10.111,7.101)%
  --(10.114,7.101)--(10.117,7.101)--(10.121,7.101)--(10.124,7.101)--(10.128,7.101)--(10.131,7.101)%
  --(10.134,7.101)--(10.138,7.101)--(10.141,7.101)--(10.145,7.101)--(10.148,7.101)--(10.152,7.101)%
  --(10.155,7.101)--(10.158,7.101)--(10.162,7.101)--(10.165,7.101)--(10.169,7.101)--(10.172,7.101)%
  --(10.175,7.101)--(10.179,7.101)--(10.182,7.101)--(10.186,7.101)--(10.189,7.101)--(10.193,7.101)%
  --(10.196,7.101)--(10.199,7.101)--(10.203,7.101)--(10.206,7.101)--(10.210,7.101)--(10.213,7.101)%
  --(10.216,7.101)--(10.220,7.101)--(10.223,7.101)--(10.227,7.101)--(10.230,7.101)--(10.234,7.101)%
  --(10.237,7.101)--(10.240,7.101)--(10.244,7.101)--(10.247,7.101)--(10.251,7.101)--(10.254,7.101)%
  --(10.257,7.101)--(10.261,7.101)--(10.264,7.101)--(10.268,7.101)--(10.271,7.101)--(10.275,7.101)%
  --(10.278,7.101)--(10.281,7.101)--(10.285,7.101)--(10.288,7.101)--(10.292,7.101)--(10.295,7.101)%
  --(10.299,7.101)--(10.302,7.101)--(10.305,7.101)--(10.309,7.101)--(10.312,7.101)--(10.316,7.101)%
  --(10.319,7.101)--(10.322,7.101)--(10.326,7.101)--(10.329,7.101)--(10.333,7.101)--(10.336,7.101)%
  --(10.340,7.101)--(10.343,7.101)--(10.346,7.101)--(10.350,1.206)--(10.353,1.206)--(10.357,1.206)%
  --(10.360,1.206)--(10.363,1.206)--(10.367,1.206)--(10.370,1.206)--(10.374,1.206)--(10.377,1.206)%
  --(10.381,1.206)--(10.384,1.206)--(10.387,1.206)--(10.391,1.206)--(10.394,1.206)--(10.398,1.206)%
  --(10.401,1.206)--(10.404,1.206)--(10.408,1.206)--(10.411,1.206)--(10.415,1.206)--(10.418,1.206)%
  --(10.422,1.206)--(10.425,1.206)--(10.428,1.206)--(10.432,1.206)--(10.435,1.206)--(10.439,1.206)%
  --(10.442,1.206)--(10.445,1.206)--(10.449,1.206)--(10.452,1.206)--(10.456,1.206)--(10.459,1.206)%
  --(10.463,1.206)--(10.466,1.206)--(10.469,1.206)--(10.473,1.206)--(10.476,1.206)--(10.480,1.206)%
  --(10.483,1.206)--(10.487,1.206)--(10.490,1.206)--(10.493,1.206)--(10.497,1.206)--(10.500,1.206)%
  --(10.504,1.206)--(10.507,1.206)--(10.510,1.206)--(10.514,1.206)--(10.517,1.206)--(10.521,1.206)%
  --(10.524,1.206)--(10.528,1.206)--(10.531,1.206)--(10.534,1.206)--(10.538,1.206)--(10.541,1.206)%
  --(10.545,1.206)--(10.548,1.206)--(10.551,1.206)--(10.555,1.206)--(10.558,1.206)--(10.562,1.206)%
  --(10.565,1.206)--(10.569,1.206)--(10.572,1.206)--(10.575,1.206)--(10.579,1.206)--(10.582,1.206)%
  --(10.586,1.206)--(10.589,1.206)--(10.592,1.206)--(10.596,1.206)--(10.599,1.206)--(10.603,1.206)%
  --(10.606,1.206)--(10.610,1.206)--(10.613,1.206)--(10.616,1.206)--(10.620,1.206)--(10.623,1.206)%
  --(10.627,1.206)--(10.630,1.206)--(10.633,1.206)--(10.637,1.206)--(10.640,1.206)--(10.644,1.206)%
  --(10.647,1.206)--(10.651,1.206)--(10.654,1.206)--(10.657,1.206)--(10.661,1.206)--(10.664,1.206)%
  --(10.668,1.206)--(10.671,1.206)--(10.675,1.206)--(10.678,1.206)--(10.681,1.206)--(10.685,1.206)%
  --(10.688,1.206)--(10.692,1.206)--(10.695,1.206)--(10.698,1.206)--(10.702,1.206)--(10.705,1.206)%
  --(10.709,1.206)--(10.712,1.206)--(10.716,1.206)--(10.719,1.206)--(10.722,1.206)--(10.726,1.206)%
  --(10.729,1.206)--(10.733,1.206)--(10.736,1.206)--(10.739,1.206)--(10.743,1.206)--(10.746,1.206)%
  --(10.750,1.206)--(10.753,1.206)--(10.757,1.206)--(10.760,1.206)--(10.763,1.206)--(10.767,1.206)%
  --(10.770,1.206)--(10.774,1.206)--(10.777,1.206)--(10.780,1.206)--(10.784,1.206)--(10.787,1.206)%
  --(10.791,1.206)--(10.794,1.206)--(10.798,1.206)--(10.801,1.206)--(10.804,1.206)--(10.808,1.206)%
  --(10.811,1.206)--(10.815,1.206)--(10.818,1.206)--(10.821,1.206)--(10.825,1.206)--(10.828,1.206)%
  --(10.832,1.206)--(10.835,1.206)--(10.839,1.206)--(10.842,1.206)--(10.845,1.206)--(10.849,1.206)%
  --(10.852,1.206)--(10.856,1.206)--(10.859,1.206)--(10.862,1.206)--(10.866,1.206)--(10.869,1.206)%
  --(10.873,1.206)--(10.876,1.206)--(10.880,1.206)--(10.883,1.206)--(10.886,1.206)--(10.890,1.206)%
  --(10.893,1.206)--(10.897,1.206)--(10.900,1.206)--(10.904,1.206)--(10.907,1.206)--(10.910,1.206)%
  --(10.914,1.206)--(10.917,1.206)--(10.921,1.206)--(10.924,1.206)--(10.927,1.206)--(10.931,1.206)%
  --(10.934,1.206)--(10.938,1.206)--(10.941,1.206)--(10.945,1.206)--(10.948,1.206)--(10.951,1.206)%
  --(10.955,1.206)--(10.958,1.206)--(10.962,1.206)--(10.965,1.206)--(10.968,1.206)--(10.972,1.206)%
  --(10.975,1.206)--(10.979,1.206)--(10.982,1.206)--(10.986,1.206)--(10.989,1.206)--(10.992,1.206)%
  --(10.996,1.206)--(10.999,1.206)--(11.003,1.206)--(11.006,1.206)--(11.009,1.206)--(11.013,1.206)%
  --(11.016,1.206)--(11.020,1.206)--(11.023,1.206)--(11.027,1.206)--(11.030,1.206)--(11.033,1.206)%
  --(11.037,1.206)--(11.040,1.206)--(11.044,1.206)--(11.047,1.206)--(11.050,1.206)--(11.054,1.206)%
  --(11.057,1.206)--(11.061,1.206)--(11.064,1.206)--(11.068,1.206)--(11.071,1.206)--(11.074,1.206)%
  --(11.078,1.206)--(11.081,1.206)--(11.085,1.206)--(11.088,1.206)--(11.092,1.206)--(11.095,1.206)%
  --(11.098,1.206)--(11.102,1.206)--(11.105,1.206)--(11.109,1.206)--(11.112,1.206)--(11.115,1.206)%
  --(11.119,1.206)--(11.122,1.206)--(11.126,1.206)--(11.129,1.206)--(11.133,1.206)--(11.136,1.206)%
  --(11.139,1.206)--(11.143,1.206)--(11.146,1.206)--(11.150,1.206)--(11.153,1.206)--(11.156,1.206)%
  --(11.160,1.206)--(11.163,1.206)--(11.167,1.206)--(11.170,1.206)--(11.174,1.206)--(11.177,1.206)%
  --(11.180,1.206)--(11.184,1.206)--(11.187,1.206)--(11.191,1.206)--(11.194,1.206)--(11.197,1.206)%
  --(11.201,1.206)--(11.204,1.206)--(11.208,1.206)--(11.211,1.206)--(11.215,1.206)--(11.218,1.206)%
  --(11.221,1.206)--(11.225,1.206)--(11.228,1.206)--(11.232,1.206)--(11.235,1.206)--(11.238,1.206)%
  --(11.242,1.206)--(11.245,1.206)--(11.249,1.206)--(11.252,1.206)--(11.256,1.206)--(11.259,1.206)%
  --(11.262,1.206)--(11.266,1.206)--(11.269,1.206)--(11.273,1.206)--(11.276,1.206)--(11.280,1.206)%
  --(11.283,1.206)--(11.286,1.206)--(11.290,1.206)--(11.293,1.206)--(11.297,1.206)--(11.300,1.206)%
  --(11.303,1.206)--(11.307,1.206)--(11.310,1.206)--(11.314,1.206)--(11.317,1.206)--(11.321,1.206)%
  --(11.324,1.206)--(11.327,1.206)--(11.331,1.206)--(11.334,1.206)--(11.338,1.206)--(11.341,1.206)%
  --(11.344,1.206)--(11.348,1.206)--(11.351,1.206)--(11.355,1.206)--(11.358,1.206)--(11.362,1.206)%
  --(11.365,1.206)--(11.368,1.206)--(11.372,1.206)--(11.375,1.206)--(11.379,1.206)--(11.382,1.206)%
  --(11.385,1.206)--(11.389,1.206)--(11.392,1.206)--(11.396,1.206)--(11.399,1.206)--(11.403,1.206)%
  --(11.406,1.206)--(11.409,1.206)--(11.413,1.206)--(11.416,1.206)--(11.420,1.206)--(11.423,1.206)%
  --(11.426,1.206)--(11.430,1.206)--(11.433,1.206)--(11.437,1.206)--(11.440,1.206)--(11.444,1.206)%
  --(11.447,1.206);
  \gpcolor{rgb color={0.000,0.620,0.451}}
  \gpsetlinewidth{2.00}
  \draw[gp path] (1.196,7.116)--(1.199,7.110)--(1.203,7.103)--(1.206,7.097)--(1.210,7.090)%
  --(1.213,7.083)--(1.217,7.077)--(1.220,7.070)--(1.223,7.064)--(1.227,7.057)--(1.230,7.051)%
  --(1.234,7.045)--(1.237,7.039)--(1.240,7.034)--(1.244,7.029)--(1.247,7.024)--(1.251,7.019)%
  --(1.254,7.014)--(1.258,7.010)--(1.261,7.007)--(1.264,7.003)--(1.268,7.000)--(1.271,6.998)%
  --(1.275,6.996)--(1.278,6.994)--(1.281,6.993)--(1.285,6.992)--(1.288,6.991)--(1.292,6.991)%
  --(1.295,6.992)--(1.299,6.993)--(1.302,6.994)--(1.305,6.996)--(1.309,6.999)--(1.312,7.001)%
  --(1.316,7.005)--(1.319,7.008)--(1.322,7.012)--(1.326,7.017)--(1.329,7.021)--(1.333,7.026)%
  --(1.336,7.032)--(1.340,7.038)--(1.343,7.044)--(1.346,7.050)--(1.350,7.057)--(1.353,7.064)%
  --(1.357,7.071)--(1.360,7.078)--(1.363,7.085)--(1.367,7.093)--(1.370,7.100)--(1.374,7.108)%
  --(1.377,7.115)--(1.381,7.123)--(1.384,7.130)--(1.387,7.138)--(1.391,7.145)--(1.394,7.152)%
  --(1.398,7.160)--(1.401,7.166)--(1.405,7.173)--(1.408,7.179)--(1.411,7.186)--(1.415,7.191)%
  --(1.418,7.197)--(1.422,7.202)--(1.425,7.207)--(1.428,7.211)--(1.432,7.215)--(1.435,7.218)%
  --(1.439,7.221)--(1.442,7.224)--(1.446,7.226)--(1.449,7.227)--(1.452,7.228)--(1.456,7.229)%
  --(1.459,7.229)--(1.463,7.228)--(1.466,7.227)--(1.469,7.226)--(1.473,7.223)--(1.476,7.221)%
  --(1.480,7.218)--(1.483,7.214)--(1.487,7.210)--(1.490,7.205)--(1.493,7.200)--(1.497,7.194)%
  --(1.500,7.188)--(1.504,7.182)--(1.507,7.175)--(1.510,7.168)--(1.514,7.161)--(1.517,7.153)%
  --(1.521,7.145)--(1.524,7.137)--(1.528,7.128)--(1.531,7.119)--(1.534,7.111)--(1.538,7.102)%
  --(1.541,7.093)--(1.545,7.084)--(1.548,7.075)--(1.551,7.066)--(1.555,7.057)--(1.558,7.048)%
  --(1.562,7.039)--(1.565,7.031)--(1.569,7.023)--(1.572,7.015)--(1.575,7.007)--(1.579,7.000)%
  --(1.582,6.993)--(1.586,6.986)--(1.589,6.980)--(1.593,6.974)--(1.596,6.969)--(1.599,6.964)%
  --(1.603,6.960)--(1.606,6.956)--(1.610,6.953)--(1.613,6.951)--(1.616,6.949)--(1.620,6.948)%
  --(1.623,6.947)--(1.627,6.947)--(1.630,6.948)--(1.634,6.949)--(1.637,6.951)--(1.640,6.953)%
  --(1.644,6.957)--(1.647,6.960)--(1.651,6.965)--(1.654,6.970)--(1.657,6.976)--(1.661,6.982)%
  --(1.664,6.989)--(1.668,6.996)--(1.671,7.004)--(1.675,7.012)--(1.678,7.021)--(1.681,7.030)%
  --(1.685,7.040)--(1.688,7.050)--(1.692,7.060)--(1.695,7.071)--(1.698,7.082)--(1.702,7.093)%
  --(1.705,7.104)--(1.709,7.115)--(1.712,7.126)--(1.716,7.138)--(1.719,7.149)--(1.722,7.160)%
  --(1.726,7.172)--(1.729,7.183)--(1.733,7.193)--(1.736,7.204)--(1.739,7.214)--(1.743,7.224)%
  --(1.746,7.233)--(1.750,7.242)--(1.753,7.251)--(1.757,7.259)--(1.760,7.266)--(1.763,7.273)%
  --(1.767,7.279)--(1.770,7.284)--(1.774,7.289)--(1.777,7.293)--(1.781,7.296)--(1.784,7.299)%
  --(1.787,7.301)--(1.791,7.302)--(1.794,7.302)--(1.798,7.301)--(1.801,7.299)--(1.804,7.297)%
  --(1.808,7.293)--(1.811,7.289)--(1.815,7.284)--(1.818,7.278)--(1.822,7.272)--(1.825,7.264)%
  --(1.828,7.256)--(1.832,7.247)--(1.835,7.237)--(1.839,7.227)--(1.842,7.215)--(1.845,7.204)%
  --(1.849,7.191)--(1.852,7.178)--(1.856,7.165)--(1.859,7.151)--(1.863,7.136)--(1.866,7.122)%
  --(1.869,7.107)--(1.873,7.091)--(1.876,7.076)--(1.880,7.060)--(1.883,7.044)--(1.886,7.029)%
  --(1.890,7.013)--(1.893,6.998)--(1.897,6.982)--(1.900,6.967)--(1.904,6.953)--(1.907,6.938)%
  --(1.910,6.924)--(1.914,6.911)--(1.917,6.898)--(1.921,6.886)--(1.924,6.875)--(1.927,6.864)%
  --(1.931,6.854)--(1.934,6.845)--(1.938,6.837)--(1.941,6.830)--(1.945,6.824)--(1.948,6.820)%
  --(1.951,6.816)--(1.955,6.813)--(1.958,6.812)--(1.962,6.812)--(1.965,6.813)--(1.968,6.815)%
  --(1.972,6.819)--(1.975,6.823)--(1.979,6.830)--(1.982,6.837)--(1.986,6.846)--(1.989,6.856)%
  --(1.992,6.867)--(1.996,6.880)--(1.999,6.894)--(2.003,6.909)--(2.006,6.925)--(2.010,6.943)%
  --(2.013,6.961)--(2.016,6.981)--(2.020,7.001)--(2.023,7.023)--(2.027,7.045)--(2.030,7.068)%
  --(2.033,7.092)--(2.037,7.116)--(2.040,7.141)--(2.044,7.167)--(2.047,7.192)--(2.051,7.218)%
  --(2.054,7.245)--(2.057,7.271)--(2.061,7.297)--(2.064,7.323)--(2.068,7.349)--(2.071,7.374)%
  --(2.074,7.399)--(2.078,7.423)--(2.081,7.447)--(2.085,7.470)--(2.088,7.491)--(2.092,7.512)%
  --(2.095,7.531)--(2.098,7.549)--(2.102,7.566)--(2.105,7.581)--(2.109,7.594)--(2.112,7.605)%
  --(2.115,7.615)--(2.119,7.622)--(2.122,7.627)--(2.126,7.630)--(2.129,7.631)--(2.133,7.629)%
  --(2.136,7.624)--(2.139,7.617)--(2.143,7.607)--(2.146,7.594)--(2.150,7.579)--(2.153,7.560)%
  --(2.156,7.538)--(2.160,7.513)--(2.163,7.485)--(2.167,7.454)--(2.170,7.419)--(2.174,7.382)%
  --(2.177,7.341)--(2.180,7.296)--(2.184,7.248)--(2.187,7.197)--(2.191,7.143)--(2.194,7.085)%
  --(2.198,7.024)--(2.201,6.959)--(2.204,6.891)--(2.208,6.821)--(2.211,6.746)--(2.215,6.669)%
  --(2.218,6.589)--(2.221,6.505)--(2.225,6.419)--(2.228,6.330)--(2.232,6.238)--(2.235,6.144)%
  --(2.239,6.046)--(2.242,5.947)--(2.245,5.845)--(2.249,5.741)--(2.252,5.635)--(2.256,5.527)%
  --(2.259,5.417)--(2.262,5.305)--(2.266,5.192)--(2.269,5.077)--(2.273,4.961)--(2.276,4.845)%
  --(2.280,4.727)--(2.283,4.608)--(2.286,4.489)--(2.290,4.369)--(2.293,4.249)--(2.297,4.129)%
  --(2.300,4.009)--(2.303,3.889)--(2.307,3.769)--(2.310,3.650)--(2.314,3.532)--(2.317,3.415)%
  --(2.321,3.298)--(2.324,3.183)--(2.327,3.069)--(2.331,2.956)--(2.334,2.845)--(2.338,2.736)%
  --(2.341,2.629)--(2.344,2.523)--(2.348,2.420)--(2.351,2.319)--(2.355,2.221)--(2.358,2.124)%
  --(2.362,2.031)--(2.365,1.940)--(2.368,1.852)--(2.372,1.767)--(2.375,1.685)--(2.379,1.606)%
  --(2.382,1.530)--(2.386,1.457)--(2.389,1.387)--(2.392,1.321)--(2.396,1.258)--(2.399,1.198)%
  --(2.403,1.142)--(2.406,1.088)--(2.409,1.039)--(2.413,0.992)--(2.416,0.949)--(2.420,0.909)%
  --(2.423,0.873)--(2.427,0.840)--(2.430,0.810)--(2.433,0.783)--(2.437,0.760)--(2.440,0.739)%
  --(2.444,0.722)--(2.447,0.707)--(2.450,0.696)--(2.454,0.687)--(2.457,0.681)--(2.461,0.677)%
  --(2.464,0.676)--(2.468,0.678)--(2.471,0.682)--(2.474,0.688)--(2.478,0.696)--(2.481,0.706)%
  --(2.485,0.718)--(2.488,0.732)--(2.491,0.748)--(2.495,0.765)--(2.498,0.784)--(2.502,0.803)%
  --(2.505,0.825)--(2.509,0.847)--(2.512,0.870)--(2.515,0.894)--(2.519,0.918)--(2.522,0.943)%
  --(2.526,0.969)--(2.529,0.995)--(2.532,1.021)--(2.536,1.047)--(2.539,1.073)--(2.543,1.099)%
  --(2.546,1.125)--(2.550,1.151)--(2.553,1.176)--(2.556,1.201)--(2.560,1.225)--(2.563,1.248)%
  --(2.567,1.271)--(2.570,1.293)--(2.574,1.314)--(2.577,1.334)--(2.580,1.353)--(2.584,1.371)%
  --(2.587,1.388)--(2.591,1.404)--(2.594,1.419)--(2.597,1.432)--(2.601,1.444)--(2.604,1.455)%
  --(2.608,1.465)--(2.611,1.473)--(2.615,1.480)--(2.618,1.486)--(2.621,1.490)--(2.625,1.493)%
  --(2.628,1.495)--(2.632,1.495)--(2.635,1.495)--(2.638,1.493)--(2.642,1.490)--(2.645,1.486)%
  --(2.649,1.480)--(2.652,1.474)--(2.656,1.466)--(2.659,1.458)--(2.662,1.449)--(2.666,1.439)%
  --(2.669,1.428)--(2.673,1.416)--(2.676,1.404)--(2.679,1.391)--(2.683,1.377)--(2.686,1.363)%
  --(2.690,1.348)--(2.693,1.334)--(2.697,1.318)--(2.700,1.303)--(2.703,1.287)--(2.707,1.272)%
  --(2.710,1.256)--(2.714,1.240)--(2.717,1.225)--(2.720,1.210)--(2.724,1.194)--(2.727,1.179)%
  --(2.731,1.165)--(2.734,1.151)--(2.738,1.137)--(2.741,1.124)--(2.744,1.111)--(2.748,1.099)%
  --(2.751,1.087)--(2.755,1.076)--(2.758,1.066)--(2.762,1.056)--(2.765,1.048)--(2.768,1.040)%
  --(2.772,1.032)--(2.775,1.026)--(2.779,1.021)--(2.782,1.016)--(2.785,1.012)--(2.789,1.009)%
  --(2.792,1.007)--(2.796,1.006)--(2.799,1.005)--(2.803,1.006)--(2.806,1.007)--(2.809,1.009)%
  --(2.813,1.012)--(2.816,1.015)--(2.820,1.020)--(2.823,1.025)--(2.826,1.031)--(2.830,1.037)%
  --(2.833,1.044)--(2.837,1.052)--(2.840,1.060)--(2.844,1.069)--(2.847,1.078)--(2.850,1.087)%
  --(2.854,1.097)--(2.857,1.108)--(2.861,1.118)--(2.864,1.129)--(2.867,1.140)--(2.871,1.151)%
  --(2.874,1.162)--(2.878,1.174)--(2.881,1.185)--(2.885,1.196)--(2.888,1.208)--(2.891,1.219)%
  --(2.895,1.230)--(2.898,1.240)--(2.902,1.251)--(2.905,1.261)--(2.908,1.271)--(2.912,1.280)%
  --(2.915,1.290)--(2.919,1.298)--(2.922,1.306)--(2.926,1.314)--(2.929,1.321)--(2.932,1.328)%
  --(2.936,1.334)--(2.939,1.339)--(2.943,1.344)--(2.946,1.348)--(2.950,1.352)--(2.953,1.355)%
  --(2.956,1.357)--(2.960,1.359)--(2.963,1.360)--(2.967,1.360)--(2.970,1.360)--(2.973,1.359)%
  --(2.977,1.357)--(2.980,1.355)--(2.984,1.353)--(2.987,1.349)--(2.991,1.345)--(2.994,1.341)%
  --(2.997,1.336)--(3.001,1.331)--(3.004,1.325)--(3.008,1.318)--(3.011,1.312)--(3.014,1.304)%
  --(3.018,1.297)--(3.021,1.289)--(3.025,1.281)--(3.028,1.273)--(3.032,1.264)--(3.035,1.255)%
  --(3.038,1.247)--(3.042,1.238)--(3.045,1.229)--(3.049,1.220)--(3.052,1.211)--(3.055,1.202)%
  --(3.059,1.193)--(3.062,1.184)--(3.066,1.175)--(3.069,1.167)--(3.073,1.159)--(3.076,1.151)%
  --(3.079,1.143)--(3.083,1.136)--(3.086,1.129)--(3.090,1.122)--(3.093,1.116)--(3.096,1.110)%
  --(3.100,1.105)--(3.103,1.100)--(3.107,1.095)--(3.110,1.092)--(3.114,1.088)--(3.117,1.085)%
  --(3.120,1.083)--(3.124,1.081)--(3.127,1.079)--(3.131,1.078)--(3.134,1.078)--(3.138,1.078)%
  --(3.141,1.079)--(3.144,1.080)--(3.148,1.082)--(3.151,1.084)--(3.155,1.087)--(3.158,1.090)%
  --(3.161,1.094)--(3.165,1.098)--(3.168,1.102)--(3.172,1.107)--(3.175,1.112)--(3.179,1.118)%
  --(3.182,1.124)--(3.185,1.130)--(3.189,1.137)--(3.192,1.143)--(3.196,1.150)--(3.199,1.157)%
  --(3.202,1.165)--(3.206,1.172)--(3.209,1.180)--(3.213,1.187)--(3.216,1.195)--(3.220,1.202)%
  --(3.223,1.210)--(3.226,1.218)--(3.230,1.225)--(3.233,1.232)--(3.237,1.239)--(3.240,1.246)%
  --(3.243,1.253)--(3.247,1.260)--(3.250,1.266)--(3.254,1.272)--(3.257,1.277)--(3.261,1.283)%
  --(3.264,1.288)--(3.267,1.292)--(3.271,1.297)--(3.274,1.300)--(3.278,1.304)--(3.281,1.307)%
  --(3.284,1.309)--(3.288,1.312)--(3.291,1.313)--(3.295,1.314)--(3.298,1.315)--(3.302,1.316)%
  --(3.305,1.315)--(3.308,1.315)--(3.312,1.314)--(3.315,1.312)--(3.319,1.311)--(3.322,1.308)%
  --(3.326,1.305)--(3.329,1.302)--(3.332,1.299)--(3.336,1.295)--(3.339,1.291)--(3.343,1.286)%
  --(3.346,1.281)--(3.349,1.276)--(3.353,1.271)--(3.356,1.265)--(3.360,1.259)--(3.363,1.253)%
  --(3.367,1.247)--(3.370,1.241)--(3.373,1.234)--(3.377,1.228)--(3.380,1.221)--(3.384,1.214)%
  --(3.387,1.208)--(3.390,1.201)--(3.394,1.194)--(3.397,1.188)--(3.401,1.182)--(3.404,1.175)%
  --(3.408,1.169)--(3.411,1.163)--(3.414,1.157)--(3.418,1.152)--(3.421,1.147)--(3.425,1.142)%
  --(3.428,1.137)--(3.431,1.132)--(3.435,1.128)--(3.438,1.125)--(3.442,1.121)--(3.445,1.118)%
  --(3.449,1.115)--(3.452,1.113)--(3.455,1.111)--(3.459,1.110)--(3.462,1.108)--(3.466,1.108)%
  --(3.469,1.107)--(3.472,1.107)--(3.476,1.108)--(3.479,1.109)--(3.483,1.110)--(3.486,1.112)%
  --(3.490,1.114)--(3.493,1.116)--(3.496,1.119)--(3.500,1.122)--(3.503,1.126)--(3.507,1.129)%
  --(3.510,1.133)--(3.513,1.138)--(3.517,1.142)--(3.520,1.147)--(3.524,1.152)--(3.527,1.158)%
  --(3.531,1.163)--(3.534,1.169)--(3.537,1.174)--(3.541,1.180)--(3.544,1.186)--(3.548,1.192)%
  --(3.551,1.198)--(3.555,1.204)--(3.558,1.210)--(3.561,1.216)--(3.565,1.222)--(3.568,1.228)%
  --(3.572,1.233)--(3.575,1.239)--(3.578,1.244)--(3.582,1.250)--(3.585,1.255)--(3.589,1.260)%
  --(3.592,1.264)--(3.596,1.268)--(3.599,1.272)--(3.602,1.276)--(3.606,1.280)--(3.609,1.283)%
  --(3.613,1.286)--(3.616,1.288)--(3.619,1.290)--(3.623,1.292)--(3.626,1.294)--(3.630,1.295)%
  --(3.633,1.295)--(3.637,1.296)--(3.640,1.296)--(3.643,1.295)--(3.647,1.294)--(3.650,1.293)%
  --(3.654,1.292)--(3.657,1.290)--(3.660,1.288)--(3.664,1.285)--(3.667,1.282)--(3.671,1.279)%
  --(3.674,1.276)--(3.678,1.272)--(3.681,1.268)--(3.684,1.264)--(3.688,1.259)--(3.691,1.255)%
  --(3.695,1.250)--(3.698,1.245)--(3.701,1.240)--(3.705,1.234)--(3.708,1.229)--(3.712,1.223)%
  --(3.715,1.218)--(3.719,1.212)--(3.722,1.207)--(3.725,1.201)--(3.729,1.196)--(3.732,1.190)%
  --(3.736,1.185)--(3.739,1.179)--(3.743,1.174)--(3.746,1.169)--(3.749,1.164)--(3.753,1.160)%
  --(3.756,1.155)--(3.760,1.151)--(3.763,1.147)--(3.766,1.143)--(3.770,1.139)--(3.773,1.136)%
  --(3.777,1.133)--(3.780,1.131)--(3.784,1.128)--(3.787,1.126)--(3.790,1.125)--(3.794,1.123)%
  --(3.797,1.122)--(3.801,1.121)--(3.804,1.121)--(3.807,1.121)--(3.811,1.122)--(3.814,1.122)%
  --(3.818,1.123)--(3.821,1.125)--(3.825,1.126)--(3.828,1.128)--(3.831,1.131)--(3.835,1.133)%
  --(3.838,1.136)--(3.842,1.140)--(3.845,1.143)--(3.848,1.147)--(3.852,1.151)--(3.855,1.155)%
  --(3.859,1.159)--(3.862,1.164)--(3.866,1.169)--(3.869,1.174)--(3.872,1.179)--(3.876,1.184)%
  --(3.879,1.189)--(3.883,1.194)--(3.886,1.199)--(3.889,1.205)--(3.893,1.210)--(3.896,1.215)%
  --(3.900,1.220)--(3.903,1.225)--(3.907,1.231)--(3.910,1.235)--(3.913,1.240)--(3.917,1.245)%
  --(3.920,1.249)--(3.924,1.254)--(3.927,1.258)--(3.931,1.262)--(3.934,1.265)--(3.937,1.269)%
  --(3.941,1.272)--(3.944,1.275)--(3.948,1.277)--(3.951,1.279)--(3.954,1.281)--(3.958,1.283)%
  --(3.961,1.284)--(3.965,1.285)--(3.968,1.286)--(3.972,1.286)--(3.975,1.286)--(3.978,1.286)%
  --(3.982,1.285)--(3.985,1.284)--(3.989,1.283)--(3.992,1.281)--(3.995,1.280)--(3.999,1.277)%
  --(4.002,1.275)--(4.006,1.272)--(4.009,1.269)--(4.013,1.266)--(4.016,1.262)--(4.019,1.258)%
  --(4.023,1.254)--(4.026,1.250)--(4.030,1.246)--(4.033,1.241)--(4.036,1.236)--(4.040,1.232)%
  --(4.043,1.227)--(4.047,1.222)--(4.050,1.217)--(4.054,1.212)--(4.057,1.206)--(4.060,1.201)%
  --(4.064,1.196)--(4.067,1.191)--(4.071,1.186)--(4.074,1.181)--(4.077,1.177)--(4.081,1.172)%
  --(4.084,1.167)--(4.088,1.163)--(4.091,1.159)--(4.095,1.155)--(4.098,1.151)--(4.101,1.148)%
  --(4.105,1.144)--(4.108,1.141)--(4.112,1.138)--(4.115,1.136)--(4.119,1.134)--(4.122,1.132)%
  --(4.125,1.130)--(4.129,1.129)--(4.132,1.128)--(4.136,1.127)--(4.139,1.127)--(4.142,1.127)%
  --(4.146,1.127)--(4.149,1.128)--(4.153,1.129)--(4.156,1.130)--(4.160,1.132)--(4.163,1.133)%
  --(4.166,1.136)--(4.170,1.138)--(4.173,1.141)--(4.177,1.144)--(4.180,1.147)--(4.183,1.150)%
  --(4.187,1.154)--(4.190,1.158)--(4.194,1.162)--(4.197,1.166)--(4.201,1.171)--(4.204,1.175)%
  --(4.207,1.180)--(4.211,1.185)--(4.214,1.190)--(4.218,1.195)--(4.221,1.200)--(4.224,1.205)%
  --(4.228,1.210)--(4.231,1.215)--(4.235,1.220)--(4.238,1.225)--(4.242,1.229)--(4.245,1.234)%
  --(4.248,1.239)--(4.252,1.243)--(4.255,1.248)--(4.259,1.252)--(4.262,1.256)--(4.265,1.259)%
  --(4.269,1.263)--(4.272,1.266)--(4.276,1.269)--(4.279,1.272)--(4.283,1.274)--(4.286,1.277)%
  --(4.289,1.279)--(4.293,1.280)--(4.296,1.282)--(4.300,1.283)--(4.303,1.283)--(4.307,1.284)%
  --(4.310,1.284)--(4.313,1.283)--(4.317,1.283)--(4.320,1.282)--(4.324,1.281)--(4.327,1.279)%
  --(4.330,1.277)--(4.334,1.275)--(4.337,1.273)--(4.341,1.270)--(4.344,1.267)--(4.348,1.264)%
  --(4.351,1.260)--(4.354,1.257)--(4.358,1.253)--(4.361,1.249)--(4.365,1.245)--(4.368,1.240)%
  --(4.371,1.236)--(4.375,1.231)--(4.378,1.226)--(4.382,1.221)--(4.385,1.216)--(4.389,1.211)%
  --(4.392,1.206)--(4.395,1.201)--(4.399,1.196)--(4.402,1.191)--(4.406,1.186)--(4.409,1.182)%
  --(4.412,1.177)--(4.416,1.172)--(4.419,1.168)--(4.423,1.163)--(4.426,1.159)--(4.430,1.155)%
  --(4.433,1.152)--(4.436,1.148)--(4.440,1.145)--(4.443,1.142)--(4.447,1.139)--(4.450,1.136)%
  --(4.453,1.134)--(4.457,1.132)--(4.460,1.130)--(4.464,1.129)--(4.467,1.128)--(4.471,1.127)%
  --(4.474,1.127)--(4.477,1.127)--(4.481,1.127)--(4.484,1.128)--(4.488,1.129)--(4.491,1.130)%
  --(4.495,1.131)--(4.498,1.133)--(4.501,1.135)--(4.505,1.138)--(4.508,1.140)--(4.512,1.143)%
  --(4.515,1.147)--(4.518,1.150)--(4.522,1.154)--(4.525,1.158)--(4.529,1.162)--(4.532,1.166)%
  --(4.536,1.171)--(4.539,1.175)--(4.542,1.180)--(4.546,1.185)--(4.549,1.190)--(4.553,1.195)%
  --(4.556,1.200)--(4.559,1.205)--(4.563,1.210)--(4.566,1.215)--(4.570,1.220)--(4.573,1.225)%
  --(4.577,1.230)--(4.580,1.235)--(4.583,1.240)--(4.587,1.244)--(4.590,1.249)--(4.594,1.253)%
  --(4.597,1.257)--(4.600,1.261)--(4.604,1.265)--(4.607,1.268)--(4.611,1.271)--(4.614,1.274)%
  --(4.618,1.277)--(4.621,1.279)--(4.624,1.281)--(4.628,1.283)--(4.631,1.284)--(4.635,1.285)%
  --(4.638,1.286)--(4.641,1.286)--(4.645,1.286)--(4.648,1.286)--(4.652,1.286)--(4.655,1.285)%
  --(4.659,1.283)--(4.662,1.282)--(4.665,1.280)--(4.669,1.278)--(4.672,1.275)--(4.676,1.273)%
  --(4.679,1.270)--(4.683,1.266)--(4.686,1.263)--(4.689,1.259)--(4.693,1.255)--(4.696,1.251)%
  --(4.700,1.246)--(4.703,1.242)--(4.706,1.237)--(4.710,1.232)--(4.713,1.227)--(4.717,1.222)%
  --(4.720,1.217)--(4.724,1.211)--(4.727,1.206)--(4.730,1.201)--(4.734,1.196)--(4.737,1.190)%
  --(4.741,1.185)--(4.744,1.180)--(4.747,1.175)--(4.751,1.170)--(4.754,1.165)--(4.758,1.161)%
  --(4.761,1.156)--(4.765,1.152)--(4.768,1.148)--(4.771,1.144)--(4.775,1.141)--(4.778,1.137)%
  --(4.782,1.134)--(4.785,1.132)--(4.788,1.129)--(4.792,1.127)--(4.795,1.125)--(4.799,1.124)%
  --(4.802,1.123)--(4.806,1.122)--(4.809,1.121)--(4.812,1.121)--(4.816,1.121)--(4.819,1.122)%
  --(4.823,1.123)--(4.826,1.124)--(4.829,1.126)--(4.833,1.128)--(4.836,1.130)--(4.840,1.132)%
  --(4.843,1.135)--(4.847,1.138)--(4.850,1.142)--(4.853,1.146)--(4.857,1.150)--(4.860,1.154)%
  --(4.864,1.158)--(4.867,1.163)--(4.870,1.168)--(4.874,1.173)--(4.877,1.178)--(4.881,1.183)%
  --(4.884,1.189)--(4.888,1.194)--(4.891,1.200)--(4.894,1.205)--(4.898,1.211)--(4.901,1.216)%
  --(4.905,1.222)--(4.908,1.227)--(4.912,1.233)--(4.915,1.238)--(4.918,1.243)--(4.922,1.248)%
  --(4.925,1.253)--(4.929,1.258)--(4.932,1.262)--(4.935,1.267)--(4.939,1.271)--(4.942,1.275)%
  --(4.946,1.278)--(4.949,1.281)--(4.953,1.284)--(4.956,1.287)--(4.959,1.289)--(4.963,1.291)%
  --(4.966,1.293)--(4.970,1.294)--(4.973,1.295)--(4.976,1.295)--(4.980,1.296)--(4.983,1.295)%
  --(4.987,1.295)--(4.990,1.294)--(4.994,1.293)--(4.997,1.291)--(5.000,1.289)--(5.004,1.286)%
  --(5.007,1.284)--(5.011,1.281)--(5.014,1.277)--(5.017,1.274)--(5.021,1.270)--(5.024,1.265)%
  --(5.028,1.261)--(5.031,1.256)--(5.035,1.251)--(5.038,1.246)--(5.041,1.241)--(5.045,1.235)%
  --(5.048,1.229)--(5.052,1.224)--(5.055,1.218)--(5.058,1.212)--(5.062,1.206)--(5.065,1.200)%
  --(5.069,1.194)--(5.072,1.188)--(5.076,1.182)--(5.079,1.176)--(5.082,1.170)--(5.086,1.165)%
  --(5.089,1.159)--(5.093,1.154)--(5.096,1.149)--(5.100,1.144)--(5.103,1.139)--(5.106,1.135)%
  --(5.110,1.131)--(5.113,1.127)--(5.117,1.123)--(5.120,1.120)--(5.123,1.117)--(5.127,1.115)%
  --(5.130,1.112)--(5.134,1.111)--(5.137,1.109)--(5.141,1.108)--(5.144,1.108)--(5.147,1.107)%
  --(5.151,1.108)--(5.154,1.108)--(5.158,1.109)--(5.161,1.111)--(5.164,1.112)--(5.168,1.115)%
  --(5.171,1.117)--(5.175,1.120)--(5.178,1.123)--(5.182,1.127)--(5.185,1.131)--(5.188,1.136)%
  --(5.192,1.140)--(5.195,1.145)--(5.199,1.150)--(5.202,1.156)--(5.205,1.161)--(5.209,1.167)%
  --(5.212,1.173)--(5.216,1.180)--(5.219,1.186)--(5.223,1.193)--(5.226,1.199)--(5.229,1.206)%
  --(5.233,1.212)--(5.236,1.219)--(5.240,1.226)--(5.243,1.232)--(5.246,1.239)--(5.250,1.245)%
  --(5.253,1.251)--(5.257,1.258)--(5.260,1.263)--(5.264,1.269)--(5.267,1.275)--(5.270,1.280)%
  --(5.274,1.285)--(5.277,1.289)--(5.281,1.294)--(5.284,1.298)--(5.288,1.301)--(5.291,1.305)%
  --(5.294,1.307)--(5.298,1.310)--(5.301,1.312)--(5.305,1.313)--(5.308,1.315)--(5.311,1.315)%
  --(5.315,1.316)--(5.318,1.315)--(5.322,1.315)--(5.325,1.314)--(5.329,1.312)--(5.332,1.310)%
  --(5.335,1.308)--(5.339,1.305)--(5.342,1.301)--(5.346,1.298)--(5.349,1.294)--(5.352,1.289)%
  --(5.356,1.284)--(5.359,1.279)--(5.363,1.273)--(5.366,1.268)--(5.370,1.261)--(5.373,1.255)%
  --(5.376,1.248)--(5.380,1.241)--(5.383,1.234)--(5.387,1.227)--(5.390,1.220)--(5.393,1.212)%
  --(5.397,1.205)--(5.400,1.197)--(5.404,1.189)--(5.407,1.182)--(5.411,1.174)--(5.414,1.167)%
  --(5.417,1.160)--(5.421,1.152)--(5.424,1.145)--(5.428,1.139)--(5.431,1.132)--(5.434,1.126)%
  --(5.438,1.120)--(5.441,1.114)--(5.445,1.109)--(5.448,1.104)--(5.452,1.099)--(5.455,1.095)%
  --(5.458,1.091)--(5.462,1.088)--(5.465,1.085)--(5.469,1.083)--(5.472,1.081)--(5.476,1.079)%
  --(5.479,1.078)--(5.482,1.078)--(5.486,1.078)--(5.489,1.079)--(5.493,1.080)--(5.496,1.082)%
  --(5.499,1.084)--(5.503,1.087)--(5.506,1.090)--(5.510,1.094)--(5.513,1.099)--(5.517,1.103)%
  --(5.520,1.109)--(5.523,1.114)--(5.527,1.120)--(5.530,1.127)--(5.534,1.134)--(5.537,1.141)%
  --(5.540,1.149)--(5.544,1.156)--(5.547,1.165)--(5.551,1.173)--(5.554,1.181)--(5.558,1.190)%
  --(5.561,1.199)--(5.564,1.208)--(5.568,1.217)--(5.571,1.226)--(5.575,1.235)--(5.578,1.244)%
  --(5.581,1.253)--(5.585,1.262)--(5.588,1.270)--(5.592,1.279)--(5.595,1.287)--(5.599,1.295)%
  --(5.602,1.302)--(5.605,1.309)--(5.609,1.316)--(5.612,1.323)--(5.616,1.329)--(5.619,1.334)%
  --(5.622,1.340)--(5.626,1.344)--(5.629,1.348)--(5.633,1.352)--(5.636,1.355)--(5.640,1.357)%
  --(5.643,1.359)--(5.646,1.360)--(5.650,1.360)--(5.653,1.360)--(5.657,1.359)--(5.660,1.358)%
  --(5.664,1.356)--(5.667,1.353)--(5.670,1.349)--(5.674,1.345)--(5.677,1.341)--(5.681,1.335)%
  --(5.684,1.330)--(5.687,1.323)--(5.691,1.316)--(5.694,1.309)--(5.698,1.301)--(5.701,1.292)%
  --(5.705,1.283)--(5.708,1.274)--(5.711,1.264)--(5.715,1.254)--(5.718,1.244)--(5.722,1.233)%
  --(5.725,1.222)--(5.728,1.211)--(5.732,1.200)--(5.735,1.189)--(5.739,1.177)--(5.742,1.166)%
  --(5.746,1.154)--(5.749,1.143)--(5.752,1.132)--(5.756,1.121)--(5.759,1.111)--(5.763,1.100)%
  --(5.766,1.090)--(5.769,1.080)--(5.773,1.071)--(5.776,1.062)--(5.780,1.054)--(5.783,1.046)%
  --(5.787,1.039)--(5.790,1.032)--(5.793,1.026)--(5.797,1.021)--(5.800,1.017)--(5.804,1.013)%
  --(5.807,1.010)--(5.810,1.007)--(5.814,1.006)--(5.817,1.005)--(5.821,1.005)--(5.824,1.007)%
  --(5.828,1.008)--(5.831,1.011)--(5.834,1.015)--(5.838,1.019)--(5.841,1.024)--(5.845,1.031)%
  --(5.848,1.037)--(5.852,1.045)--(5.855,1.054)--(5.858,1.063)--(5.862,1.073)--(5.865,1.084)%
  --(5.869,1.095)--(5.872,1.107)--(5.875,1.120)--(5.879,1.133)--(5.882,1.146)--(5.886,1.161)%
  --(5.889,1.175)--(5.893,1.190)--(5.896,1.205)--(5.899,1.220)--(5.903,1.236)--(5.906,1.252)%
  --(5.910,1.267)--(5.913,1.283)--(5.916,1.298)--(5.920,1.314)--(5.923,1.329)--(5.927,1.344)%
  --(5.930,1.359)--(5.934,1.373)--(5.937,1.387)--(5.940,1.400)--(5.944,1.412)--(5.947,1.424)%
  --(5.951,1.436)--(5.954,1.446)--(5.957,1.455)--(5.961,1.464)--(5.964,1.472)--(5.968,1.478)%
  --(5.971,1.484)--(5.975,1.489)--(5.978,1.492)--(5.981,1.494)--(5.985,1.495)--(5.988,1.495)%
  --(5.992,1.494)--(5.995,1.491)--(5.998,1.487)--(6.002,1.482)--(6.005,1.475)--(6.009,1.467)%
  --(6.012,1.458)--(6.016,1.448)--(6.019,1.436)--(6.022,1.423)--(6.026,1.409)--(6.029,1.393)%
  --(6.033,1.377)--(6.036,1.359)--(6.040,1.340)--(6.043,1.320)--(6.046,1.299)--(6.050,1.278)%
  --(6.053,1.255)--(6.057,1.232)--(6.060,1.208)--(6.063,1.183)--(6.067,1.158)--(6.070,1.133)%
  --(6.074,1.107)--(6.077,1.081)--(6.081,1.055)--(6.084,1.028)--(6.087,1.002)--(6.091,0.976)%
  --(6.094,0.951)--(6.098,0.925)--(6.101,0.901)--(6.104,0.877)--(6.108,0.853)--(6.111,0.831)%
  --(6.115,0.810)--(6.118,0.789)--(6.122,0.770)--(6.125,0.753)--(6.128,0.737)--(6.132,0.722)%
  --(6.135,0.710)--(6.139,0.699)--(6.142,0.690)--(6.145,0.683)--(6.149,0.679)--(6.152,0.676)%
  --(6.156,0.677)--(6.159,0.679)--(6.163,0.685)--(6.166,0.693)--(6.169,0.703)--(6.173,0.717)%
  --(6.176,0.734)--(6.180,0.753)--(6.183,0.776)--(6.186,0.802)--(6.190,0.831)--(6.193,0.863)%
  --(6.197,0.898)--(6.200,0.937)--(6.204,0.979)--(6.207,1.025)--(6.210,1.073)--(6.214,1.126)%
  --(6.217,1.181)--(6.221,1.240)--(6.224,1.302)--(6.228,1.367)--(6.231,1.436)--(6.234,1.508)%
  --(6.238,1.583)--(6.241,1.661)--(6.245,1.743)--(6.248,1.827)--(6.251,1.914)--(6.255,2.004)%
  --(6.258,2.097)--(6.262,2.192)--(6.265,2.290)--(6.269,2.390)--(6.272,2.492)--(6.275,2.597)%
  --(6.279,2.704)--(6.282,2.813)--(6.286,2.923)--(6.289,3.035)--(6.292,3.149)--(6.296,3.264)%
  --(6.299,3.380)--(6.303,3.497)--(6.306,3.615)--(6.310,3.734)--(6.313,3.854)--(6.316,3.973)%
  --(6.320,4.093)--(6.323,4.214)--(6.327,4.334)--(6.330,4.453)--(6.333,4.573)--(6.337,4.692)%
  --(6.340,4.810)--(6.344,4.927)--(6.347,5.043)--(6.351,5.158)--(6.354,5.272)--(6.357,5.384)%
  --(6.361,5.494)--(6.364,5.603)--(6.368,5.710)--(6.371,5.815)--(6.374,5.917)--(6.378,6.017)%
  --(6.381,6.115)--(6.385,6.210)--(6.388,6.303)--(6.392,6.393)--(6.395,6.480)--(6.398,6.564)%
  --(6.402,6.646)--(6.405,6.724)--(6.409,6.799)--(6.412,6.871)--(6.415,6.940)--(6.419,7.005)%
  --(6.422,7.067)--(6.426,7.126)--(6.429,7.181)--(6.433,7.234)--(6.436,7.282)--(6.439,7.328)%
  --(6.443,7.370)--(6.446,7.409)--(6.450,7.444)--(6.453,7.476)--(6.457,7.505)--(6.460,7.531)%
  --(6.463,7.554)--(6.467,7.573)--(6.470,7.590)--(6.474,7.604)--(6.477,7.614)--(6.480,7.622)%
  --(6.484,7.628)--(6.487,7.630)--(6.491,7.631)--(6.494,7.628)--(6.498,7.624)--(6.501,7.617)%
  --(6.504,7.608)--(6.508,7.597)--(6.511,7.585)--(6.515,7.570)--(6.518,7.554)--(6.521,7.537)%
  --(6.525,7.518)--(6.528,7.497)--(6.532,7.476)--(6.535,7.454)--(6.539,7.430)--(6.542,7.406)%
  --(6.545,7.382)--(6.549,7.356)--(6.552,7.331)--(6.556,7.305)--(6.559,7.279)--(6.562,7.252)%
  --(6.566,7.226)--(6.569,7.200)--(6.573,7.174)--(6.576,7.149)--(6.580,7.124)--(6.583,7.099)%
  --(6.586,7.075)--(6.590,7.052)--(6.593,7.029)--(6.597,7.008)--(6.600,6.987)--(6.603,6.967)%
  --(6.607,6.948)--(6.610,6.930)--(6.614,6.914)--(6.617,6.898)--(6.621,6.884)--(6.624,6.871)%
  --(6.627,6.859)--(6.631,6.849)--(6.634,6.840)--(6.638,6.832)--(6.641,6.825)--(6.645,6.820)%
  --(6.648,6.816)--(6.651,6.813)--(6.655,6.812)--(6.658,6.812)--(6.662,6.813)--(6.665,6.815)%
  --(6.668,6.818)--(6.672,6.823)--(6.675,6.829)--(6.679,6.835)--(6.682,6.843)--(6.686,6.852)%
  --(6.689,6.861)--(6.692,6.871)--(6.696,6.883)--(6.699,6.895)--(6.703,6.907)--(6.706,6.920)%
  --(6.709,6.934)--(6.713,6.948)--(6.716,6.963)--(6.720,6.978)--(6.723,6.993)--(6.727,7.009)%
  --(6.730,7.024)--(6.733,7.040)--(6.737,7.055)--(6.740,7.071)--(6.744,7.087)--(6.747,7.102)%
  --(6.750,7.117)--(6.754,7.132)--(6.757,7.146)--(6.761,7.161)--(6.764,7.174)--(6.768,7.187)%
  --(6.771,7.200)--(6.774,7.212)--(6.778,7.223)--(6.781,7.234)--(6.785,7.244)--(6.788,7.253)%
  --(6.791,7.262)--(6.795,7.270)--(6.798,7.276)--(6.802,7.283)--(6.805,7.288)--(6.809,7.292)%
  --(6.812,7.296)--(6.815,7.299)--(6.819,7.300)--(6.822,7.302)--(6.826,7.302)--(6.829,7.301)%
  --(6.833,7.300)--(6.836,7.297)--(6.839,7.294)--(6.843,7.290)--(6.846,7.286)--(6.850,7.281)%
  --(6.853,7.275)--(6.856,7.268)--(6.860,7.261)--(6.863,7.253)--(6.867,7.245)--(6.870,7.236)%
  --(6.874,7.227)--(6.877,7.217)--(6.880,7.207)--(6.884,7.196)--(6.887,7.186)--(6.891,7.175)%
  --(6.894,7.164)--(6.897,7.153)--(6.901,7.141)--(6.904,7.130)--(6.908,7.118)--(6.911,7.107)%
  --(6.915,7.096)--(6.918,7.085)--(6.921,7.074)--(6.925,7.063)--(6.928,7.053)--(6.932,7.043)%
  --(6.935,7.033)--(6.938,7.024)--(6.942,7.015)--(6.945,7.006)--(6.949,6.998)--(6.952,6.991)%
  --(6.956,6.984)--(6.959,6.977)--(6.962,6.972)--(6.966,6.966)--(6.969,6.962)--(6.973,6.958)%
  --(6.976,6.954)--(6.979,6.951)--(6.983,6.949)--(6.986,6.948)--(6.990,6.947)--(6.993,6.947)%
  --(6.997,6.947)--(7.000,6.948)--(7.003,6.950)--(7.007,6.952)--(7.010,6.955)--(7.014,6.959)%
  --(7.017,6.963)--(7.021,6.967)--(7.024,6.973)--(7.027,6.978)--(7.031,6.984)--(7.034,6.991)%
  --(7.038,6.998)--(7.041,7.005)--(7.044,7.012)--(7.048,7.020)--(7.051,7.028)--(7.055,7.037)%
  --(7.058,7.045)--(7.062,7.054)--(7.065,7.063)--(7.068,7.072)--(7.072,7.081)--(7.075,7.090)%
  --(7.079,7.099)--(7.082,7.108)--(7.085,7.117)--(7.089,7.126)--(7.092,7.134)--(7.096,7.142)%
  --(7.099,7.151)--(7.103,7.158)--(7.106,7.166)--(7.109,7.173)--(7.113,7.180)--(7.116,7.187)%
  --(7.120,7.193)--(7.123,7.198)--(7.126,7.204)--(7.130,7.208)--(7.133,7.213)--(7.137,7.217)%
  --(7.140,7.220)--(7.144,7.223)--(7.147,7.225)--(7.150,7.227)--(7.154,7.228)--(7.157,7.229)%
  --(7.161,7.229)--(7.164,7.229)--(7.167,7.228)--(7.171,7.226)--(7.174,7.224)--(7.178,7.222)%
  --(7.181,7.219)--(7.185,7.216)--(7.188,7.212)--(7.191,7.208)--(7.195,7.203)--(7.198,7.198)%
  --(7.202,7.193)--(7.205,7.187)--(7.209,7.181)--(7.212,7.175)--(7.215,7.168)--(7.219,7.162)%
  --(7.222,7.155)--(7.226,7.147)--(7.229,7.140)--(7.232,7.133)--(7.236,7.125)--(7.239,7.118)%
  --(7.243,7.110)--(7.246,7.102)--(7.250,7.095)--(7.253,7.087)--(7.256,7.080)--(7.260,7.073)%
  --(7.263,7.066)--(7.267,7.059)--(7.270,7.052)--(7.273,7.046)--(7.277,7.039)--(7.280,7.034)%
  --(7.284,7.028)--(7.287,7.023)--(7.291,7.018)--(7.294,7.013)--(7.297,7.009)--(7.301,7.006)%
  --(7.304,7.002)--(7.308,6.999)--(7.311,6.997)--(7.314,6.995)--(7.318,6.993)--(7.321,6.992)%
  --(7.325,6.992)--(7.328,6.991)--(7.332,6.992)--(7.335,6.992)--(7.338,6.994)--(7.342,6.995)%
  --(7.345,6.997)--(7.349,7.000)--(7.352,7.002)--(7.355,7.006)--(7.359,7.009)--(7.362,7.013)%
  --(7.366,7.018)--(7.369,7.022)--(7.373,7.027)--(7.376,7.032)--(7.379,7.038)--(7.383,7.044)%
  --(7.386,7.049)--(7.390,7.056)--(7.393,7.062)--(7.397,7.068)--(7.400,7.075)--(7.403,7.081)%
  --(7.407,7.088)--(7.410,7.095)--(7.414,7.101)--(7.417,7.108)--(7.420,7.114)--(7.424,7.121)%
  --(7.427,7.127)--(7.431,7.134)--(7.434,7.140)--(7.438,7.146)--(7.441,7.151)--(7.444,7.157)%
  --(7.448,7.162)--(7.451,7.167)--(7.455,7.171)--(7.458,7.176)--(7.461,7.180)--(7.465,7.184)%
  --(7.468,7.187)--(7.472,7.190)--(7.475,7.192)--(7.479,7.195)--(7.482,7.196)--(7.485,7.198)%
  --(7.489,7.199)--(7.492,7.199)--(7.496,7.200)--(7.499,7.199)--(7.502,7.199)--(7.506,7.198)%
  --(7.509,7.196)--(7.513,7.195)--(7.516,7.192)--(7.520,7.190)--(7.523,7.187)--(7.526,7.184)%
  --(7.530,7.180)--(7.533,7.176)--(7.537,7.172)--(7.540,7.168)--(7.543,7.163)--(7.547,7.158)%
  --(7.550,7.153)--(7.554,7.148)--(7.557,7.142)--(7.561,7.137)--(7.564,7.131)--(7.567,7.125)%
  --(7.571,7.119)--(7.574,7.113)--(7.578,7.107)--(7.581,7.101)--(7.585,7.095)--(7.588,7.089)%
  --(7.591,7.083)--(7.595,7.078)--(7.598,7.072)--(7.602,7.066)--(7.605,7.061)--(7.608,7.056)%
  --(7.612,7.051)--(7.615,7.046)--(7.619,7.042)--(7.622,7.037)--(7.626,7.033)--(7.629,7.030)%
  --(7.632,7.026)--(7.636,7.023)--(7.639,7.021)--(7.643,7.018)--(7.646,7.016)--(7.649,7.014)%
  --(7.653,7.013)--(7.656,7.012)--(7.660,7.012)--(7.663,7.011)--(7.667,7.012)--(7.670,7.012)%
  --(7.673,7.013)--(7.677,7.014)--(7.680,7.016)--(7.684,7.018)--(7.687,7.020)--(7.690,7.023)%
  --(7.694,7.026)--(7.697,7.029)--(7.701,7.032)--(7.704,7.036)--(7.708,7.040)--(7.711,7.045)%
  --(7.714,7.049)--(7.718,7.054)--(7.721,7.059)--(7.725,7.064)--(7.728,7.069)--(7.731,7.074)%
  --(7.735,7.080)--(7.738,7.085)--(7.742,7.091)--(7.745,7.096)--(7.749,7.102)--(7.752,7.107)%
  --(7.755,7.113)--(7.759,7.118)--(7.762,7.124)--(7.766,7.129)--(7.769,7.134)--(7.773,7.139)%
  --(7.776,7.144)--(7.779,7.149)--(7.783,7.153)--(7.786,7.157)--(7.790,7.161)--(7.793,7.165)%
  --(7.796,7.169)--(7.800,7.172)--(7.803,7.175)--(7.807,7.177)--(7.810,7.179)--(7.814,7.181)%
  --(7.817,7.183)--(7.820,7.184)--(7.824,7.185)--(7.827,7.186)--(7.831,7.186)--(7.834,7.186)%
  --(7.837,7.185)--(7.841,7.184)--(7.844,7.183)--(7.848,7.182)--(7.851,7.180)--(7.855,7.178)%
  --(7.858,7.175)--(7.861,7.173)--(7.865,7.170)--(7.868,7.166)--(7.872,7.163)--(7.875,7.159)%
  --(7.878,7.155)--(7.882,7.151)--(7.885,7.146)--(7.889,7.142)--(7.892,7.137)--(7.896,7.132)%
  --(7.899,7.127)--(7.902,7.122)--(7.906,7.117)--(7.909,7.111)--(7.913,7.106)--(7.916,7.101)%
  --(7.919,7.096)--(7.923,7.090)--(7.926,7.085)--(7.930,7.080)--(7.933,7.075)--(7.937,7.070)%
  --(7.940,7.065)--(7.943,7.061)--(7.947,7.056)--(7.950,7.052)--(7.954,7.048)--(7.957,7.044)%
  --(7.960,7.041)--(7.964,7.037)--(7.967,7.034)--(7.971,7.032)--(7.974,7.029)--(7.978,7.027)%
  --(7.981,7.025)--(7.984,7.024)--(7.988,7.022)--(7.991,7.021)--(7.995,7.021)--(7.998,7.021)%
  --(8.002,7.021)--(8.005,7.021)--(8.008,7.022)--(8.012,7.023)--(8.015,7.024)--(8.019,7.026)%
  --(8.022,7.028)--(8.025,7.030)--(8.029,7.033)--(8.032,7.036)--(8.036,7.039)--(8.039,7.042)%
  --(8.043,7.046)--(8.046,7.050)--(8.049,7.054)--(8.053,7.058)--(8.056,7.063)--(8.060,7.067)%
  --(8.063,7.072)--(8.066,7.077)--(8.070,7.082)--(8.073,7.087)--(8.077,7.092)--(8.080,7.097)%
  --(8.084,7.102)--(8.087,7.107)--(8.090,7.112)--(8.094,7.117)--(8.097,7.122)--(8.101,7.127)%
  --(8.104,7.132)--(8.107,7.136)--(8.111,7.141)--(8.114,7.145)--(8.118,7.149)--(8.121,7.153)%
  --(8.125,7.157)--(8.128,7.160)--(8.131,7.164)--(8.135,7.167)--(8.138,7.169)--(8.142,7.172)%
  --(8.145,7.174)--(8.148,7.176)--(8.152,7.177)--(8.155,7.178)--(8.159,7.179)--(8.162,7.180)%
  --(8.166,7.180)--(8.169,7.180)--(8.172,7.180)--(8.176,7.179)--(8.179,7.178)--(8.183,7.177)%
  --(8.186,7.175)--(8.190,7.173)--(8.193,7.171)--(8.196,7.168)--(8.200,7.165)--(8.203,7.162)%
  --(8.207,7.159)--(8.210,7.155)--(8.213,7.152)--(8.217,7.148)--(8.220,7.144)--(8.224,7.139)%
  --(8.227,7.135)--(8.231,7.130)--(8.234,7.125)--(8.237,7.121)--(8.241,7.116)--(8.244,7.111)%
  --(8.248,7.106)--(8.251,7.101)--(8.254,7.096)--(8.258,7.091)--(8.261,7.086)--(8.265,7.081)%
  --(8.268,7.076)--(8.272,7.071)--(8.275,7.067)--(8.278,7.062)--(8.282,7.058)--(8.285,7.054)%
  --(8.289,7.050)--(8.292,7.047)--(8.295,7.043)--(8.299,7.040)--(8.302,7.037)--(8.306,7.034)%
  --(8.309,7.032)--(8.313,7.030)--(8.316,7.028)--(8.319,7.026)--(8.323,7.025)--(8.326,7.024)%
  --(8.330,7.024)--(8.333,7.023)--(8.336,7.023)--(8.340,7.024)--(8.343,7.024)--(8.347,7.025)%
  --(8.350,7.027)--(8.354,7.028)--(8.357,7.030)--(8.360,7.033)--(8.364,7.035)--(8.367,7.038)%
  --(8.371,7.041)--(8.374,7.044)--(8.378,7.048)--(8.381,7.051)--(8.384,7.055)--(8.388,7.059)%
  --(8.391,7.064)--(8.395,7.068)--(8.398,7.073)--(8.401,7.078)--(8.405,7.082)--(8.408,7.087)%
  --(8.412,7.092)--(8.415,7.097)--(8.419,7.102)--(8.422,7.107)--(8.425,7.112)--(8.429,7.117)%
  --(8.432,7.122)--(8.436,7.127)--(8.439,7.132)--(8.442,7.136)--(8.446,7.141)--(8.449,7.145)%
  --(8.453,7.149)--(8.456,7.153)--(8.460,7.157)--(8.463,7.160)--(8.466,7.163)--(8.470,7.166)%
  --(8.473,7.169)--(8.477,7.171)--(8.480,7.174)--(8.483,7.175)--(8.487,7.177)--(8.490,7.178)%
  --(8.494,7.179)--(8.497,7.180)--(8.501,7.180)--(8.504,7.180)--(8.507,7.180)--(8.511,7.179)%
  --(8.514,7.178)--(8.518,7.177)--(8.521,7.175)--(8.524,7.173)--(8.528,7.171)--(8.531,7.169)%
  --(8.535,7.166)--(8.538,7.163)--(8.542,7.159)--(8.545,7.156)--(8.548,7.152)--(8.552,7.148)%
  --(8.555,7.144)--(8.559,7.140)--(8.562,7.135)--(8.566,7.130)--(8.569,7.126)--(8.572,7.121)%
  --(8.576,7.116)--(8.579,7.111)--(8.583,7.106)--(8.586,7.101)--(8.589,7.095)--(8.593,7.090)%
  --(8.596,7.085)--(8.600,7.080)--(8.603,7.075)--(8.607,7.071)--(8.610,7.066)--(8.613,7.061)%
  --(8.617,7.057)--(8.620,7.053)--(8.624,7.049)--(8.627,7.045)--(8.630,7.041)--(8.634,7.038)%
  --(8.637,7.035)--(8.641,7.032)--(8.644,7.030)--(8.648,7.027)--(8.651,7.026)--(8.654,7.024)%
  --(8.658,7.023)--(8.661,7.022)--(8.665,7.021)--(8.668,7.021)--(8.671,7.021)--(8.675,7.021)%
  --(8.678,7.022)--(8.682,7.023)--(8.685,7.024)--(8.689,7.026)--(8.692,7.028)--(8.695,7.030)%
  --(8.699,7.032)--(8.702,7.035)--(8.706,7.038)--(8.709,7.042)--(8.712,7.045)--(8.716,7.049)%
  --(8.719,7.053)--(8.723,7.058)--(8.726,7.062)--(8.730,7.067)--(8.733,7.072)--(8.736,7.076)%
  --(8.740,7.082)--(8.743,7.087)--(8.747,7.092)--(8.750,7.097)--(8.754,7.102)--(8.757,7.108)%
  --(8.760,7.113)--(8.764,7.118)--(8.767,7.123)--(8.771,7.128)--(8.774,7.133)--(8.777,7.138)%
  --(8.781,7.143)--(8.784,7.148)--(8.788,7.152)--(8.791,7.156)--(8.795,7.160)--(8.798,7.164)%
  --(8.801,7.167)--(8.805,7.171)--(8.808,7.174)--(8.812,7.176)--(8.815,7.179)--(8.818,7.181)%
  --(8.822,7.182)--(8.825,7.184)--(8.829,7.185)--(8.832,7.185)--(8.836,7.186)--(8.839,7.186)%
  --(8.842,7.186)--(8.846,7.185)--(8.849,7.184)--(8.853,7.182)--(8.856,7.181)--(8.859,7.179)%
  --(8.863,7.176)--(8.866,7.174)--(8.870,7.171)--(8.873,7.168)--(8.877,7.164)--(8.880,7.160)%
  --(8.883,7.156)--(8.887,7.152)--(8.890,7.147)--(8.894,7.143)--(8.897,7.138)--(8.900,7.133)%
  --(8.904,7.128)--(8.907,7.122)--(8.911,7.117)--(8.914,7.111)--(8.918,7.106)--(8.921,7.100)%
  --(8.924,7.095)--(8.928,7.089)--(8.931,7.084)--(8.935,7.078)--(8.938,7.073)--(8.942,7.067)%
  --(8.945,7.062)--(8.948,7.057)--(8.952,7.052)--(8.955,7.048)--(8.959,7.043)--(8.962,7.039)%
  --(8.965,7.035)--(8.969,7.031)--(8.972,7.028)--(8.976,7.025)--(8.979,7.022)--(8.983,7.019)%
  --(8.986,7.017)--(8.989,7.015)--(8.993,7.014)--(8.996,7.013)--(9.000,7.012)--(9.003,7.011)%
  --(9.006,7.011)--(9.010,7.012)--(9.013,7.012)--(9.017,7.013)--(9.020,7.015)--(9.024,7.017)%
  --(9.027,7.019)--(9.030,7.021)--(9.034,7.024)--(9.037,7.027)--(9.041,7.031)--(9.044,7.035)%
  --(9.047,7.039)--(9.051,7.043)--(9.054,7.047)--(9.058,7.052)--(9.061,7.057)--(9.065,7.063)%
  --(9.068,7.068)--(9.071,7.074)--(9.075,7.079)--(9.078,7.085)--(9.082,7.091)--(9.085,7.097)%
  --(9.088,7.103)--(9.092,7.109)--(9.095,7.115)--(9.099,7.121)--(9.102,7.127)--(9.106,7.133)%
  --(9.109,7.138)--(9.112,7.144)--(9.116,7.149)--(9.119,7.155)--(9.123,7.160)--(9.126,7.165)%
  --(9.130,7.169)--(9.133,7.174)--(9.136,7.178)--(9.140,7.181)--(9.143,7.185)--(9.147,7.188)%
  --(9.150,7.191)--(9.153,7.193)--(9.157,7.195)--(9.160,7.197)--(9.164,7.198)--(9.167,7.199)%
  --(9.171,7.200)--(9.174,7.200)--(9.177,7.199)--(9.181,7.199)--(9.184,7.197)--(9.188,7.196)%
  --(9.191,7.194)--(9.194,7.192)--(9.198,7.189)--(9.201,7.186)--(9.205,7.182)--(9.208,7.179)%
  --(9.212,7.175)--(9.215,7.170)--(9.218,7.165)--(9.222,7.160)--(9.225,7.155)--(9.229,7.150)%
  --(9.232,7.144)--(9.235,7.138)--(9.239,7.132)--(9.242,7.125)--(9.246,7.119)--(9.249,7.113)%
  --(9.253,7.106)--(9.256,7.099)--(9.259,7.093)--(9.263,7.086)--(9.266,7.079)--(9.270,7.073)%
  --(9.273,7.066)--(9.276,7.060)--(9.280,7.054)--(9.283,7.048)--(9.287,7.042)--(9.290,7.036)%
  --(9.294,7.031)--(9.297,7.026)--(9.300,7.021)--(9.304,7.016)--(9.307,7.012)--(9.311,7.008)%
  --(9.314,7.005)--(9.317,7.002)--(9.321,6.999)--(9.324,6.996)--(9.328,6.995)--(9.331,6.993)%
  --(9.335,6.992)--(9.338,6.992)--(9.341,6.991)--(9.345,6.992)--(9.348,6.993)--(9.352,6.994)%
  --(9.355,6.995)--(9.359,6.998)--(9.362,7.000)--(9.365,7.003)--(9.369,7.007)--(9.372,7.010)%
  --(9.376,7.015)--(9.379,7.019)--(9.382,7.024)--(9.386,7.030)--(9.389,7.035)--(9.393,7.041)%
  --(9.396,7.047)--(9.400,7.054)--(9.403,7.061)--(9.406,7.068)--(9.410,7.075)--(9.413,7.082)%
  --(9.417,7.089)--(9.420,7.097)--(9.423,7.105)--(9.427,7.112)--(9.430,7.120)--(9.434,7.127)%
  --(9.437,7.135)--(9.441,7.142)--(9.444,7.150)--(9.447,7.157)--(9.451,7.164)--(9.454,7.170)%
  --(9.458,7.177)--(9.461,7.183)--(9.464,7.189)--(9.468,7.195)--(9.471,7.200)--(9.475,7.205)%
  --(9.478,7.209)--(9.482,7.213)--(9.485,7.217)--(9.488,7.220)--(9.492,7.223)--(9.495,7.225)%
  --(9.499,7.227)--(9.502,7.228)--(9.505,7.229)--(9.509,7.229)--(9.512,7.229)--(9.516,7.228)%
  --(9.519,7.226)--(9.523,7.224)--(9.526,7.222)--(9.529,7.219)--(9.533,7.215)--(9.536,7.212)%
  --(9.540,7.207)--(9.543,7.202)--(9.547,7.197)--(9.550,7.191)--(9.553,7.185)--(9.557,7.178)%
  --(9.560,7.171)--(9.564,7.164)--(9.567,7.156)--(9.570,7.148)--(9.574,7.140)--(9.577,7.132)%
  --(9.581,7.123)--(9.584,7.114)--(9.588,7.105)--(9.591,7.096)--(9.594,7.087)--(9.598,7.078)%
  --(9.601,7.069)--(9.605,7.060)--(9.608,7.052)--(9.611,7.043)--(9.615,7.034)--(9.618,7.026)%
  --(9.622,7.018)--(9.625,7.010)--(9.629,7.003)--(9.632,6.995)--(9.635,6.989)--(9.639,6.982)%
  --(9.642,6.976)--(9.646,6.971)--(9.649,6.966)--(9.652,6.962)--(9.656,6.958)--(9.659,6.954)%
  --(9.663,6.952)--(9.666,6.950)--(9.670,6.948)--(9.673,6.947)--(9.676,6.947)--(9.680,6.947)%
  --(9.683,6.948)--(9.687,6.950)--(9.690,6.952)--(9.693,6.955)--(9.697,6.959)--(9.700,6.963)%
  --(9.704,6.968)--(9.707,6.973)--(9.711,6.979)--(9.714,6.986)--(9.717,6.993)--(9.721,7.001)%
  --(9.724,7.009)--(9.728,7.017)--(9.731,7.027)--(9.735,7.036)--(9.738,7.046)--(9.741,7.056)%
  --(9.745,7.067)--(9.748,7.077)--(9.752,7.088)--(9.755,7.099)--(9.758,7.111)--(9.762,7.122)%
  --(9.765,7.133)--(9.769,7.145)--(9.772,7.156)--(9.776,7.167)--(9.779,7.178)--(9.782,7.189)%
  --(9.786,7.199)--(9.789,7.210)--(9.793,7.220)--(9.796,7.229)--(9.799,7.238)--(9.803,7.247)%
  --(9.806,7.255)--(9.810,7.263)--(9.813,7.270)--(9.817,7.276)--(9.820,7.282)--(9.823,7.287)%
  --(9.827,7.292)--(9.830,7.295)--(9.834,7.298)--(9.837,7.300)--(9.840,7.301)--(9.844,7.302)%
  --(9.847,7.301)--(9.851,7.300)--(9.854,7.298)--(9.858,7.295)--(9.861,7.291)--(9.864,7.286)%
  --(9.868,7.281)--(9.871,7.275)--(9.875,7.267)--(9.878,7.259)--(9.881,7.251)--(9.885,7.241)%
  --(9.888,7.231)--(9.892,7.220)--(9.895,7.208)--(9.899,7.196)--(9.902,7.183)--(9.905,7.170)%
  --(9.909,7.156)--(9.912,7.142)--(9.916,7.128)--(9.919,7.113)--(9.923,7.097)--(9.926,7.082)%
  --(9.929,7.067)--(9.933,7.051)--(9.936,7.035)--(9.940,7.020)--(9.943,7.004)--(9.946,6.989)%
  --(9.950,6.973)--(9.953,6.959)--(9.957,6.944)--(9.960,6.930)--(9.964,6.916)--(9.967,6.903)%
  --(9.970,6.891)--(9.974,6.879)--(9.977,6.868)--(9.981,6.858)--(9.984,6.849)--(9.987,6.841)%
  --(9.991,6.833)--(9.994,6.827)--(9.998,6.821)--(10.001,6.817)--(10.005,6.814)--(10.008,6.812)%
  --(10.011,6.812)--(10.015,6.812)--(10.018,6.814)--(10.022,6.817)--(10.025,6.821)--(10.028,6.827)%
  --(10.032,6.834)--(10.035,6.842)--(10.039,6.852)--(10.042,6.863)--(10.046,6.875)--(10.049,6.888)%
  --(10.052,6.903)--(10.056,6.919)--(10.059,6.936)--(10.063,6.954)--(10.066,6.973)--(10.069,6.993)%
  --(10.073,7.014)--(10.076,7.036)--(10.080,7.059)--(10.083,7.082)--(10.087,7.106)--(10.090,7.131)%
  --(10.093,7.156)--(10.097,7.182)--(10.100,7.208)--(10.104,7.234)--(10.107,7.260)--(10.111,7.286)%
  --(10.114,7.312)--(10.117,7.338)--(10.121,7.364)--(10.124,7.389)--(10.128,7.413)--(10.131,7.437)%
  --(10.134,7.460)--(10.138,7.482)--(10.141,7.504)--(10.145,7.523)--(10.148,7.542)--(10.152,7.559)%
  --(10.155,7.575)--(10.158,7.589)--(10.162,7.601)--(10.165,7.611)--(10.169,7.619)--(10.172,7.625)%
  --(10.175,7.629)--(10.179,7.631)--(10.182,7.630)--(10.186,7.626)--(10.189,7.620)--(10.193,7.611)%
  --(10.196,7.600)--(10.199,7.585)--(10.203,7.568)--(10.206,7.547)--(10.210,7.524)--(10.213,7.497)%
  --(10.216,7.467)--(10.220,7.434)--(10.223,7.398)--(10.227,7.358)--(10.230,7.315)--(10.234,7.268)%
  --(10.237,7.219)--(10.240,7.165)--(10.244,7.109)--(10.247,7.049)--(10.251,6.986)--(10.254,6.920)%
  --(10.257,6.850)--(10.261,6.777)--(10.264,6.701)--(10.268,6.622)--(10.271,6.540)--(10.275,6.455)%
  --(10.278,6.367)--(10.281,6.276)--(10.285,6.183)--(10.288,6.086)--(10.292,5.988)--(10.295,5.887)%
  --(10.299,5.784)--(10.302,5.678)--(10.305,5.571)--(10.309,5.462)--(10.312,5.351)--(10.316,5.238)%
  --(10.319,5.124)--(10.322,5.009)--(10.326,4.892)--(10.329,4.775)--(10.333,4.657)--(10.336,4.538)%
  --(10.340,4.418)--(10.343,4.298)--(10.346,4.178)--(10.350,4.058)--(10.353,3.938)--(10.357,3.818)%
  --(10.360,3.699)--(10.363,3.580)--(10.367,3.462)--(10.370,3.346)--(10.374,3.230)--(10.377,3.115)%
  --(10.381,3.002)--(10.384,2.890)--(10.387,2.780)--(10.391,2.672)--(10.394,2.566)--(10.398,2.462)%
  --(10.401,2.360)--(10.404,2.261)--(10.408,2.163)--(10.411,2.069)--(10.415,1.977)--(10.418,1.888)%
  --(10.422,1.802)--(10.425,1.718)--(10.428,1.638)--(10.432,1.561)--(10.435,1.486)--(10.439,1.416)%
  --(10.442,1.348)--(10.445,1.283)--(10.449,1.222)--(10.452,1.164)--(10.456,1.110)--(10.459,1.059)%
  --(10.463,1.011)--(10.466,0.966)--(10.469,0.925)--(10.473,0.888)--(10.476,0.853)--(10.480,0.822)%
  --(10.483,0.794)--(10.487,0.769)--(10.490,0.747)--(10.493,0.728)--(10.497,0.713)--(10.500,0.700)%
  --(10.504,0.690)--(10.507,0.683)--(10.510,0.678)--(10.514,0.676)--(10.517,0.677)--(10.521,0.680)%
  --(10.524,0.685)--(10.528,0.692)--(10.531,0.702)--(10.534,0.713)--(10.538,0.726)--(10.541,0.741)%
  --(10.545,0.758)--(10.548,0.776)--(10.551,0.795)--(10.555,0.816)--(10.558,0.837)--(10.562,0.860)%
  --(10.565,0.884)--(10.569,0.908)--(10.572,0.933)--(10.575,0.958)--(10.579,0.984)--(10.582,1.010)%
  --(10.586,1.036)--(10.589,1.062)--(10.592,1.089)--(10.596,1.115)--(10.599,1.140)--(10.603,1.166)%
  --(10.606,1.191)--(10.610,1.215)--(10.613,1.239)--(10.616,1.262)--(10.620,1.284)--(10.623,1.306)%
  --(10.627,1.326)--(10.630,1.346)--(10.633,1.364)--(10.637,1.382)--(10.640,1.398)--(10.644,1.413)%
  --(10.647,1.427)--(10.651,1.440)--(10.654,1.451)--(10.657,1.461)--(10.661,1.470)--(10.664,1.477)%
  --(10.668,1.484)--(10.671,1.488)--(10.675,1.492)--(10.678,1.494)--(10.681,1.495)--(10.685,1.495)%
  --(10.688,1.494)--(10.692,1.491)--(10.695,1.487)--(10.698,1.483)--(10.702,1.477)--(10.705,1.470)%
  --(10.709,1.462)--(10.712,1.453)--(10.716,1.443)--(10.719,1.432)--(10.722,1.421)--(10.726,1.409)%
  --(10.729,1.396)--(10.733,1.383)--(10.736,1.369)--(10.739,1.354)--(10.743,1.340)--(10.746,1.325)%
  --(10.750,1.309)--(10.753,1.294)--(10.757,1.278)--(10.760,1.263)--(10.763,1.247)--(10.767,1.231)%
  --(10.770,1.216)--(10.774,1.200)--(10.777,1.185)--(10.780,1.171)--(10.784,1.156)--(10.787,1.142)%
  --(10.791,1.129)--(10.794,1.116)--(10.798,1.103)--(10.801,1.092)--(10.804,1.080)--(10.808,1.070)%
  --(10.811,1.060)--(10.815,1.051)--(10.818,1.043)--(10.821,1.035)--(10.825,1.029)--(10.828,1.023)%
  --(10.832,1.018)--(10.835,1.014)--(10.839,1.010)--(10.842,1.008)--(10.845,1.006)--(10.849,1.005)%
  --(10.852,1.005)--(10.856,1.006)--(10.859,1.008)--(10.862,1.011)--(10.866,1.014)--(10.869,1.018)%
  --(10.873,1.023)--(10.876,1.028)--(10.880,1.034)--(10.883,1.041)--(10.886,1.048)--(10.890,1.056)%
  --(10.893,1.065)--(10.897,1.074)--(10.900,1.083)--(10.904,1.093)--(10.907,1.103)--(10.910,1.114)%
  --(10.914,1.124)--(10.917,1.135)--(10.921,1.147)--(10.924,1.158)--(10.927,1.169)--(10.931,1.181)%
  --(10.934,1.192)--(10.938,1.203)--(10.941,1.214)--(10.945,1.225)--(10.948,1.236)--(10.951,1.247)%
  --(10.955,1.257)--(10.958,1.267)--(10.962,1.277)--(10.965,1.286)--(10.968,1.295)--(10.972,1.303)%
  --(10.975,1.311)--(10.979,1.318)--(10.982,1.325)--(10.986,1.331)--(10.989,1.337)--(10.992,1.342)%
  --(10.996,1.347)--(10.999,1.350)--(11.003,1.354)--(11.006,1.356)--(11.009,1.358)--(11.013,1.359)%
  --(11.016,1.360)--(11.020,1.360)--(11.023,1.359)--(11.027,1.358)--(11.030,1.356)--(11.033,1.354)%
  --(11.037,1.351)--(11.040,1.347)--(11.044,1.343)--(11.047,1.338)--(11.050,1.333)--(11.054,1.327)%
  --(11.057,1.321)--(11.061,1.314)--(11.064,1.307)--(11.068,1.300)--(11.071,1.292)--(11.074,1.284)%
  --(11.078,1.276)--(11.081,1.268)--(11.085,1.259)--(11.088,1.250)--(11.092,1.241)--(11.095,1.232)%
  --(11.098,1.223)--(11.102,1.214)--(11.105,1.205)--(11.109,1.196)--(11.112,1.188)--(11.115,1.179)%
  --(11.119,1.170)--(11.122,1.162)--(11.126,1.154)--(11.129,1.146)--(11.133,1.139)--(11.136,1.132)%
  --(11.139,1.125)--(11.143,1.119)--(11.146,1.113)--(11.150,1.107)--(11.153,1.102)--(11.156,1.097)%
  --(11.160,1.093)--(11.163,1.089)--(11.167,1.086)--(11.170,1.084)--(11.174,1.081)--(11.177,1.080)%
  --(11.180,1.079)--(11.184,1.078)--(11.187,1.078)--(11.191,1.079)--(11.194,1.080)--(11.197,1.081)%
  --(11.201,1.083)--(11.204,1.086)--(11.208,1.089)--(11.211,1.092)--(11.215,1.096)--(11.218,1.100)%
  --(11.221,1.105)--(11.225,1.110)--(11.228,1.116)--(11.232,1.121)--(11.235,1.128)--(11.238,1.134)%
  --(11.242,1.141)--(11.245,1.147)--(11.249,1.155)--(11.252,1.162)--(11.256,1.169)--(11.259,1.177)%
  --(11.262,1.184)--(11.266,1.192)--(11.269,1.199)--(11.273,1.207)--(11.276,1.214)--(11.280,1.222)%
  --(11.283,1.229)--(11.286,1.236)--(11.290,1.243)--(11.293,1.250)--(11.297,1.257)--(11.300,1.263)%
  --(11.303,1.269)--(11.307,1.275)--(11.310,1.281)--(11.314,1.286)--(11.317,1.290)--(11.321,1.295)%
  --(11.324,1.299)--(11.327,1.302)--(11.331,1.306)--(11.334,1.308)--(11.338,1.311)--(11.341,1.313)%
  --(11.344,1.314)--(11.348,1.315)--(11.351,1.316)--(11.355,1.316)--(11.358,1.315)--(11.362,1.314)%
  --(11.365,1.313)--(11.368,1.311)--(11.372,1.309)--(11.375,1.307)--(11.379,1.304)--(11.382,1.300)%
  --(11.385,1.297)--(11.389,1.293)--(11.392,1.288)--(11.396,1.283)--(11.399,1.278)--(11.403,1.273)%
  --(11.406,1.268)--(11.409,1.262)--(11.413,1.256)--(11.416,1.250)--(11.420,1.243)--(11.423,1.237)%
  --(11.426,1.230)--(11.430,1.224)--(11.433,1.217)--(11.437,1.210)--(11.440,1.204)--(11.444,1.197)%
  --(11.447,1.191);
  \gpcolor{rgb color={0.902,0.624,0.000}}
  \draw[gp path] (1.196,7.045)--(1.199,7.045)--(1.203,7.046)--(1.206,7.047)--(1.210,7.050)%
  --(1.213,7.052)--(1.217,7.056)--(1.220,7.060)--(1.223,7.065)--(1.227,7.071)--(1.230,7.076)%
  --(1.234,7.083)--(1.237,7.089)--(1.240,7.096)--(1.244,7.103)--(1.247,7.110)--(1.251,7.116)%
  --(1.254,7.123)--(1.258,7.129)--(1.261,7.135)--(1.264,7.141)--(1.268,7.146)--(1.271,7.150)%
  --(1.275,7.154)--(1.278,7.157)--(1.281,7.159)--(1.285,7.161)--(1.288,7.162)--(1.292,7.161)%
  --(1.295,7.161)--(1.299,7.159)--(1.302,7.156)--(1.305,7.153)--(1.309,7.149)--(1.312,7.144)%
  --(1.316,7.139)--(1.319,7.133)--(1.322,7.126)--(1.326,7.120)--(1.329,7.113)--(1.333,7.105)%
  --(1.336,7.098)--(1.340,7.091)--(1.343,7.084)--(1.346,7.077)--(1.350,7.070)--(1.353,7.063)%
  --(1.357,7.058)--(1.360,7.052)--(1.363,7.048)--(1.367,7.044)--(1.370,7.041)--(1.374,7.038)%
  --(1.377,7.037)--(1.381,7.036)--(1.384,7.037)--(1.387,7.038)--(1.391,7.040)--(1.394,7.043)%
  --(1.398,7.047)--(1.401,7.051)--(1.405,7.057)--(1.408,7.063)--(1.411,7.069)--(1.415,7.076)%
  --(1.418,7.084)--(1.422,7.091)--(1.425,7.099)--(1.428,7.107)--(1.432,7.115)--(1.435,7.123)%
  --(1.439,7.130)--(1.442,7.138)--(1.446,7.144)--(1.449,7.151)--(1.452,7.156)--(1.456,7.161)%
  --(1.459,7.165)--(1.463,7.168)--(1.466,7.171)--(1.469,7.172)--(1.473,7.172)--(1.476,7.172)%
  --(1.480,7.170)--(1.483,7.168)--(1.487,7.164)--(1.490,7.160)--(1.493,7.155)--(1.497,7.149)%
  --(1.500,7.142)--(1.504,7.135)--(1.507,7.127)--(1.510,7.119)--(1.514,7.110)--(1.517,7.101)%
  --(1.521,7.093)--(1.524,7.084)--(1.528,7.075)--(1.531,7.067)--(1.534,7.059)--(1.538,7.052)%
  --(1.541,7.045)--(1.545,7.039)--(1.548,7.034)--(1.551,7.030)--(1.555,7.027)--(1.558,7.024)%
  --(1.562,7.023)--(1.565,7.023)--(1.569,7.024)--(1.572,7.026)--(1.575,7.029)--(1.579,7.033)%
  --(1.582,7.038)--(1.586,7.044)--(1.589,7.051)--(1.593,7.058)--(1.596,7.067)--(1.599,7.075)%
  --(1.603,7.085)--(1.606,7.094)--(1.610,7.104)--(1.613,7.114)--(1.616,7.124)--(1.620,7.133)%
  --(1.623,7.142)--(1.627,7.151)--(1.630,7.159)--(1.634,7.166)--(1.637,7.173)--(1.640,7.178)%
  --(1.644,7.183)--(1.647,7.186)--(1.651,7.189)--(1.654,7.190)--(1.657,7.190)--(1.661,7.189)%
  --(1.664,7.186)--(1.668,7.182)--(1.671,7.177)--(1.675,7.171)--(1.678,7.164)--(1.681,7.156)%
  --(1.685,7.148)--(1.688,7.138)--(1.692,7.128)--(1.695,7.117)--(1.698,7.106)--(1.702,7.095)%
  --(1.705,7.084)--(1.709,7.073)--(1.712,7.062)--(1.716,7.052)--(1.719,7.042)--(1.722,7.033)%
  --(1.726,7.025)--(1.729,7.017)--(1.733,7.011)--(1.736,7.006)--(1.739,7.003)--(1.743,7.000)%
  --(1.746,6.999)--(1.750,7.000)--(1.753,7.002)--(1.757,7.005)--(1.760,7.009)--(1.763,7.015)%
  --(1.767,7.023)--(1.770,7.031)--(1.774,7.041)--(1.777,7.051)--(1.781,7.062)--(1.784,7.074)%
  --(1.787,7.087)--(1.791,7.100)--(1.794,7.113)--(1.798,7.126)--(1.801,7.139)--(1.804,7.151)%
  --(1.808,7.163)--(1.811,7.175)--(1.815,7.185)--(1.818,7.195)--(1.822,7.203)--(1.825,7.210)%
  --(1.828,7.215)--(1.832,7.219)--(1.835,7.222)--(1.839,7.223)--(1.842,7.222)--(1.845,7.219)%
  --(1.849,7.215)--(1.852,7.209)--(1.856,7.201)--(1.859,7.192)--(1.863,7.182)--(1.866,7.170)%
  --(1.869,7.157)--(1.873,7.144)--(1.876,7.129)--(1.880,7.114)--(1.883,7.098)--(1.886,7.082)%
  --(1.890,7.066)--(1.893,7.051)--(1.897,7.035)--(1.900,7.021)--(1.904,7.007)--(1.907,6.995)%
  --(1.910,6.984)--(1.914,6.974)--(1.917,6.966)--(1.921,6.959)--(1.924,6.955)--(1.927,6.952)%
  --(1.931,6.952)--(1.934,6.953)--(1.938,6.957)--(1.941,6.963)--(1.945,6.971)--(1.948,6.981)%
  --(1.951,6.992)--(1.955,7.006)--(1.958,7.021)--(1.962,7.038)--(1.965,7.055)--(1.968,7.074)%
  --(1.972,7.094)--(1.975,7.114)--(1.979,7.134)--(1.982,7.154)--(1.986,7.174)--(1.989,7.194)%
  --(1.992,7.212)--(1.996,7.229)--(1.999,7.245)--(2.003,7.259)--(2.006,7.272)--(2.010,7.282)%
  --(2.013,7.290)--(2.016,7.295)--(2.020,7.298)--(2.023,7.298)--(2.027,7.295)--(2.030,7.289)%
  --(2.033,7.281)--(2.037,7.270)--(2.040,7.256)--(2.044,7.239)--(2.047,7.220)--(2.051,7.199)%
  --(2.054,7.176)--(2.057,7.152)--(2.061,7.125)--(2.064,7.098)--(2.068,7.070)--(2.071,7.042)%
  --(2.074,7.013)--(2.078,6.985)--(2.081,6.958)--(2.085,6.932)--(2.088,6.908)--(2.092,6.886)%
  --(2.095,6.866)--(2.098,6.849)--(2.102,6.835)--(2.105,6.824)--(2.109,6.817)--(2.112,6.814)%
  --(2.115,6.815)--(2.119,6.820)--(2.122,6.830)--(2.126,6.844)--(2.129,6.862)--(2.133,6.884)%
  --(2.136,6.911)--(2.139,6.941)--(2.143,6.976)--(2.146,7.013)--(2.150,7.054)--(2.153,7.097)%
  --(2.156,7.142)--(2.160,7.188)--(2.163,7.236)--(2.167,7.284)--(2.170,7.331)--(2.174,7.378)%
  --(2.177,7.423)--(2.180,7.465)--(2.184,7.504)--(2.187,7.539)--(2.191,7.570)--(2.194,7.595)%
  --(2.198,7.613)--(2.201,7.625)--(2.204,7.630)--(2.208,7.626)--(2.211,7.613)--(2.215,7.591)%
  --(2.218,7.559)--(2.221,7.517)--(2.225,7.464)--(2.228,7.400)--(2.232,7.325)--(2.235,7.239)%
  --(2.239,7.142)--(2.242,7.033)--(2.245,6.914)--(2.249,6.783)--(2.252,6.642)--(2.256,6.491)%
  --(2.259,6.329)--(2.262,6.159)--(2.266,5.980)--(2.269,5.793)--(2.273,5.598)--(2.276,5.398)%
  --(2.280,5.191)--(2.283,4.980)--(2.286,4.766)--(2.290,4.548)--(2.293,4.329)--(2.297,4.108)%
  --(2.300,3.888)--(2.303,3.670)--(2.307,3.453)--(2.310,3.240)--(2.314,3.031)--(2.317,2.826)%
  --(2.321,2.628)--(2.324,2.437)--(2.327,2.253)--(2.331,2.077)--(2.334,1.911)--(2.338,1.753)%
  --(2.341,1.606)--(2.344,1.469)--(2.348,1.343)--(2.351,1.228)--(2.355,1.124)--(2.358,1.031)%
  --(2.362,0.950)--(2.365,0.880)--(2.368,0.820)--(2.372,0.772)--(2.375,0.734)--(2.379,0.706)%
  --(2.382,0.688)--(2.386,0.679)--(2.389,0.678)--(2.392,0.686)--(2.396,0.700)--(2.399,0.722)%
  --(2.403,0.749)--(2.406,0.781)--(2.409,0.818)--(2.413,0.859)--(2.416,0.902)--(2.420,0.948)%
  --(2.423,0.995)--(2.427,1.043)--(2.430,1.091)--(2.433,1.138)--(2.437,1.184)--(2.440,1.228)%
  --(2.444,1.270)--(2.447,1.310)--(2.450,1.346)--(2.454,1.379)--(2.457,1.407)--(2.461,1.432)%
  --(2.464,1.453)--(2.468,1.470)--(2.471,1.482)--(2.474,1.489)--(2.478,1.493)--(2.481,1.492)%
  --(2.485,1.487)--(2.488,1.479)--(2.491,1.467)--(2.495,1.451)--(2.498,1.433)--(2.502,1.412)%
  --(2.505,1.389)--(2.509,1.364)--(2.512,1.338)--(2.515,1.310)--(2.519,1.282)--(2.522,1.254)%
  --(2.526,1.225)--(2.529,1.198)--(2.532,1.171)--(2.536,1.145)--(2.539,1.121)--(2.543,1.099)%
  --(2.546,1.079)--(2.550,1.061)--(2.553,1.045)--(2.556,1.032)--(2.560,1.022)--(2.563,1.015)%
  --(2.567,1.011)--(2.570,1.009)--(2.574,1.010)--(2.577,1.014)--(2.580,1.020)--(2.584,1.029)%
  --(2.587,1.040)--(2.591,1.053)--(2.594,1.068)--(2.597,1.085)--(2.601,1.102)--(2.604,1.121)%
  --(2.608,1.141)--(2.611,1.161)--(2.615,1.181)--(2.618,1.201)--(2.621,1.221)--(2.625,1.241)%
  --(2.628,1.259)--(2.632,1.276)--(2.635,1.292)--(2.638,1.307)--(2.642,1.320)--(2.645,1.331)%
  --(2.649,1.340)--(2.652,1.347)--(2.656,1.352)--(2.659,1.354)--(2.662,1.355)--(2.666,1.354)%
  --(2.669,1.351)--(2.673,1.345)--(2.676,1.338)--(2.679,1.329)--(2.683,1.319)--(2.686,1.307)%
  --(2.690,1.294)--(2.693,1.280)--(2.697,1.265)--(2.700,1.250)--(2.703,1.234)--(2.707,1.218)%
  --(2.710,1.203)--(2.714,1.187)--(2.717,1.172)--(2.720,1.158)--(2.724,1.144)--(2.727,1.132)%
  --(2.731,1.121)--(2.734,1.111)--(2.738,1.102)--(2.741,1.096)--(2.744,1.090)--(2.748,1.087)%
  --(2.751,1.085)--(2.755,1.085)--(2.758,1.086)--(2.762,1.089)--(2.765,1.094)--(2.768,1.100)%
  --(2.772,1.107)--(2.775,1.116)--(2.779,1.126)--(2.782,1.137)--(2.785,1.148)--(2.789,1.161)%
  --(2.792,1.173)--(2.796,1.186)--(2.799,1.199)--(2.803,1.212)--(2.806,1.225)--(2.809,1.238)%
  --(2.813,1.249)--(2.816,1.260)--(2.820,1.270)--(2.823,1.279)--(2.826,1.287)--(2.830,1.294)%
  --(2.833,1.300)--(2.837,1.304)--(2.840,1.306)--(2.844,1.308)--(2.847,1.307)--(2.850,1.306)%
  --(2.854,1.303)--(2.857,1.299)--(2.861,1.293)--(2.864,1.287)--(2.867,1.279)--(2.871,1.271)%
  --(2.874,1.261)--(2.878,1.251)--(2.881,1.241)--(2.885,1.230)--(2.888,1.219)--(2.891,1.207)%
  --(2.895,1.196)--(2.898,1.185)--(2.902,1.175)--(2.905,1.165)--(2.908,1.156)--(2.912,1.147)%
  --(2.915,1.140)--(2.919,1.133)--(2.922,1.127)--(2.926,1.123)--(2.929,1.120)--(2.932,1.118)%
  --(2.936,1.117)--(2.939,1.117)--(2.943,1.119)--(2.946,1.122)--(2.950,1.126)--(2.953,1.131)%
  --(2.956,1.137)--(2.960,1.144)--(2.963,1.151)--(2.967,1.160)--(2.970,1.168)--(2.973,1.178)%
  --(2.977,1.187)--(2.980,1.197)--(2.984,1.207)--(2.987,1.217)--(2.991,1.226)--(2.994,1.235)%
  --(2.997,1.244)--(3.001,1.252)--(3.004,1.259)--(3.008,1.266)--(3.011,1.271)--(3.014,1.276)%
  --(3.018,1.280)--(3.021,1.282)--(3.025,1.284)--(3.028,1.284)--(3.032,1.284)--(3.035,1.282)%
  --(3.038,1.279)--(3.042,1.275)--(3.045,1.271)--(3.049,1.265)--(3.052,1.259)--(3.055,1.252)%
  --(3.059,1.245)--(3.062,1.236)--(3.066,1.228)--(3.069,1.219)--(3.073,1.211)--(3.076,1.202)%
  --(3.079,1.193)--(3.083,1.185)--(3.086,1.177)--(3.090,1.169)--(3.093,1.162)--(3.096,1.156)%
  --(3.100,1.150)--(3.103,1.145)--(3.107,1.141)--(3.110,1.138)--(3.114,1.136)--(3.117,1.135)%
  --(3.120,1.135)--(3.124,1.135)--(3.127,1.137)--(3.131,1.140)--(3.134,1.143)--(3.138,1.148)%
  --(3.141,1.153)--(3.144,1.159)--(3.148,1.165)--(3.151,1.172)--(3.155,1.180)--(3.158,1.187)%
  --(3.161,1.195)--(3.165,1.203)--(3.168,1.211)--(3.172,1.219)--(3.175,1.226)--(3.179,1.234)%
  --(3.182,1.241)--(3.185,1.247)--(3.189,1.253)--(3.192,1.258)--(3.196,1.262)--(3.199,1.265)%
  --(3.202,1.268)--(3.206,1.270)--(3.209,1.271)--(3.213,1.270)--(3.216,1.270)--(3.220,1.268)%
  --(3.223,1.265)--(3.226,1.262)--(3.230,1.257)--(3.233,1.253)--(3.237,1.247)--(3.240,1.241)%
  --(3.243,1.235)--(3.247,1.228)--(3.250,1.221)--(3.254,1.213)--(3.257,1.206)--(3.261,1.199)%
  --(3.264,1.191)--(3.267,1.185)--(3.271,1.178)--(3.274,1.172)--(3.278,1.166)--(3.281,1.161)%
  --(3.284,1.156)--(3.288,1.153)--(3.291,1.150)--(3.295,1.147)--(3.298,1.146)--(3.302,1.145)%
  --(3.305,1.146)--(3.308,1.147)--(3.312,1.149)--(3.315,1.151)--(3.319,1.155)--(3.322,1.159)%
  --(3.326,1.163)--(3.329,1.168)--(3.332,1.174)--(3.336,1.180)--(3.339,1.187)--(3.343,1.193)%
  --(3.346,1.200)--(3.349,1.207)--(3.353,1.214)--(3.356,1.220)--(3.360,1.227)--(3.363,1.233)%
  --(3.367,1.239)--(3.370,1.244)--(3.373,1.248)--(3.377,1.252)--(3.380,1.256)--(3.384,1.258)%
  --(3.387,1.260)--(3.390,1.262)--(3.394,1.262)--(3.397,1.262)--(3.401,1.260)--(3.404,1.258)%
  --(3.408,1.256)--(3.411,1.253)--(3.414,1.249)--(3.418,1.244)--(3.421,1.239)--(3.425,1.234)%
  --(3.428,1.228)--(3.431,1.222)--(3.435,1.215)--(3.438,1.209)--(3.442,1.203)--(3.445,1.196)%
  --(3.449,1.190)--(3.452,1.184)--(3.455,1.178)--(3.459,1.173)--(3.462,1.168)--(3.466,1.164)%
  --(3.469,1.160)--(3.472,1.157)--(3.476,1.155)--(3.479,1.153)--(3.483,1.153)--(3.486,1.152)%
  --(3.490,1.153)--(3.493,1.154)--(3.496,1.156)--(3.500,1.159)--(3.503,1.162)--(3.507,1.166)%
  --(3.510,1.170)--(3.513,1.175)--(3.517,1.180)--(3.520,1.186)--(3.524,1.192)--(3.527,1.198)%
  --(3.531,1.204)--(3.534,1.210)--(3.537,1.216)--(3.541,1.222)--(3.544,1.227)--(3.548,1.233)%
  --(3.551,1.238)--(3.555,1.242)--(3.558,1.246)--(3.561,1.249)--(3.565,1.252)--(3.568,1.254)%
  --(3.572,1.255)--(3.575,1.256)--(3.578,1.256)--(3.582,1.255)--(3.585,1.254)--(3.589,1.252)%
  --(3.592,1.249)--(3.596,1.246)--(3.599,1.242)--(3.602,1.238)--(3.606,1.233)--(3.609,1.228)%
  --(3.613,1.223)--(3.616,1.217)--(3.619,1.211)--(3.623,1.206)--(3.626,1.200)--(3.630,1.194)%
  --(3.633,1.189)--(3.637,1.183)--(3.640,1.178)--(3.643,1.174)--(3.647,1.170)--(3.650,1.166)%
  --(3.654,1.163)--(3.657,1.161)--(3.660,1.159)--(3.664,1.158)--(3.667,1.157)--(3.671,1.157)%
  --(3.674,1.158)--(3.678,1.159)--(3.681,1.162)--(3.684,1.164)--(3.688,1.167)--(3.691,1.171)%
  --(3.695,1.175)--(3.698,1.180)--(3.701,1.185)--(3.705,1.190)--(3.708,1.196)--(3.712,1.201)%
  --(3.715,1.207)--(3.719,1.212)--(3.722,1.218)--(3.725,1.223)--(3.729,1.228)--(3.732,1.233)%
  --(3.736,1.237)--(3.739,1.241)--(3.743,1.244)--(3.746,1.247)--(3.749,1.249)--(3.753,1.251)%
  --(3.756,1.252)--(3.760,1.252)--(3.763,1.252)--(3.766,1.251)--(3.770,1.250)--(3.773,1.248)%
  --(3.777,1.245)--(3.780,1.242)--(3.784,1.238)--(3.787,1.234)--(3.790,1.229)--(3.794,1.224)%
  --(3.797,1.219)--(3.801,1.214)--(3.804,1.208)--(3.807,1.203)--(3.811,1.198)--(3.814,1.192)%
  --(3.818,1.187)--(3.821,1.183)--(3.825,1.178)--(3.828,1.174)--(3.831,1.170)--(3.835,1.167)%
  --(3.838,1.165)--(3.842,1.163)--(3.845,1.161)--(3.848,1.160)--(3.852,1.160)--(3.855,1.161)%
  --(3.859,1.162)--(3.862,1.163)--(3.866,1.165)--(3.869,1.168)--(3.872,1.172)--(3.876,1.175)%
  --(3.879,1.179)--(3.883,1.184)--(3.886,1.189)--(3.889,1.194)--(3.893,1.199)--(3.896,1.204)%
  --(3.900,1.209)--(3.903,1.215)--(3.907,1.220)--(3.910,1.225)--(3.913,1.229)--(3.917,1.233)%
  --(3.920,1.237)--(3.924,1.241)--(3.927,1.244)--(3.931,1.246)--(3.934,1.248)--(3.937,1.249)%
  --(3.941,1.250)--(3.944,1.250)--(3.948,1.249)--(3.951,1.248)--(3.954,1.246)--(3.958,1.244)%
  --(3.961,1.241)--(3.965,1.238)--(3.968,1.234)--(3.972,1.230)--(3.975,1.226)--(3.978,1.221)%
  --(3.982,1.216)--(3.985,1.211)--(3.989,1.206)--(3.992,1.201)--(3.995,1.195)--(3.999,1.191)%
  --(4.002,1.186)--(4.006,1.181)--(4.009,1.177)--(4.013,1.174)--(4.016,1.170)--(4.019,1.168)%
  --(4.023,1.165)--(4.026,1.164)--(4.030,1.163)--(4.033,1.162)--(4.036,1.162)--(4.040,1.163)%
  --(4.043,1.164)--(4.047,1.166)--(4.050,1.168)--(4.054,1.171)--(4.057,1.175)--(4.060,1.178)%
  --(4.064,1.183)--(4.067,1.187)--(4.071,1.192)--(4.074,1.197)--(4.077,1.202)--(4.081,1.207)%
  --(4.084,1.212)--(4.088,1.217)--(4.091,1.222)--(4.095,1.226)--(4.098,1.230)--(4.101,1.234)%
  --(4.105,1.238)--(4.108,1.241)--(4.112,1.244)--(4.115,1.246)--(4.119,1.247)--(4.122,1.248)%
  --(4.125,1.249)--(4.129,1.248)--(4.132,1.248)--(4.136,1.246)--(4.139,1.244)--(4.142,1.242)%
  --(4.146,1.239)--(4.149,1.235)--(4.153,1.232)--(4.156,1.227)--(4.160,1.223)--(4.163,1.218)%
  --(4.166,1.213)--(4.170,1.208)--(4.173,1.203)--(4.177,1.198)--(4.180,1.193)--(4.183,1.189)%
  --(4.187,1.184)--(4.190,1.180)--(4.194,1.176)--(4.197,1.173)--(4.201,1.170)--(4.204,1.167)%
  --(4.207,1.165)--(4.211,1.164)--(4.214,1.163)--(4.218,1.163)--(4.221,1.163)--(4.224,1.164)%
  --(4.228,1.166)--(4.231,1.168)--(4.235,1.170)--(4.238,1.173)--(4.242,1.177)--(4.245,1.181)%
  --(4.248,1.185)--(4.252,1.190)--(4.255,1.194)--(4.259,1.199)--(4.262,1.204)--(4.265,1.209)%
  --(4.269,1.214)--(4.272,1.219)--(4.276,1.224)--(4.279,1.228)--(4.283,1.232)--(4.286,1.236)%
  --(4.289,1.239)--(4.293,1.242)--(4.296,1.244)--(4.300,1.246)--(4.303,1.247)--(4.307,1.248)%
  --(4.310,1.248)--(4.313,1.248)--(4.317,1.247)--(4.320,1.245)--(4.324,1.243)--(4.327,1.240)%
  --(4.330,1.237)--(4.334,1.233)--(4.337,1.229)--(4.341,1.225)--(4.344,1.220)--(4.348,1.216)%
  --(4.351,1.211)--(4.354,1.206)--(4.358,1.201)--(4.361,1.196)--(4.365,1.191)--(4.368,1.186)%
  --(4.371,1.182)--(4.375,1.178)--(4.378,1.174)--(4.382,1.171)--(4.385,1.168)--(4.389,1.166)%
  --(4.392,1.165)--(4.395,1.163)--(4.399,1.163)--(4.402,1.163)--(4.406,1.164)--(4.409,1.165)%
  --(4.412,1.167)--(4.416,1.169)--(4.419,1.172)--(4.423,1.175)--(4.426,1.179)--(4.430,1.183)%
  --(4.433,1.187)--(4.436,1.192)--(4.440,1.197)--(4.443,1.202)--(4.447,1.207)--(4.450,1.212)%
  --(4.453,1.217)--(4.457,1.221)--(4.460,1.226)--(4.464,1.230)--(4.467,1.234)--(4.471,1.238)%
  --(4.474,1.241)--(4.477,1.244)--(4.481,1.246)--(4.484,1.247)--(4.488,1.248)--(4.491,1.249)%
  --(4.495,1.248)--(4.498,1.248)--(4.501,1.246)--(4.505,1.244)--(4.508,1.242)--(4.512,1.239)%
  --(4.515,1.236)--(4.518,1.232)--(4.522,1.228)--(4.525,1.223)--(4.529,1.218)--(4.532,1.213)%
  --(4.536,1.208)--(4.539,1.203)--(4.542,1.198)--(4.546,1.193)--(4.549,1.188)--(4.553,1.184)%
  --(4.556,1.180)--(4.559,1.176)--(4.563,1.172)--(4.566,1.169)--(4.570,1.167)--(4.573,1.165)%
  --(4.577,1.163)--(4.580,1.162)--(4.583,1.162)--(4.587,1.162)--(4.590,1.163)--(4.594,1.165)%
  --(4.597,1.167)--(4.600,1.169)--(4.604,1.173)--(4.607,1.176)--(4.611,1.180)--(4.614,1.184)%
  --(4.618,1.189)--(4.621,1.194)--(4.624,1.199)--(4.628,1.204)--(4.631,1.209)--(4.635,1.214)%
  --(4.638,1.219)--(4.641,1.224)--(4.645,1.229)--(4.648,1.233)--(4.652,1.237)--(4.655,1.240)%
  --(4.659,1.243)--(4.662,1.246)--(4.665,1.248)--(4.669,1.249)--(4.672,1.250)--(4.676,1.250)%
  --(4.679,1.250)--(4.683,1.248)--(4.686,1.247)--(4.689,1.244)--(4.693,1.242)--(4.696,1.238)%
  --(4.700,1.235)--(4.703,1.230)--(4.706,1.226)--(4.710,1.221)--(4.713,1.216)--(4.717,1.211)%
  --(4.720,1.206)--(4.724,1.200)--(4.727,1.195)--(4.730,1.190)--(4.734,1.185)--(4.737,1.181)%
  --(4.741,1.176)--(4.744,1.173)--(4.747,1.169)--(4.751,1.166)--(4.754,1.164)--(4.758,1.162)%
  --(4.761,1.161)--(4.765,1.160)--(4.768,1.160)--(4.771,1.161)--(4.775,1.162)--(4.778,1.164)%
  --(4.782,1.166)--(4.785,1.169)--(4.788,1.173)--(4.792,1.177)--(4.795,1.181)--(4.799,1.186)%
  --(4.802,1.191)--(4.806,1.196)--(4.809,1.201)--(4.812,1.207)--(4.816,1.212)--(4.819,1.218)%
  --(4.823,1.223)--(4.826,1.228)--(4.829,1.232)--(4.833,1.237)--(4.836,1.240)--(4.840,1.244)%
  --(4.843,1.247)--(4.847,1.249)--(4.850,1.251)--(4.853,1.252)--(4.857,1.252)--(4.860,1.252)%
  --(4.864,1.251)--(4.867,1.250)--(4.870,1.248)--(4.874,1.245)--(4.877,1.242)--(4.881,1.238)%
  --(4.884,1.234)--(4.888,1.230)--(4.891,1.225)--(4.894,1.219)--(4.898,1.214)--(4.901,1.209)%
  --(4.905,1.203)--(4.908,1.197)--(4.912,1.192)--(4.915,1.187)--(4.918,1.182)--(4.922,1.177)%
  --(4.925,1.172)--(4.929,1.169)--(4.932,1.165)--(4.935,1.162)--(4.939,1.160)--(4.942,1.158)%
  --(4.946,1.157)--(4.949,1.157)--(4.953,1.157)--(4.956,1.158)--(4.959,1.160)--(4.963,1.162)%
  --(4.966,1.165)--(4.970,1.169)--(4.973,1.173)--(4.976,1.177)--(4.980,1.182)--(4.983,1.187)%
  --(4.987,1.193)--(4.990,1.198)--(4.994,1.204)--(4.997,1.210)--(5.000,1.216)--(5.004,1.221)%
  --(5.007,1.227)--(5.011,1.232)--(5.014,1.237)--(5.017,1.241)--(5.021,1.245)--(5.024,1.249)%
  --(5.028,1.251)--(5.031,1.254)--(5.035,1.255)--(5.038,1.256)--(5.041,1.256)--(5.045,1.256)%
  --(5.048,1.255)--(5.052,1.253)--(5.055,1.250)--(5.058,1.247)--(5.062,1.243)--(5.065,1.239)%
  --(5.069,1.234)--(5.072,1.229)--(5.076,1.224)--(5.079,1.218)--(5.082,1.212)--(5.086,1.206)%
  --(5.089,1.200)--(5.093,1.194)--(5.096,1.188)--(5.100,1.182)--(5.103,1.177)--(5.106,1.172)%
  --(5.110,1.167)--(5.113,1.163)--(5.117,1.160)--(5.120,1.157)--(5.123,1.155)--(5.127,1.153)%
  --(5.130,1.152)--(5.134,1.152)--(5.137,1.153)--(5.141,1.155)--(5.144,1.157)--(5.147,1.160)%
  --(5.151,1.163)--(5.154,1.167)--(5.158,1.172)--(5.161,1.177)--(5.164,1.182)--(5.168,1.188)%
  --(5.171,1.194)--(5.175,1.201)--(5.178,1.207)--(5.182,1.213)--(5.185,1.220)--(5.188,1.226)%
  --(5.192,1.232)--(5.195,1.237)--(5.199,1.243)--(5.202,1.247)--(5.205,1.251)--(5.209,1.255)%
  --(5.212,1.258)--(5.216,1.260)--(5.219,1.261)--(5.223,1.262)--(5.226,1.262)--(5.229,1.261)%
  --(5.233,1.259)--(5.236,1.257)--(5.240,1.254)--(5.243,1.250)--(5.246,1.245)--(5.250,1.240)%
  --(5.253,1.235)--(5.257,1.229)--(5.260,1.222)--(5.264,1.216)--(5.267,1.209)--(5.270,1.202)%
  --(5.274,1.195)--(5.277,1.189)--(5.281,1.182)--(5.284,1.176)--(5.288,1.170)--(5.291,1.165)%
  --(5.294,1.160)--(5.298,1.156)--(5.301,1.152)--(5.305,1.149)--(5.308,1.147)--(5.311,1.146)%
  --(5.315,1.145)--(5.318,1.146)--(5.322,1.147)--(5.325,1.149)--(5.329,1.152)--(5.332,1.155)%
  --(5.335,1.160)--(5.339,1.164)--(5.342,1.170)--(5.346,1.176)--(5.349,1.183)--(5.352,1.189)%
  --(5.356,1.196)--(5.359,1.204)--(5.363,1.211)--(5.366,1.218)--(5.370,1.226)--(5.373,1.233)%
  --(5.376,1.239)--(5.380,1.245)--(5.383,1.251)--(5.387,1.256)--(5.390,1.260)--(5.393,1.264)%
  --(5.397,1.267)--(5.400,1.269)--(5.404,1.270)--(5.407,1.271)--(5.411,1.270)--(5.414,1.269)%
  --(5.417,1.266)--(5.421,1.263)--(5.424,1.259)--(5.428,1.254)--(5.431,1.249)--(5.434,1.242)%
  --(5.438,1.236)--(5.441,1.229)--(5.445,1.221)--(5.448,1.213)--(5.452,1.205)--(5.455,1.197)%
  --(5.458,1.190)--(5.462,1.182)--(5.465,1.174)--(5.469,1.167)--(5.472,1.161)--(5.476,1.155)%
  --(5.479,1.149)--(5.482,1.145)--(5.486,1.141)--(5.489,1.138)--(5.493,1.136)--(5.496,1.135)%
  --(5.499,1.135)--(5.503,1.136)--(5.506,1.137)--(5.510,1.140)--(5.513,1.144)--(5.517,1.149)%
  --(5.520,1.154)--(5.523,1.160)--(5.527,1.167)--(5.530,1.175)--(5.534,1.182)--(5.537,1.191)%
  --(5.540,1.199)--(5.544,1.208)--(5.547,1.217)--(5.551,1.226)--(5.554,1.234)--(5.558,1.242)%
  --(5.561,1.250)--(5.564,1.257)--(5.568,1.264)--(5.571,1.269)--(5.575,1.274)--(5.578,1.278)%
  --(5.581,1.281)--(5.585,1.283)--(5.588,1.284)--(5.592,1.284)--(5.595,1.283)--(5.599,1.280)%
  --(5.602,1.277)--(5.605,1.273)--(5.609,1.267)--(5.612,1.261)--(5.616,1.254)--(5.619,1.246)%
  --(5.622,1.238)--(5.626,1.229)--(5.629,1.219)--(5.633,1.210)--(5.636,1.200)--(5.640,1.190)%
  --(5.643,1.181)--(5.646,1.171)--(5.650,1.162)--(5.653,1.154)--(5.657,1.146)--(5.660,1.139)%
  --(5.664,1.132)--(5.667,1.127)--(5.670,1.123)--(5.674,1.120)--(5.677,1.118)--(5.681,1.117)%
  --(5.684,1.117)--(5.687,1.119)--(5.691,1.122)--(5.694,1.126)--(5.698,1.131)--(5.701,1.138)%
  --(5.705,1.145)--(5.708,1.153)--(5.711,1.162)--(5.715,1.172)--(5.718,1.182)--(5.722,1.193)%
  --(5.725,1.204)--(5.728,1.215)--(5.732,1.226)--(5.735,1.237)--(5.739,1.248)--(5.742,1.258)%
  --(5.746,1.268)--(5.749,1.277)--(5.752,1.285)--(5.756,1.291)--(5.759,1.297)--(5.763,1.302)%
  --(5.766,1.305)--(5.769,1.307)--(5.773,1.308)--(5.776,1.307)--(5.780,1.305)--(5.783,1.301)%
  --(5.787,1.296)--(5.790,1.290)--(5.793,1.282)--(5.797,1.273)--(5.800,1.263)--(5.804,1.253)%
  --(5.807,1.241)--(5.810,1.229)--(5.814,1.216)--(5.817,1.203)--(5.821,1.190)--(5.824,1.177)%
  --(5.828,1.164)--(5.831,1.152)--(5.834,1.140)--(5.838,1.129)--(5.841,1.119)--(5.845,1.110)%
  --(5.848,1.102)--(5.852,1.095)--(5.855,1.090)--(5.858,1.087)--(5.862,1.085)--(5.865,1.085)%
  --(5.869,1.086)--(5.872,1.089)--(5.875,1.094)--(5.879,1.100)--(5.882,1.108)--(5.886,1.118)%
  --(5.889,1.128)--(5.893,1.141)--(5.896,1.154)--(5.899,1.168)--(5.903,1.183)--(5.906,1.198)%
  --(5.910,1.214)--(5.913,1.230)--(5.916,1.245)--(5.920,1.261)--(5.923,1.276)--(5.927,1.290)%
  --(5.930,1.303)--(5.934,1.316)--(5.937,1.326)--(5.940,1.336)--(5.944,1.343)--(5.947,1.349)%
  --(5.951,1.353)--(5.954,1.355)--(5.957,1.355)--(5.961,1.353)--(5.964,1.348)--(5.968,1.342)%
  --(5.971,1.333)--(5.975,1.323)--(5.978,1.311)--(5.981,1.297)--(5.985,1.281)--(5.988,1.264)%
  --(5.992,1.246)--(5.995,1.227)--(5.998,1.207)--(6.002,1.187)--(6.005,1.167)--(6.009,1.147)%
  --(6.012,1.127)--(6.016,1.108)--(6.019,1.090)--(6.022,1.073)--(6.026,1.057)--(6.029,1.044)%
  --(6.033,1.032)--(6.036,1.023)--(6.040,1.015)--(6.043,1.011)--(6.046,1.009)--(6.050,1.010)%
  --(6.053,1.013)--(6.057,1.020)--(6.060,1.029)--(6.063,1.041)--(6.067,1.056)--(6.070,1.073)%
  --(6.074,1.093)--(6.077,1.114)--(6.081,1.138)--(6.084,1.163)--(6.087,1.190)--(6.091,1.217)%
  --(6.094,1.245)--(6.098,1.274)--(6.101,1.302)--(6.104,1.330)--(6.108,1.356)--(6.111,1.382)%
  --(6.115,1.406)--(6.118,1.427)--(6.122,1.446)--(6.125,1.463)--(6.128,1.476)--(6.132,1.485)%
  --(6.135,1.491)--(6.139,1.493)--(6.142,1.491)--(6.145,1.484)--(6.149,1.474)--(6.152,1.458)%
  --(6.156,1.439)--(6.159,1.415)--(6.163,1.387)--(6.166,1.356)--(6.169,1.321)--(6.173,1.282)%
  --(6.176,1.241)--(6.180,1.197)--(6.183,1.152)--(6.186,1.105)--(6.190,1.057)--(6.193,1.009)%
  --(6.197,0.962)--(6.200,0.916)--(6.204,0.871)--(6.207,0.830)--(6.210,0.792)--(6.214,0.758)%
  --(6.217,0.729)--(6.221,0.706)--(6.224,0.689)--(6.228,0.680)--(6.231,0.678)--(6.234,0.684)%
  --(6.238,0.700)--(6.241,0.725)--(6.245,0.760)--(6.248,0.805)--(6.251,0.861)--(6.255,0.928)%
  --(6.258,1.006)--(6.262,1.095)--(6.265,1.196)--(6.269,1.308)--(6.272,1.431)--(6.275,1.564)%
  --(6.279,1.709)--(6.282,1.863)--(6.286,2.027)--(6.289,2.200)--(6.292,2.382)--(6.296,2.571)%
  --(6.299,2.767)--(6.303,2.970)--(6.306,3.177)--(6.310,3.390)--(6.313,3.605)--(6.316,3.824)%
  --(6.320,4.043)--(6.323,4.264)--(6.327,4.483)--(6.330,4.702)--(6.333,4.917)--(6.337,5.130)%
  --(6.340,5.337)--(6.344,5.540)--(6.347,5.736)--(6.351,5.925)--(6.354,6.107)--(6.357,6.280)%
  --(6.361,6.444)--(6.364,6.598)--(6.368,6.743)--(6.371,6.876)--(6.374,6.999)--(6.378,7.111)%
  --(6.381,7.212)--(6.385,7.301)--(6.388,7.379)--(6.392,7.446)--(6.395,7.502)--(6.398,7.547)%
  --(6.402,7.582)--(6.405,7.607)--(6.409,7.623)--(6.412,7.629)--(6.415,7.627)--(6.419,7.618)%
  --(6.422,7.601)--(6.426,7.578)--(6.429,7.549)--(6.433,7.515)--(6.436,7.477)--(6.439,7.436)%
  --(6.443,7.391)--(6.446,7.345)--(6.450,7.298)--(6.453,7.250)--(6.457,7.202)--(6.460,7.155)%
  --(6.463,7.110)--(6.467,7.066)--(6.470,7.025)--(6.474,6.986)--(6.477,6.951)--(6.480,6.920)%
  --(6.484,6.892)--(6.487,6.868)--(6.491,6.849)--(6.494,6.833)--(6.498,6.823)--(6.501,6.816)%
  --(6.504,6.814)--(6.508,6.816)--(6.511,6.822)--(6.515,6.831)--(6.518,6.844)--(6.521,6.861)%
  --(6.525,6.880)--(6.528,6.901)--(6.532,6.925)--(6.535,6.951)--(6.539,6.977)--(6.542,7.005)%
  --(6.545,7.033)--(6.549,7.062)--(6.552,7.090)--(6.556,7.117)--(6.559,7.144)--(6.562,7.169)%
  --(6.566,7.193)--(6.569,7.214)--(6.573,7.234)--(6.576,7.251)--(6.580,7.266)--(6.583,7.278)%
  --(6.586,7.287)--(6.590,7.294)--(6.593,7.297)--(6.597,7.298)--(6.600,7.296)--(6.603,7.292)%
  --(6.607,7.284)--(6.610,7.275)--(6.614,7.263)--(6.617,7.250)--(6.621,7.234)--(6.624,7.217)%
  --(6.627,7.199)--(6.631,7.180)--(6.634,7.160)--(6.638,7.140)--(6.641,7.120)--(6.645,7.100)%
  --(6.648,7.080)--(6.651,7.061)--(6.655,7.043)--(6.658,7.026)--(6.662,7.010)--(6.665,6.996)%
  --(6.668,6.984)--(6.672,6.974)--(6.675,6.965)--(6.679,6.959)--(6.682,6.954)--(6.686,6.952)%
  --(6.689,6.952)--(6.692,6.954)--(6.696,6.958)--(6.699,6.964)--(6.703,6.971)--(6.706,6.981)%
  --(6.709,6.991)--(6.713,7.004)--(6.716,7.017)--(6.720,7.031)--(6.723,7.046)--(6.727,7.062)%
  --(6.730,7.077)--(6.733,7.093)--(6.737,7.109)--(6.740,7.124)--(6.744,7.139)--(6.747,7.153)%
  --(6.750,7.166)--(6.754,7.179)--(6.757,7.189)--(6.761,7.199)--(6.764,7.207)--(6.768,7.213)%
  --(6.771,7.218)--(6.774,7.221)--(6.778,7.222)--(6.781,7.222)--(6.785,7.220)--(6.788,7.217)%
  --(6.791,7.212)--(6.795,7.205)--(6.798,7.197)--(6.802,7.188)--(6.805,7.178)--(6.809,7.167)%
  --(6.812,7.155)--(6.815,7.143)--(6.819,7.130)--(6.822,7.117)--(6.826,7.104)--(6.829,7.091)%
  --(6.833,7.078)--(6.836,7.066)--(6.839,7.054)--(6.843,7.044)--(6.846,7.034)--(6.850,7.025)%
  --(6.853,7.017)--(6.856,7.011)--(6.860,7.006)--(6.863,7.002)--(6.867,7.000)--(6.870,6.999)%
  --(6.874,7.000)--(6.877,7.002)--(6.880,7.005)--(6.884,7.010)--(6.887,7.016)--(6.891,7.022)%
  --(6.894,7.030)--(6.897,7.039)--(6.901,7.049)--(6.904,7.059)--(6.908,7.070)--(6.911,7.081)%
  --(6.915,7.092)--(6.918,7.103)--(6.921,7.114)--(6.925,7.125)--(6.928,7.135)--(6.932,7.145)%
  --(6.935,7.154)--(6.938,7.162)--(6.942,7.169)--(6.945,7.176)--(6.949,7.181)--(6.952,7.185)%
  --(6.956,7.188)--(6.959,7.190)--(6.962,7.190)--(6.966,7.189)--(6.969,7.187)--(6.973,7.184)%
  --(6.976,7.180)--(6.979,7.175)--(6.983,7.168)--(6.986,7.161)--(6.990,7.153)--(6.993,7.145)%
  --(6.997,7.136)--(7.000,7.126)--(7.003,7.117)--(7.007,7.107)--(7.010,7.097)--(7.014,7.088)%
  --(7.017,7.078)--(7.021,7.069)--(7.024,7.061)--(7.027,7.053)--(7.031,7.046)--(7.034,7.040)%
  --(7.038,7.034)--(7.041,7.030)--(7.044,7.027)--(7.048,7.024)--(7.051,7.023)--(7.055,7.023)%
  --(7.058,7.024)--(7.062,7.026)--(7.065,7.029)--(7.068,7.033)--(7.072,7.038)--(7.075,7.043)%
  --(7.079,7.050)--(7.082,7.057)--(7.085,7.065)--(7.089,7.073)--(7.092,7.081)--(7.096,7.090)%
  --(7.099,7.099)--(7.103,7.108)--(7.106,7.116)--(7.109,7.125)--(7.113,7.132)--(7.116,7.140)%
  --(7.120,7.147)--(7.123,7.153)--(7.126,7.158)--(7.130,7.163)--(7.133,7.167)--(7.137,7.170)%
  --(7.140,7.171)--(7.144,7.172)--(7.147,7.172)--(7.150,7.171)--(7.154,7.169)--(7.157,7.166)%
  --(7.161,7.162)--(7.164,7.158)--(7.167,7.152)--(7.171,7.146)--(7.174,7.140)--(7.178,7.133)%
  --(7.181,7.125)--(7.185,7.117)--(7.188,7.110)--(7.191,7.102)--(7.195,7.094)--(7.198,7.086)%
  --(7.202,7.078)--(7.205,7.071)--(7.209,7.065)--(7.212,7.058)--(7.215,7.053)--(7.219,7.048)%
  --(7.222,7.044)--(7.226,7.041)--(7.229,7.038)--(7.232,7.037)--(7.236,7.036)--(7.239,7.037)%
  --(7.243,7.038)--(7.246,7.040)--(7.250,7.043)--(7.253,7.047)--(7.256,7.051)--(7.260,7.056)%
  --(7.263,7.062)--(7.267,7.068)--(7.270,7.074)--(7.273,7.081)--(7.277,7.089)--(7.280,7.096)%
  --(7.284,7.103)--(7.287,7.111)--(7.291,7.118)--(7.294,7.124)--(7.297,7.131)--(7.301,7.137)%
  --(7.304,7.143)--(7.308,7.147)--(7.311,7.152)--(7.314,7.155)--(7.318,7.158)--(7.321,7.160)%
  --(7.325,7.161)--(7.328,7.162)--(7.332,7.161)--(7.335,7.160)--(7.338,7.158)--(7.342,7.155)%
  --(7.345,7.151)--(7.349,7.147)--(7.352,7.142)--(7.355,7.137)--(7.359,7.131)--(7.362,7.125)%
  --(7.366,7.118)--(7.369,7.112)--(7.373,7.105)--(7.376,7.098)--(7.379,7.091)--(7.383,7.085)%
  --(7.386,7.078)--(7.390,7.072)--(7.393,7.067)--(7.397,7.062)--(7.400,7.057)--(7.403,7.053)%
  --(7.407,7.050)--(7.410,7.048)--(7.414,7.046)--(7.417,7.045)--(7.420,7.045)--(7.424,7.046)%
  --(7.427,7.047)--(7.431,7.049)--(7.434,7.052)--(7.438,7.056)--(7.441,7.060)--(7.444,7.064)%
  --(7.448,7.070)--(7.451,7.075)--(7.455,7.081)--(7.458,7.087)--(7.461,7.094)--(7.465,7.100)%
  --(7.468,7.106)--(7.472,7.113)--(7.475,7.119)--(7.479,7.125)--(7.482,7.130)--(7.485,7.135)%
  --(7.489,7.140)--(7.492,7.144)--(7.496,7.147)--(7.499,7.150)--(7.502,7.152)--(7.506,7.154)%
  --(7.509,7.155)--(7.513,7.155)--(7.516,7.154)--(7.520,7.152)--(7.523,7.150)--(7.526,7.147)%
  --(7.530,7.144)--(7.533,7.140)--(7.537,7.135)--(7.540,7.130)--(7.543,7.125)--(7.547,7.119)%
  --(7.550,7.113)--(7.554,7.107)--(7.557,7.101)--(7.561,7.095)--(7.564,7.089)--(7.567,7.083)%
  --(7.571,7.078)--(7.574,7.073)--(7.578,7.068)--(7.581,7.064)--(7.585,7.060)--(7.588,7.057)%
  --(7.591,7.054)--(7.595,7.052)--(7.598,7.051)--(7.602,7.051)--(7.605,7.051)--(7.608,7.052)%
  --(7.612,7.053)--(7.615,7.056)--(7.619,7.058)--(7.622,7.062)--(7.626,7.066)--(7.629,7.070)%
  --(7.632,7.075)--(7.636,7.080)--(7.639,7.086)--(7.643,7.091)--(7.646,7.097)--(7.649,7.103)%
  --(7.653,7.109)--(7.656,7.114)--(7.660,7.120)--(7.663,7.125)--(7.667,7.130)--(7.670,7.134)%
  --(7.673,7.138)--(7.677,7.142)--(7.680,7.145)--(7.684,7.147)--(7.687,7.149)--(7.690,7.150)%
  --(7.694,7.150)--(7.697,7.150)--(7.701,7.149)--(7.704,7.147)--(7.708,7.145)--(7.711,7.142)%
  --(7.714,7.138)--(7.718,7.135)--(7.721,7.130)--(7.725,7.125)--(7.728,7.120)--(7.731,7.115)%
  --(7.735,7.110)--(7.738,7.104)--(7.742,7.098)--(7.745,7.093)--(7.749,7.088)--(7.752,7.082)%
  --(7.755,7.077)--(7.759,7.073)--(7.762,7.069)--(7.766,7.065)--(7.769,7.062)--(7.773,7.059)%
  --(7.776,7.057)--(7.779,7.056)--(7.783,7.055)--(7.786,7.055)--(7.790,7.055)--(7.793,7.056)%
  --(7.796,7.058)--(7.800,7.060)--(7.803,7.063)--(7.807,7.067)--(7.810,7.070)--(7.814,7.075)%
  --(7.817,7.079)--(7.820,7.084)--(7.824,7.089)--(7.827,7.095)--(7.831,7.100)--(7.834,7.106)%
  --(7.837,7.111)--(7.841,7.116)--(7.844,7.121)--(7.848,7.126)--(7.851,7.130)--(7.855,7.134)%
  --(7.858,7.138)--(7.861,7.141)--(7.865,7.143)--(7.868,7.145)--(7.872,7.146)--(7.875,7.147)%
  --(7.878,7.147)--(7.882,7.146)--(7.885,7.145)--(7.889,7.143)--(7.892,7.141)--(7.896,7.138)%
  --(7.899,7.134)--(7.902,7.131)--(7.906,7.126)--(7.909,7.122)--(7.913,7.117)--(7.916,7.112)%
  --(7.919,7.107)--(7.923,7.101)--(7.926,7.096)--(7.930,7.091)--(7.933,7.086)--(7.937,7.081)%
  --(7.940,7.077)--(7.943,7.072)--(7.947,7.069)--(7.950,7.065)--(7.954,7.063)--(7.957,7.060)%
  --(7.960,7.059)--(7.964,7.057)--(7.967,7.057)--(7.971,7.057)--(7.974,7.058)--(7.978,7.059)%
  --(7.981,7.061)--(7.984,7.064)--(7.988,7.067)--(7.991,7.070)--(7.995,7.074)--(7.998,7.078)%
  --(8.002,7.083)--(8.005,7.088)--(8.008,7.093)--(8.012,7.098)--(8.015,7.103)--(8.019,7.108)%
  --(8.022,7.113)--(8.025,7.118)--(8.029,7.123)--(8.032,7.127)--(8.036,7.131)--(8.039,7.134)%
  --(8.043,7.138)--(8.046,7.140)--(8.049,7.142)--(8.053,7.144)--(8.056,7.145)--(8.060,7.145)%
  --(8.063,7.145)--(8.066,7.144)--(8.070,7.142)--(8.073,7.140)--(8.077,7.138)--(8.080,7.135)%
  --(8.084,7.131)--(8.087,7.127)--(8.090,7.123)--(8.094,7.119)--(8.097,7.114)--(8.101,7.109)%
  --(8.104,7.104)--(8.107,7.099)--(8.111,7.094)--(8.114,7.089)--(8.118,7.084)--(8.121,7.079)%
  --(8.125,7.075)--(8.128,7.071)--(8.131,7.068)--(8.135,7.065)--(8.138,7.063)--(8.142,7.061)%
  --(8.145,7.059)--(8.148,7.059)--(8.152,7.058)--(8.155,7.059)--(8.159,7.060)--(8.162,7.061)%
  --(8.166,7.063)--(8.169,7.066)--(8.172,7.069)--(8.176,7.073)--(8.179,7.077)--(8.183,7.081)%
  --(8.186,7.086)--(8.190,7.090)--(8.193,7.095)--(8.196,7.100)--(8.200,7.105)--(8.203,7.110)%
  --(8.207,7.115)--(8.210,7.120)--(8.213,7.124)--(8.217,7.128)--(8.220,7.132)--(8.224,7.135)%
  --(8.227,7.138)--(8.231,7.140)--(8.234,7.142)--(8.237,7.143)--(8.241,7.144)--(8.244,7.144)%
  --(8.248,7.144)--(8.251,7.142)--(8.254,7.141)--(8.258,7.139)--(8.261,7.136)--(8.265,7.133)%
  --(8.268,7.129)--(8.272,7.125)--(8.275,7.121)--(8.278,7.116)--(8.282,7.111)--(8.285,7.106)%
  --(8.289,7.101)--(8.292,7.096)--(8.295,7.091)--(8.299,7.087)--(8.302,7.082)--(8.306,7.078)%
  --(8.309,7.074)--(8.313,7.070)--(8.316,7.067)--(8.319,7.064)--(8.323,7.062)--(8.326,7.060)%
  --(8.330,7.059)--(8.333,7.059)--(8.336,7.059)--(8.340,7.060)--(8.343,7.061)--(8.347,7.063)%
  --(8.350,7.065)--(8.354,7.068)--(8.357,7.071)--(8.360,7.075)--(8.364,7.079)--(8.367,7.083)%
  --(8.371,7.088)--(8.374,7.093)--(8.378,7.098)--(8.381,7.103)--(8.384,7.108)--(8.388,7.113)%
  --(8.391,7.117)--(8.395,7.122)--(8.398,7.126)--(8.401,7.130)--(8.405,7.134)--(8.408,7.137)%
  --(8.412,7.139)--(8.415,7.141)--(8.419,7.143)--(8.422,7.144)--(8.425,7.144)--(8.429,7.144)%
  --(8.432,7.143)--(8.436,7.142)--(8.439,7.140)--(8.442,7.137)--(8.446,7.134)--(8.449,7.131)%
  --(8.453,7.127)--(8.456,7.123)--(8.460,7.118)--(8.463,7.114)--(8.466,7.109)--(8.470,7.104)%
  --(8.473,7.099)--(8.477,7.094)--(8.480,7.089)--(8.483,7.084)--(8.487,7.080)--(8.490,7.075)%
  --(8.494,7.072)--(8.497,7.068)--(8.501,7.065)--(8.504,7.063)--(8.507,7.061)--(8.511,7.059)%
  --(8.514,7.059)--(8.518,7.058)--(8.521,7.059)--(8.524,7.060)--(8.528,7.061)--(8.531,7.063)%
  --(8.535,7.066)--(8.538,7.069)--(8.542,7.073)--(8.545,7.077)--(8.548,7.081)--(8.552,7.085)%
  --(8.555,7.090)--(8.559,7.095)--(8.562,7.100)--(8.566,7.105)--(8.569,7.110)--(8.572,7.115)%
  --(8.576,7.120)--(8.579,7.124)--(8.583,7.129)--(8.586,7.132)--(8.589,7.136)--(8.593,7.139)%
  --(8.596,7.141)--(8.600,7.143)--(8.603,7.144)--(8.607,7.145)--(8.610,7.145)--(8.613,7.144)%
  --(8.617,7.143)--(8.620,7.142)--(8.624,7.139)--(8.627,7.137)--(8.630,7.133)--(8.634,7.130)%
  --(8.637,7.126)--(8.641,7.121)--(8.644,7.116)--(8.648,7.112)--(8.651,7.106)--(8.654,7.101)%
  --(8.658,7.096)--(8.661,7.091)--(8.665,7.086)--(8.668,7.081)--(8.671,7.077)--(8.675,7.073)%
  --(8.678,7.069)--(8.682,7.066)--(8.685,7.063)--(8.689,7.061)--(8.692,7.059)--(8.695,7.058)%
  --(8.699,7.057)--(8.702,7.057)--(8.706,7.058)--(8.709,7.059)--(8.712,7.061)--(8.716,7.063)%
  --(8.719,7.066)--(8.723,7.070)--(8.726,7.074)--(8.730,7.078)--(8.733,7.082)--(8.736,7.087)%
  --(8.740,7.092)--(8.743,7.098)--(8.747,7.103)--(8.750,7.108)--(8.754,7.113)--(8.757,7.118)%
  --(8.760,7.123)--(8.764,7.128)--(8.767,7.132)--(8.771,7.135)--(8.774,7.139)--(8.777,7.142)%
  --(8.781,7.144)--(8.784,7.145)--(8.788,7.146)--(8.791,7.147)--(8.795,7.147)--(8.798,7.146)%
  --(8.801,7.144)--(8.805,7.142)--(8.808,7.140)--(8.812,7.137)--(8.815,7.133)--(8.818,7.129)%
  --(8.822,7.124)--(8.825,7.120)--(8.829,7.115)--(8.832,7.109)--(8.836,7.104)--(8.839,7.099)%
  --(8.842,7.093)--(8.846,7.088)--(8.849,7.083)--(8.853,7.078)--(8.856,7.073)--(8.859,7.069)%
  --(8.863,7.065)--(8.866,7.062)--(8.870,7.059)--(8.873,7.057)--(8.877,7.056)--(8.880,7.055)%
  --(8.883,7.055)--(8.887,7.055)--(8.890,7.056)--(8.894,7.058)--(8.897,7.060)--(8.900,7.063)%
  --(8.904,7.066)--(8.907,7.070)--(8.911,7.074)--(8.914,7.079)--(8.918,7.084)--(8.921,7.089)%
  --(8.924,7.095)--(8.928,7.100)--(8.931,7.106)--(8.935,7.111)--(8.938,7.117)--(8.942,7.122)%
  --(8.945,7.127)--(8.948,7.132)--(8.952,7.136)--(8.955,7.140)--(8.959,7.143)--(8.962,7.145)%
  --(8.965,7.148)--(8.969,7.149)--(8.972,7.150)--(8.976,7.150)--(8.979,7.149)--(8.983,7.148)%
  --(8.986,7.146)--(8.989,7.144)--(8.993,7.141)--(8.996,7.137)--(9.000,7.133)--(9.003,7.129)%
  --(9.006,7.124)--(9.010,7.118)--(9.013,7.113)--(9.017,7.107)--(9.020,7.101)--(9.024,7.096)%
  --(9.027,7.090)--(9.030,7.084)--(9.034,7.079)--(9.037,7.074)--(9.041,7.069)--(9.044,7.065)%
  --(9.047,7.061)--(9.051,7.058)--(9.054,7.055)--(9.058,7.053)--(9.061,7.052)--(9.065,7.051)%
  --(9.068,7.051)--(9.071,7.052)--(9.075,7.053)--(9.078,7.055)--(9.082,7.058)--(9.085,7.061)%
  --(9.088,7.065)--(9.092,7.069)--(9.095,7.074)--(9.099,7.080)--(9.102,7.085)--(9.106,7.091)%
  --(9.109,7.097)--(9.112,7.103)--(9.116,7.109)--(9.119,7.115)--(9.123,7.121)--(9.126,7.127)%
  --(9.130,7.132)--(9.133,7.137)--(9.136,7.141)--(9.140,7.145)--(9.143,7.148)--(9.147,7.151)%
  --(9.150,7.153)--(9.153,7.154)--(9.157,7.155)--(9.160,7.154)--(9.164,7.154)--(9.167,7.152)%
  --(9.171,7.150)--(9.174,7.147)--(9.177,7.143)--(9.181,7.139)--(9.184,7.134)--(9.188,7.129)%
  --(9.191,7.123)--(9.194,7.117)--(9.198,7.111)--(9.201,7.104)--(9.205,7.098)--(9.208,7.092)%
  --(9.212,7.085)--(9.215,7.079)--(9.218,7.073)--(9.222,7.068)--(9.225,7.063)--(9.229,7.058)%
  --(9.232,7.054)--(9.235,7.051)--(9.239,7.049)--(9.242,7.047)--(9.246,7.045)--(9.249,7.045)%
  --(9.253,7.045)--(9.256,7.047)--(9.259,7.049)--(9.263,7.051)--(9.266,7.055)--(9.270,7.059)%
  --(9.273,7.063)--(9.276,7.068)--(9.280,7.074)--(9.283,7.080)--(9.287,7.087)--(9.290,7.093)%
  --(9.294,7.100)--(9.297,7.107)--(9.300,7.114)--(9.304,7.120)--(9.307,7.127)--(9.311,7.133)%
  --(9.314,7.139)--(9.317,7.144)--(9.321,7.148)--(9.324,7.152)--(9.328,7.156)--(9.331,7.158)%
  --(9.335,7.160)--(9.338,7.161)--(9.341,7.162)--(9.345,7.161)--(9.348,7.160)--(9.352,7.157)%
  --(9.355,7.154)--(9.359,7.151)--(9.362,7.146)--(9.365,7.141)--(9.369,7.135)--(9.372,7.129)%
  --(9.376,7.122)--(9.379,7.116)--(9.382,7.108)--(9.386,7.101)--(9.389,7.094)--(9.393,7.086)%
  --(9.396,7.079)--(9.400,7.072)--(9.403,7.066)--(9.406,7.060)--(9.410,7.054)--(9.413,7.050)%
  --(9.417,7.045)--(9.420,7.042)--(9.423,7.039)--(9.427,7.037)--(9.430,7.037)--(9.434,7.036)%
  --(9.437,7.037)--(9.441,7.039)--(9.444,7.042)--(9.447,7.045)--(9.451,7.049)--(9.454,7.054)%
  --(9.458,7.060)--(9.461,7.066)--(9.464,7.073)--(9.468,7.081)--(9.471,7.088)--(9.475,7.096)%
  --(9.478,7.104)--(9.482,7.112)--(9.485,7.120)--(9.488,7.127)--(9.492,7.135)--(9.495,7.142)%
  --(9.499,7.148)--(9.502,7.154)--(9.505,7.159)--(9.509,7.164)--(9.512,7.167)--(9.516,7.170)%
  --(9.519,7.172)--(9.523,7.172)--(9.526,7.172)--(9.529,7.171)--(9.533,7.169)--(9.536,7.166)%
  --(9.540,7.162)--(9.543,7.157)--(9.547,7.151)--(9.550,7.145)--(9.553,7.138)--(9.557,7.130)%
  --(9.560,7.122)--(9.564,7.114)--(9.567,7.105)--(9.570,7.096)--(9.574,7.088)--(9.577,7.079)%
  --(9.581,7.071)--(9.584,7.062)--(9.588,7.055)--(9.591,7.048)--(9.594,7.042)--(9.598,7.036)%
  --(9.601,7.032)--(9.605,7.028)--(9.608,7.025)--(9.611,7.023)--(9.615,7.023)--(9.618,7.023)%
  --(9.622,7.025)--(9.625,7.027)--(9.629,7.031)--(9.632,7.036)--(9.635,7.041)--(9.639,7.048)%
  --(9.642,7.055)--(9.646,7.063)--(9.649,7.072)--(9.652,7.081)--(9.656,7.090)--(9.659,7.100)%
  --(9.663,7.110)--(9.666,7.120)--(9.670,7.129)--(9.673,7.139)--(9.676,7.147)--(9.680,7.156)%
  --(9.683,7.163)--(9.687,7.170)--(9.690,7.176)--(9.693,7.181)--(9.697,7.185)--(9.700,7.188)%
  --(9.704,7.190)--(9.707,7.190)--(9.711,7.189)--(9.714,7.187)--(9.717,7.184)--(9.721,7.180)%
  --(9.724,7.174)--(9.728,7.167)--(9.731,7.160)--(9.735,7.151)--(9.738,7.142)--(9.741,7.132)%
  --(9.745,7.122)--(9.748,7.111)--(9.752,7.100)--(9.755,7.088)--(9.758,7.077)--(9.762,7.066)%
  --(9.765,7.056)--(9.769,7.046)--(9.772,7.036)--(9.776,7.028)--(9.779,7.020)--(9.782,7.014)%
  --(9.786,7.008)--(9.789,7.004)--(9.793,7.001)--(9.796,7.000)--(9.799,6.999)--(9.803,7.001)%
  --(9.806,7.003)--(9.810,7.007)--(9.813,7.013)--(9.817,7.020)--(9.820,7.028)--(9.823,7.037)%
  --(9.827,7.047)--(9.830,7.058)--(9.834,7.069)--(9.837,7.082)--(9.840,7.095)--(9.844,7.108)%
  --(9.847,7.121)--(9.851,7.134)--(9.854,7.146)--(9.858,7.159)--(9.861,7.170)--(9.864,7.181)%
  --(9.868,7.191)--(9.871,7.200)--(9.875,7.207)--(9.878,7.213)--(9.881,7.218)--(9.885,7.221)%
  --(9.888,7.222)--(9.892,7.222)--(9.895,7.220)--(9.899,7.217)--(9.902,7.211)--(9.905,7.205)%
  --(9.909,7.196)--(9.912,7.186)--(9.916,7.175)--(9.919,7.163)--(9.923,7.149)--(9.926,7.135)%
  --(9.929,7.120)--(9.933,7.104)--(9.936,7.089)--(9.940,7.073)--(9.943,7.057)--(9.946,7.042)%
  --(9.950,7.027)--(9.953,7.013)--(9.957,7.000)--(9.960,6.988)--(9.964,6.978)--(9.967,6.969)%
  --(9.970,6.962)--(9.974,6.956)--(9.977,6.953)--(9.981,6.952)--(9.984,6.953)--(9.987,6.955)%
  --(9.991,6.960)--(9.994,6.967)--(9.998,6.976)--(10.001,6.987)--(10.005,7.000)--(10.008,7.015)%
  --(10.011,7.031)--(10.015,7.048)--(10.018,7.066)--(10.022,7.086)--(10.025,7.106)--(10.028,7.126)%
  --(10.032,7.146)--(10.035,7.166)--(10.039,7.186)--(10.042,7.205)--(10.046,7.222)--(10.049,7.239)%
  --(10.052,7.254)--(10.056,7.267)--(10.059,7.278)--(10.063,7.287)--(10.066,7.293)--(10.069,7.297)%
  --(10.073,7.298)--(10.076,7.296)--(10.080,7.292)--(10.083,7.285)--(10.087,7.275)--(10.090,7.262)%
  --(10.093,7.246)--(10.097,7.228)--(10.100,7.208)--(10.104,7.186)--(10.107,7.162)--(10.111,7.136)%
  --(10.114,7.109)--(10.117,7.082)--(10.121,7.053)--(10.124,7.025)--(10.128,6.997)--(10.131,6.969)%
  --(10.134,6.943)--(10.138,6.918)--(10.141,6.895)--(10.145,6.874)--(10.148,6.856)--(10.152,6.840)%
  --(10.155,6.828)--(10.158,6.820)--(10.162,6.815)--(10.165,6.814)--(10.169,6.818)--(10.172,6.825)%
  --(10.175,6.837)--(10.179,6.854)--(10.182,6.875)--(10.186,6.900)--(10.189,6.928)--(10.193,6.961)%
  --(10.196,6.997)--(10.199,7.037)--(10.203,7.079)--(10.206,7.123)--(10.210,7.169)--(10.213,7.216)%
  --(10.216,7.264)--(10.220,7.312)--(10.223,7.359)--(10.227,7.405)--(10.230,7.448)--(10.234,7.489)%
  --(10.237,7.526)--(10.240,7.558)--(10.244,7.585)--(10.247,7.607)--(10.251,7.621)--(10.254,7.629)%
  --(10.257,7.628)--(10.261,7.619)--(10.264,7.601)--(10.268,7.573)--(10.271,7.535)--(10.275,7.487)%
  --(10.278,7.427)--(10.281,7.357)--(10.285,7.276)--(10.288,7.183)--(10.292,7.079)--(10.295,6.964)%
  --(10.299,6.838)--(10.302,6.701)--(10.305,6.554)--(10.309,6.396)--(10.312,6.230)--(10.316,6.054)%
  --(10.319,5.870)--(10.322,5.679)--(10.326,5.481)--(10.329,5.276)--(10.333,5.067)--(10.336,4.854)%
  --(10.340,4.637)--(10.343,4.419)--(10.346,4.199)--(10.350,3.978)--(10.353,3.759)--(10.357,3.541)%
  --(10.360,3.327)--(10.363,3.116)--(10.367,2.909)--(10.370,2.709)--(10.374,2.514)--(10.377,2.327)%
  --(10.381,2.148)--(10.384,1.978)--(10.387,1.816)--(10.391,1.665)--(10.394,1.524)--(10.398,1.393)%
  --(10.401,1.274)--(10.404,1.165)--(10.408,1.068)--(10.411,0.982)--(10.415,0.907)--(10.418,0.843)%
  --(10.422,0.790)--(10.425,0.748)--(10.428,0.716)--(10.432,0.694)--(10.435,0.681)--(10.439,0.677)%
  --(10.442,0.682)--(10.445,0.694)--(10.449,0.712)--(10.452,0.737)--(10.456,0.768)--(10.459,0.803)%
  --(10.463,0.842)--(10.466,0.884)--(10.469,0.929)--(10.473,0.976)--(10.476,1.023)--(10.480,1.071)%
  --(10.483,1.119)--(10.487,1.165)--(10.490,1.210)--(10.493,1.253)--(10.497,1.294)--(10.500,1.331)%
  --(10.504,1.366)--(10.507,1.396)--(10.510,1.423)--(10.514,1.445)--(10.517,1.463)--(10.521,1.477)%
  --(10.524,1.487)--(10.528,1.492)--(10.531,1.493)--(10.534,1.490)--(10.538,1.483)--(10.541,1.472)%
  --(10.545,1.458)--(10.548,1.441)--(10.551,1.421)--(10.555,1.399)--(10.558,1.375)--(10.562,1.349)%
  --(10.565,1.322)--(10.569,1.294)--(10.572,1.265)--(10.575,1.237)--(10.579,1.209)--(10.582,1.182)%
  --(10.586,1.155)--(10.589,1.131)--(10.592,1.108)--(10.596,1.087)--(10.599,1.068)--(10.603,1.051)%
  --(10.606,1.037)--(10.610,1.026)--(10.613,1.018)--(10.616,1.012)--(10.620,1.009)--(10.623,1.009)%
  --(10.627,1.012)--(10.630,1.017)--(10.633,1.025)--(10.637,1.035)--(10.640,1.048)--(10.644,1.062)%
  --(10.647,1.078)--(10.651,1.095)--(10.654,1.113)--(10.657,1.133)--(10.661,1.153)--(10.664,1.173)%
  --(10.668,1.193)--(10.671,1.213)--(10.675,1.233)--(10.678,1.252)--(10.681,1.269)--(10.685,1.286)%
  --(10.688,1.301)--(10.692,1.315)--(10.695,1.326)--(10.698,1.336)--(10.702,1.344)--(10.705,1.350)%
  --(10.709,1.354)--(10.712,1.355)--(10.716,1.355)--(10.719,1.352)--(10.722,1.348)--(10.726,1.341)%
  --(10.729,1.333)--(10.733,1.323)--(10.736,1.312)--(10.739,1.300)--(10.743,1.286)--(10.746,1.272)%
  --(10.750,1.256)--(10.753,1.241)--(10.757,1.225)--(10.760,1.209)--(10.763,1.193)--(10.767,1.178)%
  --(10.770,1.163)--(10.774,1.150)--(10.777,1.137)--(10.780,1.125)--(10.784,1.115)--(10.787,1.106)%
  --(10.791,1.098)--(10.794,1.092)--(10.798,1.088)--(10.801,1.085)--(10.804,1.084)--(10.808,1.085)%
  --(10.811,1.088)--(10.815,1.092)--(10.818,1.097)--(10.821,1.104)--(10.825,1.112)--(10.828,1.122)%
  --(10.832,1.132)--(10.835,1.144)--(10.839,1.156)--(10.842,1.168)--(10.845,1.181)--(10.849,1.194)%
  --(10.852,1.207)--(10.856,1.220)--(10.859,1.233)--(10.862,1.245)--(10.866,1.256)--(10.869,1.266)%
  --(10.873,1.276)--(10.876,1.284)--(10.880,1.292)--(10.883,1.298)--(10.886,1.302)--(10.890,1.305)%
  --(10.893,1.307)--(10.897,1.308)--(10.900,1.307)--(10.904,1.304)--(10.907,1.301)--(10.910,1.296)%
  --(10.914,1.290)--(10.917,1.282)--(10.921,1.274)--(10.924,1.265)--(10.927,1.255)--(10.931,1.245)%
  --(10.934,1.234)--(10.938,1.223)--(10.941,1.212)--(10.945,1.201)--(10.948,1.190)--(10.951,1.179)%
  --(10.955,1.169)--(10.958,1.159)--(10.962,1.151)--(10.965,1.143)--(10.968,1.136)--(10.972,1.130)%
  --(10.975,1.125)--(10.979,1.121)--(10.982,1.118)--(10.986,1.117)--(10.989,1.117)--(10.992,1.118)%
  --(10.996,1.121)--(10.999,1.124)--(11.003,1.129)--(11.006,1.134)--(11.009,1.141)--(11.013,1.148)%
  --(11.016,1.156)--(11.020,1.165)--(11.023,1.174)--(11.027,1.183)--(11.030,1.193)--(11.033,1.203)%
  --(11.037,1.213)--(11.040,1.222)--(11.044,1.232)--(11.047,1.240)--(11.050,1.249)--(11.054,1.256)%
  --(11.057,1.263)--(11.061,1.269)--(11.064,1.274)--(11.068,1.278)--(11.071,1.281)--(11.074,1.283)%
  --(11.078,1.284)--(11.081,1.284)--(11.085,1.283)--(11.088,1.280)--(11.092,1.277)--(11.095,1.273)%
  --(11.098,1.268)--(11.102,1.262)--(11.105,1.255)--(11.109,1.248)--(11.112,1.240)--(11.115,1.232)%
  --(11.119,1.223)--(11.122,1.214)--(11.126,1.206)--(11.129,1.197)--(11.133,1.188)--(11.136,1.180)%
  --(11.139,1.172)--(11.143,1.165)--(11.146,1.158)--(11.150,1.152)--(11.153,1.147)--(11.156,1.143)%
  --(11.160,1.139)--(11.163,1.137)--(11.167,1.135)--(11.170,1.135)--(11.174,1.135)--(11.177,1.136)%
  --(11.180,1.139)--(11.184,1.142)--(11.187,1.146)--(11.191,1.151)--(11.194,1.156)--(11.197,1.163)%
  --(11.201,1.169)--(11.204,1.177)--(11.208,1.184)--(11.211,1.192)--(11.215,1.200)--(11.218,1.208)%
  --(11.221,1.216)--(11.225,1.223)--(11.228,1.231)--(11.232,1.238)--(11.235,1.244)--(11.238,1.250)%
  --(11.242,1.256)--(11.245,1.260)--(11.249,1.264)--(11.252,1.267)--(11.256,1.269)--(11.259,1.270)%
  --(11.262,1.271)--(11.266,1.270)--(11.269,1.269)--(11.273,1.266)--(11.276,1.263)--(11.280,1.259)%
  --(11.283,1.255)--(11.286,1.249)--(11.290,1.244)--(11.293,1.237)--(11.297,1.230)--(11.300,1.223)%
  --(11.303,1.216)--(11.307,1.209)--(11.310,1.202)--(11.314,1.194)--(11.317,1.187)--(11.321,1.181)%
  --(11.324,1.174)--(11.327,1.168)--(11.331,1.163)--(11.334,1.158)--(11.338,1.154)--(11.341,1.151)%
  --(11.344,1.148)--(11.348,1.146)--(11.351,1.146)--(11.355,1.145)--(11.358,1.146)--(11.362,1.148)%
  --(11.365,1.150)--(11.368,1.153)--(11.372,1.157)--(11.375,1.161)--(11.379,1.166)--(11.382,1.172)%
  --(11.385,1.178)--(11.389,1.184)--(11.392,1.191)--(11.396,1.197)--(11.399,1.204)--(11.403,1.211)%
  --(11.406,1.218)--(11.409,1.224)--(11.413,1.231)--(11.416,1.236)--(11.420,1.242)--(11.423,1.247)%
  --(11.426,1.251)--(11.430,1.255)--(11.433,1.257)--(11.437,1.260)--(11.440,1.261)--(11.444,1.262)%
  --(11.447,1.262);
  \gpcolor{rgb color={0.898,0.118,0.063}}
  \draw[gp path] (1.196,7.083)--(1.199,7.086)--(1.203,7.091)--(1.206,7.097)--(1.210,7.104)%
  --(1.213,7.110)--(1.217,7.116)--(1.220,7.120)--(1.223,7.122)--(1.227,7.121)--(1.230,7.119)%
  --(1.234,7.115)--(1.237,7.109)--(1.240,7.102)--(1.244,7.095)--(1.247,7.089)--(1.251,7.084)%
  --(1.254,7.081)--(1.258,7.081)--(1.261,7.082)--(1.264,7.086)--(1.268,7.091)--(1.271,7.097)%
  --(1.275,7.104)--(1.278,7.111)--(1.281,7.117)--(1.285,7.121)--(1.288,7.123)--(1.292,7.122)%
  --(1.295,7.120)--(1.299,7.115)--(1.302,7.109)--(1.305,7.102)--(1.309,7.095)--(1.312,7.089)%
  --(1.316,7.083)--(1.319,7.080)--(1.322,7.079)--(1.326,7.081)--(1.329,7.085)--(1.333,7.090)%
  --(1.336,7.097)--(1.340,7.104)--(1.343,7.111)--(1.346,7.117)--(1.350,7.122)--(1.353,7.124)%
  --(1.357,7.123)--(1.360,7.121)--(1.363,7.116)--(1.367,7.109)--(1.370,7.102)--(1.374,7.095)%
  --(1.377,7.088)--(1.381,7.082)--(1.384,7.079)--(1.387,7.078)--(1.391,7.080)--(1.394,7.084)%
  --(1.398,7.090)--(1.401,7.097)--(1.405,7.105)--(1.408,7.112)--(1.411,7.118)--(1.415,7.123)%
  --(1.418,7.125)--(1.422,7.125)--(1.425,7.122)--(1.428,7.117)--(1.432,7.110)--(1.435,7.102)%
  --(1.439,7.094)--(1.442,7.087)--(1.446,7.081)--(1.449,7.078)--(1.452,7.077)--(1.456,7.078)%
  --(1.459,7.083)--(1.463,7.089)--(1.466,7.097)--(1.469,7.105)--(1.473,7.113)--(1.476,7.120)%
  --(1.480,7.124)--(1.483,7.127)--(1.487,7.126)--(1.490,7.123)--(1.493,7.118)--(1.497,7.110)%
  --(1.500,7.102)--(1.504,7.094)--(1.507,7.086)--(1.510,7.080)--(1.514,7.076)--(1.517,7.075)%
  --(1.521,7.077)--(1.524,7.081)--(1.528,7.088)--(1.531,7.096)--(1.534,7.105)--(1.538,7.114)%
  --(1.541,7.121)--(1.545,7.126)--(1.548,7.129)--(1.551,7.128)--(1.555,7.125)--(1.558,7.119)%
  --(1.562,7.111)--(1.565,7.102)--(1.569,7.093)--(1.572,7.085)--(1.575,7.078)--(1.579,7.074)%
  --(1.582,7.073)--(1.586,7.075)--(1.589,7.080)--(1.593,7.087)--(1.596,7.096)--(1.599,7.106)%
  --(1.603,7.115)--(1.606,7.123)--(1.610,7.128)--(1.613,7.131)--(1.616,7.131)--(1.620,7.127)%
  --(1.623,7.120)--(1.627,7.112)--(1.630,7.102)--(1.634,7.092)--(1.637,7.083)--(1.640,7.076)%
  --(1.644,7.071)--(1.647,7.070)--(1.651,7.073)--(1.654,7.078)--(1.657,7.086)--(1.661,7.096)%
  --(1.664,7.106)--(1.668,7.116)--(1.671,7.125)--(1.675,7.131)--(1.678,7.134)--(1.681,7.133)%
  --(1.685,7.129)--(1.688,7.122)--(1.692,7.113)--(1.695,7.102)--(1.698,7.091)--(1.702,7.081)%
  --(1.705,7.073)--(1.709,7.068)--(1.712,7.067)--(1.716,7.070)--(1.719,7.076)--(1.722,7.084)%
  --(1.726,7.095)--(1.729,7.107)--(1.733,7.118)--(1.736,7.127)--(1.739,7.134)--(1.743,7.137)%
  --(1.746,7.137)--(1.750,7.132)--(1.753,7.124)--(1.757,7.114)--(1.760,7.102)--(1.763,7.090)%
  --(1.767,7.079)--(1.770,7.070)--(1.774,7.065)--(1.777,7.063)--(1.781,7.066)--(1.784,7.073)%
  --(1.787,7.083)--(1.791,7.095)--(1.794,7.108)--(1.798,7.120)--(1.801,7.131)--(1.804,7.138)%
  --(1.808,7.142)--(1.811,7.141)--(1.815,7.136)--(1.818,7.127)--(1.822,7.115)--(1.825,7.102)%
  --(1.828,7.088)--(1.832,7.075)--(1.835,7.066)--(1.839,7.060)--(1.842,7.058)--(1.845,7.061)%
  --(1.849,7.069)--(1.852,7.080)--(1.856,7.094)--(1.859,7.109)--(1.863,7.123)--(1.866,7.135)%
  --(1.869,7.144)--(1.873,7.148)--(1.876,7.147)--(1.880,7.141)--(1.883,7.131)--(1.886,7.117)%
  --(1.890,7.101)--(1.893,7.086)--(1.897,7.071)--(1.900,7.060)--(1.904,7.053)--(1.907,7.051)%
  --(1.910,7.055)--(1.914,7.064)--(1.917,7.077)--(1.921,7.093)--(1.924,7.110)--(1.927,7.127)%
  --(1.931,7.141)--(1.934,7.151)--(1.938,7.156)--(1.941,7.155)--(1.945,7.148)--(1.948,7.136)%
  --(1.951,7.120)--(1.955,7.101)--(1.958,7.082)--(1.962,7.065)--(1.965,7.052)--(1.968,7.043)%
  --(1.972,7.041)--(1.975,7.046)--(1.979,7.056)--(1.982,7.072)--(1.986,7.092)--(1.989,7.113)%
  --(1.992,7.133)--(1.996,7.150)--(1.999,7.162)--(2.003,7.168)--(2.006,7.166)--(2.010,7.158)%
  --(2.013,7.143)--(2.016,7.123)--(2.020,7.100)--(2.023,7.077)--(2.027,7.056)--(2.030,7.039)%
  --(2.033,7.029)--(2.037,7.027)--(2.040,7.032)--(2.044,7.046)--(2.047,7.066)--(2.051,7.090)%
  --(2.054,7.116)--(2.057,7.142)--(2.061,7.163)--(2.064,7.179)--(2.068,7.186)--(2.071,7.185)%
  --(2.074,7.174)--(2.078,7.154)--(2.081,7.128)--(2.085,7.099)--(2.088,7.069)--(2.092,7.041)%
  --(2.095,7.019)--(2.098,7.006)--(2.102,7.002)--(2.105,7.010)--(2.109,7.028)--(2.112,7.055)%
  --(2.115,7.088)--(2.119,7.123)--(2.122,7.158)--(2.126,7.188)--(2.129,7.209)--(2.133,7.220)%
  --(2.136,7.217)--(2.139,7.202)--(2.143,7.174)--(2.146,7.137)--(2.150,7.095)--(2.153,7.051)%
  --(2.156,7.011)--(2.160,6.979)--(2.163,6.959)--(2.167,6.954)--(2.170,6.965)--(2.174,6.993)%
  --(2.177,7.034)--(2.180,7.085)--(2.184,7.142)--(2.187,7.197)--(2.191,7.244)--(2.194,7.279)%
  --(2.198,7.296)--(2.201,7.291)--(2.204,7.265)--(2.208,7.218)--(2.211,7.153)--(2.215,7.077)%
  --(2.218,6.998)--(2.221,6.924)--(2.225,6.863)--(2.228,6.825)--(2.232,6.815)--(2.235,6.839)%
  --(2.239,6.897)--(2.242,6.986)--(2.245,7.101)--(2.249,7.232)--(2.252,7.366)--(2.256,7.487)%
  --(2.259,7.579)--(2.262,7.626)--(2.266,7.612)--(2.269,7.524)--(2.273,7.353)--(2.276,7.093)%
  --(2.280,6.744)--(2.283,6.313)--(2.286,5.808)--(2.290,5.246)--(2.293,4.645)--(2.297,4.027)%
  --(2.300,3.412)--(2.303,2.825)--(2.307,2.284)--(2.310,1.808)--(2.314,1.410)--(2.317,1.097)%
  --(2.321,0.874)--(2.324,0.737)--(2.327,0.681)--(2.331,0.693)--(2.334,0.761)--(2.338,0.867)%
  --(2.341,0.995)--(2.344,1.130)--(2.348,1.255)--(2.351,1.361)--(2.355,1.438)--(2.358,1.482)%
  --(2.362,1.491)--(2.365,1.470)--(2.368,1.421)--(2.372,1.354)--(2.375,1.277)--(2.379,1.198)%
  --(2.382,1.126)--(2.386,1.068)--(2.389,1.029)--(2.392,1.011)--(2.396,1.016)--(2.399,1.040)%
  --(2.403,1.081)--(2.406,1.132)--(2.409,1.189)--(2.413,1.244)--(2.416,1.291)--(2.420,1.327)%
  --(2.423,1.348)--(2.427,1.353)--(2.430,1.342)--(2.433,1.316)--(2.437,1.281)--(2.440,1.238)%
  --(2.444,1.194)--(2.447,1.154)--(2.450,1.120)--(2.454,1.098)--(2.457,1.087)--(2.461,1.090)%
  --(2.464,1.105)--(2.468,1.130)--(2.471,1.163)--(2.474,1.198)--(2.478,1.233)--(2.481,1.264)%
  --(2.485,1.288)--(2.488,1.302)--(2.491,1.305)--(2.495,1.297)--(2.498,1.280)--(2.502,1.255)%
  --(2.505,1.226)--(2.509,1.196)--(2.512,1.167)--(2.515,1.144)--(2.519,1.128)--(2.522,1.121)%
  --(2.526,1.123)--(2.529,1.133)--(2.532,1.152)--(2.536,1.175)--(2.539,1.202)--(2.543,1.227)%
  --(2.546,1.250)--(2.550,1.268)--(2.553,1.278)--(2.556,1.280)--(2.560,1.275)--(2.563,1.261)%
  --(2.567,1.243)--(2.570,1.220)--(2.574,1.197)--(2.577,1.175)--(2.580,1.157)--(2.584,1.145)%
  --(2.587,1.139)--(2.591,1.141)--(2.594,1.149)--(2.597,1.164)--(2.601,1.182)--(2.604,1.203)%
  --(2.608,1.224)--(2.611,1.242)--(2.615,1.256)--(2.618,1.264)--(2.621,1.266)--(2.625,1.261)%
  --(2.628,1.250)--(2.632,1.235)--(2.635,1.217)--(2.638,1.198)--(2.642,1.180)--(2.645,1.166)%
  --(2.649,1.155)--(2.652,1.151)--(2.656,1.152)--(2.659,1.159)--(2.662,1.171)--(2.666,1.187)%
  --(2.669,1.204)--(2.673,1.221)--(2.676,1.236)--(2.679,1.248)--(2.683,1.254)--(2.686,1.256)%
  --(2.690,1.252)--(2.693,1.243)--(2.697,1.230)--(2.700,1.215)--(2.703,1.199)--(2.707,1.184)%
  --(2.710,1.172)--(2.714,1.163)--(2.717,1.159)--(2.720,1.160)--(2.724,1.166)--(2.727,1.176)%
  --(2.731,1.190)--(2.734,1.204)--(2.738,1.219)--(2.741,1.232)--(2.744,1.242)--(2.748,1.248)%
  --(2.751,1.249)--(2.755,1.246)--(2.758,1.238)--(2.762,1.227)--(2.765,1.214)--(2.768,1.200)%
  --(2.772,1.187)--(2.775,1.176)--(2.779,1.168)--(2.782,1.165)--(2.785,1.166)--(2.789,1.171)%
  --(2.792,1.180)--(2.796,1.192)--(2.799,1.205)--(2.803,1.218)--(2.806,1.229)--(2.809,1.238)%
  --(2.813,1.243)--(2.816,1.244)--(2.820,1.241)--(2.823,1.234)--(2.826,1.224)--(2.830,1.212)%
  --(2.833,1.200)--(2.837,1.189)--(2.840,1.179)--(2.844,1.172)--(2.847,1.169)--(2.850,1.170)%
  --(2.854,1.175)--(2.857,1.183)--(2.861,1.193)--(2.864,1.205)--(2.867,1.216)--(2.871,1.227)%
  --(2.874,1.234)--(2.878,1.239)--(2.881,1.240)--(2.885,1.237)--(2.888,1.231)--(2.891,1.222)%
  --(2.895,1.212)--(2.898,1.201)--(2.902,1.190)--(2.905,1.182)--(2.908,1.176)--(2.912,1.173)%
  --(2.915,1.174)--(2.919,1.178)--(2.922,1.185)--(2.926,1.195)--(2.929,1.205)--(2.932,1.215)%
  --(2.936,1.225)--(2.939,1.232)--(2.943,1.236)--(2.946,1.237)--(2.950,1.234)--(2.953,1.229)%
  --(2.956,1.221)--(2.960,1.211)--(2.963,1.201)--(2.967,1.192)--(2.970,1.184)--(2.973,1.178)%
  --(2.977,1.176)--(2.980,1.177)--(2.984,1.181)--(2.987,1.187)--(2.991,1.196)--(2.994,1.205)%
  --(2.997,1.215)--(3.001,1.223)--(3.004,1.230)--(3.008,1.233)--(3.011,1.234)--(3.014,1.232)%
  --(3.018,1.227)--(3.021,1.219)--(3.025,1.210)--(3.028,1.201)--(3.032,1.193)--(3.035,1.185)%
  --(3.038,1.180)--(3.042,1.178)--(3.045,1.179)--(3.049,1.183)--(3.052,1.189)--(3.055,1.197)%
  --(3.059,1.205)--(3.062,1.214)--(3.066,1.222)--(3.069,1.228)--(3.073,1.231)--(3.076,1.232)%
  --(3.079,1.230)--(3.083,1.225)--(3.086,1.218)--(3.090,1.210)--(3.093,1.201)--(3.096,1.193)%
  --(3.100,1.187)--(3.103,1.182)--(3.107,1.180)--(3.110,1.181)--(3.114,1.184)--(3.117,1.190)%
  --(3.120,1.197)--(3.124,1.205)--(3.127,1.214)--(3.131,1.221)--(3.134,1.226)--(3.138,1.230)%
  --(3.141,1.230)--(3.144,1.228)--(3.148,1.224)--(3.151,1.217)--(3.155,1.210)--(3.158,1.202)%
  --(3.161,1.194)--(3.165,1.188)--(3.168,1.184)--(3.172,1.182)--(3.175,1.182)--(3.179,1.186)%
  --(3.182,1.191)--(3.185,1.198)--(3.189,1.206)--(3.192,1.213)--(3.196,1.220)--(3.199,1.225)%
  --(3.202,1.228)--(3.206,1.229)--(3.209,1.227)--(3.213,1.223)--(3.216,1.217)--(3.220,1.209)%
  --(3.223,1.202)--(3.226,1.195)--(3.230,1.189)--(3.233,1.185)--(3.237,1.183)--(3.240,1.184)%
  --(3.243,1.187)--(3.247,1.192)--(3.250,1.198)--(3.254,1.206)--(3.257,1.213)--(3.261,1.219)%
  --(3.264,1.224)--(3.267,1.227)--(3.271,1.227)--(3.274,1.226)--(3.278,1.222)--(3.281,1.216)%
  --(3.284,1.209)--(3.288,1.202)--(3.291,1.195)--(3.295,1.190)--(3.298,1.186)--(3.302,1.184)%
  --(3.305,1.185)--(3.308,1.188)--(3.312,1.193)--(3.315,1.199)--(3.319,1.206)--(3.322,1.212)%
  --(3.326,1.219)--(3.329,1.223)--(3.332,1.226)--(3.336,1.226)--(3.339,1.225)--(3.343,1.221)%
  --(3.346,1.215)--(3.349,1.209)--(3.353,1.202)--(3.356,1.196)--(3.360,1.190)--(3.363,1.187)%
  --(3.367,1.185)--(3.370,1.186)--(3.373,1.189)--(3.377,1.193)--(3.380,1.199)--(3.384,1.206)%
  --(3.387,1.212)--(3.390,1.218)--(3.394,1.222)--(3.397,1.225)--(3.401,1.225)--(3.404,1.224)%
  --(3.408,1.220)--(3.411,1.215)--(3.414,1.209)--(3.418,1.202)--(3.421,1.196)--(3.425,1.191)%
  --(3.428,1.188)--(3.431,1.186)--(3.435,1.187)--(3.438,1.189)--(3.442,1.194)--(3.445,1.199)%
  --(3.449,1.206)--(3.452,1.212)--(3.455,1.218)--(3.459,1.222)--(3.462,1.224)--(3.466,1.225)%
  --(3.469,1.223)--(3.472,1.220)--(3.476,1.215)--(3.479,1.209)--(3.483,1.202)--(3.486,1.196)%
  --(3.490,1.192)--(3.493,1.188)--(3.496,1.187)--(3.500,1.187)--(3.503,1.190)--(3.507,1.194)%
  --(3.510,1.200)--(3.513,1.206)--(3.517,1.212)--(3.520,1.217)--(3.524,1.221)--(3.527,1.224)%
  --(3.531,1.224)--(3.534,1.222)--(3.537,1.219)--(3.541,1.214)--(3.544,1.208)--(3.548,1.202)%
  --(3.551,1.197)--(3.555,1.192)--(3.558,1.189)--(3.561,1.188)--(3.565,1.188)--(3.568,1.191)%
  --(3.572,1.195)--(3.575,1.200)--(3.578,1.206)--(3.582,1.212)--(3.585,1.217)--(3.589,1.221)%
  --(3.592,1.223)--(3.596,1.223)--(3.599,1.222)--(3.602,1.219)--(3.606,1.214)--(3.609,1.208)%
  --(3.613,1.202)--(3.616,1.197)--(3.619,1.192)--(3.623,1.189)--(3.626,1.188)--(3.630,1.189)%
  --(3.633,1.191)--(3.637,1.195)--(3.640,1.200)--(3.643,1.206)--(3.647,1.211)--(3.650,1.216)%
  --(3.654,1.220)--(3.657,1.222)--(3.660,1.223)--(3.664,1.221)--(3.667,1.218)--(3.671,1.214)%
  --(3.674,1.208)--(3.678,1.203)--(3.681,1.197)--(3.684,1.193)--(3.688,1.190)--(3.691,1.189)%
  --(3.695,1.189)--(3.698,1.191)--(3.701,1.195)--(3.705,1.200)--(3.708,1.206)--(3.712,1.211)%
  --(3.715,1.216)--(3.719,1.220)--(3.722,1.222)--(3.725,1.222)--(3.729,1.221)--(3.732,1.218)%
  --(3.736,1.213)--(3.739,1.208)--(3.743,1.203)--(3.746,1.197)--(3.749,1.193)--(3.753,1.190)%
  --(3.756,1.189)--(3.760,1.190)--(3.763,1.192)--(3.766,1.196)--(3.770,1.200)--(3.773,1.206)%
  --(3.777,1.211)--(3.780,1.216)--(3.784,1.220)--(3.787,1.222)--(3.790,1.222)--(3.794,1.221)%
  --(3.797,1.217)--(3.801,1.213)--(3.804,1.208)--(3.807,1.203)--(3.811,1.198)--(3.814,1.193)%
  --(3.818,1.191)--(3.821,1.189)--(3.825,1.190)--(3.828,1.192)--(3.831,1.196)--(3.835,1.201)%
  --(3.838,1.206)--(3.842,1.211)--(3.845,1.216)--(3.848,1.219)--(3.852,1.221)--(3.855,1.222)%
  --(3.859,1.220)--(3.862,1.217)--(3.866,1.213)--(3.869,1.208)--(3.872,1.203)--(3.876,1.198)%
  --(3.879,1.194)--(3.883,1.191)--(3.886,1.190)--(3.889,1.190)--(3.893,1.192)--(3.896,1.196)%
  --(3.900,1.201)--(3.903,1.206)--(3.907,1.211)--(3.910,1.216)--(3.913,1.219)--(3.917,1.221)%
  --(3.920,1.221)--(3.924,1.220)--(3.927,1.217)--(3.931,1.213)--(3.934,1.208)--(3.937,1.203)%
  --(3.941,1.198)--(3.944,1.194)--(3.948,1.191)--(3.951,1.190)--(3.954,1.190)--(3.958,1.193)%
  --(3.961,1.196)--(3.965,1.201)--(3.968,1.206)--(3.972,1.211)--(3.975,1.215)--(3.978,1.219)%
  --(3.982,1.221)--(3.985,1.221)--(3.989,1.220)--(3.992,1.217)--(3.995,1.213)--(3.999,1.208)%
  --(4.002,1.203)--(4.006,1.198)--(4.009,1.194)--(4.013,1.191)--(4.016,1.190)--(4.019,1.191)%
  --(4.023,1.193)--(4.026,1.196)--(4.030,1.201)--(4.033,1.206)--(4.036,1.211)--(4.040,1.215)%
  --(4.043,1.219)--(4.047,1.221)--(4.050,1.221)--(4.054,1.220)--(4.057,1.217)--(4.060,1.213)%
  --(4.064,1.208)--(4.067,1.203)--(4.071,1.198)--(4.074,1.194)--(4.077,1.191)--(4.081,1.190)%
  --(4.084,1.191)--(4.088,1.193)--(4.091,1.196)--(4.095,1.201)--(4.098,1.206)--(4.101,1.211)%
  --(4.105,1.215)--(4.108,1.219)--(4.112,1.221)--(4.115,1.221)--(4.119,1.219)--(4.122,1.217)%
  --(4.125,1.212)--(4.129,1.208)--(4.132,1.203)--(4.136,1.198)--(4.139,1.194)--(4.142,1.191)%
  --(4.146,1.190)--(4.149,1.191)--(4.153,1.193)--(4.156,1.197)--(4.160,1.201)--(4.163,1.206)%
  --(4.166,1.211)--(4.170,1.215)--(4.173,1.219)--(4.177,1.220)--(4.180,1.221)--(4.183,1.219)%
  --(4.187,1.216)--(4.190,1.212)--(4.194,1.208)--(4.197,1.203)--(4.201,1.198)--(4.204,1.194)%
  --(4.207,1.192)--(4.211,1.190)--(4.214,1.191)--(4.218,1.193)--(4.221,1.197)--(4.224,1.201)%
  --(4.228,1.206)--(4.231,1.211)--(4.235,1.215)--(4.238,1.219)--(4.242,1.220)--(4.245,1.221)%
  --(4.248,1.219)--(4.252,1.216)--(4.255,1.212)--(4.259,1.208)--(4.262,1.203)--(4.265,1.198)%
  --(4.269,1.194)--(4.272,1.192)--(4.276,1.190)--(4.279,1.191)--(4.283,1.193)--(4.286,1.197)%
  --(4.289,1.201)--(4.293,1.206)--(4.296,1.211)--(4.300,1.215)--(4.303,1.219)--(4.307,1.220)%
  --(4.310,1.221)--(4.313,1.219)--(4.317,1.216)--(4.320,1.212)--(4.324,1.208)--(4.327,1.203)%
  --(4.330,1.198)--(4.334,1.194)--(4.337,1.192)--(4.341,1.190)--(4.344,1.191)--(4.348,1.193)%
  --(4.351,1.197)--(4.354,1.201)--(4.358,1.206)--(4.361,1.211)--(4.365,1.215)--(4.368,1.219)%
  --(4.371,1.220)--(4.375,1.221)--(4.378,1.219)--(4.382,1.216)--(4.385,1.212)--(4.389,1.208)%
  --(4.392,1.203)--(4.395,1.198)--(4.399,1.194)--(4.402,1.191)--(4.406,1.190)--(4.409,1.191)%
  --(4.412,1.193)--(4.416,1.197)--(4.419,1.201)--(4.423,1.206)--(4.426,1.211)--(4.430,1.215)%
  --(4.433,1.219)--(4.436,1.220)--(4.440,1.221)--(4.443,1.219)--(4.447,1.216)--(4.450,1.212)%
  --(4.453,1.208)--(4.457,1.203)--(4.460,1.198)--(4.464,1.194)--(4.467,1.191)--(4.471,1.190)%
  --(4.474,1.191)--(4.477,1.193)--(4.481,1.197)--(4.484,1.201)--(4.488,1.206)--(4.491,1.211)%
  --(4.495,1.215)--(4.498,1.219)--(4.501,1.221)--(4.505,1.221)--(4.508,1.219)--(4.512,1.216)%
  --(4.515,1.212)--(4.518,1.207)--(4.522,1.202)--(4.525,1.198)--(4.529,1.194)--(4.532,1.191)%
  --(4.536,1.190)--(4.539,1.191)--(4.542,1.193)--(4.546,1.197)--(4.549,1.201)--(4.553,1.206)%
  --(4.556,1.211)--(4.559,1.216)--(4.563,1.219)--(4.566,1.221)--(4.570,1.221)--(4.573,1.219)%
  --(4.577,1.217)--(4.580,1.212)--(4.583,1.207)--(4.587,1.202)--(4.590,1.198)--(4.594,1.194)%
  --(4.597,1.191)--(4.600,1.190)--(4.604,1.191)--(4.607,1.193)--(4.611,1.197)--(4.614,1.201)%
  --(4.618,1.206)--(4.621,1.211)--(4.624,1.216)--(4.628,1.219)--(4.631,1.221)--(4.635,1.221)%
  --(4.638,1.220)--(4.641,1.217)--(4.645,1.212)--(4.648,1.207)--(4.652,1.202)--(4.655,1.198)%
  --(4.659,1.194)--(4.662,1.191)--(4.665,1.190)--(4.669,1.191)--(4.672,1.193)--(4.676,1.196)%
  --(4.679,1.201)--(4.683,1.206)--(4.686,1.211)--(4.689,1.216)--(4.693,1.219)--(4.696,1.221)%
  --(4.700,1.221)--(4.703,1.220)--(4.706,1.217)--(4.710,1.213)--(4.713,1.207)--(4.717,1.202)%
  --(4.720,1.197)--(4.724,1.193)--(4.727,1.191)--(4.730,1.190)--(4.734,1.190)--(4.737,1.193)%
  --(4.741,1.196)--(4.744,1.201)--(4.747,1.206)--(4.751,1.212)--(4.754,1.216)--(4.758,1.219)%
  --(4.761,1.221)--(4.765,1.222)--(4.768,1.220)--(4.771,1.217)--(4.775,1.213)--(4.778,1.207)%
  --(4.782,1.202)--(4.785,1.197)--(4.788,1.193)--(4.792,1.190)--(4.795,1.189)--(4.799,1.190)%
  --(4.802,1.192)--(4.806,1.196)--(4.809,1.201)--(4.812,1.206)--(4.816,1.212)--(4.819,1.216)%
  --(4.823,1.220)--(4.826,1.222)--(4.829,1.222)--(4.833,1.220)--(4.836,1.217)--(4.840,1.213)%
  --(4.843,1.207)--(4.847,1.202)--(4.850,1.197)--(4.853,1.193)--(4.857,1.190)--(4.860,1.189)%
  --(4.864,1.190)--(4.867,1.192)--(4.870,1.196)--(4.874,1.201)--(4.877,1.206)--(4.881,1.212)%
  --(4.884,1.217)--(4.888,1.220)--(4.891,1.222)--(4.894,1.222)--(4.898,1.221)--(4.901,1.217)%
  --(4.905,1.213)--(4.908,1.207)--(4.912,1.202)--(4.915,1.197)--(4.918,1.192)--(4.922,1.190)%
  --(4.925,1.189)--(4.929,1.189)--(4.932,1.192)--(4.935,1.196)--(4.939,1.201)--(4.942,1.207)%
  --(4.946,1.212)--(4.949,1.217)--(4.953,1.221)--(4.956,1.223)--(4.959,1.223)--(4.963,1.221)%
  --(4.966,1.218)--(4.970,1.213)--(4.973,1.208)--(4.976,1.202)--(4.980,1.196)--(4.983,1.192)%
  --(4.987,1.189)--(4.990,1.188)--(4.994,1.189)--(4.997,1.191)--(5.000,1.196)--(5.004,1.201)%
  --(5.007,1.207)--(5.011,1.212)--(5.014,1.217)--(5.017,1.221)--(5.021,1.223)--(5.024,1.223)%
  --(5.028,1.222)--(5.031,1.218)--(5.035,1.213)--(5.038,1.208)--(5.041,1.202)--(5.045,1.196)%
  --(5.048,1.192)--(5.052,1.189)--(5.055,1.187)--(5.058,1.188)--(5.062,1.191)--(5.065,1.195)%
  --(5.069,1.201)--(5.072,1.207)--(5.076,1.213)--(5.079,1.218)--(5.082,1.222)--(5.086,1.224)%
  --(5.089,1.224)--(5.093,1.222)--(5.096,1.218)--(5.100,1.213)--(5.103,1.208)--(5.106,1.201)%
  --(5.110,1.196)--(5.113,1.191)--(5.117,1.188)--(5.120,1.187)--(5.123,1.188)--(5.127,1.190)%
  --(5.130,1.195)--(5.134,1.201)--(5.137,1.207)--(5.141,1.213)--(5.144,1.218)--(5.147,1.222)%
  --(5.151,1.224)--(5.154,1.225)--(5.158,1.223)--(5.161,1.219)--(5.164,1.214)--(5.168,1.208)%
  --(5.171,1.201)--(5.175,1.195)--(5.178,1.191)--(5.182,1.187)--(5.185,1.186)--(5.188,1.187)%
  --(5.192,1.190)--(5.195,1.194)--(5.199,1.200)--(5.202,1.207)--(5.205,1.213)--(5.209,1.219)%
  --(5.212,1.223)--(5.216,1.225)--(5.219,1.225)--(5.223,1.223)--(5.226,1.219)--(5.229,1.214)%
  --(5.233,1.208)--(5.236,1.201)--(5.240,1.195)--(5.243,1.190)--(5.246,1.186)--(5.250,1.185)%
  --(5.253,1.186)--(5.257,1.189)--(5.260,1.194)--(5.264,1.200)--(5.267,1.207)--(5.270,1.214)%
  --(5.274,1.219)--(5.277,1.224)--(5.281,1.226)--(5.284,1.226)--(5.288,1.224)--(5.291,1.220)%
  --(5.294,1.214)--(5.298,1.208)--(5.301,1.201)--(5.305,1.194)--(5.308,1.189)--(5.311,1.186)%
  --(5.315,1.184)--(5.318,1.185)--(5.322,1.188)--(5.325,1.193)--(5.329,1.200)--(5.332,1.207)%
  --(5.335,1.214)--(5.339,1.220)--(5.342,1.225)--(5.346,1.227)--(5.349,1.227)--(5.352,1.225)%
  --(5.356,1.221)--(5.359,1.215)--(5.363,1.208)--(5.366,1.200)--(5.370,1.194)--(5.373,1.188)%
  --(5.376,1.184)--(5.380,1.183)--(5.383,1.184)--(5.387,1.187)--(5.390,1.193)--(5.393,1.200)%
  --(5.397,1.207)--(5.400,1.215)--(5.404,1.221)--(5.407,1.226)--(5.411,1.228)--(5.414,1.229)%
  --(5.417,1.226)--(5.421,1.222)--(5.424,1.215)--(5.428,1.208)--(5.431,1.200)--(5.434,1.193)%
  --(5.438,1.187)--(5.441,1.183)--(5.445,1.182)--(5.448,1.183)--(5.452,1.186)--(5.455,1.192)%
  --(5.458,1.199)--(5.462,1.207)--(5.465,1.215)--(5.469,1.222)--(5.472,1.227)--(5.476,1.230)%
  --(5.479,1.230)--(5.482,1.228)--(5.486,1.223)--(5.489,1.216)--(5.493,1.208)--(5.496,1.200)%
  --(5.499,1.192)--(5.503,1.186)--(5.506,1.182)--(5.510,1.180)--(5.513,1.181)--(5.517,1.185)%
  --(5.520,1.191)--(5.523,1.199)--(5.527,1.208)--(5.530,1.216)--(5.534,1.223)--(5.537,1.229)%
  --(5.540,1.232)--(5.544,1.232)--(5.547,1.229)--(5.551,1.224)--(5.554,1.217)--(5.558,1.208)%
  --(5.561,1.199)--(5.564,1.191)--(5.568,1.184)--(5.571,1.180)--(5.575,1.178)--(5.578,1.179)%
  --(5.581,1.184)--(5.585,1.190)--(5.588,1.199)--(5.592,1.208)--(5.595,1.217)--(5.599,1.225)%
  --(5.602,1.231)--(5.605,1.234)--(5.609,1.234)--(5.612,1.231)--(5.616,1.225)--(5.619,1.217)%
  --(5.622,1.208)--(5.626,1.198)--(5.629,1.190)--(5.633,1.182)--(5.636,1.178)--(5.640,1.176)%
  --(5.643,1.177)--(5.646,1.182)--(5.650,1.189)--(5.653,1.198)--(5.657,1.208)--(5.660,1.218)%
  --(5.664,1.226)--(5.667,1.233)--(5.670,1.236)--(5.674,1.236)--(5.677,1.233)--(5.681,1.227)%
  --(5.684,1.218)--(5.687,1.208)--(5.691,1.198)--(5.694,1.188)--(5.698,1.180)--(5.701,1.175)%
  --(5.705,1.173)--(5.708,1.175)--(5.711,1.180)--(5.715,1.187)--(5.718,1.197)--(5.722,1.208)%
  --(5.725,1.219)--(5.728,1.229)--(5.732,1.236)--(5.735,1.239)--(5.739,1.240)--(5.742,1.236)%
  --(5.746,1.229)--(5.749,1.220)--(5.752,1.208)--(5.756,1.197)--(5.759,1.186)--(5.763,1.177)%
  --(5.766,1.171)--(5.769,1.169)--(5.773,1.171)--(5.776,1.177)--(5.780,1.186)--(5.783,1.197)%
  --(5.787,1.209)--(5.790,1.221)--(5.793,1.231)--(5.797,1.239)--(5.800,1.243)--(5.804,1.243)%
  --(5.807,1.240)--(5.810,1.232)--(5.814,1.221)--(5.817,1.209)--(5.821,1.196)--(5.824,1.183)%
  --(5.828,1.174)--(5.831,1.167)--(5.834,1.165)--(5.838,1.167)--(5.841,1.173)--(5.845,1.183)%
  --(5.848,1.196)--(5.852,1.209)--(5.855,1.223)--(5.858,1.235)--(5.862,1.244)--(5.865,1.248)%
  --(5.869,1.249)--(5.872,1.244)--(5.875,1.235)--(5.879,1.223)--(5.882,1.209)--(5.886,1.194)%
  --(5.889,1.180)--(5.893,1.169)--(5.896,1.162)--(5.899,1.159)--(5.903,1.161)--(5.906,1.169)%
  --(5.910,1.180)--(5.913,1.194)--(5.916,1.210)--(5.920,1.226)--(5.923,1.240)--(5.927,1.250)%
  --(5.930,1.255)--(5.934,1.255)--(5.937,1.250)--(5.940,1.240)--(5.944,1.226)--(5.947,1.209)%
  --(5.951,1.192)--(5.954,1.176)--(5.957,1.162)--(5.961,1.154)--(5.964,1.151)--(5.968,1.154)%
  --(5.971,1.162)--(5.975,1.176)--(5.978,1.193)--(5.981,1.212)--(5.985,1.230)--(5.988,1.246)%
  --(5.992,1.258)--(5.995,1.265)--(5.998,1.265)--(6.002,1.259)--(6.005,1.246)--(6.009,1.229)%
  --(6.012,1.209)--(6.016,1.188)--(6.019,1.169)--(6.022,1.153)--(6.026,1.143)--(6.029,1.139)%
  --(6.033,1.142)--(6.036,1.153)--(6.040,1.169)--(6.043,1.190)--(6.046,1.214)--(6.050,1.236)%
  --(6.053,1.256)--(6.057,1.271)--(6.060,1.279)--(6.063,1.280)--(6.067,1.272)--(6.070,1.256)%
  --(6.074,1.235)--(6.077,1.209)--(6.081,1.183)--(6.084,1.158)--(6.087,1.138)--(6.091,1.125)%
  --(6.094,1.120)--(6.098,1.125)--(6.101,1.138)--(6.104,1.160)--(6.108,1.187)--(6.111,1.217)%
  --(6.115,1.247)--(6.118,1.273)--(6.122,1.293)--(6.125,1.304)--(6.128,1.304)--(6.132,1.293)%
  --(6.135,1.272)--(6.139,1.243)--(6.142,1.209)--(6.145,1.173)--(6.149,1.139)--(6.152,1.112)%
  --(6.156,1.093)--(6.159,1.087)--(6.163,1.093)--(6.166,1.112)--(6.169,1.143)--(6.173,1.182)%
  --(6.176,1.225)--(6.180,1.268)--(6.183,1.307)--(6.186,1.336)--(6.190,1.351)--(6.193,1.351)%
  --(6.197,1.335)--(6.200,1.303)--(6.204,1.259)--(6.207,1.205)--(6.210,1.149)--(6.214,1.095)%
  --(6.217,1.051)--(6.221,1.021)--(6.224,1.010)--(6.228,1.021)--(6.231,1.054)--(6.234,1.107)%
  --(6.238,1.175)--(6.241,1.253)--(6.245,1.332)--(6.248,1.403)--(6.251,1.458)--(6.255,1.488)%
  --(6.258,1.488)--(6.262,1.454)--(6.265,1.387)--(6.269,1.289)--(6.272,1.168)--(6.275,1.035)%
  --(6.279,0.903)--(6.282,0.789)--(6.286,0.708)--(6.289,0.678)--(6.292,0.713)--(6.296,0.824)%
  --(6.299,1.022)--(6.303,1.308)--(6.306,1.682)--(6.310,2.136)--(6.313,2.659)--(6.316,3.235)%
  --(6.320,3.844)--(6.323,4.463)--(6.327,5.072)--(6.330,5.648)--(6.333,6.171)--(6.337,6.625)%
  --(6.340,6.999)--(6.344,7.285)--(6.347,7.483)--(6.351,7.594)--(6.354,7.629)--(6.357,7.599)%
  --(6.361,7.518)--(6.364,7.404)--(6.368,7.272)--(6.371,7.139)--(6.374,7.018)--(6.378,6.920)%
  --(6.381,6.853)--(6.385,6.819)--(6.388,6.819)--(6.392,6.849)--(6.395,6.904)--(6.398,6.975)%
  --(6.402,7.054)--(6.405,7.132)--(6.409,7.200)--(6.412,7.253)--(6.415,7.286)--(6.419,7.297)%
  --(6.422,7.286)--(6.426,7.256)--(6.429,7.212)--(6.433,7.158)--(6.436,7.102)--(6.439,7.048)%
  --(6.443,7.004)--(6.446,6.972)--(6.450,6.956)--(6.453,6.956)--(6.457,6.971)--(6.460,7.000)%
  --(6.463,7.039)--(6.467,7.082)--(6.470,7.125)--(6.474,7.164)--(6.477,7.195)--(6.480,7.214)%
  --(6.484,7.220)--(6.487,7.214)--(6.491,7.195)--(6.494,7.168)--(6.498,7.134)--(6.501,7.098)%
  --(6.504,7.064)--(6.508,7.035)--(6.511,7.014)--(6.515,7.003)--(6.518,7.003)--(6.521,7.014)%
  --(6.525,7.034)--(6.528,7.060)--(6.532,7.090)--(6.535,7.120)--(6.539,7.147)--(6.542,7.169)%
  --(6.545,7.182)--(6.549,7.187)--(6.552,7.182)--(6.556,7.169)--(6.559,7.149)--(6.562,7.124)%
  --(6.566,7.098)--(6.569,7.072)--(6.573,7.051)--(6.576,7.035)--(6.580,7.027)--(6.583,7.028)%
  --(6.586,7.036)--(6.590,7.051)--(6.593,7.071)--(6.597,7.093)--(6.600,7.117)--(6.603,7.138)%
  --(6.607,7.154)--(6.610,7.165)--(6.614,7.168)--(6.617,7.164)--(6.621,7.154)--(6.624,7.138)%
  --(6.627,7.119)--(6.631,7.098)--(6.634,7.078)--(6.638,7.061)--(6.641,7.048)--(6.645,7.042)%
  --(6.648,7.042)--(6.651,7.049)--(6.655,7.061)--(6.658,7.077)--(6.662,7.095)--(6.665,7.114)%
  --(6.668,7.131)--(6.672,7.145)--(6.675,7.153)--(6.679,7.156)--(6.682,7.153)--(6.686,7.145)%
  --(6.689,7.131)--(6.692,7.115)--(6.696,7.098)--(6.699,7.081)--(6.703,7.067)--(6.706,7.057)%
  --(6.709,7.052)--(6.713,7.052)--(6.716,7.057)--(6.720,7.067)--(6.723,7.081)--(6.727,7.097)%
  --(6.730,7.113)--(6.733,7.127)--(6.737,7.138)--(6.740,7.146)--(6.744,7.148)--(6.747,7.145)%
  --(6.750,7.138)--(6.754,7.127)--(6.757,7.113)--(6.761,7.098)--(6.764,7.084)--(6.768,7.072)%
  --(6.771,7.063)--(6.774,7.058)--(6.778,7.059)--(6.781,7.063)--(6.785,7.072)--(6.788,7.084)%
  --(6.791,7.098)--(6.795,7.111)--(6.798,7.124)--(6.802,7.134)--(6.805,7.140)--(6.809,7.142)%
  --(6.812,7.140)--(6.815,7.133)--(6.819,7.124)--(6.822,7.111)--(6.826,7.098)--(6.829,7.086)%
  --(6.833,7.075)--(6.836,7.067)--(6.839,7.064)--(6.843,7.064)--(6.846,7.068)--(6.850,7.076)%
  --(6.853,7.086)--(6.856,7.098)--(6.860,7.110)--(6.863,7.121)--(6.867,7.130)--(6.870,7.136)%
  --(6.874,7.138)--(6.877,7.136)--(6.880,7.130)--(6.884,7.121)--(6.887,7.110)--(6.891,7.099)%
  --(6.894,7.087)--(6.897,7.078)--(6.901,7.071)--(6.904,7.067)--(6.908,7.068)--(6.911,7.071)%
  --(6.915,7.078)--(6.918,7.088)--(6.921,7.099)--(6.925,7.110)--(6.928,7.120)--(6.932,7.127)%
  --(6.935,7.132)--(6.938,7.134)--(6.942,7.132)--(6.945,7.127)--(6.949,7.119)--(6.952,7.109)%
  --(6.956,7.099)--(6.959,7.089)--(6.962,7.080)--(6.966,7.074)--(6.969,7.071)--(6.973,7.071)%
  --(6.976,7.074)--(6.979,7.081)--(6.983,7.089)--(6.986,7.099)--(6.990,7.109)--(6.993,7.118)%
  --(6.997,7.125)--(7.000,7.130)--(7.003,7.131)--(7.007,7.129)--(7.010,7.125)--(7.014,7.117)%
  --(7.017,7.109)--(7.021,7.099)--(7.024,7.090)--(7.027,7.082)--(7.031,7.076)--(7.034,7.073)%
  --(7.038,7.073)--(7.041,7.076)--(7.044,7.082)--(7.048,7.090)--(7.051,7.099)--(7.055,7.108)%
  --(7.058,7.117)--(7.062,7.123)--(7.065,7.128)--(7.068,7.129)--(7.072,7.127)--(7.075,7.123)%
  --(7.079,7.116)--(7.082,7.108)--(7.085,7.099)--(7.089,7.090)--(7.092,7.083)--(7.096,7.078)%
  --(7.099,7.075)--(7.103,7.075)--(7.106,7.078)--(7.109,7.084)--(7.113,7.091)--(7.116,7.099)%
  --(7.120,7.108)--(7.123,7.116)--(7.126,7.122)--(7.130,7.126)--(7.133,7.127)--(7.137,7.125)%
  --(7.140,7.121)--(7.144,7.115)--(7.147,7.107)--(7.150,7.099)--(7.154,7.091)--(7.157,7.084)%
  --(7.161,7.079)--(7.164,7.077)--(7.167,7.077)--(7.171,7.080)--(7.174,7.085)--(7.178,7.092)%
  --(7.181,7.100)--(7.185,7.108)--(7.188,7.115)--(7.191,7.121)--(7.195,7.124)--(7.198,7.125)%
  --(7.202,7.124)--(7.205,7.120)--(7.209,7.114)--(7.212,7.107)--(7.215,7.099)--(7.219,7.092)%
  --(7.222,7.085)--(7.226,7.081)--(7.229,7.078)--(7.232,7.079)--(7.236,7.081)--(7.239,7.086)%
  --(7.243,7.092)--(7.246,7.100)--(7.250,7.107)--(7.253,7.114)--(7.256,7.120)--(7.260,7.123)%
  --(7.263,7.124)--(7.267,7.123)--(7.270,7.119)--(7.273,7.113)--(7.277,7.107)--(7.280,7.099)%
  --(7.284,7.092)--(7.287,7.086)--(7.291,7.082)--(7.294,7.080)--(7.297,7.080)--(7.301,7.082)%
  --(7.304,7.087)--(7.308,7.093)--(7.311,7.100)--(7.314,7.107)--(7.318,7.114)--(7.321,7.119)%
  --(7.325,7.122)--(7.328,7.123)--(7.332,7.121)--(7.335,7.118)--(7.338,7.113)--(7.342,7.106)%
  --(7.345,7.099)--(7.349,7.093)--(7.352,7.087)--(7.355,7.083)--(7.359,7.081)--(7.362,7.081)%
  --(7.366,7.083)--(7.369,7.088)--(7.373,7.093)--(7.376,7.100)--(7.379,7.107)--(7.383,7.113)%
  --(7.386,7.118)--(7.390,7.121)--(7.393,7.122)--(7.397,7.121)--(7.400,7.117)--(7.403,7.112)%
  --(7.407,7.106)--(7.410,7.099)--(7.414,7.093)--(7.417,7.088)--(7.420,7.084)--(7.424,7.082)%
  --(7.427,7.082)--(7.431,7.084)--(7.434,7.088)--(7.438,7.094)--(7.441,7.100)--(7.444,7.107)%
  --(7.448,7.113)--(7.451,7.117)--(7.455,7.120)--(7.458,7.121)--(7.461,7.120)--(7.465,7.116)%
  --(7.468,7.112)--(7.472,7.106)--(7.475,7.099)--(7.479,7.093)--(7.482,7.088)--(7.485,7.084)%
  --(7.489,7.082)--(7.492,7.083)--(7.496,7.085)--(7.499,7.089)--(7.502,7.094)--(7.506,7.100)%
  --(7.509,7.106)--(7.513,7.112)--(7.516,7.117)--(7.520,7.119)--(7.523,7.120)--(7.526,7.119)%
  --(7.530,7.116)--(7.533,7.111)--(7.537,7.106)--(7.540,7.099)--(7.543,7.094)--(7.547,7.089)%
  --(7.550,7.085)--(7.554,7.083)--(7.557,7.083)--(7.561,7.085)--(7.564,7.089)--(7.567,7.094)%
  --(7.571,7.100)--(7.574,7.106)--(7.578,7.112)--(7.581,7.116)--(7.585,7.119)--(7.588,7.120)%
  --(7.591,7.118)--(7.595,7.115)--(7.598,7.111)--(7.602,7.105)--(7.605,7.099)--(7.608,7.094)%
  --(7.612,7.089)--(7.615,7.085)--(7.619,7.084)--(7.622,7.084)--(7.626,7.086)--(7.629,7.090)%
  --(7.632,7.095)--(7.636,7.100)--(7.639,7.106)--(7.643,7.111)--(7.646,7.116)--(7.649,7.118)%
  --(7.653,7.119)--(7.656,7.118)--(7.660,7.115)--(7.663,7.111)--(7.667,7.105)--(7.670,7.099)%
  --(7.673,7.094)--(7.677,7.089)--(7.680,7.086)--(7.684,7.084)--(7.687,7.084)--(7.690,7.086)%
  --(7.694,7.090)--(7.697,7.095)--(7.701,7.100)--(7.704,7.106)--(7.708,7.111)--(7.711,7.115)%
  --(7.714,7.118)--(7.718,7.118)--(7.721,7.117)--(7.725,7.115)--(7.728,7.110)--(7.731,7.105)%
  --(7.735,7.100)--(7.738,7.094)--(7.742,7.090)--(7.745,7.086)--(7.749,7.085)--(7.752,7.085)%
  --(7.755,7.087)--(7.759,7.090)--(7.762,7.095)--(7.766,7.101)--(7.769,7.106)--(7.773,7.111)%
  --(7.776,7.115)--(7.779,7.117)--(7.783,7.118)--(7.786,7.117)--(7.790,7.114)--(7.793,7.110)%
  --(7.796,7.105)--(7.800,7.100)--(7.803,7.094)--(7.807,7.090)--(7.810,7.087)--(7.814,7.085)%
  --(7.817,7.085)--(7.820,7.087)--(7.824,7.091)--(7.827,7.095)--(7.831,7.101)--(7.834,7.106)%
  --(7.837,7.111)--(7.841,7.115)--(7.844,7.117)--(7.848,7.118)--(7.851,7.117)--(7.855,7.114)%
  --(7.858,7.110)--(7.861,7.105)--(7.865,7.100)--(7.868,7.094)--(7.872,7.090)--(7.875,7.087)%
  --(7.878,7.085)--(7.882,7.086)--(7.885,7.088)--(7.889,7.091)--(7.892,7.095)--(7.896,7.101)%
  --(7.899,7.106)--(7.902,7.111)--(7.906,7.114)--(7.909,7.117)--(7.913,7.117)--(7.916,7.116)%
  --(7.919,7.114)--(7.923,7.110)--(7.926,7.105)--(7.930,7.100)--(7.933,7.094)--(7.937,7.090)%
  --(7.940,7.087)--(7.943,7.086)--(7.947,7.086)--(7.950,7.088)--(7.954,7.091)--(7.957,7.096)%
  --(7.960,7.101)--(7.964,7.106)--(7.967,7.111)--(7.971,7.114)--(7.974,7.116)--(7.978,7.117)%
  --(7.981,7.116)--(7.984,7.113)--(7.988,7.109)--(7.991,7.105)--(7.995,7.100)--(7.998,7.095)%
  --(8.002,7.090)--(8.005,7.087)--(8.008,7.086)--(8.012,7.086)--(8.015,7.088)--(8.019,7.091)%
  --(8.022,7.096)--(8.025,7.101)--(8.029,7.106)--(8.032,7.110)--(8.036,7.114)--(8.039,7.116)%
  --(8.043,7.117)--(8.046,7.116)--(8.049,7.113)--(8.053,7.109)--(8.056,7.105)--(8.060,7.100)%
  --(8.063,7.095)--(8.066,7.090)--(8.070,7.088)--(8.073,7.086)--(8.077,7.086)--(8.080,7.088)%
  --(8.084,7.091)--(8.087,7.096)--(8.090,7.101)--(8.094,7.106)--(8.097,7.110)--(8.101,7.114)%
  --(8.104,7.116)--(8.107,7.117)--(8.111,7.116)--(8.114,7.113)--(8.118,7.109)--(8.121,7.105)%
  --(8.125,7.100)--(8.128,7.095)--(8.131,7.091)--(8.135,7.088)--(8.138,7.086)--(8.142,7.086)%
  --(8.145,7.088)--(8.148,7.092)--(8.152,7.096)--(8.155,7.101)--(8.159,7.106)--(8.162,7.110)%
  --(8.166,7.114)--(8.169,7.116)--(8.172,7.117)--(8.176,7.116)--(8.179,7.113)--(8.183,7.109)%
  --(8.186,7.104)--(8.190,7.099)--(8.193,7.095)--(8.196,7.091)--(8.200,7.088)--(8.203,7.086)%
  --(8.207,7.087)--(8.210,7.088)--(8.213,7.092)--(8.217,7.096)--(8.220,7.101)--(8.224,7.106)%
  --(8.227,7.110)--(8.231,7.114)--(8.234,7.116)--(8.237,7.117)--(8.241,7.116)--(8.244,7.113)%
  --(8.248,7.109)--(8.251,7.104)--(8.254,7.099)--(8.258,7.095)--(8.261,7.091)--(8.265,7.088)%
  --(8.268,7.086)--(8.272,7.087)--(8.275,7.088)--(8.278,7.092)--(8.282,7.096)--(8.285,7.101)%
  --(8.289,7.106)--(8.292,7.110)--(8.295,7.114)--(8.299,7.116)--(8.302,7.117)--(8.306,7.115)%
  --(8.309,7.113)--(8.313,7.109)--(8.316,7.104)--(8.319,7.099)--(8.323,7.095)--(8.326,7.091)%
  --(8.330,7.088)--(8.333,7.086)--(8.336,7.087)--(8.340,7.088)--(8.343,7.092)--(8.347,7.096)%
  --(8.350,7.101)--(8.354,7.106)--(8.357,7.110)--(8.360,7.114)--(8.364,7.116)--(8.367,7.117)%
  --(8.371,7.115)--(8.374,7.113)--(8.378,7.109)--(8.381,7.104)--(8.384,7.099)--(8.388,7.095)%
  --(8.391,7.091)--(8.395,7.088)--(8.398,7.086)--(8.401,7.087)--(8.405,7.088)--(8.408,7.092)%
  --(8.412,7.096)--(8.415,7.101)--(8.419,7.106)--(8.422,7.110)--(8.425,7.114)--(8.429,7.116)%
  --(8.432,7.117)--(8.436,7.115)--(8.439,7.113)--(8.442,7.109)--(8.446,7.104)--(8.449,7.099)%
  --(8.453,7.095)--(8.456,7.091)--(8.460,7.088)--(8.463,7.086)--(8.466,7.087)--(8.470,7.088)%
  --(8.473,7.092)--(8.477,7.096)--(8.480,7.101)--(8.483,7.106)--(8.487,7.110)--(8.490,7.114)%
  --(8.494,7.116)--(8.497,7.117)--(8.501,7.116)--(8.504,7.113)--(8.507,7.109)--(8.511,7.104)%
  --(8.514,7.099)--(8.518,7.095)--(8.521,7.090)--(8.524,7.088)--(8.528,7.086)--(8.531,7.086)%
  --(8.535,7.088)--(8.538,7.092)--(8.542,7.096)--(8.545,7.101)--(8.548,7.106)--(8.552,7.111)%
  --(8.555,7.114)--(8.559,7.116)--(8.562,7.117)--(8.566,7.116)--(8.569,7.113)--(8.572,7.109)%
  --(8.576,7.104)--(8.579,7.099)--(8.583,7.094)--(8.586,7.090)--(8.589,7.087)--(8.593,7.086)%
  --(8.596,7.086)--(8.600,7.088)--(8.603,7.092)--(8.607,7.096)--(8.610,7.101)--(8.613,7.106)%
  --(8.617,7.111)--(8.620,7.114)--(8.624,7.116)--(8.627,7.117)--(8.630,7.116)--(8.634,7.113)%
  --(8.637,7.109)--(8.641,7.104)--(8.644,7.099)--(8.648,7.094)--(8.651,7.090)--(8.654,7.087)%
  --(8.658,7.086)--(8.661,7.086)--(8.665,7.088)--(8.668,7.092)--(8.671,7.096)--(8.675,7.101)%
  --(8.678,7.106)--(8.682,7.111)--(8.685,7.114)--(8.689,7.117)--(8.692,7.117)--(8.695,7.116)%
  --(8.699,7.113)--(8.702,7.109)--(8.706,7.104)--(8.709,7.099)--(8.712,7.094)--(8.716,7.090)%
  --(8.719,7.087)--(8.723,7.086)--(8.726,7.086)--(8.730,7.088)--(8.733,7.091)--(8.736,7.096)%
  --(8.740,7.101)--(8.743,7.106)--(8.747,7.111)--(8.750,7.115)--(8.754,7.117)--(8.757,7.117)%
  --(8.760,7.116)--(8.764,7.113)--(8.767,7.109)--(8.771,7.104)--(8.774,7.099)--(8.777,7.094)%
  --(8.781,7.090)--(8.784,7.087)--(8.788,7.085)--(8.791,7.086)--(8.795,7.088)--(8.798,7.091)%
  --(8.801,7.096)--(8.805,7.101)--(8.808,7.106)--(8.812,7.111)--(8.815,7.115)--(8.818,7.117)%
  --(8.822,7.118)--(8.825,7.116)--(8.829,7.114)--(8.832,7.109)--(8.836,7.104)--(8.839,7.099)%
  --(8.842,7.094)--(8.846,7.090)--(8.849,7.086)--(8.853,7.085)--(8.856,7.085)--(8.859,7.087)%
  --(8.863,7.091)--(8.866,7.096)--(8.870,7.101)--(8.873,7.107)--(8.877,7.111)--(8.880,7.115)%
  --(8.883,7.117)--(8.887,7.118)--(8.890,7.117)--(8.894,7.114)--(8.897,7.110)--(8.900,7.104)%
  --(8.904,7.099)--(8.907,7.094)--(8.911,7.089)--(8.914,7.086)--(8.918,7.085)--(8.921,7.085)%
  --(8.924,7.087)--(8.928,7.091)--(8.931,7.096)--(8.935,7.101)--(8.938,7.107)--(8.942,7.112)%
  --(8.945,7.116)--(8.948,7.118)--(8.952,7.118)--(8.955,7.117)--(8.959,7.114)--(8.962,7.110)%
  --(8.965,7.104)--(8.969,7.099)--(8.972,7.093)--(8.976,7.089)--(8.979,7.086)--(8.983,7.084)%
  --(8.986,7.085)--(8.989,7.087)--(8.993,7.091)--(8.996,7.096)--(9.000,7.101)--(9.003,7.107)%
  --(9.006,7.112)--(9.010,7.116)--(9.013,7.118)--(9.017,7.119)--(9.020,7.118)--(9.024,7.115)%
  --(9.027,7.110)--(9.030,7.105)--(9.034,7.099)--(9.037,7.093)--(9.041,7.088)--(9.044,7.085)%
  --(9.047,7.084)--(9.051,7.084)--(9.054,7.086)--(9.058,7.090)--(9.061,7.095)--(9.065,7.101)%
  --(9.068,7.107)--(9.071,7.112)--(9.075,7.116)--(9.078,7.119)--(9.082,7.119)--(9.085,7.118)%
  --(9.088,7.115)--(9.092,7.110)--(9.095,7.105)--(9.099,7.099)--(9.102,7.093)--(9.106,7.088)%
  --(9.109,7.085)--(9.112,7.083)--(9.116,7.083)--(9.119,7.086)--(9.123,7.090)--(9.126,7.095)%
  --(9.130,7.101)--(9.133,7.107)--(9.136,7.113)--(9.140,7.117)--(9.143,7.120)--(9.147,7.120)%
  --(9.150,7.119)--(9.153,7.115)--(9.157,7.111)--(9.160,7.105)--(9.164,7.098)--(9.167,7.092)%
  --(9.171,7.087)--(9.174,7.084)--(9.177,7.082)--(9.181,7.083)--(9.184,7.085)--(9.188,7.089)%
  --(9.191,7.095)--(9.194,7.101)--(9.198,7.108)--(9.201,7.113)--(9.205,7.118)--(9.208,7.120)%
  --(9.212,7.121)--(9.215,7.119)--(9.218,7.116)--(9.222,7.111)--(9.225,7.105)--(9.229,7.098)%
  --(9.232,7.092)--(9.235,7.087)--(9.239,7.083)--(9.242,7.082)--(9.246,7.082)--(9.249,7.085)%
  --(9.253,7.089)--(9.256,7.095)--(9.259,7.101)--(9.263,7.108)--(9.266,7.114)--(9.270,7.118)%
  --(9.273,7.121)--(9.276,7.122)--(9.280,7.120)--(9.283,7.117)--(9.287,7.111)--(9.290,7.105)%
  --(9.294,7.098)--(9.297,7.092)--(9.300,7.086)--(9.304,7.082)--(9.307,7.081)--(9.311,7.081)%
  --(9.314,7.084)--(9.317,7.088)--(9.321,7.095)--(9.324,7.101)--(9.328,7.108)--(9.331,7.114)%
  --(9.335,7.119)--(9.338,7.122)--(9.341,7.123)--(9.345,7.121)--(9.348,7.117)--(9.352,7.112)%
  --(9.355,7.105)--(9.359,7.098)--(9.362,7.091)--(9.365,7.085)--(9.369,7.081)--(9.372,7.080)%
  --(9.376,7.080)--(9.379,7.083)--(9.382,7.088)--(9.386,7.094)--(9.389,7.101)--(9.393,7.109)%
  --(9.396,7.115)--(9.400,7.120)--(9.403,7.123)--(9.406,7.124)--(9.410,7.122)--(9.413,7.118)%
  --(9.417,7.112)--(9.420,7.105)--(9.423,7.098)--(9.427,7.090)--(9.430,7.084)--(9.434,7.080)%
  --(9.437,7.078)--(9.441,7.079)--(9.444,7.082)--(9.447,7.087)--(9.451,7.094)--(9.454,7.101)%
  --(9.458,7.109)--(9.461,7.116)--(9.464,7.121)--(9.468,7.125)--(9.471,7.125)--(9.475,7.123)%
  --(9.478,7.119)--(9.482,7.113)--(9.485,7.105)--(9.488,7.097)--(9.492,7.090)--(9.495,7.083)%
  --(9.499,7.079)--(9.502,7.077)--(9.505,7.077)--(9.509,7.081)--(9.512,7.086)--(9.516,7.093)%
  --(9.519,7.102)--(9.523,7.110)--(9.526,7.117)--(9.529,7.123)--(9.533,7.126)--(9.536,7.127)%
  --(9.540,7.125)--(9.543,7.120)--(9.547,7.114)--(9.550,7.106)--(9.553,7.097)--(9.557,7.089)%
  --(9.560,7.082)--(9.564,7.077)--(9.567,7.075)--(9.570,7.076)--(9.574,7.079)--(9.577,7.085)%
  --(9.581,7.093)--(9.584,7.102)--(9.588,7.110)--(9.591,7.118)--(9.594,7.124)--(9.598,7.128)%
  --(9.601,7.129)--(9.605,7.127)--(9.608,7.122)--(9.611,7.114)--(9.615,7.106)--(9.618,7.097)%
  --(9.622,7.088)--(9.625,7.080)--(9.629,7.075)--(9.632,7.073)--(9.635,7.074)--(9.639,7.077)%
  --(9.642,7.084)--(9.646,7.092)--(9.649,7.102)--(9.652,7.111)--(9.656,7.120)--(9.659,7.126)%
  --(9.663,7.130)--(9.666,7.131)--(9.670,7.129)--(9.673,7.123)--(9.676,7.115)--(9.680,7.106)%
  --(9.683,7.096)--(9.687,7.086)--(9.690,7.078)--(9.693,7.073)--(9.697,7.070)--(9.700,7.071)%
  --(9.704,7.075)--(9.707,7.082)--(9.711,7.092)--(9.714,7.102)--(9.717,7.112)--(9.721,7.122)%
  --(9.724,7.129)--(9.728,7.133)--(9.731,7.134)--(9.735,7.131)--(9.738,7.125)--(9.741,7.117)%
  --(9.745,7.106)--(9.748,7.095)--(9.752,7.085)--(9.755,7.076)--(9.758,7.070)--(9.762,7.067)%
  --(9.765,7.068)--(9.769,7.073)--(9.772,7.080)--(9.776,7.091)--(9.779,7.102)--(9.782,7.114)%
  --(9.786,7.124)--(9.789,7.132)--(9.793,7.137)--(9.796,7.138)--(9.799,7.135)--(9.803,7.128)%
  --(9.806,7.118)--(9.810,7.107)--(9.813,7.095)--(9.817,7.083)--(9.820,7.073)--(9.823,7.066)%
  --(9.827,7.063)--(9.830,7.064)--(9.834,7.069)--(9.837,7.078)--(9.840,7.089)--(9.844,7.102)%
  --(9.847,7.115)--(9.851,7.127)--(9.854,7.136)--(9.858,7.141)--(9.861,7.142)--(9.864,7.139)%
  --(9.868,7.131)--(9.871,7.120)--(9.875,7.107)--(9.878,7.093)--(9.881,7.080)--(9.885,7.069)%
  --(9.888,7.061)--(9.892,7.058)--(9.895,7.059)--(9.899,7.065)--(9.902,7.075)--(9.905,7.088)%
  --(9.909,7.103)--(9.912,7.117)--(9.916,7.131)--(9.919,7.141)--(9.923,7.147)--(9.926,7.148)%
  --(9.929,7.144)--(9.933,7.135)--(9.936,7.123)--(9.940,7.108)--(9.943,7.092)--(9.946,7.077)%
  --(9.950,7.064)--(9.953,7.055)--(9.957,7.051)--(9.960,7.053)--(9.964,7.059)--(9.967,7.071)%
  --(9.970,7.086)--(9.974,7.103)--(9.977,7.120)--(9.981,7.136)--(9.984,7.148)--(9.987,7.155)%
  --(9.991,7.156)--(9.994,7.152)--(9.998,7.141)--(10.001,7.127)--(10.005,7.109)--(10.008,7.090)%
  --(10.011,7.072)--(10.015,7.057)--(10.018,7.046)--(10.022,7.041)--(10.025,7.043)--(10.028,7.051)%
  --(10.032,7.065)--(10.035,7.083)--(10.039,7.104)--(10.042,7.125)--(10.046,7.143)--(10.049,7.158)%
  --(10.052,7.166)--(10.056,7.168)--(10.059,7.162)--(10.063,7.150)--(10.066,7.132)--(10.069,7.110)%
  --(10.073,7.087)--(10.076,7.064)--(10.080,7.046)--(10.083,7.032)--(10.087,7.027)--(10.090,7.029)%
  --(10.093,7.039)--(10.097,7.057)--(10.100,7.080)--(10.104,7.105)--(10.107,7.132)--(10.111,7.155)%
  --(10.114,7.174)--(10.117,7.184)--(10.121,7.186)--(10.124,7.179)--(10.128,7.163)--(10.131,7.140)%
  --(10.134,7.111)--(10.138,7.081)--(10.141,7.052)--(10.145,7.027)--(10.148,7.010)--(10.152,7.002)%
  --(10.155,7.005)--(10.158,7.019)--(10.162,7.043)--(10.165,7.074)--(10.169,7.109)--(10.172,7.144)%
  --(10.175,7.177)--(10.179,7.202)--(10.182,7.217)--(10.186,7.220)--(10.189,7.209)--(10.193,7.187)%
  --(10.196,7.153)--(10.199,7.113)--(10.203,7.069)--(10.206,7.026)--(10.210,6.991)--(10.213,6.965)%
  --(10.216,6.954)--(10.220,6.959)--(10.223,6.980)--(10.227,7.016)--(10.230,7.063)--(10.234,7.118)%
  --(10.237,7.175)--(10.240,7.226)--(10.244,7.267)--(10.247,7.291)--(10.251,7.296)--(10.254,7.278)%
  --(10.257,7.239)--(10.261,7.181)--(10.264,7.109)--(10.268,7.030)--(10.271,6.953)--(10.275,6.886)%
  --(10.278,6.837)--(10.281,6.816)--(10.285,6.825)--(10.288,6.869)--(10.292,6.946)--(10.295,7.052)%
  --(10.299,7.177)--(10.302,7.312)--(10.305,7.440)--(10.309,7.546)--(10.312,7.614)--(10.316,7.626)%
  --(10.319,7.570)--(10.322,7.433)--(10.326,7.210)--(10.329,6.897)--(10.333,6.499)--(10.336,6.023)%
  --(10.340,5.482)--(10.343,4.895)--(10.346,4.280)--(10.350,3.662)--(10.353,3.061)--(10.357,2.499)%
  --(10.360,1.994)--(10.363,1.563)--(10.367,1.214)--(10.370,0.954)--(10.374,0.783)--(10.377,0.695)%
  --(10.381,0.681)--(10.384,0.728)--(10.387,0.820)--(10.391,0.941)--(10.394,1.075)--(10.398,1.206)%
  --(10.401,1.321)--(10.404,1.410)--(10.408,1.468)--(10.411,1.492)--(10.415,1.482)--(10.418,1.444)%
  --(10.422,1.383)--(10.425,1.309)--(10.428,1.230)--(10.432,1.154)--(10.435,1.089)--(10.439,1.042)%
  --(10.442,1.016)--(10.445,1.011)--(10.449,1.028)--(10.452,1.063)--(10.456,1.110)--(10.459,1.165)%
  --(10.463,1.222)--(10.466,1.273)--(10.469,1.314)--(10.473,1.342)--(10.476,1.353)--(10.480,1.348)%
  --(10.483,1.328)--(10.487,1.296)--(10.490,1.256)--(10.493,1.212)--(10.497,1.170)--(10.500,1.133)%
  --(10.504,1.105)--(10.507,1.090)--(10.510,1.087)--(10.514,1.098)--(10.517,1.119)--(10.521,1.149)%
  --(10.524,1.184)--(10.528,1.219)--(10.531,1.252)--(10.534,1.279)--(10.538,1.297)--(10.541,1.305)%
  --(10.545,1.301)--(10.548,1.288)--(10.551,1.266)--(10.555,1.238)--(10.558,1.208)--(10.562,1.179)%
  --(10.565,1.153)--(10.569,1.133)--(10.572,1.122)--(10.575,1.121)--(10.579,1.128)--(10.582,1.144)%
  --(10.586,1.165)--(10.589,1.191)--(10.592,1.217)--(10.596,1.241)--(10.599,1.261)--(10.603,1.275)%
  --(10.606,1.280)--(10.610,1.278)--(10.613,1.268)--(10.616,1.251)--(10.620,1.230)--(10.623,1.207)%
  --(10.627,1.184)--(10.630,1.164)--(10.633,1.149)--(10.637,1.141)--(10.640,1.139)--(10.644,1.145)%
  --(10.647,1.157)--(10.651,1.174)--(10.654,1.194)--(10.657,1.215)--(10.661,1.235)--(10.664,1.251)%
  --(10.668,1.261)--(10.671,1.266)--(10.675,1.264)--(10.678,1.255)--(10.681,1.242)--(10.685,1.225)%
  --(10.688,1.206)--(10.692,1.187)--(10.695,1.171)--(10.698,1.159)--(10.702,1.152)--(10.705,1.151)%
  --(10.709,1.156)--(10.712,1.166)--(10.716,1.180)--(10.719,1.197)--(10.722,1.214)--(10.726,1.230)%
  --(10.729,1.243)--(10.733,1.252)--(10.736,1.256)--(10.739,1.254)--(10.743,1.247)--(10.746,1.236)%
  --(10.750,1.221)--(10.753,1.206)--(10.757,1.190)--(10.760,1.176)--(10.763,1.166)--(10.767,1.160)%
  --(10.770,1.159)--(10.774,1.163)--(10.777,1.172)--(10.780,1.184)--(10.784,1.198)--(10.787,1.213)%
  --(10.791,1.227)--(10.794,1.238)--(10.798,1.246)--(10.801,1.249)--(10.804,1.247)--(10.808,1.241)%
  --(10.811,1.232)--(10.815,1.219)--(10.818,1.205)--(10.821,1.192)--(10.825,1.180)--(10.828,1.171)%
  --(10.832,1.166)--(10.835,1.165)--(10.839,1.169)--(10.842,1.176)--(10.845,1.187)--(10.849,1.199)%
  --(10.852,1.212)--(10.856,1.224)--(10.859,1.234)--(10.862,1.241)--(10.866,1.244)--(10.869,1.242)%
  --(10.873,1.237)--(10.876,1.228)--(10.880,1.217)--(10.883,1.205)--(10.886,1.193)--(10.890,1.183)%
  --(10.893,1.175)--(10.897,1.170)--(10.900,1.170)--(10.904,1.173)--(10.907,1.180)--(10.910,1.189)%
  --(10.914,1.200)--(10.917,1.212)--(10.921,1.223)--(10.924,1.231)--(10.927,1.237)--(10.931,1.240)%
  --(10.934,1.239)--(10.938,1.234)--(10.941,1.226)--(10.945,1.216)--(10.948,1.205)--(10.951,1.194)%
  --(10.955,1.185)--(10.958,1.178)--(10.962,1.174)--(10.965,1.173)--(10.968,1.176)--(10.972,1.182)%
  --(10.975,1.191)--(10.979,1.201)--(10.982,1.211)--(10.986,1.221)--(10.989,1.229)--(10.992,1.234)%
  --(10.996,1.237)--(10.999,1.236)--(11.003,1.231)--(11.006,1.224)--(11.009,1.215)--(11.013,1.205)%
  --(11.016,1.195)--(11.020,1.187)--(11.023,1.180)--(11.027,1.176)--(11.030,1.176)--(11.033,1.179)%
  --(11.037,1.184)--(11.040,1.192)--(11.044,1.201)--(11.047,1.211)--(11.050,1.220)--(11.054,1.227)%
  --(11.057,1.232)--(11.061,1.234)--(11.064,1.233)--(11.068,1.229)--(11.071,1.222)--(11.074,1.214)%
  --(11.078,1.205)--(11.081,1.196)--(11.085,1.188)--(11.088,1.182)--(11.092,1.179)--(11.095,1.178)%
  --(11.098,1.181)--(11.102,1.186)--(11.105,1.193)--(11.109,1.202)--(11.112,1.211)--(11.115,1.219)%
  --(11.119,1.226)--(11.122,1.230)--(11.126,1.232)--(11.129,1.231)--(11.133,1.227)--(11.136,1.221)%
  --(11.139,1.213)--(11.143,1.205)--(11.146,1.197)--(11.150,1.189)--(11.153,1.184)--(11.156,1.181)%
  --(11.160,1.180)--(11.163,1.183)--(11.167,1.187)--(11.170,1.194)--(11.174,1.202)--(11.177,1.210)%
  --(11.180,1.218)--(11.184,1.224)--(11.187,1.229)--(11.191,1.230)--(11.194,1.229)--(11.197,1.226)%
  --(11.201,1.220)--(11.204,1.213)--(11.208,1.205)--(11.211,1.197)--(11.215,1.190)--(11.218,1.185)%
  --(11.221,1.182)--(11.225,1.182)--(11.228,1.184)--(11.232,1.189)--(11.235,1.195)--(11.238,1.202)%
  --(11.242,1.210)--(11.245,1.217)--(11.249,1.223)--(11.252,1.227)--(11.256,1.229)--(11.259,1.228)%
  --(11.262,1.225)--(11.266,1.219)--(11.269,1.212)--(11.273,1.205)--(11.276,1.198)--(11.280,1.191)%
  --(11.283,1.186)--(11.286,1.184)--(11.290,1.183)--(11.293,1.185)--(11.297,1.190)--(11.300,1.196)%
  --(11.303,1.203)--(11.307,1.210)--(11.310,1.217)--(11.314,1.222)--(11.317,1.226)--(11.321,1.228)%
  --(11.324,1.227)--(11.327,1.224)--(11.331,1.218)--(11.334,1.212)--(11.338,1.205)--(11.341,1.198)%
  --(11.344,1.192)--(11.348,1.187)--(11.351,1.185)--(11.355,1.184)--(11.358,1.186)--(11.362,1.190)%
  --(11.365,1.196)--(11.368,1.203)--(11.372,1.210)--(11.375,1.216)--(11.379,1.221)--(11.382,1.225)%
  --(11.385,1.226)--(11.389,1.226)--(11.392,1.223)--(11.396,1.218)--(11.399,1.212)--(11.403,1.205)%
  --(11.406,1.198)--(11.409,1.192)--(11.413,1.188)--(11.416,1.186)--(11.420,1.185)--(11.423,1.187)%
  --(11.426,1.191)--(11.430,1.197)--(11.433,1.203)--(11.437,1.210)--(11.440,1.216)--(11.444,1.221)%
  --(11.447,1.224);
  \gpfill{color=gpbgfillcolor} (7.137,1.448)--(9.525,1.448)--(9.525,2.680)--(7.137,2.680)--cycle;
  \gpcolor{color=gp lt color border}
  \gpsetlinewidth{1.00}
  \draw[gp path] (7.137,1.448)--(7.137,2.680)--(9.525,2.680)--(9.525,1.448)--cycle;
  \node[gp node right] at (8.241,2.526) {{$f_4(t)$}};
  \gpcolor{rgb color={0.000,0.000,0.000}}
  \gpsetlinewidth{4.00}
  \draw[gp path] (8.425,2.526)--(9.341,2.526);
  \gpcolor{color=gp lt color border}
  \node[gp node right] at (8.241,2.218) {{$S_{11}(t)$}};
  \gpcolor{rgb color={0.000,0.620,0.451}}
  \gpsetlinewidth{2.00}
  \draw[gp path] (8.425,2.218)--(9.341,2.218);
  \gpcolor{color=gp lt color border}
  \node[gp node right] at (8.241,1.910) {{$S_{21}(t)$}};
  \gpcolor{rgb color={0.902,0.624,0.000}}
  \draw[gp path] (8.425,1.910)--(9.341,1.910);
  \gpcolor{color=gp lt color border}
  \node[gp node right] at (8.241,1.602) {{$S_{61}(t)$}};
  \gpcolor{rgb color={0.898,0.118,0.063}}
  \draw[gp path] (8.425,1.602)--(9.341,1.602);
  %% coordinates of the plot area
  \gpdefrectangularnode{gp plot 1}{\pgfpoint{1.196cm}{0.616cm}}{\pgfpoint{11.447cm}{7.691cm}}
\end{tikzpicture}
%% gnuplot variables

    \end{center}
    \caption{方形波とそのフーリエ部分和.ただし$T=2\pi$,つまり$\omega=1$.}
    \label{fig:ex1-4}
\end{figure}

不連続な点$0,\pm\pi$以外では$N$が大きくなるとき$S_N(t)$は$f_4(t)$に収束することが観察できるが,$T=0,\pm\pi$の近くでは$S_N(t)$は強く振動し,$f(t)$よりも大きく飛び出した点が存在する.しかも$N$を大きくしてもこれは消えない.この現象をギッブス現象という.また,連続な点でも小さな振動は消えない.電子工学では,この飛び出しをオーバーシュート,振動をリップル,リンキングなどと呼ぶ.この飛び出しの高さは,$\Delta t=\omega t,\, (2N+1)\Delta t=\pi$とおくと
\[
    \begin{split}
        S_N(t)
        &=\frac{4}{\pi}\sum_{n=0}^N \frac{\sin(2n+1)\omega t}{2n+1} \\
        &=\frac{4}{\pi}\sum_{n=0}^N \frac{\sin(2n+1)\Delta t}{(2n+1)\Delta t}\Delta t \\
        &\overset{N\to\infty}{\longrightarrow}\frac{2}{\pi}\int_0^\pi \frac{\sin t}{t}dt=1.178978\ldots
    \end{split}
\]
に収束することが示せる.このギッブス現象は,どのような不連続点の周りでも発生する一般的な現象である.これを回避する方法として総和法がある.\qedsymbol

\paragraph{総和法}

$G(s)$を$[-1,1]$に台をもつ有界関数で$G(0)=1$かつ$s=0$で連続であるとする.このとき,重み関数$G(s)$に対応する部分和を
\[
    S_N^G(t)=\sum_{n=-N}^N G\mleft(\frac{n}{N}\mright)c[n]e^{in\omega t}\quad(t\in\mathbb{R})
\]
とおく.$G(s)$が上の条件を満たせば,形式的には
\[
    \lim_{N\to\infty} S_N^G(t)=\sum_{n=-\infty}^\infty c[n]e^{in\omega t}=f(t)
\]
となり,同じフーリエ展開を与えるはずである.しかし,収束の性質は$G(s)$の選び方によって異なってくる.このような手法を総和法という.

\subparagraph{フェイエル和(チェザロ和)}
\[
    G(s)=\begin{cases}
        1-|s| & (s\in[-1,1])         \\
        0     & (\textrm{otherwise})
    \end{cases}
\]
とおいたときの部分和
\[
    \sigma_N(t)=S_N^G(t)=\sum_{n=-N}^N\mleft(\frac{N-|n|}{N}\mright)c[n]e^{in\omega t}
\]
をフェイエル和という.

\begin{thm}\label{thm:1-14}
    $f$を連続な周期関数とすると,$\sigma_N(t)$は$N\to\infty$のとき$f$に一様収束する.
\end{thm}

\begin{proof}
    は\ref{sec:2-5}節に回す
\end{proof}
\subparagraph{ハン窓}

フーリエ級数の総和法はディジタル信号処理で実用的に大切な役割を果たす.この文脈では,重み関数$G(s)$は窓関数とよばれる.
\[
    H(s)=\begin{cases}
        \frac{1}{2}(1+\cos\pi s) & (s\in[-1,1])         \\
        0                        & (\textrm{otherwise})
    \end{cases}
\]
をハン窓という.定理\ref{thm:1-14}同様に,連続関数のフーリエ級数のハン窓による部分和は,もとの関数に一様収束する.

\newpage

\section{フーリエ級数の性質と応用}

\subsection{フーリエ級数と微分}

周期$T$の周期関数$f$が
\[
    f(t)=\sum_{n=-\infty}^\infty c[n]e^{in\omega t}
\]
とフーリエ級数展開されるとき,項別微分できるならば
\[
    f'(t)=\sum_{n=-\infty}^\infty in\omega c[n]e^{in\omega t}
\]
となる.つまり,$f'(t)$のフーリエ係数は$\{in\omega c[n]\}$となるはずだが,次のような場合にはこれが正当化できる.

\begin{thm}\label{thm:2-1}
    $f$を周期$T$の連続な周期関数で,$[0,T]$上で有限個の点を除いて微分可能,さらに$f'$は有界で積分可能であるとする.このとき$\{c[n]\}_{n=-\infty}^\infty$,$\{d[n]\}_{n=-\infty}^\infty$をそれぞれ$f,\,f'$のフーリエ係数とすれば
    \[
        d[n]=in\omega c[n] \quad (n\in\mathbb{Z})
    \]
    が成り立つ.
\end{thm}

\begin{proof}
    フーリエ係数の定義と部分積分により
    \[
        \begin{split}
            d[n]&=\frac{1}{T}\int_0^T f'(t) e^{-in\omega t}dt \\
            &=\frac{1}{T}\blueunderline{\mleft[f(t)e^{-in\omega t}\mright]_0^T}{周期性より0}-\frac{1}{T}\int_0^T f(t) (-in\omega)e^{-in\omega t}dt \\[2ex]
            &=in\omega c[n]
        \end{split}
    \]
\end{proof}

\begin{thm}\label{thm:2-2}
    $f$を$C^m$級周期関数($m\in\mathbb{N}$),$\{c[n]\}_{n=-\infty}^\infty$をフーリエ係数とする.このとき,$f$の$k$階微分$f^{(k)}$のフーリエ係数は$\{(i\omega n)^k c[n]\}_{n=-\infty}^\infty$($k=0,1,\ldots,m$)で与えられる.また,
    \begin{equation}\label{eq:2-1}
        \exists C>0\,\textrm{s.t.}\,|c[n]|\leq C(1+|n|)^{-m}\quad (n\in\mathbb{Z})
    \end{equation}
\end{thm}

\begin{proof}
    定理\ref{thm:2-1}を繰り返し用いることで前半は示せる.後半は定理\ref{thm:1-6}を利用して
    \[
        \sum_{n=-\infty}^\infty |(in\omega)^m c[n]|=\|f^{(m)}\|\leq\mleft(\sup_t|f^{(m)}(t)|\mright)^2\quad(n\in\mathbb{Z})
    \]
    ゆえに
    \[
        |(in\omega)^m c[n]|\leq\sup_t|f^{(m)}(t)|\quad(n\in\mathbb{Z})
    \]
    が成り立つ.したがって適当な$C$をとれば
    \[
        |c[n]|\leq\mleft(\omega^m\sup_t|f^{(m)}(t)|\mright)|n|^{-m}\leq C(1+|n|)^{-m}\quad(n\in\mathbb{Z})
    \]
    となる.
\end{proof}

\begin{thm}\label{thm:2-3}
    $f$を連続な周期関数,$\{c[n]\}_{n=-\infty}^\infty$をフーリエ係数とする.
    \begin{equation}\label{eq:2-2}
        \sum_{n=-\infty}^\infty |n^m||c[n]|<\infty\quad(m\in\mathbb{N})
    \end{equation}
    ならば,$f$は$C^m$級である.特に,$a>m+1$に対して
    \[
        |c[n]|<C(1+|n|)^{-a}\quad (n\in\mathbb{Z})
    \]
    が成り立つならば,$f$は$C^m$級である.
\end{thm}

\begin{proof}
    $m\geq 1$について\eqref{eq:2-2}を仮定すると
    \[
        \sum_{n=-\infty}^\infty |in\omega c[n]|=|\omega|\sum_{n=-\infty}^\infty |n||c[n]|<\infty
    \]
    であるから,$f(t)=\sum_n c[n]e^{in\omega t}$は項別微分ができる.すると
    \[
        f'(t)=\sum_{n=-\infty}^\infty in\omega c[n]e^{in\omega t}
    \]
    となり,右辺は仮定により$n$についての和が$t$に関して一様収束する.したがって$f'$は連続であり,$f$は$C^1$級である.これを繰り返し用いて前半の主張を得る.後半の主張は前半より直ちに導かれる.
\end{proof}

\begin{cor}\label{cor:2-4}
    $f$を連続な周期関数とする.$f$が$C^\infty$級関数であるための(必要)十分条件は
    \[
        \forall M>0,\exists C>0\,\textrm{s.t.}\,|c[n]|\leq C(1+|n|)^{-M}\quad(n\in\mathbb{Z})
    \]
    である.
\end{cor}

解析的な関数については,次のようなさらに強い結果が成り立つ\footnote{$C^\omega$級は$C^\infty$級よりも強い条件である.}.

\begin{thm}[ペイリー・ウィナーの定理]\label{thm:2-5}
    $f$が解析的な周期関数であるための必要十分条件は
    \begin{equation}\label{eq:2-3}
        \exists C>0,\exists \varepsilon>0\,\textrm{s.t.}\,|c[n]|\leq Ce^{-\varepsilon|n|}\quad(n\in\mathbb{Z})
    \end{equation}
    である.
\end{thm}

\begin{proof}
    $f(t)$に対し,複素平面の帯状領域$\{z\in\mathbb{C}:|\Im z|<a,\exists a>0\}$上の正則関数$f(z)$であって,$\mathbb{R}$上で$f(t)$に一致するものを考える.このとき,$f(z+2\pi)$と$f(z)$は$\mathbb{R}$上で一致するから,一致の定理によって$\{z\in\mathbb{C}:|\Im z|<a,\exists a>0\}$上で$f(z+2\pi)=f(z)$が成り立つ.ゆえに,$\forall \varepsilon\,\textrm{s.t.}\,0<\varepsilon<a$に対して
    \[
        \int_0^{\pm i\varepsilon/\omega}f(z)e^{-in\omega z}dz=\int_T^{T\pm i\varepsilon/\omega}f(z)e^{-in\omega z}dz\quad(n\in\mathbb{Z})
    \]
    となる.このことに注意すると,コーシーの積分定理より
    \[
        c[n]=\frac{1}{T}\int_0^T f(t)e^{-in\omega t}dt=\frac{1}{T}\int_0^T f(t\mp i\varepsilon/\omega)e^{-in\omega (t\mp i\varepsilon/\omega)}\quad(n\in\mathbb{Z})
    \]
    が得られる\footnote{$0\to \pm i\varepsilon/\omega\to T\pm i\varepsilon/\omega\to T \to 0$のような積分路で$f(t)e^{-in\omega t}$を積分すると$0$であることを利用する.}(複号同順).したがって,$C=\max\{|f(t\pm i\varepsilon/\omega)|:t\in[0,T]\}$とすれば,
    \[
        |c[n]|\leq\frac{e^{-\varepsilon|n|}}{T}\int_0^T|f(t\mp i\varepsilon)|dt\leq Ce^{-\varepsilon|n|}
    \]
    となり\eqref{eq:2-3}が成り立つ.

    逆に,\eqref{eq:2-3}が成り立つとき,定理3.16(何?)によって
    \[
        f(t)=\sum_{n=-\infty}^\infty c[n]e^{-in\omega t}
    \]
    である.$z\in \{z\in\mathbb{C}:|\Im z|<a,\exists a>0\}$に対して
    \[
        f(z)=\sum_{n=-\infty}^\infty c[n]e^{-in\omega z}
    \]
    と定めると,右辺は$\{z\in\mathbb{C}:|\Im z|<a,\exists a>0\}$上広義一様収束する.また,$\sum_{n=-\infty}^\infty C^{-\varepsilon|n|}<+\infty$であるので,weierstrassの解析関数に対する一様収束定理\footnote{weierstrassのM-testのことだと思う.}によって$f(z)$も$\{z\in\mathbb{C}:|\Im z|<a,\exists a>0\}$上の正則関数となる.$f(t)$は$f(z)$の実軸への制限であり,正則関数ならば解析関数である.
\end{proof}

\paragraph{定数係数の線形微分方程式の周期解}

$P$を
\[
    Pf(t)\coloneqq\sum_{j=0}^m a_j\frac{d^j f}{dt^j}(t)
\]
で定義される定数係数の線形常微分作用素とする.ここに$a_0,a_1,\ldots,a_m\in\mathbb{C}$は定数である.$g(t)$を周期$T$の周期関数とし
\[
    Pf(t)=g(t)
\]
を考える.

$f$が$m$階微分可能と仮定すると,定理\ref{thm:2-2}より,$\{c[n]\},\,\{d[n]\}$を$f,g$のフーリエ係数とすると
\[
    \sum_{j=0}^m a_j(i\omega n)^j c[n]=d[n]\quad(d\in\mathbb{Z})
\]
これは各$n$ごとに独立な方程式だから,$\{d[n]\}$から$\{c[n]\}$が求まるはずである.ここで,
\[
    Pf(t)=f''(t)+\lambda f(t)
\]
の場合を考えると
\[
    (-\omega^2 n^2+\lambda)c[n]=d[n]\quad(d\in\mathbb{Z})
\]
となる.

\subparagraph{Case1. $\lambda\notin\{n^2\omega^2|n\in\mathbb{Z}\}$の場合}

このとき
\[
    c[n]=\frac{1}{\lambda-n^2\omega^2}d[n]
\]
とかける.$g$がリプシッツ連続ならば,補題\ref{lem:1-11}から$\sum_n |d[n]|<+\infty$なので,
\[
    \sum_{n=-\infty}^\infty n^2|c[n]|=\sum_{n=-\infty}^\infty\mleft|\frac{n^2}{\lambda-n^2\omega^2}\mright||d[n]|\leq C\sum_{n=-\infty}^\infty|d[n]|<+\infty
\]
となる.よって,定理\ref{thm:2-3}より$f=\sum_n c[n]e^{in\omega t}$は$C^2$級関数である.また,唯一の周期回である.特に,$Pf=0$の周期解は$f=0$のみである.

\subparagraph{Case2. $\lambda=n_0^2\omega^2\,(\exists n_0\in\mathbb{Z})$の場合}

このとき$Pf=g$が解けるためには$d[n_0]=0$が必要である.このとき,$n\neq n_0$についてはCase1.の場合と同様に$d[n]$から$c[n]$が解ける.$c[n_0]$はどのように選んでも方程式は満たされる.すなわち,$A,B$を任意定数として
\[
    f(t)=\sum_{n\neq n_0}\frac{d[n]}{\lambda-n^2\omega^2}e^{in\omega t}+Ae^{in_0\omega t}+Be^{-in_0\omega t}
\]
となる.

さて,$g=0$とした場合は固有値問題
\[
    f''+\lambda f=0
\]
である.上の考察より,固有値は$\{n^2\omega^2 | n\in\mathbb{Z}\}$であり,固有関数は$e^{in\omega t}$であることがわかる.$n\neq 0$では固有値は二重に縮退しており,各$\lambda=n^2\omega^2$について固有関数は$\{a\cos(n\omega t)+b\sin(n\omega t)|a,b\in\mathbb{C}\}$という2次元の空間を張る.


\subsection{偏微分方程式への応用-1:熱方程式}

\paragraph{熱方程式の初期境界値問題}

端点の温度が0の,長さ$R>0$の棒を考える.

\begin{subequations}\label{eq:2-4}
    \begin{empheq}[left = {\empheqlbrace \,}, right = {}]{align}
        &\pdv{u}{t}(x,t)=\pdv[2]{u}{x}(x,t) && (x\in(0,R),t>0) \label{eq:2-4a} \\
        &u(0,t)=u(R,t)=0 && (t>0) \label{eq:2-4b} \\
        &u(x,0)=f(x) && (x\in[0,R]) \label{eq:2-4c}
    \end{empheq}
\end{subequations}

$u(x,t)$は時刻$t$,点$x$での温度を表し,$f(x)$は時刻$t=0$での初期温度である.\eqref{eq:2-4a}は熱方程式,\eqref{eq:2-4b}は境界条件,\eqref{eq:2-4c}は初期条件とよばれる.$t=0,x=0,R$では,境界条件と初期条件はつじつまが会っていなければならないので
\begin{equation}\label{eq:2-5}
    f(0)=f(R)=0
\end{equation}
が成立する必要がある.\eqref{eq:2-5}を両立条件という.

まず,$u$や$f$は十分に滑らか(少なくともリプシッツ連続)だと仮定して変数分離形の解を考える.
\[
    u(x,t)=\xi(x)\tau(t)
\]
として\eqref{eq:2-4a}に代入すると
\[
    \xi(x)\tau'(t)=\xi''(x)\tau(t)
\]
となる.$\xi(x)\neq 0,\tau(t)\neq 0$ならば,$x$に独立,$t$に独立な部分に分解でき,ある定数$a$が存在して
\[
    \frac{\tau'(t)}{\tau(t)}=\frac{\xi''(x)}{\xi(x)}=a
\]
が成り立つ.これより
\begin{empheq}[left = {\empheqlbrace \,}, right = {}]{align*}
    &\tau'(t)=a\tau(t) && (t>0)  \\
    &\xi''(x)=a\xi(x) && (x\in(0,R))
\end{empheq}
が導かれる.第1式はただちに解け
\[
    \tau(t)=\tau(0)e^{at}
\]
となる.第2式は,
\[
    \xi(x)=\begin{cases}
        \alpha e^{\sqrt{a}x}+\beta e^{-\sqrt{a}x} \quad(\alpha,\beta\in\mathbb{C}) & (a\neq 0) \\
        \alpha x+\beta \quad(\alpha,\beta \in\mathbb{R})                           & (a=0)
    \end{cases}
\]
となる.$a=0$のとき,境界条件は満たされないので以下$a\neq 0$とする.境界条件\eqref{eq:2-4b}より,
\begin{empheq}[left = {\empheqlbrace \,}, right = {}]{align*}
    & \alpha+\beta=0                              \\
    & \alpha e^{\sqrt{a}R}+\beta e^{-\sqrt{a}R}=0
\end{empheq}
が成り立つ必要がある.$(\alpha,\beta)\neq (0,0)$であるためには
\[
    \det\mqty(1 & 1 \\ e^{\sqrt{a}R} & e^{-\sqrt{a}R})=e^{-\sqrt{a}R}-e^{\sqrt{a}R}=0
\]
つまり
\[
    e^{2\sqrt{a}R}=1
\]
が満たされていなければならない.よって,$2\sqrt{a}R\in 2\pi i(\mathbb{Z}\setminus\{0\})$,すなわち
\[
    a=-\mleft(\frac{\pi n}{R}\mright)^2\quad(n\in\mathbb{Z}\setminus\{0\})
\]
また,$\beta=-\alpha$であるので
\[
    \xi(x)=\alpha e^{i\pi nx/R}-\alpha e^{-i\pi nx/R}=(2i\alpha)\sin\mleft(\frac{\pi n}{R}x\mright)
\]
となる.以上より,$n\in\mathbb{Z}\setminus\{0\}$について
\[
    u_n(x,t)=e^{-(\pi n/R)^2 t}\sin\mleft(\frac{\pi n}{R}x\mright)
\]
が解となることがわかった.このようにして得られた解の線型結合も解であるので,$\{\tilde b[n]\}$を任意定数として\begin{equation}\label{eq:2-6}
    u(x,t)=\sum_{n=1}^\infty \tilde b[n]\sin\mleft(\frac{\pi n}{R}x\mright)e^{-(\pi n/R)^2 t}
\end{equation}
も(収束すれば,形式的には)\eqref{eq:2-4}を満たすことがわかる.\eqref{eq:2-6}の$t=0$での値は
\[
    u(x,0)=\sum_{n=1}^\infty \tilde b[n]\sin\mleft(\frac{\pi n}{R}x\mright)
\]
となる.

ここで$f(t)$を周期$2R$の周期関数に拡張する.すなわち
\[
    \begin{split}
        f(x)     & =f(-x)\quad(0\leq x\leq R)                 \\
        f(x+2mR) & =f(x)\quad (-R\leq x\leq R,m\in\mathbb{Z})
    \end{split}
\]
とする.両立条件\eqref{eq:2-5}より,もとの$f$が連続であれば拡張された$f$も連続であり,$f$は奇関数だから
\[
    f(x)=\sum_{n=1}^\infty b[n]\sin\mleft(\frac{\pi n}{R}x\mright)
\]
ただし
\[
    b[n]=\frac{1}{R}\int_{-R}^R f(x)\sin\mleft(\frac{\pi n}{R}x\mright) dx=\frac{2}{R}\int_0^R f(x)\sin\mleft(\frac{\pi n}{R}x\mright) dx
\]
と$f$を正弦フーリエ変換できる.$u(x,0)$が正弦フーリエ変換された$f(x)$と等しいためには$\tilde b[n]=b[n]$とならなくてはならない.したがって,$\{b[n]\}$を上で定められた数列として
\begin{equation}\label{eq:2-6'}
    u(x,t)=\sum_{n=1}^\infty b[n]\sin\mleft(\frac{\pi n}{R}x\mright)e^{-(\pi n/R)^2 t}\tag{2.6'}
\end{equation}
が\eqref{eq:2-4}の形式的な解を与えることがわかる.実際,$f$がリプシッツ連続なら,この主張は正当化でき,さらに\eqref{eq:2-4}の解はこれ以外にないことが証明できる.

\begin{thm}\label{thm:2-5}
    $f$を$[0,R]$上のリプシッツ連続な関数で,両立条件\eqref{eq:2-5}をみたすと仮定する.このとき,$\omega=\pi/R$として
    \[
        \begin{split}
            b[n]&=\frac{2}{R}\int_0^R f(x)\sin(n\omega x)dx\quad(n\geq 1) \\
            u(x,t)&=\sum_{n=1}^\infty b[n]\sin(n\omega x)e^{-n^2\omega^2 t}
        \end{split}
    \]
    とおけば,$u(x,t)$は$[0,R]\times [0,\infty)$で連続,$(0,R)\times(0,\infty)$で無限回微分可能であり,\eqref{eq:2-4}をみたす.さらに,$[0,R]\times [0,\infty)$で連続,$(0,R)\times(0,\infty)$で2回連続微分可能な唯一の\eqref{eq:2-4}の解である.
\end{thm}

\begin{proof}
    これの証明には次の補題が必要となるので後で書く.
\end{proof}

\begin{lem}[エネルギー不等式]\label{lem:2-7}
    $w(x,t)$が$x$について2回連続微分可能,$t$について1回連続微分可能で,熱方程式の初期境界値問題\eqref{eq:2-4}を満たすならば
    \begin{equation}
        \int_0^R w(x,t)^2 dx\leq \int_0^R f(x)^2 dx\quad(t>0)
    \end{equation}
    が成り立つ.
\end{lem}

\begin{proof}
    \[
        \begin{split}
            \pdv{t}\int_0^R w(x,t)^2 dx
            &=2\int_0^R\mleft(\pdv{t}w(x,t)\mright)w(x,t)dx \\
            &=2\int_0^R \mleft(\pdv[2]{x} w(x,t)\mright)w(x,t)dx \\
            &\bluenoteunderleft{=}{部分積分と境界条件\eqref{eq:2-4b}}-2\int_0^R\mleft|\pdv{x}w(x,t) \mright|^2 dx \\
            &\leq 0
        \end{split}
    \]
    したがって
    \[
        \int_0^R w(x,t)^2 dx\leq  \int_0^R w(x,0)^2 dx =\int_0^R f(x)^2 dx
    \]
\end{proof}

\setcounter{definition}{5}

\begin{thm}[(再掲)]
    $f$を$[0,R]$上のリプシッツ連続な関数で,両立条件\eqref{eq:2-5}をみたすと仮定する.このとき,$\omega=\pi/R$として
    \[
        \begin{split}
            b[n]&=\frac{2}{R}\int_0^R f(x)\sin(n\omega x)dx\quad(n\geq 1) \\
            u(x,t)&=\sum_{n=1}^\infty b[n]\sin(n\omega x)e^{-n^2\omega^2 t}
        \end{split}
    \]
    とおけば,$u(x,t)$は$[0,R]\times [0,\infty)$で連続,$(0,R)\times(0,\infty)$で無限回微分可能であり,\eqref{eq:2-4}をみたす.さらに,$[0,R]\times [0,\infty)$で連続,$(0,R)\times(0,\infty)$で2回連続微分可能な唯一の\eqref{eq:2-4}の解である.
\end{thm}

\begin{proof}
    あとでしっかり書く
    \[
        \forall M>0,\exists C_M>0\,\textrm{s.t.}\, e^{-s}(1+s)^M\leq C_M
    \]
    と補題\ref{lem:1-11}とから
    \[
        |b[n]e^{-n^2\omega^2 t}|\leq C_M(1+n^2\omega^2 t)^{-M}
    \]
    系\ref{cor:2-4}により$u(x,t)$は$x$について$C^\infty$級

    また$\forall K\in\mathbb{N},\forall T>0,\exists C_{K,T}>0\,(\textrm{s.t.}\, \forall t\in[0,T])$
    \[
        |(n^2 \omega^2)^K b[n]e^{-n^2\omega^2 t}\sin(n\omega x)|\leq C_{K,T}(1+n^2\omega^2 t)^{-2}\quad(n\in\mathbb{Z})
    \]
    より,$\forall t\in[0,T]$について
    \[
        \pdv[K]{u}{t}=\sum_{n=1}^\infty (-n^2\omega^2)^K b[n]e^{-n^2\omega^2 t}\sin(n\omega x)
    \]
    は絶対収束(定理2.3の証明を参照)
    $u(x,t)$は$t$について$C^\infty$級
\end{proof}

\subsection{偏微分方程式への応用-2:ディリクレ問題}

省略

\setcounter{equation}{14}
\setcounter{definition}{10}

\subsection{積のフーリエ級数展開とたたみこみ}

$f$を周期$T$の周期関数とするとき,$f$のフーリエ係数を
\[
    (\mathcal{F}f)[n]=\frac{1}{T}\int_0^T f(t)e^{-in\omega t}dt\quad(n\in\mathbb{Z})
\]
とかくことにする\footnote{つまり$c[n]=(\mathcal{F}f)[n]$.}.すなわち,$\mathcal{F}$は周期関数の空間$X$から数列の集合への,フーリエ係数を対応させる線形写像\footnote{作用素ということもある.}である.逆に,$c=\{c[n]\}_{n=-\infty}^\infty$とするとき
\[
    (\mathcal{F}^* c)(t)=\sum_{n=-\infty}^\infty c[n]e^{in\omega t}\quad(t\in\mathbb{R})
\]
と書くことにする.$\mathcal{F}^* c$は一般には収束するとは限らないが,$l^1$-条件:
\begin{equation}\label{eq:2-15}
    \sum_{n=-\infty}^\infty |c[n]|<+\infty
\end{equation}
が満たされれば$\mathcal{F}^* c$は一様収束する.

ここで,$l^1(\mathbb{Z})=\{c=\{c[n]\}_{n=-\infty}^{\infty} | \sum_{n=-\infty}^\infty |c[n]|<+\infty  \}$とかく\footnote{つまり$l^1(\mathbb{Z})$は$l^1$-条件をみたす数列全体である.同様に$l^2(\mathbb{Z})$数列空間なども考えることができる.}.
$c\in l^1(\mathbb{Z})$ならば$\mathcal{F}^* c$は一様収束し,$\mathcal{F}\mathcal{F}^* c=c$となる(補題\ref{lem:1-12}).また,$f$がリプシッツ連続ならば$\mathcal{F}f\in l^1(\mathbb{Z})$,$\mathcal{F}^*\mathcal{F} f=f$(補題\ref{lem:1-11}).$f\in X$ならば平均収束の意味で$\mathcal{F}^*\mathcal{F} f=f$(定理\ref{thm:1-6}).以上の意味で
\[
    \mathcal{F}^*=\mathcal{F}^{-1}
\]
が成り立つ.\footnote{写像だと思えば逆写像があると捉えられる.ただし必ずしも逆写像があるわけではなく,定義域を制限させるなどする必要がある.例えばリプシッツ連続な関数に$\mathcal{F}$を作用させた後に$\mathcal{F}^*$を作用させてもリプシッツ連続な関数になるとは限らない.}

以下では$X$を周期$T$のリプシッツ連続な関数全体とし,$\mathcal{F}:X\to l^1(\mathbb{Z})$を考える.

\paragraph{数列のたたみこみ}

\begin{dfn*}[数列のたたみこみ]
    一般に,数列$c,d$に対して
    \begin{equation}\label{eq:2-16}
        p[n]=\sum_{m=-\infty}^\infty c[m]d[n-m]\quad(n\in\mathbb{Z})
    \end{equation}
    で与えられる数列$p=\{p[n]\}$をたたみこみと言い,$c*d$とかく.
\end{dfn*}

$c,d\in l^1(\mathbb{Z})$ならば$c*d\in l^1(\mathbb{Z})$である\footnote{
$\sum_{n=-\infty}^\infty|(c*d)[n]|\leq \sum_{n=-\infty}^\infty \sum_{m=-\infty}^\infty |c[m]||d[n-m]|=\sum_{m=-\infty}^\infty |c[m]|\sum_{n=-\infty}^\infty |d[n]|<\infty$.
}.また,$c*d=d*c$が成り立ち,たたみこみは可換である\footnote{
$(c*d)[n]=\sum_{m=-\infty}^\infty c[m]d[n-m]=\sum_{k=-\infty}^\infty c[n-k]d[k]=(d*c)[n]$.
}.

\begin{thm}\label{thm:2-11}
    $f,g$を周期$T$の連続な周期関数で,$\mathcal{F}f,\mathcal{F}g\in l^1(\mathbb{Z})$をみたすとする.このとき,
    \[
        \mathcal{F}(fg)=(\mathcal{F}f)*(\mathcal{F}g)
    \]
    が成り立つ.
\end{thm}

\begin{proof}
    $c=\mathcal{F}f,d=\mathcal{F}g\in l^1(\mathbb{Z})$とする.このとき
    \[
        \begin{split}
            f(t)g(t)&=\sum_{m=-\infty}^\infty c[m]e^{im\omega t}\sum_{n=-\infty}^\infty d[n]e^{in\omega t} \\
            &=\sum_{m=-\infty}^\infty\sum_{n=-\infty}^\infty c[m]d[n]e^{i(n+m)\omega t}
        \end{split}
    \]
    とかける.仮定より絶対収束し,無限和の取り方は自由に変えられるので
    \[
        f(t)g(t)=\sum_{k=-\infty}^\infty\mleft(\sum_{m=-\infty}^\infty c[m]d[k-m]\mright)e^{ik\omega t}
    \]
    となる.ゆえに積$f(t)g(t)$のフーリエ係数は
    \[
        p[n]=\sum_{m=-\infty}^\infty c[m]d[n-m]\quad(n\in\mathbb{Z})
    \]
    となるが,これは$p=c*d$を意味しており,$p=\mathcal{F}(fg),c=\mathcal{F}f,d=\mathcal{F}g$だったので$\mathcal{F}(fg)=(\mathcal{F}f)*(\mathcal{F}g)$であることが示せた.
\end{proof}

\paragraph{関数のたたみこみ}

\begin{dfn*}[関数のたたみこみ]
    $f,g$を周期$T$の周期関数とする.この$f,g$に対して$f$と$g$のたたみこみ$f*g(t)$を
    \[
        f*g(t)=\int_0^T f(s)g(t-s)ds\quad(t\in\mathbb{R})
    \]
    で定義する.
\end{dfn*}

数列の場合と同様に$f*g=g*f$が成り立ち,$f,g$が有界ならば$f*g$も有界である.また,$f,g$が有界でなくても$\|f\|,\|g\|<+\infty$ならば$f*g$は有界.

\begin{thm}\label{thm:2-12}
    $f,g$を周期$T$の有界な周期関数とする.このとき
    \[
        (\mathcal{F}f)[n]\cdot (\mathcal{F}g)[n]=\frac{1}{T}\mathcal{F}(f*g)[n]\quad(n\in\mathbb{Z})
    \]
    が成り立つ.また,$c,d\in l^1(\mathbb{Z})$ならば,
    \[
        \mathcal{F}^*(cd)(t)=\frac{1}{T}(\mathcal{F}^*c)*(\mathcal{F}^*d)(t)\quad(t\in\mathbb{R})
    \]
    が成り立つ.
\end{thm}

\begin{proof}
    前半は
    \[
        \begin{split}
            (\mathcal{F}f)[n]\cdot (\mathcal{F}g)[n]
            &=\frac{1}{T}\int_0^T f(t)e^{-in\omega t}dt\cdot\frac{1}{T}\int_0^T g(s)e^{-in\omega s}ds \\
            &=\frac{1}{T^2}\int_0^T\int_0^T f(t)g(s)e^{-im\omega(t+s)}dsdt \\
            &=\frac{1}{T^2}\int_0^T\int_{t-T}^t f(t)g(u-t)e^{-in\omega u}dudt \\
            &=\frac{1}{T}\int_0^T\mleft(\frac{1}{T}\int_0^T f(t)g(u-t)dt\mright)e^{-in\omega u}du \\
            &=\frac{1}{T}\int_0^T \mleft(\frac{1}{T}(f*g)(u)\mright) e^{-in\omega u}du \\
            &=\mathcal{F}\mleft(\frac{1}{T}(f*g)\mright)[n] \\
            &=\frac{1}{T}\mathcal{F}(f*g)[n]
        \end{split}
    \]
    後半は
    \[
        \begin{split}
            \mathcal{F}^*(cd)(t)&=\sum_{n=-\infty}^\infty c[n]d[n]e^{in\omega t} \\
            &=\frac{1}{T}\int_0^T \sum_{m=-\infty}^\infty\sum_{n=-\infty}^\infty c[m]d[n] e^{i (m-n)\omega s}ds \cdot e^{in\omega t} \footnotemark \\
            &=\frac{1}{T}\int_0^T \sum_{m=-\infty}^\infty c[m]e^{im\omega s}\sum_{n=-\infty}^\infty d[n]e^{in\omega (t-s)} ds \\
            &=\frac{1}{T}\int_0^T(\mathcal{F}^*c)(s)\,(\mathcal{F}^*d)(t-s)ds \\
            &=\frac{1}{T}(\mathcal{F}^*c)*(\mathcal{F}^*d)(t)
        \end{split}
    \]
    からわかる.
    \footnotetext{
        直交性(定理\ref{thm:1-1})により$m\neq n$のとき,この積分は$0$になる.$m=n$のときは積分値が$T$になることにも注意.また,$c,d\in l^1(\mathbb{Z})$なので$\|\mathcal{F}^*c\|$と$\|\mathcal{F}^*d\|$は有界で$(\mathcal{F}^*c)*(\mathcal{F}^*d)(t)$も有界.
    }
\end{proof}

\subsection{フーリエ級数の総和法・再論}\label{sec:2-5}

$G(s)$を重み関数($\supp G=[-1,1],G(0)=1,s=0$で連続)とすると
\[
    g_N(t)=\frac{1}{T}\mathcal{F}^*\mleft(G\mleft(\frac{n}{N}\mright)\mright)(t)=\frac{1}{T}\sum_{n=-N}^NG\mleft(\frac{n}{N}\mright)e^{in\omega t}
\]
とおけば,$f$のフーリエ級数の重み関数$G$に対応する部分和$S_N^G(t)$は
\[
    S_N^G(t)=\mathcal{F}^*\mleft(G\mleft(\frac{n}{N}\mright)(\mathcal{F}f)[n]\mright)(t)=(g_N*f)(t)=\int_0^T g_N(s)f(t-s)ds
\]
となる.

\setcounter{example}{6}

\begin{ex}[フーリエ部分和]
    フーリエ部分和の対応する重み関数$G$は
    \[
        G(s)=\begin{cases}
            1 & (|s|\leq 1) \\
            0 & (|s|>1)
        \end{cases}
    \]
    で与えられた.このとき対応する周期関数はディリクレ核
    \[
        D_N(t)=\frac{1}{T}\sum_{n=-N}^N e^{in\omega t}=\begin{cases}
            \frac{1}{T}\frac{\sin(N+\frac{1}{2})\omega t}{\sin\frac{1}{2}\omega t} & (t\notin T\mathbb{Z}) \\
            \frac{1}{T}(2N+1)                                                      & (t\in T\mathbb{Z})
        \end{cases}
    \]
    となり,これを用いると
    \[
        S_N(t)=D_N*f(t)
    \]
    と書くことができる.
\end{ex}

\begin{ex}[フェイエル和]
    フェイエル和の対応する重み関数$G$は
    \[
        G(s)=\begin{cases}
            1-|s| & (|s|\leq 1) \\
            0     & (|s|>1)
        \end{cases}
    \]
    で与えられた.このとき対応する周期関数はフェイエル核
    \[
        F_N(t)=\frac{1}{T}\sum_{n=-N}^N \frac{N-|n|}{N}e^{in\omega t}=\begin{cases}
            \frac{1}{TN}\mleft(\frac{\sin\frac{1}{2}N\omega t}{\sin\frac{1}{2}\omega t}\mright)^2 & (t\notin T\mathbb{Z}) \\
            \frac{N}{T}                                                                           & (t\in T\mathbb{Z})
        \end{cases}
    \]
    となり,これを用いると
    \[
        S_N(t)=F_N*f(t)
    \]
    と書くことができる.
\end{ex}

memo:定理2.6未完成,関数のたたみこみの性質を追記する,例2.7,2.8を追記する












\end{document}
